%
\chapter
{FÁGUÐ FÖLL}
 
%\kaflahaus{Fáguð föll}



\section{Markgildi og samfelld föll}

\subsection*{Skífur og hringir}

Áður en lengra er haldið skulum við innleiða rithátt fyrir skífur.  
{\it Opna skífu\index{opin skífa}\index{skífa!opin}} með miðju $\alpha$ og geisla
$\varrho$ táknum við með
$$ S(\alpha,\varrho)=\set{z\in \C; |z-\alpha|<\varrho}, $$
{\it lokaða skífu\index{lokuð skífa}\index{skífa!lokuð}} með miðju $\alpha$ og geisla 
$\varrho$ táknum við með
$$ \overline S(\alpha,\varrho)=\set{z\in \C; |z-\alpha|\leq\varrho} $$
og {\it gataða opna skífu\index{götuð opin skífa}\index{skífa!götuð opin}} með miðju $\alpha$ og
geisla $\varrho$ táknum við með
$$ S^*(\alpha,\varrho)=\set{z\in \C; 0<|z-\alpha|<\varrho}. $$


Athugið að fallið $[a,b]\ni \theta\mapsto \alpha+\varrho e^{i\theta}$ stikar
hringboga með miðju $\alpha$ og geislann $\varrho$ frá punktinum
$\alpha+\varrho e^{ia}$
til punktsins $\alpha+\varrho e^{ib}$ og að það stikar heilan hring ef 
$b-a=2\pi$.

\subsection*{Opin og lokuð mengi}

Hlutmengi $X$ í $\C$ er sagt vera {\it opið} ef um sérhvern punkt $a\in X$
gildir að til er opin skífa $S(a,r)$ sem er innihaldin í $X$.
Hlutmengi  $X$ í $\C$ er sagt vera {\it lokað } ef fyllimengi þess
$\C\setminus X$ er opið.  Þá er ljóst að mengi $X$ er lokað þá og því
aðeins að um sérhvern punkt $a$ í fyllimenginu $\C\setminus X$ gildir
að til er $r>0$ þannig að $S(a,r)\subset \C\setminus X$.  


{\it Jaðar} hlutmengis $X$ í $\C$ samanstendur af öllum punktum
$a\in \C$ þannig að sérhver opin skífa $S(a,r)$ með $r>0$ sker bæði
$X$ og $\C\setminus X$.  Við táknum jaðar $X$ með $\partial X$.  
Ef $X$ er opið, þá er $\partial X\subset \C\setminus X$.
Ef $X$ er lokað, þá er $\partial X\subset X$.   

Punktur $a\in \C$ nefnist {\it þéttipunktur} mengisins $X$ ef um
sérhvert $r>0$ gildir að gataða opna skífan $S^*(a,r)$ inniheldur
punkta úr $X$.  


Hlutmengi $X$ í $\C$ er sagt vera
{\it samanhangangi} ef um sérhverja tvo punkta $a$ og $b$ í 
$X$ gildir að til er samfelldur ferill  $[0,1]\ni t\mapsto \gamma(t)\in \C$ sem er
innihaldinn í $X$.   Opið samanhangandi mengi nefnist {\it svæði}.  


Athugið að sérhver opin skífa er svæði, því sérhverja tvo punkta í
henni má
tengja saman með línustriki.  Lokaðar skífur eru ferilsamanhangandi,
og sama er að segja um gataðar skífur.


\subsection*{Markgildi}

Látum nú $X$ vera hlutmengi í $\C$ og $f:X\to \C$ vera fall.  Við
segjum að $f(z)$ stefni á tvinntöluna $L$ þegar  $z$ stefnir á $a$, ef $a$ er
þéttipunktur í $X$ og fyrir sérhvert $\varepsilon>0$ gildir að til er
$\delta>0$ þannig að 
$$
|f(z)-L|<\varepsilon \qquad \text{ fyrir öll } z\in X\cap S^*(a,\delta).
$$
Við köllum þá töluna $L$ {\it markgildi $f$ þegar $z$ stefnir á $a$}
og skrifum 
$$
\lim_{z\to a}f(z)=L  \qquad \text{ eða } \quad f(z)\to L \text{ ef }
z\to a.
$$
Við höfum nokkrar reiknireglur fyrir markgildi:  Ef $f$ og $g$ eru
tvinngild föll sem skilgreind eru á menginu $X$, $\lim_{z\to a}f(z)=L$
og $\lim_{z\to a}g(z)=M$, þá er 
\begin{gather*}
\lim_{z\to a}(f(z)+g(z))=\lim_{x\to a}f(z)+\lim_{x\to a}g(z)=L+M,\\
\lim_{z\to a}(f(z)-g(z))=\lim_{x\to a}f(z)-\lim_{x\to a}g(z)=L-M,\\
\lim_{z\to a}(f(z)g(z))=\big(\lim_{x\to a}f(z)\big)\big(\lim_{x\to
a}g(z)\big)=LM\\
\lim_{z\to a}\dfrac{f(z)}{g(z)}=\dfrac{\lim_{x\to a}f(z)}{\lim_{x\to
a}g(z)}=\dfrac LM.
\end{gather*}
Í síðustu formúlunni þarf að gera ráð fyrir að $M\neq 0$.


\subsection*{Samfelldni}

Fallið $f:X\to \C$ er sagt vera samfellt í punktinum $a\in X$ ef
$$
\lim_{z\to a}f(z)=f(a).
$$


Af reiknireglunum fyrir markgildi leiðir að ef $f$ og $g$ eru föll
á mengi $X$ með gildi í $\C$ sem eru samfelld í punktinum $a\in X$, þá
eru $f+g$, $f-g$, $fg$ og $f/g$ samfelld í $a$ og 
\begin{gather*}
\lim_{x\to a}(f(z)+g(z))=f(a)+g(a),\\
\lim_{x\to a}(f(z)-g(z))=f(a)-g(a),\\
\lim_{x\to a}(f(z)g(z))=f(a)g(a),\\
\lim_{x\to a}\dfrac{f(z)}{g(z)}=\dfrac{f(a)}{g(a)}, 
\qquad \text{ef } \ g(a)\neq 0.
\end{gather*}
Ef $f:X\to \C$ og $g:Y\to \C$ eru föll,  $f(X)\subset Y$,
$a$ er þéttipunkur $X$, $b=\lim_{z\to a}f(z)$ er
þéttipunktur mengisins $Y$ og $g$ er samfellt í $b$, þá er
$$
\lim_{z\to a} g\circ f(z)=g(\lim_{z\to a}f(z)).
$$


\subsection*{Ritháttur fyrir hlutafleiður}



Ef $f$ er fall af breytistærðunum $x,y,z,\dots$, þá skrifum við
$$
{\partial}_xf=\dfrac{\partial f}{\partial x}, \qquad
{\partial}_yf=\dfrac{\partial f}{\partial y}, \qquad
{\partial}_zf=\dfrac{\partial f}{\partial z}, \ \dots
$$
og hærri afleiður táknum við með
$$
{\partial}_x^2f=\dfrac{\partial^2f}{\partial x^2}, \qquad
{\partial}_{xy}^2f=\dfrac{\partial^2f}{\partial x\partial y}, \qquad
{\partial}_{xxy}^3f=\dfrac{\partial^3f}{\partial x^2\partial y}, \ \dots.
$$

Í mörgum bókum eru hlutafleiður skrifaðar sem $f_{x}$, $f_y$ o.s.frv.
 Þessi
ritháttur hentar okkur illa, því við notum lágvísinn til þess að tákna
ýmislegt annað en hlutafleiður.  Mun skýrari ritháttur, sem við notum
þó ekki,  er að tákna
hlutafleiður með $f_x'$, $f_y'$ o.s.frv.  

\subsection*{Samfellt deildanleg föll}

\medskip\noindent
Við fjöllum mikið  um
samfelld og deildanleg föll  og 
þess vegna er mjög hagkvæmt að innleiða rithátt fyrir mengi allra falla
sem eru samfelld á einhverju mengi.
Ef $X$ er opið hlutmengi í $\C$ þá látum við $C(X)$ tákna mengi
allra samfelldra falla $f:X\to \C$.  Það er til mikilla þæginda að
gera frá byrjun ráð fyrir að föllin séu tvinntölugild.  Við látum
$C\sp m(X)$ tákna mengi allra $m$ sinnum\index{samfellt
deilanlegur!$m$ sinnum} samfellt
deildanlegra\index{samfellt deilanlegur} falla.
Hér er átt við að allar hlutafleiður fallsins $f$ af stigi $\leq m$
eru til og þar að auki samfelldar.  Við skrifum $C^0(X)=C(X)$ og
táknum mengi óendanlega oft deildanlegra falla með $C^{\infty}(X)$.


\section{Fáguð föll}


\noindent
Látum $f:X\to \C$ vera fall á opnu hlutmengi $X$ af $\C$.  
Ef við látum $z$ tákna tvinnbreytistærð með gildi í $X$, þá getum við
skrifað 
 \begin{equation*}f(z)=u(z)+iv(z)=u(x,y)+iv(x,y), \qquad z=x+iy=(x,y) \in X,
\label{4.2.1}
 \end{equation*}
þar sem föllin $u=\Re f$ og $v=\Im f$ eru raunhluti og þverhluti
fallsins $f$.
Við getum þá jafnframt litið á $f$ sem vigurgilt fall
af tveimur raunbreytistærðum
 \begin{equation*}f:X\to \R\sp 2, \qquad f(x,y)=(u(x,y), v(x,y)).
\label{4.2.2}
 \end{equation*}
Hugtök eins og samfelldni, deildanleiki og heildanleiki  eru
skilgreind eins og venjulega fyrir vigurgild föll.  Þetta þýðir að
$f$ er samfellt á $X$, $f\in C(X)$, þá og því aðeins að föllin $u$ og
$v$ séu samfelld á $X$, $u,v\in C(X)$.  Eins er $f$ $k$--sinnum
samfellt deildanlegt á $X$, $f\in C\sp k(X)$ þá og því aðeins að
$u,v\in C\sp k(X)$  og við skilgreinum hlutafleiður af $f$ sem tvinnföllin
\begin{gather*}
\partial_xf=\partial_xu+i\partial_xv, \qquad
\partial_yf=\partial_yu+i\partial_yv,\\
\partial\sp 2_xf=\partial\sp 2_xu+i\partial\sp 2_xv, \qquad
\partial\sp 2_{xy}f=\partial\sp 2_{xy}u+i\partial\sp 2_{xy}v,\qquad
\partial\sp 2_yf=\partial\sp 2_yu+i\partial\sp 2_yv.
\end{gather*}
Þannig er síðan haldið áfram eftir því sem deildanleiki $u$ og $v$
endist.  Nú ætlum við að innleiða nýtt deildanleikahugtak, þar sem
við lítum á breytistærðina sem {\it tvinntölu\index{tvinntala}} en ekki sem vigur:

\subsection*{$\C$-deildanleg föll}

\begin{sk}
Látum $f:X\to \C$ vera fall á opnu hlutmengi $X$ af $\C$.  
Við segjum
að $f$ sé {\it $\C$--deildanlegt\index{$\C$-deildanlegur}} í punktinum $a\in X$ ef markgildið
 \begin{equation*}\lim _{\substack{ h\to 0\\ h\in\C}}
 \dfrac{f(a+h)-f(a)}h  \label{4.2.3}
 \end{equation*}
er til.  Markgildið táknum við með $f\dash(a)$ og köllum það
{\it $\C$--afleiðu\index{$\C$-afleiða}} fallsins $f$ í punktinum $a$.  
Fall $f:X\to \C$ er sagt vera {\it fágað\index{fágað fall}} á opna menginu $X$ ef $f\in
C^1(X)$ og $f$ er $\C$--deildanlegt í sérhverjum punkti í $X$.  Við
látum $\O(X)$ tákna mengi allra fágaðra falla á $X$.  Við segjum að
$f$ sé {\it fágað í punktinum $a$} ef til er opin grennd $U$ um $a$ þannig
að $f$ sé fágað í $U$.  Fallið $f$ er sagt vera {\it heilt fall} ef
það er fágað á  öllu $\C$.
\end{sk}


Þessi skilgreining er eins og skilgreiningin af afleiðu falls af einni
raun\-breyti\-stærð.  


\begin{se}\label{se:sammfelldni}  Ef $f$ er $\C$--deildanlegt í $a$, þá er $f$ samfellt í $a$.
\end{se}



\subsection*{Reiknireglur fyrir $\C$-afleiður}

Reiknireglurnar fyrir $\C$-afleiður eru nánast þær sömu og
reiknireglurnar fyrir afleiður falla af einni raunbreytistærð.
Við tökum sannanirnar á þeim fyrir aftast í kaflanum:


\begin{se}\label{set4.2.3}
Látum $f,g:X\to \C$ vera föll, $a\in X$, $\alpha,\beta\in \C$ og
gerum ráð fyrir að $f$ og $g$ séu $\C$--deildanleg í $a$.  
Þá gildir

\smallskip\noindent
(i) $\alpha f+\beta g$ er $\C$--deildanlegt í $a$ og 
 $$
(\alpha f+\beta g)\dash(a)=\alpha f\dash(a)+\beta g\dash(a).
 $$

\smallskip\noindent
(ii) ({\it Leibniz-regla\index{regla Leibniz}\index{regla
Leibniz!fyrir tvinnföll}\index{Leibniz}}). $fg$ er $\C$--deildanlegt í $a$ og
$$
(fg)\dash(a)=f\dash(a)g(a)+f(a)g\dash(a).
$$

\smallskip\noindent
(iii) Ef $g(a)\neq 0$, þá er $f/g$ $\C$--deildanlegt í $a$ og 
$$(f/g)\dash(a)=\dfrac{f\dash(a)g(a)-f(a)g\dash(a)}{g(a)^2}.$$
\end{se}



\begin{fs}
$\O(X)$ er línulegt rúm yfir $\C$.
\end{fs}

Ef $f_1,f_2,\dots, f_n$ eru
$\C$--deildanleg í  $a$ og $\alpha_1,\dots,\alpha_n\in \C$, þá fáum við
með þrepun að
$f=\alpha_1f_1+\cdots+\alpha_nf_n$ er $\C$--deildanlegt í $a$ og
 $$f\dash(a)=\alpha_1 f_1\dash(a)+\cdots+\alpha_nf_n\dash(a).
 $$
Eins fáum við með þrepun að margfeldið $f=f_1f_2\cdots f_n$ er
$\C$--deildanlegt í $a$ og
 $$f\dash(a)= \sum_{j=1}^n f_j\dash(a)\bigg(\prod_{\substack{ k=1\\ k\neq
 j}}^n f_k(a)\bigg).
 $$
Athugið að af þessu leiðir formúlan
 $$\dfrac{f\dash(a)}{f(a)} =  \dfrac{f_1\dash(a)}{f_1(a)}+\cdots+
\dfrac{f_n\dash(a)}{f_n(a)}.
 $$


\begin{sy} (i) Allar margliður
 $$P(z)= a_0+a_1z+\cdots+a_mz^m, \qquad z\in \C,
 $$
eru fáguð föll á öllu $\C$ og afleiðan er
 $$P\dash(z)= a_1+2a_2z+\cdots+ma_mz^{m-1}, \qquad z\in \C.
 $$
Til þess að sjá þetta, þá athugum við fyrst að sérhvert fastafall
$f(z)=c$ er $\C$--deildanlegt í $z$ og 
 $$\dfrac {f(z+h)-f(z)}h=0,
 $$
sem gefur að $f\dash(z)=0$ fyrir öll $z\in \C$.  Næst athugum við 
fallið $g(z)=z$. Jafnan
 $$\dfrac{g(z+h)-g(z)}h=1
 $$
gefur að $g$ er $\C$--deildanlegt í sérhverjum punkti og
$g\dash(z)=1$.  Með því að beita setningu \ref{set4.2.3} (ii) og
þrepun fáum við síðan að fallið $h(z)=z^n$ er
$\C$--deildanlegt í sérhverjum punkti $z\in \C$ og að afleiða þess er
$h\dash(z)=nz^{n-1}$.  Að lokum fæst að sérhver margliða er fágað
fall, því línulegar samantektir af fáguðum föllum eru fáguð föll.


(ii) Sérhvert rætt fall $R=P/Q$, þar sem $P$ og $Q$ eru margliður, er
fágað fall á menginu $\set{z\in \C; Q(z)\neq 0}$
og
 $$R\dash(z)= \dfrac{P\dash(z)Q(z)-P(z)Q\dash(z)}{Q(z)^2}.
 $$
\end{sy}


Keðjureglan\index{keðjuregla fyrir fáguð föll} fyrir $\C$--deildanleg
föll er  eins og keðjureglan fyrir  raunföll:

\begin{se}\label{se:2.2.6}  Látum $X$ og $Y$ vera opin hlutmengi af $\C$, $f:X\to \C$ og
$g:Y\to \C$ vera föll, þannig að $f(X)\subset Y$, $a\in X$, $b\in Y$,
$b=f(a)$ og setjum
$$h=g\circ f.
$$ 
(i) Ef $f$ er $\C$--deildanlegt í $a$ og $g$ er $\C$--deildanlegt í
$b$, þá er $h$ $\C$--deildanlegt í $a$ og
 $$h\dash(a)=g\dash(b)f\dash(a).
 $$
(ii) Ef $g$ er $\C$--deildanlegt í $b$, $g\dash(b)\neq 0$, $h$ er
$\C$--deildanlegt í $a$ og $f$ er samfellt í $a$, 
þá er $f$ $\C$--deildanlegt í $a$ og 
$$ f\dash(a)=h\dash(a)/g\dash(b)$$.
 
\end{se}

Mikilvæg afleiðing af þessari setningu er:

\begin{fs}\label{fs:2.2.7}  
Látum $X$ og $Y$ vera opin hlutmengi af $\C$, $f:X\to Y$ 
vera gagntækt fall.  Ef $f$ er $\C$--deildanlegt í $a$ og
$f\dash(a)\neq 0$, þá er andhverfa fallið 
$f^{[-1]}$ $\C$--deildanlegt í $b=f(a)$ og
 \begin{equation*}\left(f^{[-1]}\right)\dash(b)= \dfrac 1{f\dash(a)}.\label{4.2.4}
 \end{equation*}
\end{fs}



\subsection*{Cauchy-Riemann-jöfnur}

Nú skulum við gera ráð fyrir því að $f$ sé $\C$--deildanlegt í punktinum
$a$ og huga að sambandinu milli $f\dash(a)$, ${\partial}_xf(a)$ og
${\partial}_yf(a)$. 
Ef við skrifum $a=\alpha+i\beta=(\alpha, \beta)$ og látum $h\to
0$ eftir  rauntölunum, þá fáum við 
\begin{align*}
f\dash(a)=&\lim_{\substack{h\to 0\\ h\in \R}}
\dfrac{u(\alpha+h,\beta)-u(\alpha,\beta)}h+i
\dfrac{v(\alpha+h,\beta)-v(\alpha,\beta)}h\\
=&\partial_xu(a)+i\partial_xv(a)=\partial_xf(a).\nonumber
\end{align*}
Ef við látum hins vegar $h\to 0$ eftir þvertölum, $h=ik$,
$k\in \R$, þá fáum við
\begin{align*}
f\dash(a)&=\lim_{\substack{k\to 0\\ k\in \R}}
\dfrac{u(\alpha,\beta+k)-u(\alpha,\beta)}{ik}+i
\dfrac{v(\alpha,\beta+k)-v(\alpha,\beta)}{ik}\\
&=-i(\partial_yu(a)+i\partial_yv(a))=-i\partial_yf(a).\nonumber
\end{align*}
Við höfum því:  

\begin{se}\label{set4.2.8}  Látum $f=u+iv:X\to \C$ vera fall af $z=x+iy$ á opnu hlutmengi
$X$ í $\C$.  Ef $f$ er $\C$--deildanlegt í $a\in X$, þá eru báðar
hlutafleiðurnar $\partial_xf(a)$ og $\partial_yf(a)$ til og
 \begin{equation*}f\dash(a)=\partial_xf(a)=-i\partial_yf(a).
\label{4.2.7}
 \end{equation*}
Þar með gildir {\it
Cauchy--Riemann--jafnan\index{Cauchy--Riemann!jafna}
\index{Cauchy--Riemann!jöfnur}\index{jafna!Cauchy--Riemann}}
\begin{equation}
\tfrac 12\big(\partial_xf(a)+i\partial_yf(a)\big)=0,
\label{4.2.8}
\end{equation}
og  hún jafngildir hneppinu
\begin{equation}\partial_xu(a)=\partial_yv(a), \qquad \partial_yu(a)=-\partial_xv(a).
\label{4.2.9}
\end{equation}
\end{se}


Hlutafleiðujafnan (\ref{4.2.8}) nefnist Cauchy--Riemann--jafna eins og
áður er getið.  Venja er að tala um jafngilda 
jöfnuhneppið (\ref{4.2.9}) sem  Cauchy--Riemann--jöfnur, í fleirtölu.


\bigskip\hrule\bigskip

\begin{sy}  Kannið hvort fallið 
$$
f(z)=(x^3-3xy^2+x-4)+i(3x^2y-y^3+y)
$$
 er fágað með því að athuga hvort
Cauchy-Riemann-jöfnurnar séu uppfylltar

\smallskip\noindent
{\it Lausn}:\   Við höfum
\begin{gather*}
u(x,y)=\Re f(x,y)=x^3-3xy^2+x-4,\\
v(x,y)=\Im f(x,y)=3x^2y-y^3+y,\\
\end{gather*}
Lítum nú á hlutafleiðurnar
\begin{alignat*}{2}
\dfrac{{\partial} u}{{\partial} x}&= 3x^2-3y^2+1, & \qquad
\dfrac{{\partial} u}{{\partial} y}&= -6xy,\\ 
\dfrac{{\partial} v}{{\partial} x}&= 6xy, &\qquad
\dfrac{{\partial} v}{{\partial} y}&= 3x^2-3y^2+1.
\end{alignat*}
Greinilegt er að Cauchy-Riemann-jöfnurnar eru uppfylltar,
${\partial} u/{\partial}x={\partial}v/{\partial}y$ og
${\partial} u/{\partial}y=-{\partial}v/{\partial}x$, og þar með er $f$
fágað fall. Athugið að $f(z)=z^3+z-4$, sem er margliða og þar með
$\C$-deildanlegt fall.
\end{sy}

\bigskip\hrule\bigskip

\subsection*{Wirtinger-afleiður}


Til þess að glöggva okkur betur á Cauchy--Riemann--jöfnunni, þá skulum
við rifja það upp að fall $f:X\to \R^2$ er sagt vera deildanlegt í
punktinum $a$, ef til er línuleg vörpun $L:\R^2\to \R^2$ þannig að
\begin{equation} \label{4.2.10}
\lim_{\substack{h\to 0\\ h\in \R^2}}
\dfrac{\| f(a+h)-f(a)-L(h)\|}{\|h\|}= 0,
\end{equation}
þar sem $\|z\|$ táknar lengd vigursins $z$.  Vörpunin $L$ er ótvírætt
ákvörðuð.  Hún nefnist afleiða $f$ í punktinum $a$ og er oftast táknuð
með $d_af$, $df_a$ eða $Df(a)$.    Með því að velja vigurinn $h$ af
gerðinni $t(1,0)$ og $t(0,1)$ og láta síðan $t\to 0$, þá sjáum við að hlutafleiðurnar 
${\partial}_xu(a)$, ${\partial}_yu(a)$, ${\partial}_xv(a)$ og 
 ${\partial}_yv(a)$ eru allar til og að fylki vörpunarinnar $d_af$ miðað
við grunninn $\set{(1,0), (0,1)}$ er
\begin{equation*}
\left[\begin{matrix} 
{\partial}_xu(a) & {\partial}_yu(a)\\
{\partial}_xv(a) & {\partial}_yv(a)
\end{matrix}\right].
\label{4.2.11}
\end{equation*} 
Þetta fylki nefnist {\it Jacobi--fylki\index{Jacobi-fylki}} $f$ í punktinum $a$.  Nú skrifum
við $z=(x,y)$, $a=({\alpha},{\beta})$ og sjáum að (\ref{4.2.10}) jafngildir
því að hægt sé að rita
\begin{equation*}
f(z)=\left[\begin{matrix}
u(a) \\ v(a)
\end{matrix}\right]+
\left[\begin{matrix}
{\partial}_xu(a) \\ {\partial}_xv(a)
\end{matrix}\right](x-{\alpha})+
\left[\begin{matrix}
{\partial}_yu(a) \\ {\partial}_yv(a)
\end{matrix}\right](y-{\beta})+
\|z-a\|F_a(z), \label{4.2.12}
\end{equation*}
þar sem $F_a:X\to \R^2$ er samfellt í $a$ og $F_a(a)=0$.  Nú skulum við
líta á $f$ sem tvinngilt fall $f=u+iv$.  Þá er þessi jafna jafngild
\begin{equation}
f(z)=f(a)+ {\partial}_xf(a)(x-{\alpha})+{\partial}_yf(a)(y-{\beta})
+(z-a)\varphi_a(z),
\label{4.2.13}
\end{equation}
þar sem $\varphi_a:X\to \C$ er samfellt í $a$ og $\varphi_a(a)=0$.  Nú
skrifum við 
$$
x-{\alpha}=\big((z-a)+\overline{(z-a)}\big)/2, \qquad
y-{\beta}=\big((z-a)-\overline{(z-a)}\big)/2i
$$ 
og fáum því 
\begin{multline*}
{\partial}_xf(a)(x-{\alpha})+{\partial}_yf(a)(y-{\beta})  \\
=\tfrac 12\big({\partial}_xf(a)-i{\partial}_yf(a)\big)(z-a)
+\tfrac 12\big({\partial}_xf(a)+i{\partial}_yf(a)\big)\overline{(z-a)}.
\end{multline*}

\begin{sk}  Við skilgreinum fyrsta stigs hlutafleiðuvirkjana
${\partial}_z={\partial}/{\partial}z$ og 
${\partial}_{\bar z}={\partial}/{\partial}\bar z$ með
\begin{equation*}
{\partial}_zf=\dfrac{{\partial}f}{{\partial} z}
=\tfrac 12\big({\partial}_xf-i{\partial}_yf\big) \quad \text{ og } \quad
{\partial}_{\bar z}f=\dfrac{{\partial}f}{{\partial}\bar z}
=\tfrac 12\big({\partial}_xf+i{\partial}_yf\big)
\label{4.2.14}
\end{equation*}
Tölurnar ${\partial}_zf(a)$ og ${\partial}_{\bar z}f(a)$ nefnast
{\it Wirtinger--afleiður\index{Wirtinger-afleiður}} fallsins $f$ í punktinum $a$ og virkinn
${\partial}_{\bar z}$ nefnist {\it
Cauchy--Riemann--virki\index{virki!Cauchy--Riemann}\index{Cauchy--Riemann!virki}}
\end{sk}

Nú höfum við umritað (\ref{4.2.13}) yfir í
\begin{equation}
f(z)=f(a)+{\partial}_zf(a)(z-a)+{\partial}_{\bar z}f(a)\overline{(z-a)}
+ (z-a)\varphi_a(z).
\label{4.2.15}
\end{equation}



Hugsum okkur nú að $f:X\to \C$ sé eitthvert fall og að til séu
tvinntölur $A$, $B$ og fall $\varphi_a:X\to \C$, samfellt í $a$ með
$\varphi_a(a)=0$, þannig að
\begin{equation}
f(z)=f(a)+A(z-a)+B\overline{(z-a)}+(z-a)\varphi_a(z).
\label{4.2.16}
\end{equation} 
Þá er greinilegt að $f$ er deildanlegt í $a$ með afleiðuna
$d_af(h)=Ah+B\bar h$ og
\begin{align*}
{\partial}_xf(a) &=
\lim_{\substack{ h\to 0\\ h\in \R}} \dfrac{f(a+h)-f(a)}h
=\lim_{\substack{ h\to 0\\ h\in \R}} A+B+\varphi_a(a+h) = A+B,\\
{\partial}_yf(a) &=
\lim_{\substack{ h\to 0\\ h\in \R}} \dfrac{f(a+ih)-f(a)}h
=\lim_{\substack{ h\to 0\\ h\in \R}} iA-iB+\varphi_a(a+ih) = i(A-B).
\end{align*}
Ef við leysum $A$ og $B$ út úr þessum jöfnum, þá fáum við
\begin{align*}
A&= \tfrac 12\big({\partial}_xf(a)-i{\partial}_yf(a)\big)
={\partial}_zf(a),\\
B&= \tfrac 12\big({\partial}_xf(a)+i{\partial}_yf(a)\big)
={\partial}_{\bar z}f(a).
\end{align*}
Við höfum nú sannað:

\begin{se}  Látum $X$ vera opið hlutmengi í $\C$, $a\in X$ og $f:X\to \C$ vera
fall.  Þá gildir:

\smallskip\noindent
(i) $f$ er deildanlegt í $a$ þá og því aðeins að til séu tvinntölur $A$,
$B$ og fall $\varphi_a:X\to \C$, samfellt í $a$ og með
$\varphi(a)=0$, þannig að (\ref{4.2.16}) sé uppfyllt.

\smallskip\noindent
(ii) $f$ er $\C$--deildanlegt í $a$ þá og því aðeins að $f$ sé
deildanlegt í $a$ og ${\partial}_{\bar z}f(a)=0$.  Þá er
$f\dash(a)={\partial}_zf(a)$.

\smallskip\noindent
(iii) $f$ er fágað í $X$ þá og því aðeins að $f$ sé samfellt deildanlegt
í $X$ og uppfylli Cauchy--Riemann--jöfnuna ${\partial}_{\bar z}f=0$ í
$X$. Við höfum þá
\begin{equation*}
f\dash=\dfrac{df}{dz}=\dfrac{\partial f}{\partial z}=\dfrac 12\bigg(
\dfrac{\partial f}{\partial x}-i\dfrac{\partial f}{\partial y}\bigg).
\label{4.2.17}
\end{equation*}
\end{se}



Reikningur með hlutafleiðunum með tilliti til $z$ og $\bar z$ er alveg
eins of reikningur með óháðu breytunum $x$ og $y$.  Ef fallið 
$f(z)=f(x+iy)$ er gefið með formúlu í $x$ og $y$, þá notum við
formúlurnar $x=(z+\bar z)/2$ og $y=(z-\bar z)/(2i)$ til þess að skipta
á óháðu breytunum $x$ og $y$ yfir í breyturnar $z$ og $\bar z$.  Til
þess að kanna hvort fall er fágað þá deildum við eins og þetta séu
óháðar breytur og könnum hvort 
$$
\dfrac{\partial f}{\partial\bar z}=0.
$$
Ef $\bar z$ kemur alls ekki fyrir í formúlunni, þá er $f$ fágað.

 
\bigskip\hrule\bigskip

\begin{sy}  Kannið hvort fallið $f(z)= |z|^2$ er fágað og reiknið út
$\C$-afleiðuna ef hún er til.  

\smallskip\noindent
{\it Lausn}: \  Við skrifum  $f(z)=|z|^2=z\bar z$ og
sjáum að  ${\partial}f/{\partial}\bar z=z$.  Cauchy-Riemann-jafnan er aðeins
uppfyllt í punktinum $z=0$ og því er fallið $f$ ekki fágað í grennd við
neinn punkt.
\end{sy}

\bigskip\hrule\bigskip

\begin{sy}  Kannið hvort fallið $f(z)= \Re z + i$ er fágað og reiknið
út $\C$-afleiðuna ef hún er til.

\smallskip
{\it Lausn}: \ Hér er $f(z)=\tfrac 12(z+\bar z)+i$ og 
${\partial} f/{\partial}\bar z=\tfrac 12$.  Fallið $f$ er því ekki
fágað.
\end{sy}


\bigskip\hrule\bigskip

 
\begin{sy}\label{sy:polhnit} ({\it Cauchy-Riemann-jöfnur á pólformi}): \  Leiðið út 
jöfnurnar 
$$\dfrac{\partial u}{\partial r}=\dfrac 1r\dfrac{\partial v}{\partial
\theta} \qquad \text{ og } \qquad 
\dfrac{\partial v}{\partial r}=-\dfrac 1r\dfrac{\partial u}{\partial
\theta} 
$$ 
frá jöfnunum
$$
\dfrac{\partial u}{\partial x}=\dfrac{\partial v}{\partial y}
\qquad \text{  og } \qquad 
\dfrac{\partial v}{\partial y}=-\dfrac{\partial v}{\partial x}.
$$      

\smallskip\noindent
{\it Lausn}: Samkvæmt skilgreiningu á afleiðunni $\partial u/\partial
r$ er 
$$
\dfrac{\partial u}{\partial r}=\dfrac{\partial}{\partial r} u(r\cos
\theta,r\sin \theta)=\dfrac{\partial u}{\partial x} \cdot \cos \theta
+\dfrac{\partial u}{\partial y}\cdot \sin \theta=\nabla u\cdot 
{\mathbf e}_r
$$
þar sem ${\mathbf e}_r=(\cos \theta,\sin \theta)$ er einingarvigurinn
í $r$-stefnu, og 
$$
\dfrac 1 r\dfrac{\partial u}{\partial \theta}=\dfrac 1r
\dfrac{\partial}{\partial \theta} u(r\cos
\theta,r\sin \theta)=\dfrac{\partial u}{\partial x}\cdot (-\sin \theta)+
\dfrac{\partial u}{\partial y}\cdot \cos \theta=\nabla u\cdot 
{\mathbf e}_\theta
$$
þar sem ${\mathbf e}_\theta=(-\sin \theta,\cos \theta)$ er einingarvigurinn
í $\theta$-stefnu.  Cauchy-Riemann-jöfnurnar segja að 
$\nabla u=(\partial v/\partial y,-\partial v/\partial x)$.  Þar með er
$$
\dfrac{\partial u}{\partial r}=\dfrac{\partial v}{\partial
y}\cdot \cos\theta -\dfrac{\partial v}{\partial x}\cdot \sin \theta= \nabla v\cdot
{\mathbf e}_\theta=\dfrac 1r \dfrac{\partial v}{\partial \theta}.
$$
og 
$$
\dfrac 1r\dfrac{\partial u}{\partial \theta}=\dfrac{\partial v}{\partial
y}\cdot (-\sin\theta) -\dfrac{\partial v}{\partial x}\cdot \cos \theta= -\nabla v\cdot
{\mathbf e}_r=-\dfrac{\partial v}{\partial r}.
$$
 Auðvelt er að snúa röksemdafærslunni við til þess að sanna að
af jöfnunum $\partial u/\partial r=(1/r)\partial v/\partial \theta$
og $(1/r)\partial u/\partial \theta =-\partial v/\partial r$,
leiði Cauchy-Riemann-jöfnurnar.

\end{sy}

\bigskip\hrule\bigskip

\begin{sy}
Það er áhugaverð staðreynd að $f(z)=e^z$ er ótvírætt
ákvarðað af tveimur eiginleikum: \ 
\begin{enumerate}
\item[(i)] $f(x+i0)=\lim_{y\to 0}f(x+iy)=e^x$, 
\item[(ii)] $f'(z)=f(z)$,
\end{enumerate}
þar sem gert er ráð fyrir að 
$f$ sé heilt fágað fall. Sannið þetta með því að nota
Cauchy-Riemann-jöfnurnar.

\smallskip\noindent
{\it Lausn:} \ Gefur okkur að $f$ sé heilt fágað fall, þ.e.a.s.~fágað
fall á öllu $\C$, sem uppfyllir (i) og (ii).  Við ætlum að sanna að þá
sé $f(z)=e^z$.  Setjum $u(x,y)=\Re f(z)$ og $v(x,y)=\Im f(z)$ þar sem
$z=x+iy$ og $x,y\in \R$.  Fyrst $f$ er $\C$-deildanlegt, þá fáum við jöfnurar
$f'(z)=\partial f/\partial x=-i\partial f/\partial y$ þetta gefur ásamt jöfnunni
$f'(z)=f(z)$ að
$$
\dfrac {\partial u}{\partial x}+i\dfrac {\partial v}{\partial x}=
-i\dfrac {\partial u}{\partial y}+\dfrac {\partial v}{\partial y}=u+iv.
$$ 
Við getum eins skrifað þetta sem tvær raunjöfnur
$$
\dfrac {\partial u}{\partial x}=\dfrac {\partial v}{\partial y}=u
\qquad \text{ og } \qquad \dfrac {\partial u}{\partial y}=-\dfrac
{\partial v}{\partial x}=-v.
$$
(Athugið að fyrri hlutinn af þessum jöfnum er Cauchy-Riemann
jöfnurnar.)  Nú ætlum við að sýna fram á að þessar jöfnur ásamt
skilyrðinu $f(x+i0)=e^x$ gefi okkur að $u(x,y)=e^x\cos y$ og
$v(x,y)=e^x\sin y$.  Til þess reiknum við út $\partial^2u/\partial
y^2$ og $\partial^2v/\partial y^2$,
$$
\dfrac{\partial^2u}{\partial y^2}=-\dfrac{\partial v}{\partial y}=-u
\quad \text{ og } \quad 
\dfrac{\partial^2v}{\partial y^2}=\dfrac{\partial u}{\partial y}=-v.
$$
Við höfum því sýnt að fyrir fast $x$ uppfylla $u$ og $v$ afleiðujöfnu
sem föll af $y$.  Við vitum hvernig lausnin á þessari jöfnu er
$$
u(x,y)=A(x)\cos y+ B(x)\sin y \qquad \text{ og } \qquad
v(x,y)=C(x)\cos y+D(x)\sin y.
$$
Skilyrðið $f(x+i0)=e^x$ gefur okkur að $u(x,0)=A(x)=e^x$ og
$v(x,0)=C(x)=0$.  Skilyrðið $\partial u(x,0)/\partial y=-v(x,0)=0$
gefur okkur að $B(x)=0$ og skilyrðið $\partial v(x,0)/\partial
y=u(x,0)=e^x$ gefur okkur $D(x)=e^x$.  Niðurstaðan er því að
$$
f(z)=u(x,y)+iv(x,y)=e^x\cos y+ie^x\sin y=e^z,\qquad z\in \C.
$$
\end{sy}

\bigskip\hrule\bigskip

\section{Samleitnar veldaraðir} 

\noindent
Einu dæmin um fáguð föll sem við höfum nefnt til þessa eru margliður
$P$, en þær eru fágaðar á öllu $\C$, og ræð föll $R=P/Q$, en þau eru
fáguð á $\C\setminus\set{z\in \C; Q(z)=0}$.  Nú ætlum við að bæta
verulega við dæmaforðann með því að sanna að öll föll, sem unnt er að
setja fram með samleitnum veldaröðum, séu fáguð á samleitniskífu
raðarinnar. 


Ef fallið $f$ er skilgreint á einhverju opnu mengi $Y$ á $\R$ og er
gefið með samleitinni veldaröð á $]a-{\varrho},a+{\varrho}[\subset Y$,
$$
f(x)=\sum\limits_{n=0}^{\infty} a_n(x-a)^n, \qquad 
x\in  ]a-{\varrho},a+{\varrho}[,
$$
þá er röðin samleitin á opnu skífunni $S(a,{\varrho})\subseteq \C$ og við getum 
framlengt skilgreiningarsvæði $f$ yfir á $S(a,{\varrho})$ með því að setja
$$
f(z)=\sum\limits_{n=0}^{\infty} a_n(z-a)^n, \qquad 
z\in  S(a,{\varrho}).
$$

\begin{sk}  Fall sem skilgreint er á opnu mengi $U$ á rauntalnaásnum
er sagt vera {\it raunfágað } ef það  hefur þann eiginleika að  
í grennd um sérhvern punkt í $U$ er hægt að setja $f$ fram með
samleitinni  veldaröð.
\end{sk}

Fallið $z\mapsto 1/(1-z)$ er fágað á $\C\setminus\{1\}$ og það 
gefið með geómetrísku röðinni
$$
\dfrac 1{1-z}=\sum_{n=0}^\infty z^n, \qquad z\in S(0,1). 
$$
Veldisvísisfallið, hornaföllin og breiðbogaföllin eru öll gefin með
samleitnum veldaröðum á $\R$ og fáguðu framlengingar þeirra eru því 
gefnar með sömu röðum á öllu $\C$
\begin{gather*}
\exp z =e\sp z = \sum_{n=0}^\infty \dfrac 1{n!}z^n, \\
\cos z = \sum_{k=0}\sp \infty \dfrac {(-1)\sp k}{(2k)!}z\sp{2k}, \quad
\sin z = \sum_{k=0}\sp \infty \dfrac {(-1)\sp k}{(2k+1)!}z\sp{2k+1},
\quad\\
\cosh z = \sum_{k=0}\sp \infty \dfrac {1}{(2k)!}z\sp{2k}, \quad
\sinh z = \sum_{k=0}\sp \infty \dfrac {1}{(2k+1)!}z\sp{2k+1}.
\end{gather*}


\begin{se}\label{set4.3.1} Gerum ráð fyrir að $X$ sé opið hlutmengi af $\C$, 
$S(\alpha,\varrho)\subset X$, að $f:X\to \C$
sé fall, sem  gefið er á $S(\alpha,\varrho)$ með samleitinni veldaröð,
 $$f(z)=\sum_{n=0}^\infty a_n(z-\alpha)^n, \qquad z\in S(\alpha,\varrho).
 $$
Þá er $f$ fágað á $S(\alpha,\varrho)$ og
 $$f\dash(z)=\sum_{n=1}^\infty na_n(z-\alpha)^{n-1}, \qquad z\in
S(\alpha,\varrho). 
 $$
\end{se}

Sönnunina tökum við fyrir í grein  2.6.  Ef $\sum_{n=0}a_nz^n$ og
$\sum_{n=0}^\infty b_nz^n$ eru tvær samleitnar veldaraðir með
samleitnigeisla $\varrho_a$ og $\varrho_b$, þá höfum við fáguð föll
$f$ og $g$ í $S(\alpha,\varrho_a)$ og $S(\alpha,\varrho_b)$ sem gefin
eru með 
$$
f(z)=\sum_{n=0}^\infty a_n(z-\alpha)^n, \qquad \text{ og } \qquad
g(z)=\sum_{n=0}^\infty b_n(z-\alpha)^n.
$$
Ef við setjum $\varrho=\min\{\varrho_a,\varrho_b\}$, þá eru fáguðu
föllin $f+g$ og $fg$ einnig gefin veldaröðum á skífunni
$S(\alpha,\varrho)$ með
$$
f(z)+g(z)=\sum_{n=0}^\infty (a_n+b_n)(z-\alpha)^n 
\qquad \text{ og } \qquad f(z)g(z)=\sum_{n=0}^\infty c_n(z-\alpha)^n,
$$
þar sem stuðlarnir $c_n$ eru gefnir með
$$
c_n=\sum_{k=0}^n a_kb_{n-k}, \qquad n=0,1,2,\dots. 
$$

Eftirfarandi setning nefnist {\it
samsemdarsetning\index{samsemdarsetning} fyrir samleitnar
veldaraðir\index{samsemdarsetning!fyrir samleitnar veldaraðir}}: 

\begin{se}
Gerum ráð fyrir að $f,g\in \O(S(\alpha,\varrho))$ séu gefin með
samleitnum veldaröðum
 $$f(z)=\sum\limits_{n=0}^\infty a_n(z-\alpha)^n, \qquad
g(z)=\sum\limits_{n=0}^\infty b_n(z-\alpha)^n, \qquad
z\in S(\alpha,\varrho),
 $$
og gerum ráð fyrir að til sé runa $\set{\alpha_j}$ af ólíkum punktum
í $S(\alpha,\varrho)$ þannig að $\alpha_j\to \alpha$ og
$f(\alpha_j)=g(\alpha_j)$ fyrir öll $j$.  Þá er $a_n=b_n$ fyrir öll
$n$ og þar með $f(z)=g(z)$ fyrir öll $z\in S(\alpha,\varrho)$.
\end{se} 
 
\begin{so}
Fallið $h=f-g\in \O(S(\alpha,\varrho))$ hefur
veldaraðar\-fram\-setninguna
 $$h(z)=\sum\limits_{n=0}^\infty c_n(z-\alpha)^n, \qquad z\in
S(\alpha,\varrho), 
 $$
þar sem $c_n=a_n-b_n$.  Við þurfum að sanna að $c_n=0$ fyrir öll $n$.
  Gerum ráð fyrir að til sé $N$ þannig að $c_N\neq 0$ og veljum $N$
eins lítið og kostur er.  Þá er 
\begin{align*}
h(z)&= \sum\limits_{n=N}^\infty c_n(z-\alpha)^n = 
(z-\alpha)^N\sum\limits_{n=0}^\infty c_{N+n}(z-\alpha)^n\\
&= (z-\alpha)^N k(z), \qquad 
k(z) = \sum\limits_{n=0}^\infty c_{N+n}(z-\alpha)^n.
\end{align*}
Fallið $k$ er fágað á $S(\alpha,\varrho)$ og þar með er það samfellt,
$k(\alpha)=c_{N}$, svo til er opin grennd $U$ um $\alpha$ þar sem
$k(z)\neq 0$.  Fyrst $\alpha_j\to \alpha$, þá er $\alpha_j\in U$ ef
$j$ er nógu stórt og þar með er
$h(\alpha_j)=(\alpha_j-\alpha)^Nk(\alpha_j)\neq 0$.  Þetta er hins
vegar í mótsögn við forsendu okkar að $h(\alpha_j)=0$.  
\end{so}


\begin{fs}  Ef \ $\sum_{n=0}^{\infty} a_nx^n$ er samleitin veldaröð,
$I$ er opið bil sem inniheldur $0$ og   $\sum_{n=0}^{\infty}
a_nx^n=0$  fyrir öll $x\in I$, þá er $a_n=0$ fyrir öll $n=0,1,2,\dots$.
\end{fs}


Í setningu \ref{set4.3.1}
sönnuðum við að sérhvert fall sem gefið er með veldaraðaframsetningu á
einhverri skífu sé fágað.  Nú hugum við að andhverfu þessarar
staðhæfingar:

\begin{se}
Látum $X\subset \C$ vera opið og $f\in \O(X)$.  Ef $\alpha\in X$,
$0<\varrho<d(\alpha,\partial X)$, þar sem  $d(\alpha,\partial X)$
táknar fjarlægð punktsins $\alpha$ frá jaðrinum $\partial X$ á
menginu $X$, þá er hægt að setja $f$ fram í $S(\alpha,\varrho)$ með
samleitinni veldaröð  
 $$f(z) = \sum\limits_{n=0}^\infty a_n(z-\alpha)^n, \qquad z\in
S(\alpha,\varrho). 
 $$
 \end{se}

\figura {fig031}{{\small Mynd: Skífa í skilgreiningarsvæði $f$}}


Þessa setningu sönnum við ekki fyrr en í kafla 3, en við skulum skoða
nokkrar afleiðingar hennar.  

\begin{fs}
Ef $f\in \O(X)$, þá er $f\dash\in \O(X)$.
\end{fs}

\begin{so}
Ef $f$ er sett fram með veldaröðinni 
$f(z)=\sum_{n=0}^\infty a_n(z-\alpha)^n$ fyrir öll $z\in
S(\alpha,\varrho)$,
þá er $f\dash$ sett fram með veldaröðinni
$f'(z)=\sum_{n=1}^\infty na_n(z-\alpha)^{n-1}$,
samkvæmt setningu \ref{set4.3.1}, og er því fágað.
\end{so}

\bigskip
Nú sjáum við að fallið $f\dash$ er fágað og afleiða þess $f\ddash$ er
einnig fáguð og þannig áfram út í hið óendanlega.  Fyrir sérhvert
fágað fall $f\in \O(X)$ skilgreinum við hærri afleiður $f^{(k)}$ með
þrepun $f^{(0)}=f$ og $f^{(k)}=\big(f^{(k-1)}\big)\dash$, fyrir
$k\geq 1$. Við fáum síðan:

\begin{se}\label{se:2.3.7}  
Látum $X$ vera opið hlutmengi af $\C$, $f\in \O(X)$, $\alpha\in
X$ og $0<\varrho<d(\alpha,\partial X)$.  Þá er 
 $$f(z)= \sum\limits_{n=0}^\infty \dfrac
{f^{(n)}(\alpha)}{n!}(z-\alpha)^n, \qquad z\in S(\alpha,\varrho).
 $$
Þessi veldaröð kallast {\it
Taylor--röð\index{Taylor-röð}\index{Taylor-röð!falls í punkti}
fallsins $f$ í punktinum $\alpha$}.
\end{se}

Ef við byrjum með raunfágað fall á bili á rauntalnaásnum, þá vitum við
að við getum stækkað skilgreiningarmengi þess yfir í opið mengi í $\C$.
Sú aðgerð fellur undir eftirfarandi almenna skilgreiningu:

\begin{sk}
Látum $f:Y\to \C$ vera raunfágað fall á opnu mengi $Y$ á $\R$
og gerum ráð fyrir að $F:X\to \C$ sé fágað fall á opnu hlutmengi $X$ af
$\C$, þannig að $Y\subset X$ og $F(x)=f(x)$ fyrir öll $x\in Y$.  Þá
kallast $F$ {\it fáguð framlenging\index{fáguð framlenging}} eða {\it
fáguð útvíkkun\index{fáguð útvíkkun}}
á fallinu $f$.  
\end{sk}

Í kafla 3 eigum við eftir sjá, að ef mengið $X$ er
samanhangandi,  þá er fáguð framlenging $F$ á $f$ ótvírætt ákvörðuð.
Við megum því nota sama tákn $f$ fyrir upprunalega fallið og
fyrir útvíkkunina.
 
 
\section{Veldaröð veldisvísisfallsins}



\noindent
Enginn vafi leikur á því að veldisvísisfallið er merkilegasta fall
stærðfræðigreiningarinnar.    Við skilgreindum það með formúlunni
$$
\exp z=e^x(\cos y+i\sin y), \qquad z=x+iy \in \C.
$$
Við hefðum eins getað notað veldaraðarframsetninguna á $x\mapsto e^x$
til þess að skilgreina fágaða framlengingu veldisvísisfallsins.
Við skulum nú kanna nokkra eiginleika veldisvísisfallsins út frá
veldaröðinni.   


Með því að
deilda röðina lið fyrir lið fáum við 
 $$\exp\dash z=\exp z, \qquad \text{eða} \qquad \dfrac d{dz}e^z=e^z.
 $$
Undirstöðueiginleiki veldisvísisfallsins er {\it
samlagningarformúla\index{samlagningarformúla!veldisvísisfallsins}
\index{veldisvísisfallið!samlagningarformúla}} þess
$$ e^{z+w}=e^ze^w, \qquad z,w\in \C. $$
Hún leiðir af tvíliðureglunni \index{tvíliðuregla},
\begin{align*}
e^{z+w}&=\sum_{n=0}^\infty\dfrac 1{n!}(z+w)^n\\
&=\sum_{n=0}^\infty\dfrac 1{n!}\sum_{k=0}^n \dfrac{n!}{k!(n-k)!}z^kw^{n-k}\\
&=\sum_{n=0}^\infty\sum_{k=0}^n \dfrac {z^k}{k!}\dfrac {w^{n-k}}{(n-k)!}\\
&=\bigg(\sum_{n=0}^\infty \dfrac {z^n}{n!}\bigg)\bigg(\sum_{n=0}^\infty\dfrac
{w^{n}}{n!}\bigg)=e^ze^w. 
\end{align*}
Flestir eiginleikar veldisvísisfallsins er leiddir út frá
samlagningarformúlunni.  Til dæmis sjáum við að 
 \begin{equation*}e^{-z}=\dfrac 1{e^z}, \qquad z\in \C.\label{4.5.1}
 \end{equation*}
Á rauntalnaásnum er veldisvísisfallið $x\mapsto e^x$
stranglega vaxandi því afleiða þess er $e^x$ og hún er jákvæð.
Við höfum líka $e^x\to+\infty$ ef
$x\to \infty$, því sérhver liður í veldaröðinni með
númer $n\geq 1$ er stranglega vaxandi og stefnir á óendanlegt. Af
þessu leiðir síðan að $e^{x}=1/e^{-x}\to 0$ ef $x\to -\infty$.
Milligildissetningin segir okkur nú að veldisvísisfallið tekur öll
jákvæð gildi á rauntalnaásnum.  



Snúum okkur þá að gildunum á  þverásnum $\set{ix\in \C;
 x\in \R}$. Reglurnar um reikning með samoka tvinntölum gefa
okkur
$$\overline{e^z}=e^{\overline z},\qquad z\in \C,
$$
og síðan
 $$|e^z|^2=e^z\overline{e^{z}}=e^ze^{\overline z}=e^{x+iy}e^{x-iy}=e^{2x}
 $$
Þar með er
 $$|e^z|=e^{\Re z}, \qquad z\in \C,
 $$
og sérstaklega gildir 
$$
|e^{iy}|=1, \qquad y\in \R.
$$
Af þessu leiðir  að veldisvísisfallið hefur enga
núllstöð\index{veldisvísisfallið!núllstöð}
$e^z=e^xe^{iy}$ og  hvorugur þátturinn í hægri hliðinni getur verið
núll.  

Með því að stinga $iz$ inn í veldaröðina fyrir veldis\-vísis\-fallið
sjáum við að formúlan $e^{ix}=\cos x+i\sin x$ gildir áfram um 
tvinntölur $z\in\C$,
 $$
e^{iz}=\sum\limits_{n=0}^\infty\dfrac{i^n}{n!}z^n
=\sum\limits_{n=0}^\infty\dfrac{(-1)^n}{(2n)!}z^{2n}
+i\sum\limits_{n=0}^\infty\dfrac{(-1)^n}{(2n+1)!}z^{2n+1}
=\cos z +i \sin z.
 $$
Allir liðirnir í kósínus--röðinni hafa jöfn veldi og allir liðirnir í
sínus--röðinni hafa oddatöluveldi, svo $\cos$ er jafnstætt, en
$\sin$ er oddstætt.  Þar með er
 $$e^{-iz}=\cos z-i\sin z, \qquad z\in \C.
 $$
Við leysum nú $\cos z$ og $\sin z$ út úr síðustu tveimur
jöfnunum og fáum {\it jöfnur
Eulers\index{Euler}\index{Euler!jöfnur}\index{jöfnur Eulers}}
 $$\cos z =\frac 12(e^{iz}+e^{-iz}), \qquad
\sin z =\frac 1{2i}(e^{iz}-e^{-iz}).
 $$
Afleiðurnar af $\cos$ og $\sin$ getum við annað hvort reiknað með því
að deilda veldaraðirnar eða með því að deilda jöfnur Eulers,
 $$\cos\dash z=-\sin z, \qquad \sin\dash z=\cos z, \qquad z\in \C.
 $$


\section{Lograr, rætur\index{rót} og horn\index{horn}}


\noindent
Veldisvísisfallið $e^z$ er lotubundið með lotuna $2\pi i$,
$$
\exp(z+2{\pi}i) = \exp z, \qquad z\in \C.
$$
Þetta leiðir beint af þeirri staðreynd að kósínus og sínus eru
lotubundin með lotuna $2{\pi}$.
Þar með getur $\exp$ ekki haft neina andhverfu á öllu menginu $\C$.  
Veldisföllin 
$z^n$, $n\geq 2$ geta ekki heldur haft neina andhverfu á öllu $\C$.
Hins vegar hafa þessi föll andhverfur {\it frá hægri } 
á minni hlutmengjum í $\C$:

\begin{sk}
Látum $X$ vera opið hlutmengi af $\C$.  Samfellt fall $\lambda:X\to
\C$ kallast  {\it logri á $X$\index{logri}} ef
 \begin{equation*}e^{\lambda(z)}=z, \qquad z\in X.
\label{4.6.1}
 \end{equation*}
Samfellt fall $\varrho:X\to \C$ kallast {\it $n$--ta
rót\index{$n$--ta rót}} á $X$ ef
 \begin{equation*}\big(\varrho(z)\big)^n=z, \qquad z\in X.
\label{4.6.2}
 \end{equation*}
Samfellt fall $\theta:X\to \R$ kallast {\it horn á $X$} ef 
 \begin{equation*}z=|z|e^{i\theta(z)}, \qquad z\in X.
\label{4.6.3}
 \end{equation*}
\end{sk}


Helstu eiginleikar logra, róta\index{rót} og horna\index{horn} eru:

\begin{se} (i) Ef $\lambda$ er logri á $X$, þá er $0\not\in X$, $\lambda\in \O(X)$ og
 $$\lambda\dash(z)=\frac 1z, \qquad z\in X.
 $$
Föllin $\lambda(z)+i2\pi k$, $k\in \Z$ eru einnig lograr á $X$.

\smallskip\noindent 
(ii) Ef $\lambda$ er logri á $X$, þá er  $$\lambda(z)=\ln
|z|+i\theta(z), \qquad z\in X,
 $$
þar sem  $\theta:X\to \R$ er horn á $X$.  Öfugt, ef 
$\theta:X\to \R$ er horn á $X$, þá er $\lambda(z)=\ln|z|+i\theta(z)$
logri á $X$.

\smallskip\noindent
(iii)  Ef $\varrho$ er $n$--ta rót á $X$ þá er $\varrho\in \O(X)$ og
 $$\varrho\dash(z)=\frac {\varrho(z)}{nz}, \qquad z\in X.
 $$
(iv) Ef $\lambda$ er logri á $X$, þá er
$\varrho(z)=e\sp{\lambda(z)/n}$ $n$--ta rót á $X$.
\end{se}


\begin{so} (i) Veldisvísisfallið tekur ekki gildið $0$, svo
$z=\exp(\lambda(z))\neq 0$ fyrir öll $z\in X$.  Setning 4.2.6 (ii)
gefur að $\lambda$ er fágað og af jöfnunni $z=\exp(\lambda(z))$,
leiðir $1=\exp(\lambda(z))\lambda\dash(z)=z\lambda\dash(z)$.  Þar með
er $\lambda\dash(z)=1/z$.  Síðasta staðhæfingin er augljós, því
veldisvísisfallið hefur lotuna $2\pi i$.  

\smallskip
(ii) Við höfum $|z|=|e^{\lambda(z)}|=e^{\Re \lambda(z)}$ 
fyrir öll $z\in X$.  Þetta
gefur $\Re \lambda(z)=\ln |z|$ og þar með að $z=|z|e^{i\Im \lambda(z)}$. 
Samkvæmt skilgreiningu segir þetta að $\theta(z)=\Im\lambda(z)$
sé horn á $X$. Öfugt, ef $\theta$ er horn á $X$ og við setjum
$\lambda(z)=\ln |z|+i\theta(z)$, þá er
$\exp(\lambda(z))=\exp(\ln|z|)\exp(i\theta(z))=|z|e^{i\theta(z)}=z$,
fyrir öll $z\in X$.

\smallskip
(iii)  Við höfum að $(\varrho(z))^n=z$ fyrir öll $z\in X$, svo
setning 4.2.6 (ii) gefur okkur að $\varrho\in \O(X)$.  Ef við deildum
þessa jöfnu, fáum við $n(\varrho(z))^{n-1}\varrho\dash(z)=1$.  Við
margföldum nú í gegn með $\varrho(z)$ og fáum 
$\varrho(z)=n(\varrho(z))^n\varrho\dash(z)=nz\varrho\dash(z)$.

\smallskip
(iv)   $\varrho(z)^n=(\exp(\lambda(z)/n))^n=\exp(\lambda(z))=z$, $z\in X$.
\end{so}



\bigskip\hrule\bigskip

Athugið að fyrir sérhverja tvinntölu ${\alpha}$ 
getum við skilgreint fágað {\it veldisfall með veldisvísi}
$\alpha$ með 
 $$z^\alpha=\exp(\alpha\lambda(z)), \qquad z\in X,
 $$
þar sem  $\lambda$ er gefinn logri á $X$ og við fáum að
 $$\dfrac d{dz}z^\alpha=\dfrac d{dz}e^{\alpha\lambda(z)}=e^{\lambda(z)}\frac
\alpha z =\alpha e^{\alpha\lambda(z)}e^{-\lambda(z)}=
\alpha e^{(\alpha-1)\lambda(z)}=\alpha z^{\alpha-1}.
 $$
Þetta er sem sagt gamalkunn regla, sem gildir áfram fyrir
$\C$--afleiður.  Hér verðum við að hafa í huga að skilgreiningin  er
algerlega háð því hvernig logrinn er skilgreindur.  Ef við skiptum
til dæmis á logranum $\lambda(z)$ og $\lambda(z)+2\pi i$, þá verður 
 $$e^{\alpha(\lambda(z)+2\pi i)}=e^{\alpha\lambda(z)}e^{2\pi i\alpha}.
 $$
Ef $\alpha$ er heiltala þá er $z^\alpha$ samkvæmt þessari
skilgreininingu það sama og fæst út úr velda\-reglunum með
heiltöluveldi, en ef $\alpha$ er ekki heiltala, þá  er
skilgreiningin háð valinu á logranum.


Ef $\alpha \in X$, þá  skilgreinum við 
{\it veldisvísisfall með grunntölu $\alpha$} sem 
fágaða fallið á $\C$, sem gefið er með
$$
\alpha^z=e^{z\lambda(\alpha)}.
$$  
Athugið að skilgreiningin er háð valinu á logranum.
Keðjureglan gefur 
$$\dfrac d{dz}\alpha^z=
\dfrac d{dz}e^{z\lambda(\alpha)}=e^{z\lambda(\alpha)}\cdot
\lambda(\alpha)=\alpha^z\lambda(\alpha).
$$  

\bigskip\hrule\bigskip

\begin{sy}  Hvernig er  $i^i$ skilgreint?

\smallskip\noindent
{\it Lausn}: \ Talan $i$ hefur lengdina $1$ og horngildið $\tfrac 12
\pi+2\pi k$, þar sem $k\in \Z$.  Ef $X$ er svæði sem inniheldur $i$ og
$\lambda$ er logri á $X$, þá er $\lambda(i)=i(\tfrac 12 \pi+2\pi k)$
fyrir eitthvert $k\in \Z$.  Með þessu vali á $\lambda$ verður
$$
i^i=e^{i\lambda(i)}=e^{-(\tfrac 12\pi+2\pi k)}.
$$
Gildið $i^i$ er því háð því hvaða logra við veljum og við höfum óendanlega
 marga valmöguleika.  
\end{sy}

\bigskip\hrule\bigskip


%%%%%%%%%%%%%%%%%%%
\setbox0\vbox{{
\input fig032
}}
\setbox2\vbox{\hsize \wd0 { \strut \noindent Mynd: Höfuðgrein hornsins \strut}}
\setbox1\vbox{\hbox{\box0}\hbox{\box2}}
\dimen0 = -\ht1
\advance\dimen0 by-\dp1
\dimen1 = \wd1
\dimen2 = -\dimen0
\divide\dimen2 by\baselineskip
\count100 = 1
\advance\count100 by\dimen2
\advance\count100 by1
\hfill\box1
\vskip\dimen0
\dimen0 = \hsize
\advance\dimen0 by-\dimen1
\parshape 2 0pt \dimen0 0pt \dimen0
%%%%%%%%%%%%%%%%%%%
\noindent 
Lítum nú á mengið  $X=\C\setminus \R_-$, sem fæst með því að skera
neikvæða raunásinn og $0$ út úr
tvinntalnaplaninu.  Við skilgreinum síðan pólhnit í $X$ eins og
myndin sýnir og veljum hornið $\theta(z)$ þannig að
 $-\pi<\theta(z)<\pi$, $z\in X$.  Fallið 


 $$\Arg :\C\setminus \R_-\to \R, \qquad
\Arg z=\theta(z),\quad z\in X
 $$
\parshape 0  % Thetta er her vegna myndar 4.2
er kallað {\it höfuðgrein
hornsins\index{logri!höfuðgrein}\index{höfuðgrein!horns}} og 
við reiknuðum út formúlu fyrir því í kafla 1,
$$
\Arg\, z=2\arctan\bigg(\dfrac y{|z|+x}\bigg), \qquad z=x+iy\in X. 
$$
Fallið
 $$\Log :\C\setminus \R_-\to \C, \qquad
\Log z=\ln |z| +i\Arg(z),\quad z\in X,
 $$
er kallað {\it höfuðgrein
lografallsins\index{höfuðgrein}\index{höfuðgrein!lografallsins}}.  Fallið 
 $$z^\alpha = e^{\alpha\Log z}, \qquad z\in \C\setminus \R_-,
 $$
kallast {\it höfuðgrein
veldisfallsins\index{veldisfall}\index{veldisfall!höfuðgrein} með
veldisvísi $\alpha$\index{höfuðgrein!veldisfallsins með veldisvísi
$\alpha$}}. Tvö síðastnefndu föllin eru fágaðar framlengingar á föllunum
$\ln x$  og $x^\alpha$ frá jákvæða raunásnum  yfir
í opna mengið $\C\setminus \R_-$ í tvinntalnaplaninu.


\bigskip\hrule\bigskip

\begin{sy}
Skrifið eftirfarandi  tvinntölur á forminu $x+iy$, þar sem 
$x$ og $y$ eru rauntölur.

\smallskip\noindent
{\bf a)} \quad  $\Log{(-i)}$. \qquad 
{\bf b)} \quad  $\Log({1+i})$. \qquad 
{\bf c)} \quad  $\Log((1+i)^{\pi i})$. 


\smallskip\noindent
{\it Lausn:} \   {\bf a)} \ Höfuðgrein lograns
er $\Log z=\ln |z|+i \Arg(z)$, $\ln|-i|=\ln 1=0$ og 
$\Arg(-i)=-\pi/2$.  Svarið verður því $\Log(-i)=-i\pi/2$.

\smallskip\noindent
{\bf b)}\  Við höfum  $\ln|1+i|=\ln\sqrt 2$ og $\Arg(1+i)=\pi/4$
og því er $\Log(1+i)=\ln \sqrt 2+i\pi/4$.

\smallskip\noindent
{\bf c)} \  Við notum niðurstöðuna úr síðasta lið til þess að
reikna $(1+i)^{i\pi}=e^{i\pi(\ln\sqrt 2+i\pi/4)}=e^{-\pi^2/4+i\pi
\ln\sqrt 2}$.  Af því leiðir að $\ln|(1+i)^{i\pi}|=\ln
e^{-\pi^2/4}=-\pi^2/4$ og þar sem $0<\pi\ln \sqrt 2<\pi$, þá fáum við einnig
$\Arg((1+i)^{i\pi})=\pi\ln\sqrt 2$.  Niðurstaðan verður því 
$$
\Log((1+i)^{i\pi})=-\pi^2/4+i\pi\ln\sqrt 2=i\pi(\ln\sqrt
2+i\pi/4)=i\pi\Log(1+i)
$$
\end{sy}

\smallskip
Athugið að gamla góða lograreglan $\Log (z^\alpha)=\alpha \Log z$, gildir
ekki almennt.

\smallskip

\begin{sy}  Sýnið að $\Log(1+\sqrt 3 i)^4\neq 4 \Log(1+\sqrt 3 i)$.


\smallskip\noindent
{\it Lausn}: \  Við byrjum á að setja töluna $1+\sqrt 3$ fram á
pólformi með horngildi á bilinu $]-\pi,\pi[$, sem er
$1+\sqrt 3 i=2e^{i{\pi}/3}$.  Því er
$$
4\Log (1+\sqrt 3 i)=4(\ln 2+i{\pi}/3)=4\ln 2+i4{\pi}/3.
$$
Á hinn bóginn er
$(1+\sqrt 3 i)^4=2^4e^{i4{\pi}/3}=
2^4e^{-i2{\pi}/3}$ og því er
$$
\Log (1+\sqrt 3 i)^4=\ln 2^4-i2{\pi}/3=4\ln 2-i2{\pi}/3.
$$
Niðurstaðan er að jafnaðarmerki gildir ekki. 
\end{sy}



\bigskip\hrule\bigskip

\begin{sy}
Í pólhnitum verður höfuðgrein lograns
$$
\Log z =\ln r+i\theta,  \qquad z=re^{i\theta}, \ -\pi<\theta<\pi.
$$
Þar með eru raun- og þverhluti $\Log z$ gefin í pólhnitum með
$u(x,y)=\ln r$ og $v(x,y)=\theta$. Við getum því auðveldlega
staðfest að $\Log$ sé fágað fall, með því að beita niðurstöðunni 
í sýnidæmi \ref{sy:polhnit}
$$
\dfrac{\partial u}{\partial r}=\dfrac 1r=\dfrac 1r\cdot \dfrac 
{\partial v}{\partial \theta}\qquad \text{ og } \qquad
\dfrac 1r\cdot \dfrac{\partial u}{\partial \theta}=0=\dfrac {\partial
v}{\partial r}.
$$
Það er aðeins meiri fyrirhöfn að sýna fram
á að $\Log$ sé fágað með því að deilda með tilliti til $x$ og $y$.
\end{sy}

\bigskip\hrule\bigskip

\begin{sy}\label{sy:2.5.7}
Nú ætlum við að finna fágaða framlengingu á fallinu
 $f(x)=\arccos x$ frá bilinu $]-1,1[$ yfir á
opið mengi $X$ í $\C$  þannig að $f(z)$ verði andhverfa
tvinngilda $\cos$--fallsins, $\cos f(z)=z$.  
Við þurfum þá að byrja á því að leysa jöfnuna
$z=\cos w$,
\begin{gather*}
z=\cos w =\frac 12(e^{iw}+e^{-iw}),\\
e^{iw}-2z+e^{-iw}=0, \\
e^{2iw}-2ze^{iw}+1=(e^{iw}-z)^2-z^2+1=0,\\
(e^{iw}-z)^2=-(1-z^2).
\end{gather*}
Nú þurfum við að taka kvaðratrót, svo við látum $\Log$ tákna
höfuðgrein lografallsins.  Þá er höfuðgrein kvaðratrótarinnar
fallið 
 $$\C\setminus\R_- \to \C, \qquad
z\mapsto z\sp{\frac 12}=\exp(\tfrac 12\Log z).
 $$  
Þar með er 
 $$(1-z^2)^{1/2}=\exp(\tfrac 12\Log(1-z^2)), \qquad z\in
X=\C\setminus\set{x\in \R; |x|\geq 1},
 $$

\figura{fig033}{Mynd: Skilgreiningarsvæði $\arccos$}

\noindent
því $z\in X$ þá og því aðeins að $1-z\sp 2\in \C\setminus\set{x\in \R;
x\leq 0}$.  Við höldum nú áfram með útreikningana,
 $$e^{iw}=z\pm i(1-z^2)^{1/2}.
 $$
Til þess að sjá hvort formerkið á að taka, þá athugum  við að
$\cos {\pi}/2=0$ segir að $z=0$
svari til $w=\pi/2$, svo $-$ er útilokað og við sjáum jafnframt að
 $$f(z) = w=-i\lambda(z+i(1-z^2)^{\frac 12}), 
\qquad z\in X, 
 $$
þar sem $\lambda$ er einhver logri.  Til þess að sýna að við megum taka
${\lambda}$ sem höfuðgrein lografallsins, þá þurfum við að
vita að  myndmengið af vörpuninni
\begin{equation*}
X\to \C, \qquad z\mapsto z+i(1-z^2)^{\frac 12},\label{4.6.4}
\end{equation*}
sé í skilgreiningarsvæði höfuðgreinarinnar.  Við sjáum að þessi vörpun
varpar $x\in ]-1,1[$ á $x+i\sqrt{1-x^2}$ sem er punktur í efra
hálfplaninu $\set{z\in \C; \Im z>0}$.  


Nú kemur í ljós að
engin rauntala liggur í myndmenginu.  Það sjáum við með því að 
gera ráð fyrir að $z+i(1-z\sp 2)\sp{\frac
12}=t\in \R$ og fáum  
$$
i(1-z\sp 2)\sp{\frac 12}=t-z,\qquad
-(1-z\sp 2)=t\sp 2+z\sp 2-2tz,\qquad
z=\dfrac{t\sp 2+1}{2t}\in \R.
$$
%%%%%%%%%%%%%%%%%%%
\setbox0\vbox{{
\input fig034
}}
\setbox2\vbox{\hsize \wd0 { \strut \noindent Mynd: $\theta=\arccos x$ \strut}}
\setbox1\vbox{\hbox{\box0}\hbox{\box2}}
\dimen0 = -\ht1
\advance\dimen0 by-\dp1
\dimen1 = \wd1
\dimen2 = -\dimen0
\divide\dimen2 by\baselineskip
\count100 = 1
\advance\count100 by\dimen2
\advance\count100 by1
\hfill\box1
\vskip\dimen0
\dimen0 = \hsize
\advance\dimen0 by-\dimen1
\parshape 9 0pt \dimen0 0pt \dimen0 0pt \dimen0 0pt \dimen0 0pt \dimen0 
0pt \dimen0 0pt \dimen0 0pt \dimen0 0pt \hsize
%%%%%%%%%%%%%%%%%%%
\noindent
Jafnan 
$z^2=1+(t-z)^2$ segir okkur að
$|z|\geq 1$.  Við höfum því  $z\not\in X$. 


Þar sem mengið \  $X$ \ er saman\-hangandi, þá er myndmengi þess við
vörpunina (\ref{4.6.4}) hlutmengi í efra hálfplaninu $\set{z\in \C; \Im
z>0}$, en það er aftur hlutmengi af skil\-grein\-ingar\-mengi $\Log$.
Við höfum því 
$$f(z) = -i\Log (z+i(1-z^2)^{\frac 12}), 
\qquad z\in \C\setminus\set{x\in \R;
|x|\geq 1}. $$
Til þess að staðfesta að þessi formúla gefi okkur virkilega fágaða
framlengingu 
á $\arccos$--fallinu, þá setjum við $z=x\in ]-1,1[$ inn í formúluna
og við sjáum á myndinni að 
\begin{align*}
f(x)&=-i\Log(x+i(1-x^2)^{1/2})\\ 
&=\Arg (x+i\sqrt{1-x^2})-i\ln|\sqrt{1-x^2}+ix|\\
&=\Arg(x+i\sqrt{1-x^2})=\arccos x, \qquad x\in [-1,1].
\end{align*}
\end{sy}


\bigskip\hrule\bigskip


\begin{sy} Með sama hætti getum við fundið fágaða framlengingu á 
fallinu $f(x)=\arcsin x$ frá bilinu $]-1,1[$ yfir á
opið mengi $X$ í $\C$, þannig að $f(z)$ verði andhverfa
tvinngilda $\sin$--fallsins, $\sin f(z)= z$.  
Eins og í síðasta dæmi, þá byrjum við á því að leysa jöfnuna
$z=\sin w$,
\begin{gather*}
z=\sin w =\frac 1{2i}(e^{iw}-e^{-iw}),\\
e^{iw}-2iz-e^{-iw}=0, \\
e^{2iw}-2ize^{iw}-1=(e^{iw}-iz)^2+z^2-1=0,\\
(e^{iw}-iz)^2=1-z^2.
\end{gather*}
Nú þurfum við að taka kvaðratrót. Það gerum við með sama hætti og í
síðasta sýndæmi og við skilgreinum $X$ eins og þar.
Þá gildir
 $$e^{iw}-iz=\pm(1-z^2)^{1/2}.
 $$
Til þess að sjá hvort formerkið á að taka, þá athugum við að $\sin 0=0$
segir að $z=0$
svari til $w=0$, svo $-$ er útilokað og við sjáum jafnframt að
 $$f(z)= w=-i\lambda((1-z^2)^{1/2}+iz), \qquad z\in \C\setminus\set{x\in \R;
|x|\geq 1},
 $$
þar sem ${\lambda}$ er einhver logri. Nú er eftir að sýna ${\lambda}$ sé
höfuðgrein lograns.  Til þess
þurfum við að vita að  myndmengið af vörpuninni
 \begin{equation}
 X\to \C, \qquad z\mapsto (1-z^2)^{\frac 12}+iz,
\label{4.6.5}
 \end{equation}
\noindent
sé í skilgreiningarsvæði höfuðgreinarinnar.  


Nú kemur í ljós að
engin hrein þvertala liggur í myndmenginu.  Það sjáum við með því að 
gera ráð fyrir að $(1-z\sp 2)\sp{\frac 12}+iz=it$, $t\in \R$ og fáum  
 $$
(1-z\sp 2)\sp{\frac 12}=i(t-z),\qquad
(1-z\sp 2)=-(t\sp 2+z\sp 2-2tz),\qquad
z=\dfrac{-t\sp 2-1}{2t}.
 $$
%%%%%%%%%%%%%%%%%%
\setbox0\vbox{{
\input fig035
}}
\setbox2\vbox{\hsize \wd0 { \strut \noindent Mynd: $\theta=\arcsin x$ \strut}}
\setbox1\vbox{\hbox{\box0}\hbox{\box2}}
\dimen0 = -\ht1
\advance\dimen0 by-\dp1
\dimen1 = \wd1
\dimen2 = -\dimen0
\divide\dimen2 by\baselineskip
\count100 = 1
\advance\count100 by\dimen2
\advance\count100 by1
\hfill\box1
\vskip\dimen0
\dimen0 = \hsize
\advance\dimen0 by-\dimen1
\parshape 12 0pt \dimen0 0pt \dimen0 0pt \dimen0 0pt \dimen0
0pt \dimen0 0pt \dimen0 0pt \dimen0 0pt \dimen0
0pt \dimen0 0pt \dimen0 0pt \dimen0 0pt \hsize
%%%%%%%%%%%%%%%%%%%
\noindent
Þar með er $z$ rauntala og jafnan $z^2=1+(t-z)^2$ segir okkur að
$|z|\geq 1$.  Þar með höfum við að $z\not\in X$. 

Við sjáum að vörpunin \ref{4.6.5}
varpar $x\in ]-1,1[$ á $\sqrt{1-x^2}+ix$ sem er punktur í hægra
hálfplaninu $\set{z\in \C; \Re z>0}$.  
Þar sem mengið $X$ er samanhangandi þá er myndmengi 
vörpunarinnar hlutmengi í hægra hálfplaninu $\set{z\in \C; \Re
z>0}$, en það er aftur hlutmengi af skilgreiningarmengi $\Log$.
Við höfum því 
\begin{multline*}
f(z) = -i\Log ((1-z^2)^{\frac 12}+iz), \\
z\in \C\setminus\set{x\in \R;|x|\geq 1}.
\end{multline*}
Til þess að staðfesta að þessi formúla gefi okkur útvíkkun
á $\arcsin$--fallinu, þá setjum við $z=x\in ]-1,1[$ inn í formúluna
og fáum 
$$
f(x)=-i\Log((1-x^2)^{1/2}+ix)=\Arg(\sqrt{1-x^2}+ix)-i\ln|\sqrt{1-x^2}+ix|.$$
Við sjáum á myndinni að 
$\Arg(\sqrt{1-x^2}+ix)=\arcsin x$ og við höfum einnig að
$|\sqrt{1-x^2}+ix|=1$, svo 
$$\arcsin x=-i\Log((1-x^2)^{1/2}+ix),\qquad x\in ]-1,1[.
$$
\end{sy}

\bigskip\hrule\bigskip

\begin{sy}
Að lokum skulum við líta á fágaða útvíkkun á fallinu $\arctan x$, en
það er raunfágað á öllu menginu $\R$.  Við byrjum á því að leysa
jöfnuna $z=\tan w$, en hún gefur
$$
z=\dfrac{\sin w}{\cos
w}=\dfrac{e\sp{iw}-e\sp{-iw}}{i(e\sp{iw}+e\sp{-iw})} =
\dfrac{e\sp{2iw}-1}{i(e\sp{2iw}+1)}, \qquad
e\sp{2iw}=\dfrac{iz+1}{-iz+1} =\dfrac{i-z}{i+z}.
$$
Nú skulum við kanna fyrir hvaða mengi $X$ formúlan
\begin{equation*}
f(z)=\dfrac{-i}2 \Log\bigg(\dfrac{i-z}{i+z}\bigg), \qquad z\in X,
\label{4.6.6}
\end{equation*}
%%%%%%%%%%%%%%%%%%
\setbox0\vbox{{
\input fig036
}}
\setbox2\vbox{\hsize \wd0 { \strut \noindent Mynd: Svæði
$\arctan$\strut}}
\setbox3\vbox{\hbox{\box0}\hbox{\box2}}
\setbox0\vbox{{
\input fig037
}}
\setbox4\vbox{\hsize \wd0 { \strut \noindent Mynd: $\theta=\arctan x$\strut}}
\setbox1\vbox{\hbox{\box3}\hbox{\box0}\hbox{\box4}}
\dimen0 = -\ht1
\advance\dimen0 by-\dp1
\dimen1 = \wd1
\dimen2 = -\dimen0
\divide\dimen2 by\baselineskip
\count100 = 1
\advance\count100 by\dimen2
\advance\count100 by1
\hfill\box1
\vskip\dimen0
\dimen0 = \hsize
\advance\dimen0 by-\dimen1
\parshape 15 0pt \dimen0 0pt \dimen0 0pt \dimen0 0pt \dimen0 0pt \dimen0
50pt \dimen0 0pt \dimen0 0pt \dimen0 0pt \dimen0 0pt \dimen0
0pt \dimen0 0pt \dimen0 0pt \dimen0 0pt \dimen0 0pt \hsize
%%%%%%%%%%%%%%%%%%%
\noindent gildir. Hún gildir um öll $z$ þannig að $(i-z)/(i+z)=t$ er
ekki á neikvæða raunásnum.  Við skulum taka $t\leq
0$  og sjá hvaða punktar $z$ eru þar með útilokaðir.  Við leysum $z$
út, $z=i(1-t)/(1+t)$ og sjáum þar með að 
$z$ er hrein þvertala og
$$|z|=(1+|t|)/(1-|t|)\geq 1.
$$  Hér með höfum  við séð að fallið $f$ er
vel skilgreint með formúlunni hér að framan  ef  $X=\C\setminus\set{ix; x\in
\R, |x|\geq 1}$.
Nú er einungis eftir að sýna fram á að $f(x)=\arctan x$ fyrir öll
$x\in \R$.   Við athugum fyrst að
$$f(x)=\dfrac {-i}2 \Log\bigg(\dfrac{i-x}{i+x}\bigg) = \dfrac 12 \Arg
\bigg( \dfrac{i-x}{i+x}\bigg), \ \  x\in \R,
$$
því $|i-x|=|i+x|$ fyrir öll $x\in \R$, svo $\ln (|i-x|/|i+x|)=0$.
Talan $\theta= \Arg((i-x)/(i+x))$ er hornið milli 
tvinntalnanna $i-x$ og $i+x$ og
greinilega gildir $\tan(\theta/2)=x$.  Þar með er niðurstaðan
$f(x)=\theta/2=\arctan x$.
\end{sy}

\bigskip\hrule\bigskip

\begin{sy} Sýnið að  $\dfrac d{dz}\arcsin z= \dfrac 1{(1-z^2)^{\frac
12}}$, \ \ $\{x\in \R \,;\, |x|\geq 1 \}$.


\smallskip\noindent 
{\it Lausn:} \  Skrifum $w=\arcsin z$.  Þá er $z=\sin w$.  Ef við
setjum $f(w)=\sin w$, þá er 
$$
f\dash(w)=\cos w=(1-\sin ^2 w)^{\frac 12}
=(1-z^2)^{\frac 12}
$$
Hér er höfuðgrein kvaðratrótarinnar tekin.  Í sýnidæmi \ref{sy:2.5.7}
vorum við búin að sannfæra okkur um að $1-z^2$ væri punktur í $\C\setminus
\R_-$, sem er skilgreiningarsvæði höfuðgreinarinnar.  Samkvæmt
reiknireglunni um afleiður af andhverfu falls er
$$
\dfrac d{dz} \arcsin z=\big(f^{[-1]}\big)\dash (z)=
\dfrac 1{f\dash(w)} =\dfrac 1{(1-z^2)^{\frac 12}}.
$$
\end{sy}



\section{Sannanir á nokkrum niðurstöðum}



\noindent
Nú tökum við fyrir sannanir á nokkrum niðurstöðum sem sleppt var í
fyrr í kaflanum.  Það er hyggilegt að byrja á því að setja fram skilyrðið um
$\C$-deildanleika fram á nýjan hátt:



\begin{hs}\label{hs4.8.1}  Látum $X$ vera opið mengi 
í $\C$, $a\in X$ og $f:X\to \C$ vera fall.  Þá
gildir:

\smallskip
(i) $f$ er $\C$--deildanlegt í punktinum $a$ þá og því aðeins
að til sé tvinntala $A$ og fall $\varphi_a:X\to \C$,  samfellt í
$a$ og með $\varphi_a(a)=0$,  þannig að
 \begin{equation}f(z)=f(a)+A(z-a)+(z-a)\varphi_a(z), \qquad z\in X.
\label{4.8.1}
 \end{equation}
Talan $A$ er ótvírætt ákvörðuð, $A=f\dash(a)$.

\smallskip
(ii) $f$ er $\C$--deildanlegt í punktinum $a$ þá og því aðeins
að til sé fall $F_a:X\to \C$, sem er samfellt í $a$, þannig að
 \begin{equation*}f(z)=f(a)+(z-a)F_a(z), \qquad z\in X.
\label{4.8.2}
 \end{equation*}
Fallið $F_a$ er ótvírætt ákvarðað og  $F_a(a)=f\dash(a)$.
\end{hs}

\begin{so} (i) Ef $f$ er $\C$--deildanlegt í $a$, þá skilgreinum við 
$$
\varphi_a(z)=\begin{cases}
\dfrac{f(z)-f(a)}{z-a}-f\dash(a),& z\neq a,\\ 
0,& x=a.\end{cases}
$$
Þá er ljóst að (\ref{4.8.1}) gildir og $\lim_{z\to a}\varphi(z)=\varphi(a)=0$. 
Öfugt, ef (\ref{4.8.1}) gildir, þá er 
$$
\lim\limits_{h\to 0} \dfrac{f(a+h)-f(a)}h=
\lim\limits_{h\to 0}A+\varphi_a(a+h)=A
$$
og því er $f$ $\C$--deildanlegt í $a$ og $f\dash(a)=A$.

\smallskip
(ii) leiðir beint af (i).  Við tökum  $F_a(z)=A+\varphi_a(z)$.
\end{so}

\begin{sotx}{Sönnun á setningu 2.2.2}  Samkvæmt hjálparsetningu
\ref{hs4.8.1} (ii) er
$$\lim_{z\to a} f(z)=f(a)+\lim_{z\to a} (z-a)F_a(z)=f(a).
$$
\end{sotx}


\begin{sotx}{Sönnun á setningu 2.2.3}   Við beitum hjálparsetningu
\ref{hs4.8.1}
 í öllum  liðunum.
Látum því $F_a,G_a:X\to \C$ vera föll sem eru samfelld í $a$ og uppfylla
\begin{equation}
f(z)=f(a)+(z-a)F_a(z), \quad g(z)=g(a)+(z-a)G_a(z), \qquad z\in X.
\label{4.8.3}
\end{equation}

(i) Setjum $h=\alpha f+\beta g$.  Samkvæmt hjálparsetningu
\ref{hs4.8.1}
dugir að sanna að til sé fall $H_a$, sem er samfellt í $a$, þannig að
$h(z)=h(a)+(z-a)H_a(z)$.  Af (\ref{4.8.3}) leiðir að 
$H_a=\alpha F_a+\beta G_a$ og að við fáum $h\dash(a)=H_a(a)=\alpha F_a(a)+\beta
G_a(a)=\alpha f\dash(a)+\beta g\dash(a)$. 

(ii)  Setjum nú $h=fg$.  Hér dugir að sanna að
$h(z)=h(a)+(z-a)H_a(z)$
þar sem $H_a$ er samfellt í $a$.  Samkvæmt (\ref{4.8.3}) er
 $$h(z)=h(a)+(z-a)\left(F_a(z)g(a)+f(a)G_a(z)+(z-a)F_a(z)G_a(z)\right).
 $$
Þar með er 
 $$H_a(z)=F_a(z)g(a)+f(a)G_a(z)+(z-a)F_a(z)G_a(z)
 $$
sem er samfellt í $a$ og
$H_a(a)=F_a(a)g(a)+f(a)G_a(a)=f\dash(a)g(a)+f(a)g\dash(a)$. 

(iii)  Fyrst $g(a)\neq 0$ og $g$ er samfellt í $a$, þá er 
mengið $Y=\set{z\in X; g(z)\neq 0}$ grennd um punktinn $a$
og við athugum að 
 $$\dfrac 1{g(z)} =\dfrac 1{g(a)}+(z-a)\dfrac {-G_a(z)}{g(a)g(z)}, \qquad
z\in Y.  
 $$
Fyrst $g(a)\neq 0$ og $g$ er samfellt í $a$, þá 
sjáum við að fallið
 $$z\mapsto \dfrac {-G_a(z)}{g(a)g(z)}, \qquad z\in Y,
 $$
er samfellt í $a$ og þar með er $1/g$ $\C$--deildanlegt í $a$ og
 $$\left(\dfrac 1g\right)\dash(a)= \dfrac{-G_a(a)}{g(a)^2} =
\dfrac {-g\dash(a)}{g(a)\sp 2}.
 $$
Reglan leiðir nú af þessari formúlu og (ii).
\end{sotx}


\begin{sotx}{Sönnun á setningu 2.2.6} (i)  
Við ætlum að beita hjálparsetningu \ref{hs4.8.1} (ii) og 
látum því $F_a:X\to \C$ vera samfellt í $a$ og $G_b:Y\to \C$ vera
samfellt í $b$ þannig að 
 $$f(z)=f(a)+(z-a)F_a(z), \quad z\in X, \qquad g(w)=g(b)+(w-b)G_b(w), \quad
w\in Y. 
 $$
Þá er
\begin{align*}
h(z)&=g(f(z))=g(b)+(f(z)-b)G_b(f(z))\\
&=g(f(a))+(f(z)-f(a))G_b(f(z))\\
&=h(a)+(z-a)F_a(z)G_b(f(z))\\
&=h(a)+(z-a)H_a(z).
\end{align*}
þar sem fallið $H_a(z)=G_b(f(z))F_a(z)$ er greinilega samfellt í $a$, því
$F_a$ er samfellt í $a$, $f$ er samfellt í $a$  og $G_b$ er samfellt í
$b=f(a)$.  Við höfum $H_a(a)=G_b(b)F_a(a)=g\dash(b)f\dash(a)$. 


(ii)  Látum nú $G_b:Y\to \C$ vera samfellt í $b$,  $H_a:X\to \C$ vera
samfellt í $a$ og gerum ráð fyrir að
 $$g(w)=g(b)+(w-b)G_b(w), \quad
w\in Y, \qquad h(z)=h(a)+(z-a)H_a(z), \quad z\in X. 
 $$
Ef við stingum $w=f(z)$ inn í fyrri jöfnuna og notum að
$g(f(z))=h(z)$, þá fáum við 
 \begin{equation*}(f(z)-f(a))G_b(f(z))=(z-a)H_a(z),\qquad z\in X. \label{4.8.4}
 \end{equation*}
Fyrst $G_b$ er samfellt í $b$, $G_b(b)=g\dash(b)\neq 0$ og $f$ er
samfellt í $a$,  þá er til
grennd $U$ um $a$ þannig að $G_b(f(z))\neq 0$ fyrir öll $z\in U$.  Nú
setjum við 
 $$F_a(z)=\begin{cases}
\dfrac{H_a(z)}{G_b(f(z))}, &z\in U,\\
\dfrac{f(z)-f(a)}{z-a}, &z\in X\setminus U.
\end{cases}
 $$
Þá gefur (\ref{4.8.4}) að $f(z)=f(a)+(z-a)F_a(z)$.  Greinilega er $F_a$
samfellt í punktinum $a$ og
 $$F_a(a)=\dfrac{H_a(a)}{G_b(f(a))}=\dfrac{h\dash(a)}{g\dash(b)}.
 $$
 \end{sotx}


\begin{sotx}{Sönnun á fylgisetningu 2.2.7}
  Setjum $h(z)=z$, $z\in \C$.  Þá er $h$ fágað á $\C$ og 
 $$z=h(z)=f\circ f^{[-1]}(z), \qquad z\in Y.
 $$
Hér er $f$ í hlutverki $g$ og $f^{[-1]}$ í hlutverki $f$ í setningu
\ref{se:2.2.6} (ii).  Þar með er $f^{[-1]}$ $\C$--deildanlegt í $b$ og  
formúlan $(f^{[-1]})'(b)=1/f'(a)$ gildir.  
\end{sotx}



\begin{sotx}{Sönnun á setningu 2.3.2}   
  Við sönnum setninguna í sértilfellinu $\alpha=0$.
Almenna tilfellið fæst síðan með því að skipta á fallinu $f(z)$ og
$f(z+\alpha)$.    Við tökum $a\in S(0,\varrho)$.
Samkvæmt hjálparsetningu \ref{hs4.8.1} (ii) dugir að sanna að
til sé fall $F_a:X\to \C$, þannig að  $f(z)=f(a)+(z-a)F_a(z)$, 
 $F_a$ samfellt í $a$ og $F_a(a)=\sum_{n=1}^\infty na_na^{n-1}$.  Við
athugum fyrst að 
\begin{gather*}
z^n-a^n=(z-a)(z^{n-1}+az^{n-2}+\cdots+a^{n-2}z+a^{n-1}),\\
 \lim_{z\to a}(z^{n-1}+az^{n-2}+\cdots+a^{n-2}z+a^{n-1}) = na^{n-1}.
\end{gather*}
Af fyrri formúlunni leiðir
\begin{align*}
f(z)&=f(a)+\sum_{n=0}^\infty a_n(z^n-a^n)\\
&=f(a)+(z-a)\sum_{n=1}^\infty 
a_n (z^{n-1}+az^{n-2}+\cdots+a^{n-2}z+a^{n-1})
\end{align*}
þar sem síðasta röðin er samleitin fyrir öll $z\in S(0,\varrho)$.
Við setjum því 
 $$F_a(z)=\begin{cases}    
\sum_{n=1}^\infty a_n
(z^{n-1}+az^{n-2}+\cdots+a^{n-2}z+a^{n-1}),
&z\in S(0,\varrho),\\
\dfrac{f(z)-f(a)}{z-a}, \qquad z\in X\setminus S(0,\varrho).
\end{cases}
 $$
Við þurfum nú einungis að sanna að 
 $$\lim_{z\to a} F_a(z)=F_a(a)=\sum_{n=1}^\infty na_na^{n-1}.
 $$
Til þess að gera það, þá tökum við $r$ sem uppfyllir $|a|<r<\varrho$,
og athugum að 
 $$
|a_n(z^{n-1}+az^{n-2}+\cdots+a^{n-2}z+a^{n-1})|\leq n|a_n|r^{n-1},
\qquad |z|\leq r,
 $$
og jafnframt að  
 $$\sum_{n=1}^\infty n|a_n|r^{n-1}<+\infty,
 $$
því samleitnigeisli raðarinnar er $\geq \varrho>r$.
Samleitnipróf Weierstrass (setning 3.5.3) 
segir okkur nú að röðin sem skilgreinir 
$F_a$ sé samleitin í jöfnum mæli á menginu $\overline S(0,r)$. 
Þar með er $F_a$ samfellt í $S(0,\varrho)$ og 
\begin{align*}
F_a(a)&=\lim_{z\to a} F_a(z)\\
&=  \lim_{z\to a}
\sum_{n=1}^\infty a_n (z^{n-1}+az^{n-2}+\cdots+a^{n-2}z+a^{n-1})\\
&=\sum_{n=1}^\infty na_na^{n-1}.
\end{align*}
\end{sotx}


\vfill\eject


\aefing

\daemi  Staðfestið að fallið $e^z=e^x(\cos y+i\sin y)$ sé fágað með því
að sýna að Cauchy-Riemann-jöfnurnar séu uppfylltar.

\daemi
 Sýnið að einungis sé til eitt fall $f\in \O(\C)$ sem uppfyllir
$f(z+w)=f(z)f(w)$ fyrir öll $z,w\in \C$ og $f(x)=e^x$ fyrir öll $x\in
\R$. 

\daemi Hver eftirtalinna falla eru fáguð föll af 
$z=x+iy=r(\cos {\theta} +i\sin {\theta})$?  
Reiknið út Wirtinger-afleiðurnar $\partial_z f$  og 
$\partial_{\bar z}f$.

\smallskip\noindent
\begin{tabular}{ll}
a) $f(z)= 1/(z-2)$,
&b) $f(z)= z+1/z$, \\
c) $f(z)= 1/(z^2-1)$,
&d) $f(z)= z^2|z|^2$,\\
e) $f(z)= (\Im z)^2$,
&f) $f(z)= r(\cos {\theta}-i\sin {\theta})$,
\end{tabular}

\daemi Kannið hvort eftirtalin föll eru fáguð með því að athuga hvort
Cauchy-Riemann- jöfnurnar séu uppfylltar

\begin{tabular}{l}
a) $f(z)=x^2-y^2-2ixy$, \\
b) $f(z)=\frac 12\ln(x^2+y^2)+i\arctan(y/x)$,\\
c) $f(z)=\frac 12\ln(x^2+y^2)+i\arccot(x/y)$.
\end{tabular}

\daemi Látum $f: X\to \C$, $f=u+iv$ vera fágað fall þar sem $u=\Re f$
og $v=\Im f$ tákna raun- og þverhluta.  
Sýnið að stiglarnir $\nabla u$ og $\nabla v$ séu innbyrðis 
hornréttir í sérhverjum punkti $z\in X$.


\daemi Hlutafleiðuvirkinn
$$
{\Delta}=\dfrac {\partial^2 }{\partial x^2}+  
\dfrac {\partial^2 }{\partial y^2}
$$
nefnist {\it Laplace-virki\index{Laplace!virki}\index{virki!Laplace}},
óhliðraða hlutafleiðujafnan
${\Delta}u=0$ nefnist {\it
Laplace-jafna\index{Laplace!jafna}\index{jafna!Laplace}}
og lausn $u:X\to \C$ á  henni er sögð vera {\it þýtt fall\index{þýtt
fall}} á $X$.
Látum $f: X\to \C$, $f=u+iv$ vera fágað fall þar sem $u=\Re f$
og $v=\Im f$ tákna raun- og þverhluta. 
 Sýnið að bæði $u$ og $v$ séu
þýð föll, þ.e.~að þau uppfylli Laplace--jöfnuna
$$
\dfrac {\partial^2 u}{\partial x^2}+  
\dfrac {\partial^2 u}{\partial y^2}=  
\dfrac {\partial^2 v}{\partial x^2}+  
\dfrac {\partial^2 v}{\partial y^2}=0.
$$

\daemi Sýnið að 
${\Delta}=4\dfrac{{\partial}^2}{{\partial}z{\partial}\bar z}$.


\daemi Sýnið að í pólhnitum $z=re^{i{\theta}}$ séu
hlutafleiðuvirkjarnir  ${\partial}/{\partial}z$,
${\partial}/{\partial}\bar z$ og ${\Delta}$  gefnir með formúlunum:

\begin{gather*} 
\dfrac{\partial}{\partial z} =
\dfrac {e^{-i\theta}}2\bigg(\dfrac{\partial}{\partial r} -\dfrac ir
\dfrac{\partial}{\partial \theta}\bigg), \qquad
\dfrac{\partial}{\partial \bar z} =
\dfrac {e^{i\theta}}2\bigg(\dfrac{\partial}{\partial r} +\dfrac ir
\dfrac{\partial}{\partial \theta}\bigg)\\
\Delta
=\dfrac{\partial^2}{\partial r^2}+\dfrac 1r
\dfrac{\partial}{\partial r}
+\dfrac 1{r^2}\dfrac{\partial^2}{\partial\theta^2}
=\dfrac 1r\dfrac{\partial}{\partial r}\bigg(
r\dfrac{\partial}{\partial r}\bigg) +\dfrac 1{r^2}
\dfrac{\partial^2}{\partial \theta^2}.
\end{gather*}



\daemi Sýnið að um sérhvert fágað fall $f: X\to \C$ gildi
$$
\bigg( {\partial}_x|f(z)|\bigg)^2 +
\bigg( {\partial}_y|f(z)|\bigg)^2 = |f\dash(z)|^2, 
$$
fyrir  öll $z$ þannig að $f(z)\neq 0$.


\daemi Látum $f:X\to \C$ vera fágað fall á opnu hlutmengi $X$ í $\C$.
Sýnið að Jacobi-ákveðan af fallinu $f$ í punktinum $z$ sé
$|f\dash(z)|^2$.  Látum $M$ vera lokað og takmarkað hlutmengi af $\C$ og
gerum ráð fyrir að $f$ varpi einhverri opinni grennd um $M$ gagntækt á opna
grennd um $N=f(M)$.  Notið formúluna fyrir breytuskipti í tvöföldu
heildi til þess að sýna að 
$$
\iint\limits_{N}{\varphi}({\zeta})\, d{\xi}d{\eta}=
\iint\limits_{M}\varphi(f(z))|f\dash(z)|^2\, dxdy,
$$
fyrir sérhvert fall sem er samfellt í grennd um $N$, 
${\zeta}={\xi}+i{\eta}$ og $z=x+iy$.



\daemi Leiðið út eftirfarandi 
formúlur fyrir andhverfur breiðbogafallanna og finnið heppilegt mengi
þar sem þær gilda:

\smallskip
\begin{tabular}{l}
a) $\arcsinh z= \Log \big( z+(z^2+1)^{\frac 12}\big)$, \\
b) $\arccosh z= \Log \big( z+(z^2-1)^{\frac 12}\big)$,\\
c) $\arctanh z= \dfrac 12 \Log\bigg(\dfrac {1+z}{1-z}\bigg)$\\
\end{tabular}


\daemi Leiðið út eftirfarandi formúlur fyrir afleiður andhverfu
hornafallanna og  breiðbogafallanna:

\smallskip
\begin{tabular}{ll}
a) $\dfrac d{dz}\arcsin z= \dfrac 1{(1-z^2)^{\frac 12}}$,
&b) $\dfrac d{dz} \arccos z= \dfrac {-1}{(1-z^2)^{\frac 12}}$,\\
c) $\dfrac d{dz} \arctan z= \dfrac 1{1+z^2}$,
&d) $\dfrac d{dz}\arcsinh z= \dfrac 1{(z^2+1)^{\frac 12}}$,\\
e) $\dfrac d{dz} \arccosh z= \dfrac 1{(z^2-1)^{\frac 12}}$,
&f) $\dfrac d{dz} \arctanh z= \dfrac 1{1-z^2}$.\\ 
\end{tabular}


\daemi Skilgreinum
${\alpha}^z=e^{z\Log\alpha}$.  Skrifið eftirfarandi tvinntölur á forminu
$x+iy$, þar sem $x$ og $y$ eru rauntölur, og teiknið þær á mynd:

\begin{tabular}{lllll}
a) $\Log(-i)$,   
&b) $(-i)^i$,
&c) $2^{{\pi}i}$,
&d) $(1+i)^{(1+i)}$,
&e) $(1+i)^i(1+i)^{-i}$.\\
\end{tabular}


\daemi Finnið allar lausnir jafnanna:
\smallskip

\begin{tabular}{llll}
a) $\Log z={\pi}/4$,
&b) $e^z=i$,
&c)$^*$ $\sin z=i$,
&d) $\tan^2 z=-1$.\\
\end{tabular}


