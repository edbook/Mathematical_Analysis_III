
%
%Allir pakkar sem þarf að nota.
%
\usepackage[utf8]{inputenc}
\usepackage[T1]{fontenc}
\usepackage[icelandic]{babel}
\usepackage{amsmath}
\usepackage{amssymb}
\usepackage{pictex}
\usepackage{epsfig,psfrag}
\usepackage{makeidx}
%\selectlanguage{icelandic}
%----------------------------

%
\hoffset=-0.4truecm
\voffset=-1truecm
\textwidth=16truecm 
%\textwidth=12truecm 
\textheight=23truecm
\evensidemargin=0truecm
%
% Gömlu gildin á bókinni 
%
%\voffset 1.4truecm
%\hoffset .25truecm
%\vsize  16.0truecm
%\hsize  15truecm
%
%
% Skilgreiningar á ýmsum skipunum.
%
%\newcommand{\Sb}{
%$$
%\sum_{\footnotesize\begin{array}{l} j=1 \\ j\neq k \end{array}}
%$$
%}
\newcommand{\bolddot}{{\mathbf \cdot}}
\newcommand{\C}{{\mathbb  C}}
\newcommand{\Cn}{{\mathbb  C\sp n}}
\newcommand{\crn}{{{\mathbb  C\mathbb  R^n}}}
\newcommand{\R}{{\mathbb  R}}
\newcommand{\Rn}{{\mathbb  R\sp n}}
\newcommand{\Rnn}{{\mathbb  R\sp{n\times n}}}
\newcommand{\Z}{{\mathbb  Z}}
\newcommand{\N}{{\mathbb  N}}
\renewcommand{\P}{{\mathbb  P}}
\newcommand{\Q}{{\mathbb  Q}}
\newcommand{\K}{{\mathbb  K}}
\newcommand{\U}{{\mathbb  U}}
\newcommand{\D}{{\mathbb  D}}
\newcommand{\T}{{\mathbb  T}}
\newcommand{\A}{{\cal A}}
\newcommand{\E}{{\cal E}}
\newcommand{\F}{{\cal F}}
\renewcommand{\H}{{\cal H}}
\renewcommand{\L}{{\cal L}}
\newcommand{\M}{{\cal M}}
\renewcommand{\O}{{\cal O}}
\renewcommand{\S}{{\cal S}}
\newcommand{\dash}{{\sp{\prime}}}
\newcommand{\ddash}{{\sp{\prime\prime}}}
\newcommand{\tdash}{{\sp{\prime\prime\prime}}}
\newcommand{\set }[1]{{\{#1\}}}
\newcommand{\scalar}[2]{{\langle#1,#2\rangle}}
\newcommand{\arccot}{{\operatorname{arccot}}}
\newcommand{\arccoth}{{\operatorname{arccoth}}}
\newcommand{\arccosh}{{\operatorname{arccosh}}}
\newcommand{\arcsinh}{{\operatorname{arcsinh}}}
\newcommand{\arctanh}{{\operatorname{arctanh}}}
\newcommand{\Log}{{\operatorname{Log}}}
\newcommand{\Arg}{{\operatorname{Arg}}}
\newcommand{\grad}{{\operatorname{grad}}}
\newcommand{\graf}{{\operatorname{graf}}}
\renewcommand{\div}{{\operatorname{div}}}
\newcommand{\rot}{{\operatorname{rot}}}
\newcommand{\curl}{{\operatorname{curl}}}
\renewcommand{\Im}{{\operatorname{Im\, }}}
\renewcommand{\Re}{{\operatorname{Re\, }}}
\newcommand{\Res}{{\operatorname{Res}}}
\newcommand{\vp}{{\operatorname{vp}}}
\newcommand{\mynd}[1]{{{\operatorname{mynd}(#1)}}}
\newcommand{\dbar}{{{\overline\partial}}}
\newcommand{\inv}{{\operatorname{inv}}}
\newcommand{\sign}{{\operatorname{sign}}}
\newcommand{\trace}{{\operatorname{trace}}}
\newcommand{\conv}{{\operatorname{conv}}}
\newcommand{\Span}{{\operatorname{Sp}}}
\newcommand{\stig}{{\operatorname{stig}}}
\newcommand{\Exp}{{\operatorname{Exp}}}
\newcommand{\diag}{{\operatorname{diag}}}
\newcommand{\adj}{{\operatorname{adj}}}
\newcommand{\erf}{{\operatorname{erf}}}
\newcommand{\erfc}{{\operatorname{erfc}}}
\newcommand{\Lloc}{{L_{\text{loc}}\sp 1}}
\newcommand{\boldcdot}{{\mathbb \cdot}}
%\newcommand{\Cinf0}[1]{{C_0\sp{\infty}(#1)}}
\newcommand{\supp}{{\text{supp}\, }}
\newcommand{\chsupp}{{\text{ch supp}\, }}
\newcommand{\singsupp}{{\text{sing supp}\, }}
\newcommand{\SL}[1]{{\dfrac {1}{\varrho} 
\bigg(-\dfrac d{dx}\bigg(p\dfrac {d#1}{dx}\bigg)+q#1\bigg)}}
\newcommand{\SLL}[1]{-\dfrac d{dx}\bigg(p\dfrac {d#1}{dx}\bigg)+q#1}
\newcommand{\Laplace}[1]{\dfrac{\partial^2 #1}{\partial x^2}+\dfrac{\partial^2 #1}{\partial y^2}}
\newcommand{\polh}[1]{{\widehat #1_{\C^n}}}
\newcommand{\tilv}{{}}
%
\renewcommand{\chaptername}{Kafli}
%
% Númering á formæulum.
%
\numberwithin{equation}{section}
%
%  Innsetning á myndum.
%
\def\figura#1#2{
\vbox{\centerline{
\input #1
}
\centerline{#2}
}\medskip}
\def\vfigura#1#2{
\setbox0\vbox{{
\input #1
}}
\setbox1\vbox{\hbox{\box0}\hbox{{\obeylines #2}}}
\dimen0 = -\ht1
\advance\dimen0 by-\dp1
\dimen1 = \wd1
\dimen2 = -\dimen0
\divide\dimen2 by\baselineskip
\count100 = 1
\advance\count100 by\dimen2
\advance\count100 by1
\box1
\hangindent\dimen1
\hangafter=-\count100
\vskip\dimen0
}
%
%  Setningar, skilgreiningar, o.s.frv. 
%
\newtheorem{setning+}           {Setning}      [section]
\newtheorem{skilgreining+}  [setning+]  {Skilgreining}
\newtheorem{setningogskilgreining+}  [setning+]  {Setning og
skilgreining}
\newtheorem{hjalparsetning+}  [setning+]  {Hjálparsetning}
\newtheorem{fylgisetning+}  [setning+]  {Fylgisetning}
\newtheorem{synidaemi+}  [setning+]  {Sýnidæmi}
\newtheorem{forrit+}  [setning+]  {Forrit}

\newcommand{\tx}[1]{{\rm({\it #1}). \ }}

\newenvironment{se}{\begin{setning+}\sl}{\hfill$\square$\end{setning+}\rm}
\newenvironment{sex}{\begin{setning+}\sl}{\hfill$\blacksquare$\end{setning+}\rm}
\newenvironment{sk}{\begin{skilgreining+}\rm}{\hfill$\square$\end{skilgreining+}\rm}
\newenvironment{sesk}{\begin{setningogskilgreining+}\rm}{\hfill$\square$\end{setningogskilgreining+}\rm}
\newenvironment{hs}{\begin{hjalparsetning+}\sl}{\hfill$\square$\end{hjalparsetning+}\rm}
\newenvironment{fs}{\begin{fylgisetning+}\sl}{\hfill$\square$\end{fylgisetning+}\rm}
\newenvironment{sy}{\begin{synidaemi+}\rm}{\hfill$\square$\end{synidaemi+}\rm}
\newenvironment{fo}{\begin{forrit+}\rm}{\hfill\end{forrit+}\rm}
\newenvironment{so}{\medbreak\noindent{\it Sönnun:}\rm}{\hfill$\blacksquare$\rm}
\newenvironment{sotx}[1]{\medbreak\noindent{\it #1:}\rm}{\hfill$\blacksquare$\rm}
\newcounter{daemateljari}
\newcommand{\aefing}{\section{Æfingardæmi} \setcounter{daemateljari}{1}}
\newcommand{\daemi}{
{\medskip\noindent{\bf \thedaemateljari.}}
\addtocounter{daemateljari}{1}
}

%\def\aefing{{\large\bf\bigskip\bigskip\noindent Æfingardæmi}}
%\def\daemi#1{\medskip\noindent{\bf #1.}}
\def\svar#1{\smallskip\noindent{\bf #1.} \ }
\def\lausn#1{\smallskip\noindent{\bf #1.} \ }
\def\ugrein#1{\medbreak\noindent{\bf #1.} }
\newcommand{\samantekt}{\noindent{\bf Samantekt.} }
%\newcommand{\proclaimbox}{\hfill$\square$}


\chapter
{VELDARAÐALAUSNIR Á AFLEIÐUJÖFNUM}
 
%\kaflahaus{Veldaraðalausnir á afleiðujöfnum}

\section{Raunfáguð föll}


\subsection*{Raunfáguð föll og fágaðar útvíkkanir}

\subsubsection{Skilgreining}
Fall $f:X\to \C$ skilgreint á opnu mengi $X$ á raunásnum,  er sagt
vera {\it raunfágað :hover:`raunfágaður`} á $X$ ef hægt er að skrifa $X$
sem sammengi af opnum bilum $]a-\varrho,a+\varrho[$  og fyrir sérhvert
þessara bila er  til samleitin veldaröð $\sum_{n=0}^\infty c_nx^n$
þannig að
\begin{equation*}f(x)= \sum\limits_{n=0}^\infty c_n(x-a)^n, \qquad x\in
]a-\varrho,a+\varrho[.


.. _3.1.2:

\end{equation*}


--------------



Ef $f$ er fall sem sett fram með veldaröð á bilinu
$]a-\varrho,a+\varrho[$, þá framlengist $f$ sjálfkrafa í fágað fall
á skífunni opnu $S(a,\varrho)$ og gildi þess eru gefin með 
$$f(z)=\sum\limits_{n=0}\sp\infty c_n(z-a)\sp n, \qquad z\in
S(a,\varrho).
$$
Af þessu leiðir að um sérhvert raunfágað fall $f$ á opnu mengi $X$ í
$\R$ gildir að til er fágað fall $F$ á opnu mengi $Y$ í $\C$ sem
inniheldur $X$ þannig að $F(x)=f(x)$ fyrir öll $x\in X$. 
Fallið $F$ er þá nefnt {\it fáguð útvíkkun} eða {\it fáguð
framlenging}  af $f$ yfir á $Y$.  Ef $Y$ er svæði og $F_1$ og $F_2$ eru
tvær fágaðar útvíkkanir  af $f$ yfir á $Y$, þá gefur
samsemdarsetningin  að $F_1=F_2$.  Þetta segir okkur að fágaðar
útvíkkanir  yfir á svæði séu ótvírætt ákvarðaðar og því notum við
bókstafinn $f$ líka fyrir útvíkkunina. 

Ef fallið $f$ er raunfágað á menginu $X$ og $f$ er gefið með
veldaröðinni
hér að framan  á bilinu $I=]a-{\varrho},a+{\varrho}[$, þá er $f\in
C^{\infty}(I)$ og afleiður $f$ eru reiknaðar með því að deilda röðina
lið fyrir lið,
$$f\dash(x)= \sum\limits_{n=1}^\infty nc_n(x-a)^{n-1}
= \sum\limits_{n=0}^\infty (n+1)c_{n+1}(x-a)^n.
$$
Athugið að í seinni summunni hliðruðum við til númeringu liðanna
með því að setja $k=n-1$ í stað $n$.  Þá svarar $n=1$ til $k=0$
og $n$ svarar til $k+1$.  Þetta þurfum við oft að gera í útreikningum
í þessu kafla.  Hærri afleiður eru nú reiknaðar á sama hátt
\begin{align*}
f\ddash(x)&= \sum\limits_{n=2}^\infty n(n-1)c_n(x-a)^{n-2}
= \sum\limits_{n=0}^\infty (n+1)(n+2)c_{n+2}(x-a)^n,\\
f^{(k)}(x)&= \sum\limits_{n=k}^\infty n(n-1)\cdots (n-k+1)c_n(x-a)^{n-k}\\
&= \sum\limits_{n=0}^\infty (n+1)(n+2)\cdots(n+k)c_{n+k}(x-a)^n.
\end{align*}
Út frá þessu sést að veldaröðin  er
Taylor-röð :hover:`Taylor-röðun` fallsins $f$ í
punktinum :hover:`Taylor-röðun!falls í punkti` $a$
\begin{equation*}
f(x)=\sum\limits_{n=0}^\infty \dfrac{f^{(n)}(a)}{n!}(x-a)^{n}.


.. _3.1.3:

\end{equation*}
Við þekkjum ótal dæmi um raunfáguð föll sem gefin eru með
Taylor-röðum og við fengumst við fágaðar útvíkkanir þeirra í 
kafla 2,
\begin{align*}
e^x&=\sum\limits_{n=0}^\infty\dfrac 1{n!}{x^n}
=1+x+\dfrac {x^2}{2!}+\dfrac{x^3}{3!}+\cdots,\\
\cos x&= \sum\limits_{n=0}^\infty \dfrac{(-1)^n}{(2n)!}x^{2n}
=1-\dfrac{x^2}{2!}+\dfrac{x^4}{4!}-\cdots,\\
\sin x &=\sum\limits_{n=0}^\infty\dfrac{(-1)^n}{(2n+1)!}x^{2n+1}
= x-\dfrac {x^3}{3!}+\dfrac{x^5}{5!}-\cdots,\\
\cosh x&=\sum\limits_{n=0}^\infty\dfrac{1}{(2n)!}x^{2n}
=1+\dfrac{x^2}{2!}+\dfrac{x^4}{4!}+\cdots,\\
\sinh x &=\sum\limits_{n=0}^\infty\dfrac{1}{(2n+1)!}x^{2n+1}
= x+\dfrac {x^3}{3!}+\dfrac{x^5}{5!}+\cdots,\\
\ln (1+x) &= \sum\limits_{n=1}^\infty\dfrac{(-1)^{n+1}}{n}x^n
=x-\dfrac{x^2}{2}+\frac{x^3}3-\cdots,\\
\dfrac 1{1-x}&=\sum\limits_{n=0}^\infty x^n
=1+x+x^2+\cdots, \\
(1+x)^\alpha&= 1+\alpha x+ \dfrac{\alpha(\alpha-1)}{2!}x^2 + 
\dfrac {\alpha(\alpha-1)(\alpha-2)}{3!}x^3+\cdots.
\end{align*}
Í veldaraðarframsetningum af þessu tagi setjum við alltaf $0!=1$
og $x^0=1$ fyrir öll $x$.  Fimm fyrstu raðirnar eru samleitnar á
öllu
$\R$ en hinar eru samleitnar á $]-1,1[$.


\subsection*{Aðgerðir á veldaröðum}

Framsetning á
föllum með veldaröðum er sérstaklega þægileg vegna þess að aðgerðir á
þeim eru nánast þær sömu og aðgerðir á margliðum.  Gerum nú ráð fyrir
því að föllin $f$ og $g$ séu gefin með veldaröðum á bilinu
$]a-{\varrho},a+{\varrho}[$,
$$
f(x)=\sum\limits_{n=0}^{\infty} a_n(x-a)^n,\qquad
g(x)=\sum\limits_{n=0}^{\infty} b_n(x-a)^n.
$$
Þá er summa þeirra gefin með veldaröðinni
\begin{equation*}
f(x)+g(x)=\sum\limits_{n=0}^{\infty} (a_n+b_n)(x-a)^n,

.. _3.1.4:

\end{equation*}
og margfeldið er gefið með röðinni
\begin{equation*}
f(x)g(x)=\sum\limits_{n=0}^{\infty} c_n(x-a)^n, 
\qquad
c_n=a_0b_n+a_1b_{n-1}+\cdots+a_nb_0.

.. _3.1.5:

\end{equation*}
Ef $g(a)=b_0\neq 0$, þá er til ${\varrho}_1\leq {\varrho}$
þannig að $g(x)\neq 0$ fyrir öll $x$ á bilinu
$]a-{\varrho}_1,a+{\varrho}_1[$.
Kvótinn $f(x)/g(x)$ er þá gefinn með veldaröð
$\sum\limits_{n=0}^{\infty} d_n(x-a)^n$.  Til þess að reikna út stuðlana
$d_n$ þá beitum við (:ref:`3.1.5`) á margfeldið
$$
\sum\limits_{n=0}^{\infty} d_n(x-a)^n
\sum\limits_{n=0}^{\infty} b_n(x-a)^n
=\sum\limits_{n=0}^{\infty} a_n(x-a)^n.
$$
Formúlan fyrir stuðlana í margfeldinu gefur
$$
d_0b_0=a_0, \quad
d_0b_1+d_1b_0=a_1, \quad \dots, \quad 
d_0b_n+d_1b_{n-1}+\cdots+d_nb_0=a_n.
$$
Við fáum því rakningarformúlu fyrir stuðlana
\begin{align*}
f(x)/g(x)&=\sum\limits_{n=0}^{\infty} d_n(x-a)^n

.. _3.1.6:
\\
d_0&=a_0/b_0,\\
d_1&=(a_1-d_0b_1)/b_0,\\
&\quad \vdots\qquad\qquad \vdots\\
d_n&=(a_n-d_0b_n-d_1b_{n-1}-\cdots-d_{n-1}b_1)/b_0.
\end{align*}

 
\section{Raðalausnir :hover:`afleiðujafna!raðalausn` umhverfis
venjulega punkta :hover:`venjulegur punktur`}

\noindent
Nú skulum við snúa okkur að almennum  afleiðuvirkja.
Við vitum    
að ef öll stuðlaföllin $a_0(x),\dots,a_{m}(x)$  eru
raunfáguð á bilinu $I$ og $a_m(x)\neq 0$ fyrir öll $x\in I$, 
þá hefur afleiðujafnan $P(x,D)u=0$
$m$ línulega óháðar lausnir, sem eru fágaðar á $I$ og
unnt er að ákvarða stuðlana í veldaraðarframsetningu þessara falla út
frá stuðlunum í veldaraðarframsetningu $a_0,\dots,a_{m-1}$.  
Við ætlum nú að ganga út frá þessari setningu og reikna út lausnir með 
veldaröðum.

\subsection*{Nokkur dæmi um veldaraðalausnir}

Hugmyndin bakvið veldaraðalausnir á afleiðujöfnum er einföld.  
Við göngum út frá þeirri lausnartilgátu að til sé lausn sem
gefin er með veldaröð :hover:`afleiðujafna!veldaraðalausn`,
$$
u(x)=\sum\limits_{n=0}^{\infty} c_n(x-a)^n.
$$
Síðan stingum við röðinni inn í jöfnuna og leiðum út formúlu fyrir
stuðlana $c_n$.  


\subsection*{Einangraðir sérstöðupunktar}

Við rifjum nú upp þekkt hugtök fyrir fáguð föll:

\subsubsection{Skilgreining}
Látum $f$ vera raunfágað fall á opnu mengi $X$ í $\R$, $a\in X$, gerum ráð
fyrir að punkturinn  $a\in X$ sé núllstöð fallsins $f$ og 
$$ f(x)=\sum_{n=0}\sp \infty c_n(x-a)\sp n. $$
Þá kallast minnsta gildið á $n$ þannig að $c_n\neq 0$ {\it
margfeldni :hover:`margfeldni`} eða
{\it stig :hover:`stig` :hover:`núllstöð!stig` :hover:`stig!núllstöðvar`}
núllstöðvarinnar :hover:`margfeldni!núllstöðvar`
 :hover:`núllstöð` :hover:`núllstöð!margfeldni` $a$.


--------------



Ef $a$ er núllstöð fallsins $f$ af stigi $N$ og við setjum
$b_n=c_{N+n}$, þá er $b_0\neq 0$ og
$$
f(x)=\sum_{n=N}\sp \infty c_n(x-a)\sp n=
(x-a)\sp N\sum_{n=N}\sp \infty c_n(x-a)\sp {n-N} =
(x-a)\sp N\sum_{n=0}\sp \infty b_n(x-a)\sp n.
$$
Það er því greinilega jafngilt að fallið $f$ hafi núllstöð af stigi
$N$ í punktinum $a$ og að hægt sé að skrifa $f$ í grennd um $a$ með
formúlu af gerðinni
$$ f(x)=(x-a)\sp N\sum_{n=0}\sp \infty b_n(x-a)\sp n, $$
þar sem $b_0\neq 0$.   

\subsubsection{Skilgreining}
Látum $f$ vera raunfágað fall á opnu mengi  $X$ í $\R$, gerum ráð
fyrir að $a\not\in X$ og  að $\{x; 0<|x-a|<r\}\subset X$ fyrir
eitthvert $r>0$.  Þá kallast punkturinn $a$ {\it einangraður
sérstöðupunktur :hover:`einangraður
sérstöðupunktur` :hover:`sérstöðupunktur` :hover:`sérstöðupunktur!einangraður`} 
raunfágaða fallsins $f$.  Við segjum að einangraður
sérstöðupunktur sé {\it afmáanlegur :hover:`afmáanlegur
sérstöðupunktur` :hover:`sérstöðupunktur!afmáanlegur` :hover:`einangraður
sérstöðupunktur!afmáanlegur`} ef til er $\varrho>0$, þannig að
$\{x; 0<|x-a|<{\varrho}\}\subset X$ og raunfágað fall
$g$ á $\{x; |x-a|<{\varrho}\}$ 
þannig að  $f(x)=g(x)$ ef $0<|x-a|<{\varrho}$.


--------------




Skilgreiningin segir að $a$ sé afmáanlegur sérstöðupunktur 
raunfágaða fallsins $f$ þá og því aðeins að 
hægt sé að bæta punktinum $a$ við skilgreiningarsvæði $f$ þannig að $f$
verði raunfágað á $X\cup \set a$.


\subsection*{Venjulegir punktar}

Nú skulum við líta á jöfnuna 
 \begin{equation*}a_2(x)u\ddash+a_1(x)u\dash+a_0(x)u=0,

.. _3.2.1:

 \end{equation*}
þar sem föllin $a_0$, $a_1$ og $a_2$ eru raunfáguð á bili $I$ á $\R$.
Það þýðir að fyrir sérhvern punkt $a\in I$ má skrifa föllin með
veldaröðum í $(x-a)$,  sem eru samleitnar í grennd um punktinn $a$,
 $$a_j(x)=\sum_{n=0}\sp \infty a_{jn}(x-a)\sp n, \qquad j=0,1,2.
 $$
Við skilgreinum nú  
 \begin{equation*}P(x)=\dfrac{a_1(x)}{a_2(x)}, \qquad 
Q(x)=\dfrac{a_0(x)}{a_2(x)}.

.. _3.2.2:

 \end{equation*}
Þessi föll eru greinilega vel skilgreind í sérhverjum punkti þar sem
$a_2(x)\neq 0$, en í núllstöðvunum þurfa þau ekki að vera skilgreind.
Þar sem föllin $P$ og $Q$ eru skilgreind fáum við jafngilda
afleiðujöfnu
 \begin{equation*}u\ddash+P(x)u\dash+Q(x)u=0,

.. _3.2.3:

 \end{equation*}

\subsubsection{Skilgreining}
Við segjum að punkturinn $a\in I$ sé {\it venjulegur
punktur :hover:`venjulegur punktur` :hover:`afleiðujafna!venjulegur
punktur`} annars stigs
afleiðujöfnu, ef
$a_2(a)\neq 0$ eða $a_2(a)=0$ og $a$ er afmáanlegur sérstöðupunktur
fallanna $P$ og $Q$.  Ef $a$ er ekki venjulegur punktur, þá kallast $a$ {\it
sérstöðupunktur :hover:`sérstöðupunktur` :hover:`afleiðujafna!sérstöðupunktur`}
jöfnunnar. 


--------------




Lítum nú á afleiðujöfnuna, umritum hana eins og hér að framan
og gerum ráð fyrir að stuðlarnir $P(x)$ og $Q(x)$ hafi
veldaraðaframsetningu 
 \begin{equation*}
P(x)=\dfrac{a_1(x)}{a_2(x)}= \sum_{n=0}\sp \infty P_n(x-a)\sp n,
\qquad
Q(x)=\dfrac{a_0(x)}{a_2(x)}= \sum_{n=0}\sp \infty Q_n(x-a)\sp n,

.. _3.2.4:

 \end{equation*}
Við göngum út frá þeirri lausnartilgátu að $u$ sé gefið með veldaröð 
umhverfis punktinn $a$,
$$
u(x)=\sum\limits_{n=0}\sp\infty c_n(x-a)\sp n, \quad
u'(x)=\sum\limits_{n=0}\sp\infty (n+1)c_{n+1}(x-a)\sp n, \quad
u\ddash(x)=\sum\limits_{n=0}\sp\infty (n+2)(n+1)c_{n+2}(x-a)\sp n.
 $$
Ef við stingum þessum röðum  inn í afleiðujöfnuna, þá fáum við 
 $$
0= \sum_{n=0}\sp \infty (n+2)(n+1)c_{n+2}(x-a)\sp n +
P(x)\sum_{n=0}\sp \infty (n+1)c_{n+1}(x-a)\sp n +
Q(x)\sum_{n=0}\sp \infty c_n(x-a)\sp n.
 $$
Með því að margfalda saman raðirnar fyrir $P$ og $u\dash$ annars vegar 
og $Q$ og $u$ hins vegar í (:ref:`3.2.4`), þá fáum við 
\begin{gather*}
P(x)\sum_{n=0}\sp \infty (n+1)c_{n+1}(x-a)\sp n=
\sum_{n=0}\sp\infty  
\bigg(\sum_{k=0}\sp n (k+1)P_{n-k}c_{k+1}\bigg)(x-a)\sp n,\\
Q(x)\sum_{n=0}\sp \infty c_n(x-a)\sp n=
 \sum_{n=0}\sp\infty  
\bigg( \sum_{k=0}\sp n  Q_{n-k}c_k\bigg) (x-a)\sp n,
\end{gather*}
svo afleiðujafnan verður
$$
0= \sum_{n=0}\sp \infty 
\bigg((n+2)(n+1)c_{n+2} +
\sum_{k=0}\sp{n} \big((k+1)P_{n-k}c_{k+1}+
Q_{n-k} c_k\big)\bigg)(x-a)\sp n.
$$
Val okkar á $c_0$ og $c_1$ er frjálst og við fáum rakningarformúluna
 \begin{equation*} c_{n+2} = \dfrac{-1}{(n+2)(n+1)}

.. _3.2.5:

\sum_{k=0}\sp n \big[(k+1)P_{n-k}c_{k+1} +  Q_{n-k}c_k\big],
 \end{equation*}
fyrir $n=0,1,2,\dots$. 

\subsubsection{Setning}
Gerum ráð fyrir að $a$ sé venjulegur  punktur afleiðujöfnunnar 
 \begin{equation*}a_2(x)u\ddash+a_1(x)u\dash+a_0(x)u=0,


.. _3.2.6:

 \end{equation*}
og látum föllin $P(x)=a_1(x)/a_2(x)$ og $Q(x)=a_0(x)/a_2(x)$ 
vera gefin með velda\-röð\-unum 
$P(x)=\sum_{n=0}\sp \infty P_n(x-a)\sp n$
og $Q(x)= \sum_{n=0}\sp \infty Q_n(x-a)\sp n$.
Þá eru sérhver lausn $u$ á afleiðujöfnunni  gefin með veldaröð
 $$u(x)=\sum_{n=0}\sp \infty c_n(x-a)\sp n
 $$
þar sem stuðlarnir $c_n$  uppfylla rakningarformúluna.
Samleitnigeislinn er að minnsta kosti jafn stór og minni
samleitnigeisli raðanna tveggja.


--------------



Útreikningar okkar hér að framan byggðu á þeirri lausnartilgátu að $u$
væri raunfágað. 

\bigskip\hrule\bigskip


.. _syn3.2.9:

\subsubsection{Sýnidæmi}\tx{Jafna
Legendre :hover:`jafna!Legendre` :hover:`Legendre!jafna`}  Gerum ráð fyrir að jafnan
$$\dfrac {d}{dx}((1-x\sp 2)\dfrac{du}{dx})+\lambda u=
(1-x\sp 2)u\ddash-2xu\dash+\lambda u=0
$$
hafi veldaraðalausn umhverfis punktinn $a=0$, 
\begin{gather*}
u(x)=\sum\limits_{n=0}\sp\infty c_nx\sp n, \quad
u\dash(x)=\sum\limits_{n=1}\sp\infty nc_nx\sp{n-1}, \quad 
xu\dash(x)=\sum\limits_{n=0}\sp\infty nc_nx\sp n, \quad
\\
u\ddash(x)
=\sum\limits_{n=2}\sp\infty n(n-1)c_nx\sp {n-2}=
\sum\limits_{n=0}\sp\infty (n+2)(n+1)c_{n+2}x\sp n,\\
x\sp 2u\ddash(x)=\sum\limits_{n=0}\sp\infty n(n-1)c_nx\sp n.
\end{gather*}
Við stingum síðan þessum röðum inn í afleiðujöfnuna og fáum
\begin{align*}
0&=
\sum\limits_{n=0}\sp\infty (n+2)(n+1)c_{n+2}x\sp n -
\sum\limits_{n=0}\sp\infty n(n-1)c_nx\sp n\\
&-2\sum\limits_{n=0}\sp\infty nc_nx\sp n+
\lambda\sum\limits_{n=0}\sp\infty c_nx\sp n
\\
&=\sum\limits_{n=0}\sp\infty
((n+2)(n+1)c_{n+2} +(\lambda-n(n-1)-2n)c_n)x\sp n.
\end{align*}
 
Stuðlarnir verða því að uppfylla
$$
c_{n+2}=- \dfrac{\lambda-(n+1)n}{(n+2)(n+1)}c_n.
$$
Valið á fyrstu tveimur stuðlunum er frjálst og
við fáum
\begin{gather*}
c_2= -\dfrac{\lambda}{2\cdot 1}c_0, \quad
c_4= \dfrac{(\lambda-3\cdot 2)\lambda}{4\cdot 3\cdot 2\cdot
1}c_0,\quad \dots, \\
c_{2k}=(-1)\sp
k\dfrac{(\lambda-(2k-1)(2k-2))(\lambda-(2k-3)(2k-4))\cdots
(\lambda-3\cdot 2)\lambda}{(2k)!}c_0\\
c_3=- \dfrac{\lambda-2\cdot 1}{3\cdot 2}c_1, \quad
c_5= \dfrac{(\lambda-4\cdot 3)(\lambda-2\cdot 1)}{5\cdot 4\cdot 3\cdot 2}
c_1,\quad \dots,\\
c_{2k+1}=(-1)\sp
k\dfrac{(\lambda-2k(2k-1))(\lambda-(2k-2)(2k-3))\cdots
(\lambda-2\cdot 1)}{(2k+1)!}c_1.
\end{gather*}
Ef við skrifum $\lambda=\alpha(\alpha+1)$ og notfærum okkur að 
$$
\alpha(\alpha+1)-n(n+1)=(\alpha-n)(\alpha+n+1),
$$ 
þá verður rakningarformúlan fyrir röðina
$$c_{n+2}= -\dfrac{(\alpha-n)(\alpha+n+1)}{(n+2)(n+1)}c_n
$$
og almenn lausn jöfnunnar verður því 
\begin{gather*}
u(x) = c_0\sum\limits_{k=0}\sp\infty
a_{2k}
x\sp{2k}
+
c_1\sum\limits_{k=0}\sp\infty
a_{2k+1}
x\sp {2k+1},\\
a_0=a_1=1,\\
\\
a_{2k}= (-1)\sp k 
\dfrac{\alpha(\alpha-2)\cdots(\alpha-2k+2)
(\alpha+1)(\alpha+3)\cdots(\alpha+2k-1)}{(2k)!},\\
a_{2k+1}= (-1)\sp k 
\dfrac{(\alpha-1)(\alpha-3)\cdots(\alpha-2k+1)
(\alpha+2)(\alpha+4)\cdots(\alpha+2k)}{(2k+1)!}.
\end{gather*}
Nú tökum við eftir því að ef $\alpha$ er jöfn heiltala þá eru allir
liðir í fyrri summunni með númer $2k\geq \alpha+2$ jafnir núll og fyrri
summan er því margliða af stigi $\alpha$.  Ef hins vegar $\alpha$  er
oddatala þá er seinni veldaröðin margliða.  Við fáum því að fyrir
sérhvert $n$ er til margliðulausn á jöfnu
Legendre, ef $\lambda$ er
valið sem $\lambda=n(n+1)$. Venja er að skilgreina
{Legendre--margliðurnar} :hover:`Legendre!margliður` :hover:`margliða!Legendre`
sem þessar lausnir eftir að hafa valið ákveðin gildi á stuðlunum
$c_0$ og $c_1$.  Legendre--margliðurnar koma fyrir í ýmsum
útreikningum, meðal annars í rafsegulfræði.  Við höfum
ekki tök á því að gera þeim nein skil hér.


--------------



\bigskip\hrule\bigskip


\subsubsection{Sýnidæmi}\tx{Jafna Hermite :hover:`Hermite-jafna` :hover:`jafna!Hermite`}  
Við lítum nú á afleiðujöfnuna $u\ddash-2xu\dash+\lambda u=0$ og
leysum hana  með því að gera ráð fyrir að lausnin sé gefin með
veldaröð. Við notum formúlurnar fyrir $u\ddash$ og $xu\dash$ úr
sýnidæmi :ref:`syn3.2.9`.  Til einföldunar setjum við $\lambda=2\alpha$. Það
gefur okkur 
\begin{align*}
0&=
\sum\limits_{n=0}\sp\infty (n+2)(n+1)c_{n+2}x\sp n
-2\sum\limits_{n=0}\sp\infty nc_nx\sp n+
2\alpha\sum\limits_{n=0}\sp\infty c_nx\sp n=
\\
&=\sum\limits_{n=0}\sp\infty
((n+2)(n+1)c_{n+2} +2(\alpha-n)c_n)x\sp n.
\end{align*}
Stuðlarnir verða því að uppfylla
$$
c_{n+2}=- \dfrac{2(\alpha-n)}{(n+2)(n+1)}c_n.
$$
Við fáum nú formúlu fyrir lausnina
$$
u(x) = c_0\sum\limits_{k=0}\sp\infty
a_{2k}
x\sp{2k}
+
c_1\sum\limits_{k=0}\sp\infty
a_{2k+1}
x\sp {2k+1},
$$
þar sem stuðlarnir $a_k$ eru gefnir með formúlunum
\begin{gather*}
a_0=a_1=1,\\
a_2=-2\dfrac{\alpha}{2\cdot 1}, \qquad
a_4=4\dfrac{(\alpha-2)\alpha}{4\cdot 3\cdot 2\cdot 1},  \quad\dots,
\\
a_{2k}=(-1)\sp k 2\sp k \dfrac{(\alpha-2k+2)\cdots(\alpha-2)\alpha}{(2k)!},\\
a_3=-2\dfrac{(\alpha-1)}{3\cdot 2}, \qquad
a_5=4\dfrac{(\alpha-3)(\alpha-1)}{5\cdot 4\cdot 3\cdot 2},  \quad\dots,\\
a_{2k+1}= (-1)\sp k 2\sp k
\dfrac{(\alpha-2k+1)\cdots(\alpha-3)(\alpha-1)}{(2k+1)!}.
\end{gather*}
Við sjáum nú að ef $\alpha$ er heiltala $>0$ þá fæst lausn sem er
margliða.   Fyrir ákveðið val á  $c_0$ og $c_1$ fæst runa af margliðum,
en þær nefnast {\it
Hermite--margliður :hover:`margliða!Hermite` :hover:`Hermite-margliður`}.


--------------




\section{$\Gamma$--fallið :hover:`Gamma-fall`}  

\noindent
Þegar rakningarformúlur eru  notaðar til að finna beinar formúlur
fyrir stuðlana í raða\-lausnum afleiðujafna koma endurtekin
margfeldi oft fyrir.  Þá er  þægilegt að grípa til
$\Gamma$--fallsins, en það er skilgreint með formúlunni 
 \begin{equation*}\Gamma(z)=\int_0^\infty e^{-t}t^{z-1}\, dt, \qquad z\in \C, \quad \Re
z>0.


.. _3.3.1:

\end{equation*}
Greinilegt er að fyrir þessi gildi á $z$ er heildið alsamleitið.
Athugum nú að hlutheildunin
 $$\int_0^\infty e^{-t}t^{z}\, dt =\left[ -e^{-t}t^z\right]_0^\infty +
\int_0^\infty e^{-t}zt^{z-1}\, dt= z\int_0^\infty e^{-t}t^{z-1}\, dt  
 $$
gefur okkur formúluna
 \begin{equation*}\Gamma(z+1)=z\Gamma(z),


.. _3.3.2:

 \end{equation*}
og með þrepun fáum við síðan
 \begin{equation*}\Gamma(z+n)= z(z+1)\cdots(z+n-1)\Gamma(z), 
\qquad n=1,2,3,\dots.


.. _3.3.3:

 \end{equation*}
Þessa formúlu getum við síðan notað til að framlengja skilgreiningarsvæði
$\Gamma$ yfir á mengið
 $$\C\setminus\{0,-1, -2, -3,\dots\}.
 $$
Við veljum $n$ það stórt að $\Re z+n>0$ og notum  
 \begin{equation*}\Gamma(z)=\dfrac{\Gamma(z+n)}{z(z+1)\cdots(z+n-1)},


.. _3.3.4:

 \end{equation*}
til að skilgreina ${\Gamma}(z)$ 
fyrir $z$ með $\Re z\leq 0$. 

Við getum auðveldlega reiknað út $\Gamma(1)$, því 
 $$\Gamma(1)=\int_0^\infty e^{-t}\, dt=\left[-e^{-t}\right]_0^\infty=1,
 $$
en formúlan hér að framan  gefur okkur síðan
 \begin{equation*}\Gamma(n)=(n-1)!

.. _3.3.5:

 \end{equation*}
Niðurstaðan er því sú að  ${\Gamma}$ er framlenging á fallinu 
$n\mapsto (n-1)!$ frá mengi náttúrlegra talna
$\{1,2,3,\dots\}$ yfir á mengið 
$\C\setminus\{0,-1, -2, -3,\dots\}$.


Við getum líka reiknað  út $\Gamma(1/2)$, en það er gert með því
að skipta fyrst um breytistærð í heildinu
 $$\Gamma(1/2)=\int_0^\infty e^{-t}t^{-1/2}\, dt =
2\int_0^\infty e^{-x^2}\, dx= \int_{-\infty}^\infty e^{-x^2}\, dx.
 $$
Síðan athugum við að $\Gamma(1/2)^2$ má skrifa sem tvöfalt heildi
 $$\Gamma(1/2)^2= 
\int_{-\infty}^\infty e^{-x^2}\, dx\int_{-\infty}^\infty e^{-y^2}\,dy=
\int_{-\infty}^\infty \int_{-\infty}^\infty e^{-(x^2+y^2)}\, dxdy.
 $$
Næsta skref er að skipta yfir í pólhnit
 $$\Gamma(1/2)^2=\int_0^\infty\int_0^{2\pi}e^{-r^2} \, rdrd\theta =
\pi \int_0^\infty e^{-r^2} \, 2rdr= \pi\left[-e^{-r^2}\right]_0^\infty=\pi.
 $$
Við höfum því 
 \begin{equation*}

.. _3.3.6:

\Gamma(1/2)=\sqrt\pi, \qquad \Gamma(-1/2)=-2\sqrt\pi,
 \end{equation*}
og í framhaldi af því 
 $$\Gamma(n+1/2) =\frac 12\frac 32\cdots (n-\frac 12)\sqrt \pi=
\dfrac{(2n-1)!}{2^{2n-1}(n-1)!}\sqrt \pi.
 $$

.. figure:: ./myndir/fig038.svg

    :align: center

    :alt: Gamma--fallið.

    2BeRemovedMynd: Gamma--fallið.





\section{Aðferð Frobeniusar :hover:`aðferð Frobeniusar` :hover:`Frobenius`}


\subsection*{Reglulegir sérstöðupunktar}

Í þessari grein ætlum við að líta á raðalausnir á jöfnunni
 \begin{equation*}a_2(x)u\ddash+a_1(x)u\dash+a_0(x) u=0


.. _3.4.1:

 \end{equation*}
í grennd um sérstöðupunkta.  Ef $a$ er sérstöðupunktur, þá kemur í
ljós að ekki er alltaf hægt að skrifa lausnirnar sem veldaraðir.
Hins vegar er stundum hægt að skrifa  þær sem margfeldi af veldaröð
og veldisfalli
 \begin{equation*}u(x)= |x-a|\sp r\sum_{n=0}\sp \infty c_n(x-a)\sp n.


.. _3.4.2:

 \end{equation*}
Aðferð Frobeniusar gengur út á að leita að lausn af þessari gerð og
ákvarða bæði veldið $r$ og stuðlana $c_n$ út frá veldaröðum
stuðlafallanna í afleiðujöfnunni.

\subsubsection{Skilgreining}
Látum $f$ vera raunfágað fall á opnu mengi  $X$ í $\R$.
Við segjum að einangraður
sérstöðupunktur $a$ raunfágaða fallsins $f$ sé 
{\it skaut :hover:`skaut` :hover:`einangraður sérstöðupunktur!skaut`
 :hover:`sérstöðupunktur!skaut` af
stigi $m>0$}, ef til er $\varrho>0$ og raunfágað fall $g$ á
$\{x; |x-a|<\varrho\}$, þannig að
$\{x; 0<|x-a|<{\varrho}\}\subset X$,  $g(a)\neq 0$ og
$$ f(x)=\dfrac {g(x)}{(x-a)^m}\qquad 0<|x-a|<\varrho. $$


--------------




Látum $a$ vera sérstöðupunkt fyrir jöfnuna (:ref:`3.4.1`)
og skrifum 
 \begin{equation*}P(x)=\dfrac{a_1(x)}{a_2(x)}=\dfrac{p(x)}{x-a}, \qquad
Q(x)=\dfrac{a_0(x)}{a_2(x)}=\dfrac{q(x)}{(x-a)^2}.

.. _3.4.3:

 \end{equation*}

\subsubsection{Skilgreining}
Við segjum að $a$ sé {\it reglulegur sérstöðupunktur :hover:`reglulegur
sérstöðupunktur` :hover:`sérstöðupunktur!reglulegur`} 
afleiðujöfnunnar (:ref:`3.4.1`), ef $a$ er sérstöðupunktur jöfnunnar,
fallið $P$ hefur annað hvort afmáanlegan sérstöðupunkt í $a$ eða skaut
af stigi $\leq 1$ og $Q$ hefur annað hvort afmáanlegan sérstöðupunkt
í $a$ eða skaut af stigi $\leq 2$. 


--------------



Punkturinn $a$ er reglulegur sérstöðupunktur afleiðujöfnunnar þá og
því aðeins að föllin $p$ og $q$, sem skilgreind eru hér fyrir ofan, 
séu bæði fáguð í grennd um $a$.

\bigskip\hrule\bigskip

\subsection*{Útfærsla á aðferð Forbeniusar}

Nú skulum við gera ráð fyrir að við höfum afleiðujöfnu með reglulegan
sérstöðupunkt $a$ og að við umritum hana yfir á formið
 $$(x-a)^2u\ddash+(x-a)p(x)u\dash+q(x)u=0,
 $$
þar sem föllin $p$ og $q$ eru sett fram með veldaröðum
 $$p(x)= \sum_{n=0}^\infty p_n(x-a)^n, \quad
q(x)= \sum_{n=0}^\infty q_n(x-a)^n.
 $$
Við  gerum ráð fyrir því að unnt sé að skrifa lausnina sem 

.. _3.4.4:

\begin{equation*}u(x)= (x-a)^r\sum_{n=0}^\infty a_n(x-a)^n=
\sum_{n=0}^\infty a_n(x-a)^{n+r}, \qquad a<x<a+\varrho.
\end{equation*}
Við stingum röðinni inn í jöfnuna og fáum 
\begin{multline*}
\sum_{n=0}^\infty (n+r)(n+r-1)a_n(x-a)^{n+r} +
p(x)\sum_{n=0}^\infty (n+r)a_n(x-a)^{n+r} \\
+ q(x)\sum_{n=0}^\infty a_n(x-a)^{n+r} = 0.
\end{multline*}
Við stingum nú röðunum fyrir $p$ og $q$ inn í jöfnuna og 
margföldum síðan raðirnar saman 
\begin{gather*}
p(x)\sum_{n=0}^\infty (n+r)a_n(x-a)^{n+r}= \sum_{n=0}^\infty
\sum_{k=0}^n(k+r)p_{n-k}a_{k} (x-a)^{n+r},\\
q(x)\sum_{n=0}^\infty a_n(x-a)^{n+r}= \sum_{n=0}^\infty
\sum_{k=0}^n q_{n-k}a_{k} (x-a)^{n+r}.
\end{gather*}
Til þess að jafnan gildi, þá þurfa stuðlarnir við öll veldin í
liðuninni að vera núll, en það jafngildir
 \begin{equation*}(n+r)(n+r-1)a_n+\sum_{k=0}^n\big((k+r)p_{n-k}+q_{n-k}\big)a_k=0,
\qquad n=0,1,2,\dots.


.. _3.4.5:

 \end{equation*}
Athugum nú sérstaklega tilfellið $n=0$, en það er jafnan
 $$(r(r-1)+p_0r+q_0)a_0=0.
 $$
Til þess að við getum valið stuðulinn $a_0$ frjálst, þá þarf talan
$r$ að uppfylla annars stigs jöfnuna
 \begin{equation*}r(r-1)+p_0r+q_0=r(r-1)+ p(a)r+q(a)=0.


.. _3.4.6:

 \end{equation*}
\subsubsection{Skilgreining}
Gerum ráð fyrir að $a$ sé reglulegur sérstöðupunktur afleiðujöfnunnar
 \begin{equation*}(x-a)^2u\ddash+(x-a)p(x)u\dash+q(x)u=0.

.. _3.4.7:

 \end{equation*}
Þá kallast margliðan 
 $$\varphi(\lambda)=\lambda(\lambda-1)+p(a)\lambda+q(a)
 $$
{\it vísamargliða afleiðujöfnunnar í punktinum :hover:`vísamargliða
afleiðujöfnu` :hover:`afleiðujafna!vísamargliða` :hover:`margliða!vísamargliða`}
$a$, jafnan $\varphi(\lambda)=0$ kallast {\it vísajafna afleiðujöfnunnar í
punktinum :hover:`vísajafna afleiðujöfnu` :hover:`afleiðujafna!vísajafna`}
$a$. Núllstöðvar hennar kallast {\it vísar jöfnunnar í
punktinum :hover:`vísir aðleiðujöfnu` :hover:`afleiðujafna!vísir`} $a$.


--------------



Við höfum sem sagt komist að því í útreikningum okkar, að til þess að
fallið $u(x)$ sem gefið er með formúlunni, geti verið lausn
á afleiðujöfnunni, þá þarf talan $r$ að vera vísir jöfnunnar í
punktinum $a$.  

Lítum nú á jöfnuna aftur í tilfellinu
$n>0$, en hún er
 \begin{gather*}
(n+r)(n+r-1)a_n+\sum_{k=0}^n\big((k+r)p_{n-k}+q_{n-k}\big)a_k\\
=\big((n+r)(n+r-1)+p_0(n+r)+q_0 \big)a_n 
+\sum_{k=0}^{n-1}\big((k+r)p_{n-k}+q_{n-k}\big)a_k\\
= \varphi(n+r)a_n + \sum_{k=0}^{n-1}\big((k+r)p_{n-k}+q_{n-k}\big)a_k=0.
\end{gather*}
Ef  $r$ er vísir jöfnunnar og $\varphi(n+r)\neq 0$ fyrir öll $n>0$,
þá fáum við rakningarformúluna
 $$a_n=\dfrac{-1}{\varphi(r+n)}\sum_{k=0}^{n-1}\big((k+r)p_{n-k}+q_{n-k}\big)a_k.
 $$
Við erum nú komin að meginniðurstöðu kaflans:

\subsubsection{Setning}\tx{Frobenius}   :hover:`aðferð Frobeniusar` :hover:`Frobenius`
 :hover:`setning!Frobenius`
Gerum ráð fyrir því að $a$ sé reglulegur sérstöðupunktur
afleiðujöfnunnar 
 \begin{equation*}(x-a)^2u\ddash+ (x-a)p(x)u\dash+q(x)u=0


.. _3.4.8:

 \end{equation*}
og gerum ráð fyrir að föllin $p$ og $q$ séu sett fram með
veldaröðunum 
 \begin{equation*}p(x)=\sum_{n=0}^\infty p_n(x-a)^n, \qquad
q(x)=\sum_{n=0}^\infty q_n(x-a)^n,


.. _3.4.9:

 \end{equation*}
og að þær séu samleitnar ef $|x-a|<\varrho$.  Látum $r_1$ og $r_2$
vera núllstöðvar vísajöfnunnar
 $$\varphi(\lambda)=\lambda(\lambda-1)+p(a)\lambda+q(a)=0
 $$
og gerum ráð fyrir að $\Re r_1\geq \Re r_2$.  Þá gildir:

(i) Til er lausn $u_1$ á afleiðujöfnunni  sem gefin er með 
 $$u_1(x)=|x-a|^{r_1}\sum_{n=0}^\infty a_n(x-a)^n.
 $$
Röðin er samleitin fyrir öll $x$ sem uppfylla $0<|x-a|<\varrho$.
Valið á $a_0$ er frjálst, en hinir 
stuðlar raðarinnar  fást með rakningarformúlunni
 $$
a_n=\dfrac{-1}{\varphi(n+r_1)}
\sum_{k=0}^{n-1}((k+r_1)p_{n-k}+q_{n-k})a_k, \qquad n=1,2,3,\dots.
 $$
(ii) Ef $r_1-r_2\neq 0,1,2,\dots$, þá er til önnur línulega óháð 
lausn $u_2$,  sem gefin er með 
 $$u_2(x)=|x-a|^{r_2}\sum_{n=0}^\infty b_n(x-a)^n.
 $$
Röðin er samleitin fyrir öll $x$ sem uppfylla $0<|x-a|<\varrho$.
Valið á $b_0$ er frjálst, en hinir
stuðlar raðarinnar fást með rakningarformúlunni
 $$
b_n=\dfrac{-1}{\varphi(n+r_2)}
\sum_{k=0}^{n-1}((k+r_2)p_{n-k}+q_{n-k})b_k, \qquad n=1,2,3,\dots.
 $$
(iii) Ef $r_1-r_2=0$, þá er til önnur línulega óháð lausn $u_2$,
sem gefin er með 
 $$u_2(x)=|x-a|^{r_1+1}\sum_{n=0}^\infty b_n(x-a)^n+
u_1(x)\ln|x-a|.
 $$
Röðin er samleitin fyrir öll $x$ sem uppfylla $0<|x-a|<\varrho$ og
stuðlar raðarinnar  fást með innsetningu í jöfnuna.

\smallskip\noindent
(iv) Ef $r_1-r_2=N$, þar sem $N$ er jákvæð heiltala, þá er til önnur
línulega óháð lausn, sem gefin er með 
 $$u_2(x)=|x-a|^{r_2}\sum_{n=0}^\infty b_n(x-a)^n+
\gamma u_1(x)\ln|x-a|.
 $$
Röðin er samleitin fyrir öll $x$ sem uppfylla $0<|x-a|<\varrho$.
Stuðlar raðarinnar og $\gamma$  fást með innsetningu í jöfnuna.


--------------



Við höfum aðeins sannað lítið brot af setningunni, en látum það duga.


\section{Bessel--jafnan :hover:`Bessel-jafnan`}

\subsection{Bessel--jafnan :hover:`Bessel-jafnan`}


\noindent
Við skulum nú taka fyrir aðferð Frobeniusar
til þess að leysa Bessel--jöfnuna
 \begin{equation*}P(x,D)u=x^2u\ddash+xu\dash+(x^2-\alpha^2)u=0


.. _3.5.1:

 \end{equation*}
í grennd um reglulega sérstöðupunktinn $a=0$.  
Hér er  $p(x)=1$ og $q(x)=x^2-\alpha^2$, svo vísajafnan er
 \begin{equation*}\varphi(\lambda)=\lambda(\lambda-1)+\lambda-\alpha^2=
\lambda^2-\alpha^2=0


.. _3.5.2:

 \end{equation*}
og núllstöðvar hennar eru $r_1=\alpha$ og $r_2=-\alpha$.  Við hugsum
okkur að $\Re \alpha\geq 0$.  Setning Frobeniusar segir okkur að við
fáum lausn af gerðinni 
$$ u_1(x)=|x|^\alpha\sum_{n=0}^\infty a_n x^n, $$
þar sem við getum valið stuðulinn $a_0$ frjálst og hina stuðlana út
frá rakningarformúlunni
$$ \varphi(\alpha+1)a_1=0, \qquad \varphi(\alpha+n)a_n=-a_{n-2}. $$
Þar sem $\varphi(\alpha+1)\neq 0$ þá verður $a_1=0$ og í framhaldi af
því fæst $0=a_3=a_5=\cdots$.  Til þess að finna formúluna fyrir
$a_{2k}$ þá athugum við að 
$$\varphi(\alpha+2k)=(\alpha+2k)^2-\alpha^2= 4k\alpha+4k^2=
2^2k(\alpha+k),
$$
og þar með verður 
\begin{gather*}
a_2=\dfrac{-a_0}{2^2(\alpha+1)}, \quad
a_4=\dfrac{a_0}{2^42(\alpha+1)(\alpha+2)}, \dots  \\
a_{2k}=\dfrac{(-1)^ka_0}{2^{2k}k!(\alpha+1)\cdots(\alpha+k)}.
\end{gather*}
Athugum nú að 
$$
(\alpha+1)\cdots(\alpha+k)={\Gamma}({\alpha}+k+1)/{\Gamma}({\alpha}+1).
$$
Það er því eðlilegt að velja
 $$a_0=\dfrac 1{2^\alpha\Gamma(\alpha+1)}.
 $$
\subsubsection{Skilgreining}
Lausnin á Bessel--jöfnunni $x^2u\ddash+xu\dash+(x^2-\alpha^2)u=0$, sem
gefin er með formúlunni
 \begin{equation*}J_\alpha(x)=\left|\dfrac x2\right|^\alpha\sum_{k=0}^\infty
\dfrac{(-1)^k}{k!\Gamma(\alpha+k+1)}\left( \dfrac x2\right)^{2k}


.. _3.5.3:

 \end{equation*}
er kölluð {\it fall Bessels :hover:`fall Bessels` :hover:`Bessel-fall`
af fyrstu gerð :hover:`Bessel-fall!af fyrstu gerð` með vísi $\alpha$}.


--------------



\smallskip
Nú þurfum við að finna línulega óháða lausn og skiptum í tilfelli:

\smallskip
\ugrein{Tilfellið $\alpha\neq 0,1,2\dots$}  Talan $-{\alpha}$ er vísir
Bessel-jöfnunnar og með því að skipta á ${\alpha}$ og $-{\alpha}$ í
rakningarformúlunum hér að framan, þá fáum við aðra línulega óháða lausn
 \begin{equation*}
J_{-\alpha}(x)=\left|\dfrac x2\right|^{-\alpha}\sum_{k=0}^\infty
\dfrac{(-1)^k}{k!\Gamma(-\alpha+k+1)}\left( \dfrac x2\right)^{2k}


.. _3.5.4:

 \end{equation*}
og sérhverja lausn má síðan skrifa sem línulega samantekt af 
$J_{\alpha}$ og $J_{-\alpha}$.

\ugrein{Tilfellið $\alpha=0$} 
Bessel-jafnan í tilfellinu ${\alpha}=0$ er jafngild jöfnunni
\begin{equation*}
xu\ddash+u\dash+xu=0,


.. _3.5.5:

\end{equation*}
og við erum búin að finna eina lausn á henni
$$
u_1(x)=J_0(x)=\sum\limits_{k=0}^{\infty}
\dfrac{(-1)^k}{2^{2k}(k!)^2}x^{2k}.
$$
Samkvæmt tilfelli (iii) í setningu Frobeniusar vitum við að
til er önnur línulega óháð lausn $u_2$, sem gefin er á jákvæða raunásnum
með formúlu af gerðinni
\begin{equation*}
u_2(x)=J_0(x)\ln x+x\sum\limits_{n=0}^{\infty} b_nx^n
=J_0(x)\ln x+\sum\limits_{m=1}^{\infty} A_mx^m.


.. _3.5.6:

\end{equation*}
Við reiknum út afleiðurnar af $u_2$ 
\begin{align*}
u_2\dash(x)&=J_0\dash(x)\ln x +\dfrac{J_0(x)}x+
\sum\limits_{m=1}^{\infty} mA_mx^{m-1},\\
u_2\ddash(x)&= J_0\ddash(x)\ln x+\dfrac{2J_0\dash(x)}x-\dfrac{J_0(x)}{x^2}
+\sum\limits_{m=1}^{\infty} m(m-1)A_mx^{m-2},
\end{align*}
stingum þeim inn í afleiðujöfnuna  og notfærum okkur að $J_0$
er lausn.  Þá fáum við
$$
2J_0\dash(x)+\sum\limits_{m=1}^{\infty} m(m-1)A_mx^{m-1}
+\sum\limits_{m=1}^{\infty} mA_mx^{m-1}
+\sum\limits_{m=1}^{\infty} A_mx^{m+1}=0.
$$
Til þess að fá formúlu fyrir stuðlana $A_m$, þá verðum við að stinga
röðinni fyrir $J_0\dash$ inn í þessa jöfnu,
$$
J_0\dash(x)=\sum \limits_{k=1}^{\infty}
\dfrac{(-1)^k2k}{2^{2k}(k!)^2}x^{2k-1}
=\sum \limits_{k=1}^{\infty}
\dfrac{(-1)^kx^{2k-1}}{2^{2k-1}k!(k-1)!}
$$
og taka summurnar þrjár saman í eina.  Við fáum þá jöfnuna 
$$
A_1x^0+4A_2x+\sum\limits_{m=2}^{\infty} 
\big((m+1)^2A_{m+1}+A_{m-1}\big)x^m
=\sum \limits_{k=1}^{\infty}
\dfrac{(-1)^{k-1}x^{2k-1}}{2^{2k-2}k!(k-1)!}.
$$
Nú eru allir stuðlarnir í hægri hliðinni við slétt veldi af $x$ 
 jafnir $0$ og því fáum við
$$
A_1=0, \qquad   (2k+1)^2A_{2k+1}+A_{2k-1}=0.
$$
Þessar jöfnur gefa að $A_m=0$ ef $m$ er oddatala.  Snúum okkur nú  að $A_m$
þar sem $m$ er slétt tala.  Við höfum
$$
4A_2=1, \qquad (2k)^2A_{2k}+A_{2k-2}=
\dfrac{(-1)^{k-1}}{2^{2k-2}k!(k-1)!}.
$$
Með þrepun fæst síðan formúlan
$$
A_{2k}=\dfrac{(-1)^{k-1}}{2^{2k}(k!)^2} h_k, \qquad k=1,2,3,\dots,
$$
þar sem $h_k=1+1/2+1/3+\cdots+1/k$.
Við getum því skrifað lausnina  sem
$$
u_2(x)= J_0(x)\ln x+
\sum\limits_{k=1}^{\infty}
\dfrac{(-1)^{k-1}h_k}{2^{2k}(k!)^2} x^{2k}.
$$
Það er venja að nota annað fall en $u_2$ sem grunnfall:


\subsubsection{Skilgreining}  Fallið $Y_0$, sem skilgreint er með 
\begin{equation*}
Y_0(x)=\dfrac 2{\pi}\left[J_0(x)\bigg(\ln \dfrac {|x|}2+{\gamma}\bigg)
+\sum\limits_{k=0}^{\infty}

.. _3.5.7:

\dfrac{(-1)^{k-1}h_k}{2^{2k}(k!)^2} x^{2k}\right],
\end{equation*}
þar sem $h_k=1+1/2+1/3+\cdots+1/k$ og ${\gamma}$ táknar fasta
Eulers :hover:`Euler!fasti` :hover:`fasti Eulers`
\begin{align*}
{\gamma}&=\lim\limits_{k\to {\infty}} \big(1+1/2+\cdots+1/k-\ln k\big)
\\
&\approx 0.577 \,  215 \,  644 \, 90 \dots,\nonumber
\end{align*}
nefnist {\it fall Bessels :hover:`Bessel-fall` :hover:`Bessel-fall!af
annarri gerð` af annarri gerð með vísi $0$}.


--------------



Það er ljóst að föllin $J_0$ og $Y_0$ eru línulega óháð, svo sérhverja
lausn á Bessel-jöfnunni með vísi ${\alpha}=0$ er unnt að skrifa sem
línulega samantekt af þeim. 


\ugrein{Tilfellið  $\alpha=1,2,3,\dots$}
Hér er gengið út frá lausnarformúlunni í tilfelli (iv) í setningu
Frobeniusar.  Lausnaraðferðin er sú sama og í tilfellinu ${\alpha}=0$,
en útfærslan er töluvert snúnari og förum við ekki út í hana hér. 
Niðurstaðan er alla vega sú, að  til sögunnar kemur nýtt fall:


\subsubsection{Skilgreining}  Fallið $Y_{\alpha}$, ${\alpha}=1,2,3,\dots$ sem skilgreint er með 
\begin{align*}
Y_{\alpha}(x)=\dfrac 2{\pi}\bigg[
J_{\alpha}(x)\bigg(\ln \dfrac {|x|}2+{\gamma}\bigg)
&+x^{\alpha}\sum\limits_{k=0}^{\infty}
\dfrac{(-1)^{k-1}\big(h_k+h_{k+\alpha}\big)}
{2^{2k+\alpha+1}k!(k+{\alpha})!} x^{2k}

.. _3.5.9:
\\
&-x^{-\alpha}\sum\limits_{k=0}^{\alpha-1}
\dfrac{(\alpha-k-1)!}{2^{2k-\alpha+1}k!}x^{2k}\bigg],\nonumber
\end{align*}
þar sem $h_k=1+1/2+1/3+\cdots+1/k$  og ${\gamma}$ táknar fasta
Eulers,
nefnist {\it fall Bessels af annarri gerð með vísi ${\alpha}$}.


--------------




Almenn lausn á Bessel-jöfnunni með vísi ${\alpha}$ er línuleg samantekt
af $J_{\alpha}$ og $Y_{\alpha}$, ${\alpha}=1,2,3,\dots$. Það er hægt að
skilgreina $Y_{\alpha}$ fyrir önnur gildi á ${\alpha}$.  Það er
gert með formúlunni
$$
Y_{\alpha}(x)=\dfrac 1{\sin {\alpha}{\pi}}\left[
J_{\alpha}(x)\cos{\alpha}{\pi} -J_{-{\alpha}}(x)
\right], \qquad {\alpha}\in \C, \ \Re {\alpha}\geq 0, {\alpha}\neq
1,2,3,\dots. 
$$
Þá fæst nokkuð merkileg formúla
$$
Y_{\alpha}(x)=\lim_{{\beta}\to {\alpha}} Y_{\beta}(x), \qquad 
{\alpha}=1,2,3,\dots .
$$
Við  höldum ekki lengra inn á þessa braut og endum kaflann
 með gröfum fallanna 
$J_0$, $Y_0$, $J_1$, $Y_1$, $J_2$ og  $Y_2$. 

\figura{fig039}{}

\figura{fig0310}{}

\figura{fig0311}{}


