
%
%Allir pakkar sem þarf að nota.
%
\usepackage[utf8]{inputenc}
\usepackage[T1]{fontenc}
\usepackage[icelandic]{babel}
\usepackage{amsmath}
\usepackage{amssymb}
\usepackage{pictex}
\usepackage{epsfig,psfrag}
\usepackage{makeidx}
%\selectlanguage{icelandic}
%----------------------------

%
\hoffset=-0.4truecm
\voffset=-1truecm
\textwidth=16truecm 
%\textwidth=12truecm 
\textheight=23truecm
\evensidemargin=0truecm
%
% Gömlu gildin á bókinni 
%
%\voffset 1.4truecm
%\hoffset .25truecm
%\vsize  16.0truecm
%\hsize  15truecm
%
%
% Skilgreiningar á ýmsum skipunum.
%
%\newcommand{\Sb}{
%$$
%\sum_{\footnotesize\begin{array}{l} j=1 \\ j\neq k \end{array}}
%$$
%}
\newcommand{\bolddot}{{\mathbf \cdot}}
\newcommand{\C}{{\mathbb  C}}
\newcommand{\Cn}{{\mathbb  C\sp n}}
\newcommand{\crn}{{{\mathbb  C\mathbb  R^n}}}
\newcommand{\R}{{\mathbb  R}}
\newcommand{\Rn}{{\mathbb  R\sp n}}
\newcommand{\Rnn}{{\mathbb  R\sp{n\times n}}}
\newcommand{\Z}{{\mathbb  Z}}
\newcommand{\N}{{\mathbb  N}}
\renewcommand{\P}{{\mathbb  P}}
\newcommand{\Q}{{\mathbb  Q}}
\newcommand{\K}{{\mathbb  K}}
\newcommand{\U}{{\mathbb  U}}
\newcommand{\D}{{\mathbb  D}}
\newcommand{\T}{{\mathbb  T}}
\newcommand{\A}{{\cal A}}
\newcommand{\E}{{\cal E}}
\newcommand{\F}{{\cal F}}
\renewcommand{\H}{{\cal H}}
\renewcommand{\L}{{\cal L}}
\newcommand{\M}{{\cal M}}
\renewcommand{\O}{{\cal O}}
\renewcommand{\S}{{\cal S}}
\newcommand{\dash}{{\sp{\prime}}}
\newcommand{\ddash}{{\sp{\prime\prime}}}
\newcommand{\tdash}{{\sp{\prime\prime\prime}}}
\newcommand{\set }[1]{{\{#1\}}}
\newcommand{\scalar}[2]{{\langle#1,#2\rangle}}
\newcommand{\arccot}{{\operatorname{arccot}}}
\newcommand{\arccoth}{{\operatorname{arccoth}}}
\newcommand{\arccosh}{{\operatorname{arccosh}}}
\newcommand{\arcsinh}{{\operatorname{arcsinh}}}
\newcommand{\arctanh}{{\operatorname{arctanh}}}
\newcommand{\Log}{{\operatorname{Log}}}
\newcommand{\Arg}{{\operatorname{Arg}}}
\newcommand{\grad}{{\operatorname{grad}}}
\newcommand{\graf}{{\operatorname{graf}}}
\renewcommand{\div}{{\operatorname{div}}}
\newcommand{\rot}{{\operatorname{rot}}}
\newcommand{\curl}{{\operatorname{curl}}}
\renewcommand{\Im}{{\operatorname{Im\, }}}
\renewcommand{\Re}{{\operatorname{Re\, }}}
\newcommand{\Res}{{\operatorname{Res}}}
\newcommand{\vp}{{\operatorname{vp}}}
\newcommand{\mynd}[1]{{{\operatorname{mynd}(#1)}}}
\newcommand{\dbar}{{{\overline\partial}}}
\newcommand{\inv}{{\operatorname{inv}}}
\newcommand{\sign}{{\operatorname{sign}}}
\newcommand{\trace}{{\operatorname{trace}}}
\newcommand{\conv}{{\operatorname{conv}}}
\newcommand{\Span}{{\operatorname{Sp}}}
\newcommand{\stig}{{\operatorname{stig}}}
\newcommand{\Exp}{{\operatorname{Exp}}}
\newcommand{\diag}{{\operatorname{diag}}}
\newcommand{\adj}{{\operatorname{adj}}}
\newcommand{\erf}{{\operatorname{erf}}}
\newcommand{\erfc}{{\operatorname{erfc}}}
\newcommand{\Lloc}{{L_{\text{loc}}\sp 1}}
\newcommand{\boldcdot}{{\mathbb \cdot}}
%\newcommand{\Cinf0}[1]{{C_0\sp{\infty}(#1)}}
\newcommand{\supp}{{\text{supp}\, }}
\newcommand{\chsupp}{{\text{ch supp}\, }}
\newcommand{\singsupp}{{\text{sing supp}\, }}
\newcommand{\SL}[1]{{\dfrac {1}{\varrho} 
\bigg(-\dfrac d{dx}\bigg(p\dfrac {d#1}{dx}\bigg)+q#1\bigg)}}
\newcommand{\SLL}[1]{-\dfrac d{dx}\bigg(p\dfrac {d#1}{dx}\bigg)+q#1}
\newcommand{\Laplace}[1]{\dfrac{\partial^2 #1}{\partial x^2}+\dfrac{\partial^2 #1}{\partial y^2}}
\newcommand{\polh}[1]{{\widehat #1_{\C^n}}}
\newcommand{\tilv}{{}}
%
\renewcommand{\chaptername}{Kafli}
%
% Númering á formæulum.
%
\numberwithin{equation}{section}
%
%  Innsetning á myndum.
%
\def\figura#1#2{
\vbox{\centerline{
\input #1
}
\centerline{#2}
}\medskip}
\def\vfigura#1#2{
\setbox0\vbox{{
\input #1
}}
\setbox1\vbox{\hbox{\box0}\hbox{{\obeylines #2}}}
\dimen0 = -\ht1
\advance\dimen0 by-\dp1
\dimen1 = \wd1
\dimen2 = -\dimen0
\divide\dimen2 by\baselineskip
\count100 = 1
\advance\count100 by\dimen2
\advance\count100 by1
\box1
\hangindent\dimen1
\hangafter=-\count100
\vskip\dimen0
}
%
%  Setningar, skilgreiningar, o.s.frv. 
%
\newtheorem{setning+}           {Setning}      [section]
\newtheorem{skilgreining+}  [setning+]  {Skilgreining}
\newtheorem{setningogskilgreining+}  [setning+]  {Setning og
skilgreining}
\newtheorem{hjalparsetning+}  [setning+]  {Hjálparsetning}
\newtheorem{fylgisetning+}  [setning+]  {Fylgisetning}
\newtheorem{synidaemi+}  [setning+]  {Sýnidæmi}
\newtheorem{forrit+}  [setning+]  {Forrit}

\newcommand{\tx}[1]{{\rm({\it #1}). \ }}

\newenvironment{se}{\begin{setning+}\sl}{\hfill$\square$\end{setning+}\rm}
\newenvironment{sex}{\begin{setning+}\sl}{\hfill$\blacksquare$\end{setning+}\rm}
\newenvironment{sk}{\begin{skilgreining+}\rm}{\hfill$\square$\end{skilgreining+}\rm}
\newenvironment{sesk}{\begin{setningogskilgreining+}\rm}{\hfill$\square$\end{setningogskilgreining+}\rm}
\newenvironment{hs}{\begin{hjalparsetning+}\sl}{\hfill$\square$\end{hjalparsetning+}\rm}
\newenvironment{fs}{\begin{fylgisetning+}\sl}{\hfill$\square$\end{fylgisetning+}\rm}
\newenvironment{sy}{\begin{synidaemi+}\rm}{\hfill$\square$\end{synidaemi+}\rm}
\newenvironment{fo}{\begin{forrit+}\rm}{\hfill\end{forrit+}\rm}
\newenvironment{so}{\medbreak\noindent{\it Sönnun:}\rm}{\hfill$\blacksquare$\rm}
\newenvironment{sotx}[1]{\medbreak\noindent{\it #1:}\rm}{\hfill$\blacksquare$\rm}
\newcounter{daemateljari}
\newcommand{\aefing}{\section{Æfingardæmi} \setcounter{daemateljari}{1}}
\newcommand{\daemi}{
{\medskip\noindent{\bf \thedaemateljari.}}
\addtocounter{daemateljari}{1}
}

%\def\aefing{{\large\bf\bigskip\bigskip\noindent Æfingardæmi}}
%\def\daemi#1{\medskip\noindent{\bf #1.}}
\def\svar#1{\smallskip\noindent{\bf #1.} \ }
\def\lausn#1{\smallskip\noindent{\bf #1.} \ }
\def\ugrein#1{\medbreak\noindent{\bf #1.} }
\newcommand{\samantekt}{\noindent{\bf Samantekt.} }
%\newcommand{\proclaimbox}{\hfill$\square$}

%
\chapter
{FÁGUÐ FÖLL}
 
%\kaflahaus{Fáguð föll}



\section{Markgildi og samfelld föll}

\subsection*{Skífur og hringir}


{\it Opna skífu :hover:`opin skífa` :hover:`skífa!opin`} með miðju $\alpha$ og geisla
$\varrho$ táknum við með
$$ S(\alpha,\varrho)=\{z\in \C; |z-\alpha|<\varrho\}, $$
{\it lokaða skífu :hover:`lokuð skífa` :hover:`skífa!lokuð`} með miðju $\alpha$ og geisla 
$\varrho$ táknum við með
$$ \overline S(\alpha,\varrho)=\{z\in \C; |z-\alpha|\leq\varrho\} $$
og {\it gataða opna skífu :hover:`götuð opin skífa` :hover:`skífa!götuð opin`} með miðju $\alpha$ og
geisla $\varrho$ táknum við með
$$ S^\ast(\alpha,\varrho)=\{z\in \C; 0<|z-\alpha|<\varrho\}. $$


Athugið að fallið $[a,b]\ni \theta\mapsto \alpha+\varrho e^{i\theta}$ stikar
hringboga með miðju $\alpha$ og geislann $\varrho$ frá punktinum
$\alpha+\varrho e^{ia}$
til punktsins $\alpha+\varrho e^{ib}$ og að það stikar heilan hring ef 
$b-a=2\pi$.

\subsection*{Opin og lokuð mengi}

Hlutmengi $X$ í $\C$ er sagt vera {\it opið} ef um sérhvern punkt $a\in X$
gildir að til er opin skífa $S(a,r)$ sem er innihaldin í $X$.

Hlutmengi  $X$ í $\C$ er sagt vera {\it lokað } ef fyllimengi þess
$\C\setminus X$ er opið.  

Ljóst er að mengi $X$ er lokað þá og því
aðeins að um sérhvern punkt $a$ í fyllimenginu $\C\setminus X$ gildir
að til er $r>0$ þannig að $S(a,r)\subset \C\setminus X$.  


{\it Jaðar} hlutmengis $X$ í $\C$ samanstendur af öllum punktum
$a\in \C$ þannig að sérhver opin skífa $S(a,r)$ með $r>0$ sker bæði
$X$ og $\C\setminus X$.  Við táknum jaðar $X$ með $\partial X$.  

Ef $X$ er opið, þá er $\partial X\subset \C\setminus X$.

Ef $X$ er lokað, þá er $\partial X\subset X$.   

Punktur $a\in \C$ nefnist {\it þéttipunktur} mengisins $X$ ef um
sérhvert $r>0$ gildir að gataða opna skífan $S^\ast(a,r)$ inniheldur
punkta úr $X$.  


Hlutmengi $X$ í $\C$ er sagt vera
{\it samanhangangi} ef um sérhverja tvo punkta $a$ og $b$ í 
$X$ gildir að til er samfelldur ferill  
$[0,1]\ni t\mapsto \gamma(t)\in \C$ sem er
innihaldinn í $X$.   

Opið samanhangandi mengi nefnist {\it svæði}.  


Athugið að sérhver opin skífa er svæði, því sérhverja tvo punkta í
henni má
tengja saman með línustriki.  Lokaðar skífur eru samanhangandi,
og sama er að segja um gataðar skífur.


\subsection*{Markgildi}

Látum nú $X$ vera hlutmengi í $\C$ og $f:X\to \C$ vera fall.  

Við segjum að $f(z)$ stefni á tvinntöluna $L$ þegar  $z$ stefnir á
$a$,  ef $a$ er
þéttipunktur í $X$ og fyrir sérhvert $\varepsilon>0$ gildir að til er
$\delta>0$ þannig að 
$$
|f(z)-L|<\varepsilon \qquad \text{ fyrir öll } z\in X\cap S^\ast(a,\delta).
$$
Við köllum þá töluna $L$ {\it markgildi $f$ þegar $z$ stefnir á $a$}
og skrifum 
$$
\lim_{z\to a}f(z)=L  \qquad \text{ eða } \quad f(z)\to L \text{ ef }
z\to a.
$$
Við höfum nokkrar reiknireglur fyrir markgildi:  Ef $f$ og $g$ eru
tvinngild föll sem skilgreind eru á menginu $X$, $\lim_{z\to a}f(z)=L$
og $\lim_{z\to a}g(z)=M$, þá er 
\begin{gather*}
\lim_{z\to a}(f(z)+g(z))=\lim_{x\to a}f(z)+\lim_{x\to a}g(z)=L+M,\\
\lim_{z\to a}(f(z)-g(z))=\lim_{x\to a}f(z)-\lim_{x\to a}g(z)=L-M,\\
\lim_{z\to a}(f(z)g(z))=\big(\lim_{x\to a}f(z)\big)\big(\lim_{x\to
a}g(z)\big)=LM\\
\lim_{z\to a}\dfrac{f(z)}{g(z)}=\dfrac{\lim_{x\to a}f(z)}{\lim_{x\to
a}g(z)}=\dfrac LM.
\end{gather*}
Í síðustu formúlunni þarf að gera ráð fyrir að $M\neq 0$.



\subsection*{Samfelldni}

Fallið $f:X\to \C$ er sagt vera samfellt í punktinum $a\in X$ ef
$$
\lim_{z\to a}f(z)=f(a).
$$


Af reiknireglunum fyrir markgildi leiðir að ef $f$ og $g$ eru föll
á mengi $X$ með gildi í $\C$ sem eru samfelld í punktinum $a\in X$, þá
eru $f+g$, $f-g$, $fg$ og $f/g$ samfelld í $a$ og 
\begin{gather*}
\lim_{x\to a}(f(z)+g(z))=f(a)+g(a),\\
\lim_{x\to a}(f(z)-g(z))=f(a)-g(a),\\
\lim_{x\to a}(f(z)g(z))=f(a)g(a),\\
\lim_{x\to a}\dfrac{f(z)}{g(z)}=\dfrac{f(a)}{g(a)}, 
\qquad \text{ef } \ g(a)\neq 0.
\end{gather*}
Ef $f:X\to \C$ og $g:Y\to \C$ eru föll,  $f(X)\subset Y$,
$a$ er þéttipunkur $X$, $b=\lim_{z\to a}f(z)$ er
þéttipunktur mengisins $Y$ og $g$ er samfellt í $b$, þá er
$$
\lim_{z\to a} g\circ f(z)=g(\lim_{z\to a}f(z)).
$$



\subsection*{Ritháttur fyrir hlutafleiður}


Ef $f$ er fall af breytistærðunum $x,y,z,\dots$, þá skrifum við
$$
{\partial}_xf=\dfrac{\partial f}{\partial x}, \qquad
{\partial}_yf=\dfrac{\partial f}{\partial y}, \qquad
{\partial}_zf=\dfrac{\partial f}{\partial z}, \ \dots
$$
og hærri afleiður táknum við með
$$
{\partial}_x^2f=\dfrac{\partial^2f}{\partial x^2}, \qquad
{\partial}_{xy}^2f=\dfrac{\partial^2f}{\partial x\partial y}, \qquad
{\partial}_{xxy}^3f=\dfrac{\partial^3f}{\partial x^2\partial y}, \ \dots.
$$

Í mörgum bókum eru hlutafleiður skrifaðar sem $f_{x}$, $f_y$ o.s.frv.
 Þessi
ritháttur hentar okkur illa, því við notum lágvísinn til þess að tákna
ýmislegt annað en hlutafleiður.  Mun skýrari ritháttur, sem við notum
þó ekki,  er að tákna
hlutafleiður með $f_x'$, $f_y'$ o.s.frv.  

\subsection*{Samfellt deildanleg föll}

\medskip\noindent
Við fjöllum mikið  um
samfelld og deildanleg föll  og 
þess vegna er mjög hagkvæmt að innleiða rithátt fyrir mengi allra falla
sem eru samfelld á einhverju mengi.


Ef $X$ er opið hlutmengi í $\C$ þá látum við $C(X)$ tákna mengi
allra samfelldra falla $f:X\to \C$.  Það er til mikilla þæginda að
gera frá byrjun ráð fyrir að föllin séu tvinntölugild.  Við látum
$C\sp m(X)$ tákna mengi allra $m$ sinnum :hover:`samfellt
deilanlegur!$m$ sinnum` samfellt
deildanlegra :hover:`samfellt deilanlegur` falla.
Hér er átt við að allar hlutafleiður fallsins $f$ af stigi $\leq m$
eru til og þar að auki samfelldar.  Við skrifum $C^0(X)=C(X)$ og
táknum mengi óendanlega oft deildanlegra falla með $C^{\infty}(X)$.


\section{Fáguð föll}


\noindent
Látum $f:X\to \C$ vera fall á opnu hlutmengi $X$ af $\C$.  
Ef við látum $z$ tákna tvinnbreytistærð með gildi í $X$, þá getum við
skrifað 
 \begin{equation*}f(z)=u(z)+iv(z)=u(x,y)+iv(x,y), \qquad z=x+iy=(x,y) \in X,


.. _4.2.1:

 \end{equation*}
þar sem föllin $u=\Re f$ og $v=\Im f$ eru raunhluti og þverhluti
fallsins $f$.
Við getum þá jafnframt litið á $f$ sem vigurgilt fall
af tveimur raunbreytistærðum
 \begin{equation*}f:X\to \R\sp 2, \qquad f(x,y)=(u(x,y), v(x,y)).


.. _4.2.2:

 \end{equation*}
Hugtök eins og samfelldni, deildanleiki og heildanleiki  eru
skilgreind eins og venjulega fyrir vigurgild föll.  Þetta þýðir að
$f$ er samfellt á $X$, $f\in C(X)$, þá og því aðeins að föllin $u$ og
$v$ séu samfelld á $X$, $u,v\in C(X)$.  Eins er $f$ $k$--sinnum
samfellt deildanlegt á $X$, $f\in C\sp k(X)$ þá og því aðeins að
$u,v\in C\sp k(X)$  og við skilgreinum hlutafleiður af $f$ sem tvinnföllin
\begin{gather*}
\partial_xf=\partial_xu+i\partial_xv, \qquad
\partial_yf=\partial_yu+i\partial_yv,\\
\partial\sp 2_xf=\partial\sp 2_xu+i\partial\sp 2_xv, \qquad
\partial\sp 2_{xy}f=\partial\sp 2_{xy}u+i\partial\sp 2_{xy}v,\qquad
\partial\sp 2_yf=\partial\sp 2_yu+i\partial\sp 2_yv.
\end{gather*}
Þannig er síðan haldið áfram eftir því sem deildanleiki $u$ og $v$
endist.  Nú ætlum við að innleiða nýtt deildanleikahugtak, þar sem
við lítum á breytistærðina sem {\it tvinntölu :hover:`tvinntala`} en ekki sem vigur:

\subsection*{$\C$-deildanleg föll}

\subsubsection{Skilgreining}
Látum $f:X\to \C$ vera fall á opnu hlutmengi $X$ af $\C$.  
Við segjum
að $f$ sé {\it $\C$--deildanlegt :hover:`$\C$-deildanlegur`} í punktinum $a\in X$ ef markgildið
 \begin{equation*}\lim _{\substack{ h\to 0\\ h\in\C}}
 \dfrac{f(a+h)-f(a)}h  

.. _4.2.3:

 \end{equation*}
er til.  Markgildið táknum við með $f\dash(a)$ og köllum það
{\it $\C$--afleiðu :hover:`$\C$-afleiða`} fallsins $f$ í punktinum $a$.  
Fall $f:X\to \C$ er sagt vera {\it fágað :hover:`fágað fall`} á opna menginu $X$ ef $f\in
C^1(X)$ og $f$ er $\C$--deildanlegt í sérhverjum punkti í $X$.  Við
látum $\O(X)$ tákna mengi allra fágaðra falla á $X$.  Við segjum að
$f$ sé {\it fágað í punktinum $a$} ef til er opin grennd $U$ um $a$ þannig
að $f$ sé fágað í $U$.  Fallið $f$ er sagt vera {\it heilt fall} ef
það er fágað á  öllu $\C$.


--------------




Þessi skilgreining er eins og skilgreiningin af afleiðu falls af einni
raun\-breyti\-stærð.  



.. _se:sammfelldni:

\subsubsection{Setning}  Ef $f$ er $\C$--deildanlegt í $a$, þá er $f$ samfellt í $a$.


--------------





\subsection*{Reiknireglur fyrir $\C$-afleiður}

Reiknireglurnar fyrir $\C$-afleiður eru nánast þær sömu og
reiknireglurnar fyrir afleiður falla af einni raunbreytistærð.
Við tökum sannanirnar á þeim fyrir aftast í kaflanum:



.. _set4.2.3:

\subsubsection{Setning}
Látum $f,g:X\to \C$ vera föll, $a\in X$, $\alpha,\beta\in \C$ og
gerum ráð fyrir að $f$ og $g$ séu $\C$--deildanleg í $a$.  
Þá gildir

\smallskip\noindent
(i) $\alpha f+\beta g$ er $\C$--deildanlegt í $a$ og 
 $$
(\alpha f+\beta g)\dash(a)=\alpha f\dash(a)+\beta g\dash(a).
 $$

\smallskip\noindent
(ii) ({\it Leibniz-regla :hover:`regla Leibniz` :hover:`regla
Leibniz!fyrir tvinnföll` :hover:`Leibniz`}). $fg$ er $\C$--deildanlegt í $a$ og
$$
(fg)\dash(a)=f\dash(a)g(a)+f(a)g\dash(a).
$$

\smallskip\noindent
(iii) Ef $g(a)\neq 0$, þá er $f/g$ $\C$--deildanlegt í $a$ og 
$$(f/g)\dash(a)=\dfrac{f\dash(a)g(a)-f(a)g\dash(a)}{g(a)^2}.$$


--------------





\subsubsection{Fylgisetning}
$\O(X)$ er línulegt rúm yfir $\C$.


--------------



Ef $f_1,f_2,\dots, f_n$ eru
$\C$--deildanleg í  $a$ og $\alpha_1,\dots,\alpha_n\in \C$, þá fáum við
með þrepun að
$f=\alpha_1f_1+\cdots+\alpha_nf_n$ er $\C$--deildanlegt í $a$ og
 $$f\dash(a)=\alpha_1 f_1\dash(a)+\cdots+\alpha_nf_n\dash(a).
 $$
Eins fáum við með þrepun að margfeldið $f=f_1f_2\cdots f_n$ er
$\C$--deildanlegt í $a$ og
 $$f\dash(a)= \sum_{j=1}^n f_j\dash(a)\bigg(\prod_{\substack{ k=1\\ k\neq
 j}}^n f_k(a)\bigg).
 $$
Athugið að af þessu leiðir formúlan
 $$\dfrac{f\dash(a)}{f(a)} =  \dfrac{f_1\dash(a)}{f_1(a)}+\cdots+
\dfrac{f_n\dash(a)}{f_n(a)}.
 $$


\subsubsection{Sýnidæmi} (i) Allar margliður
 $$P(z)= a_0+a_1z+\cdots+a_mz^m, \qquad z\in \C,
 $$
eru fáguð föll á öllu $\C$ og afleiðan er
 $$P\dash(z)= a_1+2a_2z+\cdots+ma_mz^{m-1}, \qquad z\in \C.
 $$

(ii) Sérhvert rætt fall $R=P/Q$, þar sem $P$ og $Q$ eru margliður, er
fágað fall á menginu $\{z\in \C; Q(z)\neq 0\}$
og
 $$R\dash(z)= \dfrac{P\dash(z)Q(z)-P(z)Q\dash(z)}{Q(z)^2}.
 $$


--------------




Keðjureglan :hover:`keðjuregla fyrir fáguð föll` fyrir $\C$--deildanleg
föll er  eins og keðjureglan fyrir  raunföll:


.. _se:2.2.6:

\subsubsection{Setning}  Látum $X$ og $Y$ vera opin hlutmengi af $\C$, $f:X\to \C$ og
$g:Y\to \C$ vera föll, þannig að $f(X)\subset Y$, $a\in X$, $b\in Y$,
$b=f(a)$ og setjum
$$h=g\circ f.
$$ 
(i) Ef $f$ er $\C$--deildanlegt í $a$ og $g$ er $\C$--deildanlegt í
$b$, þá er $h$ $\C$--deildanlegt í $a$ og
 $$h\dash(a)=g\dash(b)f\dash(a).
 $$
(ii) Ef $g$ er $\C$--deildanlegt í $b$, $g\dash(b)\neq 0$, $h$ er
$\C$--deildanlegt í $a$ og $f$ er samfellt í $a$, 
þá er $f$ $\C$--deildanlegt í $a$ og 
$$ f\dash(a)=h\dash(a)/g\dash(b)$$.
 


--------------



Mikilvæg afleiðing af þessari setningu er:


.. _fs:2.2.7:

\subsubsection{Fylgisetning}  
Látum $X$ og $Y$ vera opin hlutmengi af $\C$, $f:X\to Y$ 
vera gagntækt fall.  Ef $f$ er $\C$--deildanlegt í $a$ og
$f\dash(a)\neq 0$, þá er andhverfa fallið 
$f^{[-1]}$ $\C$--deildanlegt í $b=f(a)$ og
 \begin{equation*}\left(f^{[-1]}\right)\dash(b)= \dfrac 1{f\dash(a)}.

.. _4.2.4:

 \end{equation*}


--------------





\subsection*{Cauchy-Riemann-jöfnur}

Nú skulum við gera ráð fyrir því að $f$ sé $\C$--deildanlegt í punktinum
$a$ og huga að sambandinu milli $f\dash(a)$, ${\partial}_xf(a)$ og
${\partial}_yf(a)$. 
Ef við skrifum $a=\alpha+i\beta=(\alpha, \beta)$ og látum $h\to
0$ eftir  rauntölunum, þá fáum við 
\begin{align*}
f\dash(a)=&\lim_{\substack{h\to 0\\ h\in \R}}
\dfrac{u(\alpha+h,\beta)-u(\alpha,\beta)}h+i
\dfrac{v(\alpha+h,\beta)-v(\alpha,\beta)}h\\
=&\partial_xu(a)+i\partial_xv(a)=\partial_xf(a).\nonumber
\end{align*}
Ef við látum hins vegar $h\to 0$ eftir þvertölum, $h=ik$,
$k\in \R$, þá fáum við
\begin{align*}
f\dash(a)&=\lim_{\substack{k\to 0\\ k\in \R}}
\dfrac{u(\alpha,\beta+k)-u(\alpha,\beta)}{ik}+i
\dfrac{v(\alpha,\beta+k)-v(\alpha,\beta)}{ik}\\
&=-i(\partial_yu(a)+i\partial_yv(a))=-i\partial_yf(a).\nonumber
\end{align*}
Við höfum því:  


.. _set4.2.8:

\subsubsection{Setning}  Látum $f=u+iv:X\to \C$ vera fall af $z=x+iy$ á opnu hlutmengi
$X$ í $\C$.  Ef $f$ er $\C$--deildanlegt í $a\in X$, þá eru báðar
hlutafleiðurnar $\partial_xf(a)$ og $\partial_yf(a)$ til og
 \begin{equation*}f\dash(a)=\partial_xf(a)=-i\partial_yf(a).


.. _4.2.7:

 \end{equation*}
Þar með gildir {\it
Cauchy--Riemann--jafnan :hover:`Cauchy--Riemann!jafna`
 :hover:`Cauchy--Riemann!jöfnur` :hover:`jafna!Cauchy--Riemann`}
\begin{equation*}
\tfrac 12\big(\partial_xf(a)+i\partial_yf(a)\big)=0,


.. _4.2.8:

\end{equation*}
og  hún jafngildir hneppinu
\begin{equation*}
\partial_xu(a)=\partial_yv(a), \qquad \partial_yu(a)=-\partial_xv(a),


.. _4.2.9:

\end{equation*}
sem venja er að kalla Cauchy--Riemann--jöfnur, í fleirtölu.


--------------




\subsection*{Wirtinger-afleiður}


Til þess að glöggva okkur betur á Cauchy--Riemann--jöfnunni, þá skulum
við rifja það upp að fall $f:X\to \R^2$ er sagt vera deildanlegt í
punktinum $a$, ef til er línuleg vörpun $L:\R^2\to \R^2$ þannig að

.. _4.2.10:

\begin{equation} 
\lim_{\substack{h\to 0\\ h\in \R^2}}
\dfrac{\| f(a+h)-f(a)-L(h)\|}{\|h\|}= 0,
\end{equation}
þar sem $\|z\|$ táknar lengd vigursins $z$.  Vörpunin $L$ er ótvírætt
ákvörðuð.  Hún nefnist afleiða $f$ í punktinum $a$ og er oftast táknuð
með $d_af$, $df_a$ eða $Df(a)$.    Með því að velja vigurinn $h$ af
gerðinni $t(1,0)$ og $t(0,1)$ og láta síðan $t\to 0$, þá sjáum við að hlutafleiðurnar 
${\partial}_xu(a)$, ${\partial}_yu(a)$, ${\partial}_xv(a)$ og 
 ${\partial}_yv(a)$ eru allar til og að fylki vörpunarinnar $d_af$ miðað
við grunninn $\{(1,0), (0,1)\}$ er
\begin{equation*}
\left[\begin{matrix} 
{\partial}_xu(a) & {\partial}_yu(a)\\
{\partial}_xv(a) & {\partial}_yv(a)
\end{matrix}\right].


.. _4.2.11:

\end{equation*} 
Þetta fylki nefnist {\it Jacobi--fylki :hover:`Jacobi-fylki`} $f$ í punktinum $a$.  Nú skrifum
við $z=(x,y)$, $a=({\alpha},{\beta})$ og sjáum að (:ref:`4.2.10`) jafngildir
því að hægt sé að rita
\begin{equation*}
f(z)=\left[\begin{matrix}
u(a) \\ v(a)
\end{matrix}\right]+
\left[\begin{matrix}
{\partial}_xu(a) \\ {\partial}_xv(a)
\end{matrix}\right](x-{\alpha})+
\left[\begin{matrix}
{\partial}_yu(a) \\ {\partial}_yv(a)

.. _4.2.12:

\end{matrix}\right](y-{\beta})+
\|z-a\|F_a(z), 
\end{equation*}
þar sem $F_a:X\to \R^2$ er samfellt í $a$ og $F_a(a)=0$.  Nú skulum við
líta á $f$ sem tvinngilt fall $f=u+iv$.  Þá er þessi jafna jafngild
\begin{equation}
f(z)=f(a)+ {\partial}_xf(a)(x-{\alpha})+{\partial}_yf(a)(y-{\beta})
+(z-a)\varphi_a(z),


.. _4.2.13:

\end{equation}
þar sem $\varphi_a:X\to \C$ er samfellt í $a$ og $\varphi_a(a)=0$.  Nú
skrifum við 
$$
x-{\alpha}=\big((z-a)+\overline{(z-a)}\big)/2, \qquad
y-{\beta}=\big((z-a)-\overline{(z-a)}\big)/2i
$$ 
og fáum því 
\begin{multline*}
{\partial}_xf(a)(x-{\alpha})+{\partial}_yf(a)(y-{\beta})  \\
=\tfrac 12\big({\partial}_xf(a)-i{\partial}_yf(a)\big)(z-a)
+\tfrac 12\big({\partial}_xf(a)+i{\partial}_yf(a)\big)\overline{(z-a)}.
\end{multline*}

\subsubsection{Skilgreining}  Við skilgreinum fyrsta stigs hlutafleiðuvirkjana
${\partial}_z={\partial}/{\partial}z$ og 
${\partial}_{\bar z}={\partial}/{\partial}\bar z$ með
\begin{equation*}
{\partial}_zf=\dfrac{{\partial}f}{{\partial} z}
=\tfrac 12\big({\partial}_xf-i{\partial}_yf\big) \quad \text{ og } \quad
{\partial}_{\bar z}f=\dfrac{{\partial}f}{{\partial}\bar z}
=\tfrac 12\big({\partial}_xf+i{\partial}_yf\big)


.. _4.2.14:

\end{equation*}
Tölurnar ${\partial}_zf(a)$ og ${\partial}_{\bar z}f(a)$ nefnast
{\it Wirtinger--afleiður :hover:`Wirtinger-afleiður`} fallsins $f$ í punktinum $a$ og virkinn
${\partial}_{\bar z}$ nefnist {\it
Cauchy--Riemann--virki :hover:`virki!Cauchy--Riemann` :hover:`Cauchy--Riemann!virki`}


--------------







Hugsum okkur nú að $f:X\to \C$ sé eitthvert fall og að til séu
tvinntölur $A$, $B$ og fall $\varphi_a:X\to \C$, samfellt í $a$ með
$\varphi_a(a)=0$, þannig að
\begin{equation}
f(z)=f(a)+A(z-a)+B\overline{(z-a)}+(z-a)\varphi_a(z).


.. _4.2.16:

\end{equation} 
Þá er greinilegt að $f$ er deildanlegt í $a$ með afleiðuna
$d_af(h)=Ah+B\bar h$ og
\begin{align*}
{\partial}_xf(a) &=
\lim_{\substack{ h\to 0\\ h\in \R}} \dfrac{f(a+h)-f(a)}h
=\lim_{\substack{ h\to 0\\ h\in \R}} A+B+\varphi_a(a+h) = A+B,\\
{\partial}_yf(a) &=
\lim_{\substack{ h\to 0\\ h\in \R}} \dfrac{f(a+ih)-f(a)}h
=\lim_{\substack{ h\to 0\\ h\in \R}} iA-iB+\varphi_a(a+ih) = i(A-B).
\end{align*}
Ef við leysum $A$ og $B$ út úr þessum jöfnum, þá fáum við
\begin{align*}
A&= \tfrac 12\big({\partial}_xf(a)-i{\partial}_yf(a)\big)
={\partial}_zf(a),\\
B&= \tfrac 12\big({\partial}_xf(a)+i{\partial}_yf(a)\big)
={\partial}_{\bar z}f(a).
\end{align*}
Við höfum nú sannað:



\subsubsection{Setning}  Látum $X\subset \C$ vera opið, $a\in X$ og $f:X\to \C$ vera
fall.  Þá gildir:

\smallskip\noindent
(i) $f$ er deildanlegt í $a$ þá og því aðeins að til séu tvinntölur $A$,
$B$ og fall $\varphi_a:X\to \C$, samfellt í $a$ og með
$\varphi(a)=0$, þannig að 
\begin{equation*}
f(z)=f(a)+A(z-a)+B\overline{(z-a)}+(z-a)\varphi_a(z).
\end{equation*} 

\smallskip\noindent
(ii) $f$ er $\C$--deildanlegt í $a$ þá og því aðeins að $f$ sé
deildanlegt í $a$ og ${\partial}_{\bar z}f(a)=0$.  Þá er
$f\dash(a)={\partial}_zf(a)$.

\smallskip\noindent
(iii) $f$ er fágað í $X$ þá og því aðeins að $f$ sé samfellt deildanlegt
í $X$ og uppfylli Cauchy--Riemann--jöfnuna ${\partial}_{\bar z}f=0$ í
$X$. Við höfum þá
\begin{equation*}
f\dash=\dfrac{df}{dz}=\dfrac{\partial f}{\partial z}=\dfrac 12\bigg(
\dfrac{\partial f}{\partial x}-i\dfrac{\partial f}{\partial y}\bigg).


.. _4.2.17:

\end{equation*}


--------------





Reikningur með hlutafleiðunum með tilliti til $z$ og $\bar z$ er alveg
eins of reikningur með óháðu breytunum $x$ og $y$.  


Ef fallið 
$f(z)=f(x+iy)$ er gefið með formúlu í $x$ og $y$, þá notum við
formúlurnar $x=(z+\bar z)/2$ og $y=(z-\bar z)/(2i)$ til þess að skipta
á óháðu breytunum $x$ og $y$ yfir í breyturnar $z$ og $\bar z$.  Til
þess að kanna hvort fall er fágað þá deildum við eins og þetta séu
óháðar breytur og könnum hvort 
$$
\dfrac{\partial f}{\partial\bar z}=0.
$$
Ef $\bar z$ kemur alls ekki fyrir í formúlunni, þá er $f$ fágað.

 

\section{Samleitnar veldaraðir}

\subsection{Samleitnar veldaraðir} 

\noindent
Einu dæmin um fáguð föll sem við höfum nefnt til þessa eru margliður
$P$, en þær eru fágaðar á öllu $\C$, og ræð föll $R=P/Q$, en þau eru
fáguð á $\C\setminus\{z\in \C; Q(z)=0\}$.  Nú ætlum við að bæta
verulega við dæmaforðann með því að sanna að öll föll, sem unnt er að
setja fram með samleitnum veldaröðum, séu fáguð á samleitniskífu
raðarinnar. 


Ef fallið $f$ er skilgreint á einhverju opnu mengi $Y$ á $\R$ og er
gefið með samleitinni veldaröð á $]a-{\varrho},a+{\varrho}[\subset Y$,
$$
f(x)=\sum\limits_{n=0}^{\infty} a_n(x-a)^n, \qquad 
x\in  ]a-{\varrho},a+{\varrho}[,
$$
þá er röðin samleitin á opnu skífunni $S(a,{\varrho})\subseteq \C$ og við getum 
framlengt skilgreiningarsvæði $f$ yfir á $S(a,{\varrho})$ með því að setja
$$
f(z)=\sum\limits_{n=0}^{\infty} a_n(z-a)^n, \qquad 
z\in  S(a,{\varrho}).
$$

\subsubsection{Skilgreining}  Fall sem skilgreint er á opnu mengi $U$ á rauntalnaásnum
er sagt vera {\it raunfágað } ef það  hefur þann eiginleika að  
í grennd um sérhvern punkt í $U$ er hægt að setja $f$ fram með
samleitinni  veldaröð.


--------------



Fallið $z\mapsto 1/(1-z)$ er fágað á $\C\setminus\{1\}$ og það 
gefið með geómetrísku röðinni
$$
\dfrac 1{1-z}=\sum_{n=0}^\infty z^n, \qquad z\in S(0,1). 
$$
Veldisvísisfallið, hornaföllin og breiðbogaföllin eru öll gefin með
samleitnum veldaröðum á $\R$ og fáguðu framlengingar þeirra eru því 
gefnar með sömu röðum á öllu $\C$
\begin{gather*}
\exp z =e\sp z = \sum_{n=0}^\infty \dfrac 1{n!}z^n, \\
\cos z = \sum_{k=0}\sp \infty \dfrac {(-1)\sp k}{(2k)!}z\sp{2k}, \quad
\sin z = \sum_{k=0}\sp \infty \dfrac {(-1)\sp k}{(2k+1)!}z\sp{2k+1},
\quad\\
\cosh z = \sum_{k=0}\sp \infty \dfrac {1}{(2k)!}z\sp{2k}, \quad
\sinh z = \sum_{k=0}\sp \infty \dfrac {1}{(2k+1)!}z\sp{2k+1}.
\end{gather*}



.. _set4.3.1:

\subsubsection{Setning} Gerum ráð fyrir að $X$ sé opið hlutmengi af $\C$, 
$S(\alpha,\varrho)\subset X$, að $f:X\to \C$
sé fall, sem  gefið er á $S(\alpha,\varrho)$ með samleitinni veldaröð,
 $$f(z)=\sum_{n=0}^\infty a_n(z-\alpha)^n, \qquad z\in S(\alpha,\varrho).
 $$
Þá er $f$ fágað á $S(\alpha,\varrho)$ og
 $$f\dash(z)=\sum_{n=1}^\infty na_n(z-\alpha)^{n-1}, \qquad z\in
S(\alpha,\varrho). 
 $$


--------------



Ef $\sum_{n=0}^\infty a_nz^n$ og
$\sum_{n=0}^\infty b_nz^n$ eru tvær samleitnar veldaraðir með
samleitnigeisla $\varrho_a$ og $\varrho_b$, þá höfum við fáguð föll
$f$ og $g$ í $S(\alpha,\varrho_a)$ og $S(\alpha,\varrho_b)$ sem gefin
eru með 
$$
f(z)=\sum_{n=0}^\infty a_n(z-\alpha)^n, \qquad \text{ og } \qquad
g(z)=\sum_{n=0}^\infty b_n(z-\alpha)^n.
$$
Ef við setjum $\varrho=\min\{\varrho_a,\varrho_b\}$, þá eru fáguðu
föllin $f+g$ og $fg$ einnig gefin veldaröðum á skífunni
$S(\alpha,\varrho)$ með
$$
f(z)+g(z)=\sum_{n=0}^\infty (a_n+b_n)(z-\alpha)^n 
\qquad \text{ og } \qquad f(z)g(z)=\sum_{n=0}^\infty c_n(z-\alpha)^n,
$$
þar sem stuðlarnir $c_n$ eru gefnir með
$$
c_n=\sum_{k=0}^n a_kb_{n-k}, \qquad n=0,1,2,\dots. 
$$

Eftirfarandi setning nefnist {\it
samsemdarsetning :hover:`samsemdarsetning` fyrir samleitnar
veldaraðir :hover:`samsemdarsetning!fyrir samleitnar veldaraðir`}: 

\subsubsection{Setning}
Gerum ráð fyrir að $f,g\in \O(S(\alpha,\varrho))$ séu gefin með
samleitnum veldaröðum
 $$f(z)=\sum\limits_{n=0}^\infty a_n(z-\alpha)^n, \qquad
g(z)=\sum\limits_{n=0}^\infty b_n(z-\alpha)^n, \qquad
z\in S(\alpha,\varrho),
 $$
og gerum ráð fyrir að til sé runa $\{\alpha_j\}$ af ólíkum punktum
í $S(\alpha,\varrho)$ þannig að $\alpha_j\to \alpha$ og
$f(\alpha_j)=g(\alpha_j)$ fyrir öll $j$.  Þá er $a_n=b_n$ fyrir öll
$n$ og þar með $f(z)=g(z)$ fyrir öll $z\in S(\alpha,\varrho)$.


--------------

 
 
\subsubsection{Fylgisetning}  Ef \ $\sum_{n=0}^{\infty} a_nx^n$ er samleitin veldaröð,
$I$ er opið bil sem inniheldur $0$ og   $\sum_{n=0}^{\infty}
a_nx^n=0$  fyrir öll $x\in I$, þá er $a_n=0$ fyrir öll $n=0,1,2,\dots$.


--------------




Við sáum hér að framan að 
sérhvert fall sem gefið er með veldaraðaframsetningu á
einhverri skífu er fágað.  Nú hugum við að andhverfu þessarar
staðhæfingar:

\subsubsection{Setning}
Látum $X\subset \C$ vera opið og $f\in \O(X)$.  Ef $\alpha\in X$,
$0<\varrho<d(\alpha,\partial X)$, þar sem  $d(\alpha,\partial X)$
táknar fjarlægð punktsins $\alpha$ frá jaðrinum $\partial X$ á
menginu $X$, þá er hægt að setja $f$ fram í $S(\alpha,\varrho)$ með
samleitinni veldaröð  
 $$f(z) = \sum\limits_{n=0}^\infty a_n(z-\alpha)^n, \qquad z\in
S(\alpha,\varrho). 
 $$
 

--------------



.. figure:: ./myndir/fig031.svg

    :align: center

    :alt: Skífa í skilgreiningarsvæði $f$

    2BeRemovedMynd: Skífa í skilgreiningarsvæði $f$



Þessa setningu sönnum við ekki fyrr en í kafla 3, en við skulum skoða
nokkrar afleiðingar hennar.  

\subsubsection{Fylgisetning}
Ef $f\in \O(X)$, þá er $f\dash\in \O(X)$.


--------------



\bigskip
Nú sjáum við að fallið $f\dash$ er fágað og afleiða þess $f\ddash$ er
einnig fáguð og þannig áfram út í hið óendanlega.  Fyrir sérhvert
fágað fall $f\in \O(X)$ skilgreinum við hærri afleiður $f^{(k)}$ með
þrepun $f^{(0)}=f$ og $f^{(k)}=\big(f^{(k-1)}\big)\dash$, fyrir
$k\geq 1$. Við fáum síðan:


.. _se:2.3.7:

\subsubsection{Setning}  
Látum $X$ vera opið hlutmengi af $\C$, $f\in \O(X)$, $\alpha\in
X$ og $0<\varrho<d(\alpha,\partial X)$.  Þá er 
 $$f(z)= \sum\limits_{n=0}^\infty \dfrac
{f^{(n)}(\alpha)}{n!}(z-\alpha)^n, \qquad z\in S(\alpha,\varrho).
 $$
Þessi veldaröð kallast {\it
Taylor--röð :hover:`Taylor-röð` :hover:`Taylor-röð!falls í punkti`
fallsins $f$ í punktinum $\alpha$}.


--------------



\subsubsection{Skilgreining}
Látum $f:Y\to \C$ vera raunfágað fall á opnu mengi $Y$ á $\R$
og gerum ráð fyrir að $F:X\to \C$ sé fágað fall á opnu hlutmengi $X$ af
$\C$, þannig að $Y\subset X$ og $F(x)=f(x)$ fyrir öll $x\in Y$.  Þá
kallast $F$ {\it fáguð framlenging :hover:`fáguð framlenging`} eða {\it
fáguð útvíkkun :hover:`fáguð útvíkkun`}
á fallinu $f$.  


--------------


 
 
\section{Veldaröð veldisvísisfallsins}



\noindent
Við skilgreindum  veldisvísisfallið  með formúlunni
$$
\exp z=e^x(\cos y+i\sin y), \qquad z=x+iy \in \C.
$$
Við hefðum eins getað notað veldaraðarframsetninguna á $x\mapsto e^x$
til þess að skilgreina fágaða framlengingu veldisvísisfallsins.

\smallskip
Við skulum nú kanna nokkra eiginleika veldisvísisfallsins út frá
veldaröðinni.   
Með því að
deilda röðina lið fyrir lið fáum við 
 $$\exp\dash z=\exp z, \qquad \text{eða} \qquad \dfrac d{dz}e^z=e^z.
 $$
Undirstöðueiginleiki veldisvísisfallsins er {\it
samlagningarformúla :hover:`samlagningarformúla!veldisvísisfallsins`
 :hover:`veldisvísisfallið!samlagningarformúla`} þess
$$ e^{z+w}=e^ze^w, \qquad z,w\in \C. $$
Hún leiðir af tvíliðureglunni  :hover:`tvíliðuregla`,
\begin{align*}
e^{z+w}&=\sum_{n=0}^\infty\dfrac 1{n!}(z+w)^n\\
&=\sum_{n=0}^\infty\dfrac 1{n!}\sum_{k=0}^n \dfrac{n!}{k!(n-k)!}z^kw^{n-k}\\
&=\sum_{n=0}^\infty\sum_{k=0}^n \dfrac {z^k}{k!}\dfrac {w^{n-k}}{(n-k)!}\\
&=\bigg(\sum_{n=0}^\infty \dfrac {z^n}{n!}\bigg)\bigg(\sum_{n=0}^\infty\dfrac
{w^{n}}{n!}\bigg)=e^ze^w. 
\end{align*}
Flestir eiginleikar veldisvísisfallsins er leiddir út frá
samlagningarformúlunni.  Til dæmis sjáum við að 
 \begin{equation*}e^{-z}=\dfrac 1{e^z}, \qquad z\in \C.

.. _4.5.1:

 \end{equation*}
Á rauntalnaásnum er veldisvísisfallið $x\mapsto e^x$
stranglega vaxandi því afleiða þess er $e^x$ og hún er jákvæð.
Við höfum líka $e^x\to+\infty$ ef
$x\to \infty$, því sérhver liður í veldaröðinni með
númer $n\geq 1$ er stranglega vaxandi og stefnir á óendanlegt. Af
þessu leiðir síðan að $e^{x}=1/e^{-x}\to 0$ ef $x\to -\infty$.
Milligildissetningin segir okkur nú að veldisvísisfallið tekur öll
jákvæð gildi á rauntalnaásnum.  



Snúum okkur þá að gildunum á  þverásnum $\{ix\in \C;
 x\in \R\}$. Reglurnar um reikning með samoka tvinntölum gefa
okkur
$$\overline{e^z}=e^{\overline z},\qquad z\in \C,
$$
og síðan
 $$|e^z|^2=e^z\overline{e^{z}}=e^ze^{\overline z}=e^{x+iy}e^{x-iy}=e^{2x}
 $$
Þar með er
 $$|e^z|=e^{\Re z}, \qquad z\in \C,
 $$
og sérstaklega gildir 
$$
|e^{iy}|=1, \qquad y\in \R.
$$
Af þessu leiðir  að veldisvísisfallið hefur enga
núllstöð :hover:`veldisvísisfallið!núllstöð`
$e^z=e^xe^{iy}$ og  hvorugur þátturinn í hægri hliðinni getur verið
núll.  

Með því að stinga $iz$ inn í veldaröðina fyrir veldis\-vísis\-fallið
sjáum við að formúlan $e^{ix}=\cos x+i\sin x$ gildir áfram um 
tvinntölur $z\in\C$,
 $$
e^{iz}=\sum\limits_{n=0}^\infty\dfrac{i^n}{n!}z^n
=\sum\limits_{n=0}^\infty\dfrac{(-1)^n}{(2n)!}z^{2n}
+i\sum\limits_{n=0}^\infty\dfrac{(-1)^n}{(2n+1)!}z^{2n+1}
=\cos z +i \sin z.
 $$
Allir liðirnir í kósínus--röðinni hafa jöfn veldi og allir liðirnir í
sínus--röðinni hafa oddatöluveldi, svo $\cos$ er jafnstætt, en
$\sin$ er oddstætt.  Þar með er
 $$e^{-iz}=\cos z-i\sin z, \qquad z\in \C.
 $$
Við leysum nú $\cos z$ og $\sin z$ út úr síðustu tveimur
jöfnunum og fáum {\it jöfnur
Eulers :hover:`Euler` :hover:`Euler!jöfnur` :hover:`jöfnur Eulers`}
 $$\cos z =\frac 12(e^{iz}+e^{-iz}), \qquad
\sin z =\frac 1{2i}(e^{iz}-e^{-iz}).
 $$
Afleiðurnar af $\cos$ og $\sin$ getum við annað hvort reiknað með því
að deilda veldaraðirnar eða með því að deilda jöfnur Eulers,
 $$\cos\dash z=-\sin z, \qquad \sin\dash z=\cos z, \qquad z\in \C.
 $$


\section{Lograr, rætur :hover:`rót` og horn :hover:`horn`}

\subsection{Lograr, rætur :hover:`rót` og horn :hover:`horn`}


\noindent
Veldisvísisfallið $e^z$ er lotubundið með lotuna $2\pi i$,
$$
\exp(z+2{\pi}i) = \exp z, \qquad z\in \C.
$$
Þetta leiðir beint af þeirri staðreynd að kósínus og sínus eru
lotubundin með lotuna $2{\pi}$.
Þar með getur $\exp$ ekki haft neina andhverfu á öllu menginu $\C$.  
Veldisföllin 
$z^n$, $n\geq 2$ geta ekki heldur haft neina andhverfu á öllu $\C$.
Hins vegar hafa þessi föll andhverfur {\it frá hægri } 
á minni hlutmengjum í $\C$:

\subsubsection{Skilgreining}
Látum $X$ vera opið hlutmengi af $\C$.  Samfellt fall $\lambda:X\to
\C$ kallast  {\it logri á $X$ :hover:`logri`} ef
 \begin{equation*}e^{\lambda(z)}=z, \qquad z\in X.


.. _4.6.1:

 \end{equation*}
Samfellt fall $\varrho:X\to \C$ kallast {\it $n$--ta
rót :hover:`$n$--ta rót`} á $X$ ef
 \begin{equation*}\big(\varrho(z)\big)^n=z, \qquad z\in X.


.. _4.6.2:

 \end{equation*}
Samfellt fall $\theta:X\to \R$ kallast {\it horn á $X$} ef 
 \begin{equation*}z=|z|e^{i\theta(z)}, \qquad z\in X.


.. _4.6.3:

 \end{equation*}


--------------




Helstu eiginleikar logra, róta :hover:`rót` og horna :hover:`horn` eru:

\subsubsection{Setning} (i) Ef $\lambda$ er logri á $X$, þá er $0\not\in X$, $\lambda\in \O(X)$ og
 $$\lambda\dash(z)=\frac 1z, \qquad z\in X.
 $$
Föllin $\lambda(z)+i2\pi k$, $k\in \Z$ eru einnig lograr á $X$.

\smallskip\noindent 
(ii) Ef $\lambda$ er logri á $X$, þá er  $$\lambda(z)=\ln
|z|+i\theta(z), \qquad z\in X,
 $$
þar sem  $\theta:X\to \R$ er horn á $X$.  Öfugt, ef 
$\theta:X\to \R$ er horn á $X$, þá er $\lambda(z)=\ln|z|+i\theta(z)$
logri á $X$.

\smallskip\noindent
(iii)  Ef $\varrho$ er $n$--ta rót á $X$ þá er $\varrho\in \O(X)$ og
 $$\varrho\dash(z)=\frac {\varrho(z)}{nz}, \qquad z\in X.
 $$
(iv) Ef $\lambda$ er logri á $X$, þá er
$\varrho(z)=e\sp{\lambda(z)/n}$ $n$--ta rót á $X$.


--------------





Fyrir sérhverja tvinntölu ${\alpha}$ 
skilgreinum við fágað {\it veldisfall með veldisvísi}
$\alpha$ með 
 $$z^\alpha=\exp(\alpha\lambda(z)), \qquad z\in X,
 $$
þar sem  $\lambda$ er gefinn logri á $X$ og við fáum að
 $$\dfrac d{dz}z^\alpha=\dfrac d{dz}e^{\alpha\lambda(z)}=e^{\lambda(z)}\frac
\alpha z =\alpha e^{\alpha\lambda(z)}e^{-\lambda(z)}=
\alpha e^{(\alpha-1)\lambda(z)}=\alpha z^{\alpha-1}.
 $$
Þetta er sem sagt gamalkunn regla, sem gildir áfram fyrir
$\C$--afleiður.  Hér verðum við að hafa í huga að skilgreiningin  er
algerlega háð því hvernig logrinn er skilgreindur.  Ef við skiptum
til dæmis á logranum $\lambda(z)$ og $\lambda(z)+2\pi i$, þá verður 
 $$e^{\alpha(\lambda(z)+2\pi i)}=e^{\alpha\lambda(z)}e^{2\pi i\alpha}.
 $$
Ef $\alpha$ er heiltala þá er $z^\alpha$ samkvæmt þessari
skilgreininingu það sama og fæst út úr velda\-reglunum með
heiltöluveldi, en ef $\alpha$ er ekki heiltala, þá  er
skilgreiningin háð valinu á logranum.


Ef $\alpha \in X$, þá  skilgreinum við 
{\it veldisvísisfall með grunntölu $\alpha$} sem 
fágaða fallið á $\C$, sem gefið er með
$$
\alpha^z=e^{z\lambda(\alpha)}.
$$  
Athugið að skilgreiningin er háð valinu á logranum.
Keðjureglan gefur 
$$\dfrac d{dz}\alpha^z=
\dfrac d{dz}e^{z\lambda(\alpha)}=e^{z\lambda(\alpha)}\cdot
\lambda(\alpha)=\alpha^z\lambda(\alpha).
$$  



%%%%%%%%%%%%%%%%%%%
MYND VANTAR HÉR!!!(setja texta undir mynd (finna í .tex) og fjarlægja ónotaðan kóða)
.. figure:: ./myndir/fig032.svg

    :align: center

\setbox0\vbox{{
\input fig032
}}
\setbox2\vbox{\hsize \wd0 { \strut \noindent Mynd: Höfuðgrein hornsins \strut}}
\setbox1\vbox{\hbox{\box0}\hbox{\box2}}
\dimen0 = -\ht1
\advance\dimen0 by-\dp1
\dimen1 = \wd1
\dimen2 = -\dimen0
\divide\dimen2 by\baselineskip
\count100 = 1
\advance\count100 by\dimen2
\advance\count100 by1
\hfill\box1
\vskip\dimen0
\dimen0 = \hsize
\advance\dimen0 by-\dimen1
\parshape 2 0pt \dimen0 0pt \dimen0
%%%%%%%%%%%%%%%%%%%
\noindent 
Lítum nú á mengið  $X=\C\setminus \R_-$, sem fæst með því að skera
neikvæða raunásinn og $0$ út úr
tvinntalnaplaninu.  Við skilgreinum síðan pólhnit í $X$ eins og
myndin sýnir og veljum hornið $\theta(z)$ þannig að
 $-\pi<\theta(z)<\pi$, $z\in X$.  Fallið 


 $$\Arg :\C\setminus \R_-\to \R, \qquad
\Arg z=\theta(z),\quad z\in X
 $$
\parshape 0  % Thetta er her vegna myndar 4.2
er kallað {\it höfuðgrein
hornsins :hover:`logri!höfuðgrein` :hover:`höfuðgrein!horns`} og 
við reiknuðum út formúlu fyrir því í kafla 1,
$$
\Arg\, z=2\arctan\bigg(\dfrac y{|z|+x}\bigg), \qquad z=x+iy\in X. 
$$
Fallið
 $$\Log :\C\setminus \R_-\to \C, \qquad
\Log z=\ln |z| +i\Arg(z),\quad z\in X,
 $$
er kallað {\it höfuðgrein
lografallsins :hover:`höfuðgrein` :hover:`höfuðgrein!lografallsins`}.  Fallið 
 $$z^\alpha = e^{\alpha\Log z}, \qquad z\in \C\setminus \R_-,
 $$
kallast {\it höfuðgrein
veldisfallsins :hover:`veldisfall` :hover:`veldisfall!höfuðgrein` með
veldisvísi $\alpha$ :hover:`höfuðgrein!veldisfallsins með veldisvísi
$\alpha$`}. Tvö síðastnefndu föllin eru fágaðar framlengingar á föllunum
$\ln x$  og $x^\alpha$ frá jákvæða raunásnum  yfir
í opna mengið $\C\setminus \R_-$ í tvinntalnaplaninu.

