%

\chapter {LAPLACE--UMMYNDUN}
 

\section{ Skilgreiningar og reiknireglur}  

\noindent
Látum $f$ vera fall sem skilgreint er á 
$\R_+=\set{t\in \R; t\geq 0}$ með gildi í $\C$ og gerum ráð fyrir að
$f$ sé heildanlegt á sérhverju lokuðu og takmörkuðu bili $[0,b]$.


{\it Laplace--mynd} $f$, 
sem við táknum með $\L f$ eða $\L\set{f}$, er skilgreind með
formúlunni
 \begin{equation*}\L f(s)=\int_0\sp \infty e\sp{-st}f(t)\, dt.\label{7.1.1}
 \end{equation*}
Skilgreiningarmengi fallsins $\L f$ samanstendur af öllum
tvinntölum $s$ þannig að heildið í hægri hliðinni sé samleitið.  


{\it Laplace-ummyndun} er vörpunin $\L$ sem úthlutar falli $f$ 
Laplace-mynd sinni $\L f$.


\begin{sk}\label{skil10.1.1a}
Við segjum að fallið $f:\R_+\to \C$ sé af
veldisvísisgerð\index{veldisvísisgerð} ef til eru
jákvæðir fastar $M$ og $c$ þannig að 
 \begin{equation*}|f(t)|\leq Me\sp{c t}, \qquad t\in \R_+.\label{7.1.2}
 \end{equation*}
\end{sk}

Ef $f$ er heildanlegt á sérhverju takmörkuðu bili $[0,b]$ og uppfyllir 
ójöfnuna, þá er $\L f$ skilgreint fyrir öll $s\in \C$ með $\Re s
>c$.   Við fáum að auki
vaxtartakmarkanir á $\L f$,
 \begin{equation*}
|\L f(s) |\leq \int_0\sp\infty e\sp{-\Re st} Me\sp{c t} \, dt =
\dfrac M{\Re\,  s-c}, \qquad \Re\,  s>c.
\label{7.1.3}
 \end{equation*}


Það er augljóst að Laplace-ummyndun er línuleg vörpun, en það þýðir að
$$
\L\{\alpha f+\beta g\}(s)=\alpha\L\{f\}(s)+\beta\L\{g\}(s)
$$
ef $f$ og $g$ eru föll af veldisvísisgerð, $\alpha$ og $\beta$ eru
tvinntölur og $s\in \C$ liggur í skilgreiningarmengi fallanna
$\L\{f\}$ og $\L\{g\}$.

\smallskip
Við þurfum að leiða út nokkrar reiknireglur fyrir Laplace-ummyndun.
Sú fyrsta segir okkur að Laplace-myndir falla af veldisvísisgerð séu
fáguð föll og hún segir okkur einnig að afleiður af Laplace-myndum 
af slíkum föllum  séu  einnig Laplace myndir: 


\begin{se} \label{se10.1.2a} Látum $f:\R_+\to \C$ vera fall sem er heildanlegt á
sérhverju bili $[0,b]$ og   uppfyllir (\ref{7.1.2}).  Þá er $\L f$ 
fágað á menginu $\set{s\in \C;\Re s>c}$ og 
 \begin{equation*}\dfrac{d^k}{ds^k}\L\set{f}(s)=
(-1)^k\L\set{t^kf(t)}(s), \qquad \Re s>c.
\label{7.1.4}
 \end{equation*}
\end{se}

\subsection*{Nokkur mikilvæg dæmi}


Reiknum nú út nokkrar Laplace-myndir:


Ef $\alpha\in \R$ og  $\alpha>-1$, þá er
\begin{align*}
\L\set{t^\alpha}(s)
&=\int_0^\infty e^{-st}t^\alpha \, dt =
\dfrac 1{s^{\alpha+1}} \int_0^\infty e^{-st}(st)^\alpha \, s dt \\
&=
\dfrac 1{s^{\alpha+1}} \int_0^\infty e^{-\tau}\tau^\alpha \,  d\tau =
\dfrac {\Gamma(\alpha+1)}{s^{\alpha+1}}.
\end{align*}
Ef ${\alpha}$ er heiltala, þá verður þessi formúla
$$
\L\set{t^\alpha}(s)
=\dfrac {\alpha!}{s^{\alpha+1}}.
 $$

\smallskip
Fyrir sérhvert $\alpha\in \C$ gildir
 $$
\L\set{e^{\alpha t}}(s)=
\int_0^{\infty}e^{-st}e^{\alpha t}\, dt =
\int_0^{\infty}e^{-(s-\alpha)t}\, dt =
\left[\dfrac {-e^{-(s-\alpha)t}} {s-\alpha}\right]_0^{\infty}=
\dfrac 1{s-\alpha},
 $$
og í framhaldi af þessu fáum við
\begin{align*}
\L\set{\cos\beta t}(s) &=
\frac 12 \L\set{e^{i\beta t}}(s) +\frac 12\L\set{e^{-i\beta t}}(s)\\
&=\frac 12\left[\dfrac 1{s-i\beta}+\dfrac 1{s+i\beta}\right]
=\dfrac s{s^2+\beta^2},\\
\L\set{\sin\beta t}(s) &=
\frac 1{2i}\L\set{e^{i\beta t}}(s) -\frac 1{2i}\L\set{e^{-i\beta t}}(s)\\
&=\frac 1{2i}\left[\dfrac 1{s-i\beta}-\dfrac 1{s+i\beta}\right]
=\dfrac {\beta}{s^2+\beta^2},\\
\L\set{\cosh \beta t}(s) &= 
\frac 12 \L\set{e^{\beta t}}(s) +\frac 12\L\set{e^{-\beta t}}(s)\\
&=\frac 12\left[\dfrac 1{s-\beta}+\dfrac 1{s+\beta}\right]
=\dfrac s{s^2-\beta^2},\\
\L\set{\sinh \beta t}(s) &= 
\frac 1{2}\L\set{e^{\beta t}}(s) -\frac 1{2}\L\set{e^{-i\beta t}}(s)\\
&=\frac 1{2}\left[\dfrac 1{s-\beta}-\dfrac 1{s+\beta}\right]
=\dfrac \beta{s^2-\beta^2}.
\end{align*}


Við höfum almenna reiknireglu:

\begin{se} $\L\set{e^{\alpha t}f}(s) = \L\set{f}(s-\alpha)$.
\end{se}

Útreikninga okkar hér að framan getum við  nú  tekið saman í litla
töflu: 
\begin{align*}
\L\set{e^{\alpha t}t^{\beta}}(s)
&=\dfrac{\Gamma(\beta+1)}{(s-\alpha)^{\beta+1}},\\
\L\set{e^{\alpha t}\cos \beta t}(s)
&=\dfrac{s-\alpha}{(s-\alpha)^2+\beta^2},\\
\L\set{e^{\alpha t}\sin \beta t}(s)
&=\dfrac{\beta}{(s-\alpha)^2+\beta^2},\\
\L\set{e^{\alpha t}\cosh \beta t}(s)
&=\dfrac{s-\alpha}{(s-\alpha)^2-\beta^2},\\
\L\set{e^{\alpha t}\sinh \beta t}(s)
&=\dfrac{\beta}{(s-\alpha)^2-\beta^2}.
\end{align*}


\subsection*{Laplace-ummyndun er eintæk vörpun}


\begin{se}\label{set:10.1.2}
Gerum ráð fyrir að föllin $f,g\in C(\R_+)$ séu bæði af
veldisvísisgerð og að til sé fasti $c$ þannig að 
 $$\L f(s)=\L g(s), \qquad s\in \C, \quad \Re s\geq c.
 $$
Þá er $f(t)=g(t)$ fyrir öll $t\in \R_+$. 
\end{se}


Þessa setningu má einnig orða þannig að Laplace-ummyndun er eintæk
vörpun á mengi allra samfelldra falla af veldisvísisgerð.  
Ef við sjáum að eitthvert fall $F(s)$ er Laplace-mynd af samfelldu 
falli $f$, 
þá segir setningin okkur að $f$ er ótvírætt ákvarðað og við köllum þá
$f$ {\it andhverfa Laplace-mynd } af fallinu $F$ og skrifum
$f(t)=\L^{-1}\{F\}(t)$.


\subsection*{Heaviside-fallið}

Fallið $H:\R\to \R$, sem skilgreint er með
\begin{equation*}
H(t)=\begin{cases} 1, &t\geq 0,\\ 0, & t<0,\end{cases}\label{7.1.5}
\end{equation*}
kallast {\it Heaviside--fall\index{Heaviside-fall}}.  Athugum að
hliðrun þess $H_a(t)=H(t-a)$ uppfyllir
\begin{equation*}
H_a(t)=\begin{cases} 1, &t\geq a,\\ 0, & t<a,\end{cases}\label{7.1.6}
\end{equation*}
og því er Laplace-mynd þess
\begin{equation*}
\L H_a(s)= \int_a^{\infty} e^{-st}\, dt= \dfrac{e^{-as}} s, \qquad a>0.
\label{7.1.7}
\end{equation*}
Við fáum reyndar almenna reiknireglu:

\begin{se}  Látum $f:\R_+\to \C$ vera fall af veldisvísisgerð.  Þá gildir um
sérhvert $a\geq 0$ að
$$
\L\set{H(t-a)f(t-a)}(s) = e^{-as}\L\set{f}(s).
$$
þar sem fallið $t\mapsto H(t-a)f(t-a)$ tekur gildið $0$ fyrir öll
$t<a$. 
\end{se}

\subsection*{Laplace-ummyndun af vigur- og fylkjagildum vörpunum}

Ef $u=(u_1,\dots,u_m): \R_+\to \C^m$ er vigurgilt fall á jákvæða
raunásnum, þá skilgreinum við Laplace-mynd $u$ með því að taka
Laplace-mynd af hnitaföllunum, 
$$\L u(s)=(\L u_1,\dots,\L u_m).
$$ 
Við
förum eins að við að skilgreina Laplace-mynd af $p\times m$-fylkjagildu
falli $U=(u_{jk})_{j,k=1}^{p,m}$, þar sem við skilgreinum $\L U(s)$ sem
$p\times m$ fylkjagilda fallið  
$$\L U(s)=(\L u_{jk}(s))_{j,k=1}^{p,m}.
$$
Ef $A$ er $p\times m$ fylki, þá er 
\begin{equation*}
\L\set{Au}(s)=A\L u(s).
\label{7.1.8}
\end{equation*}
Þessa reglu sönnum við  með því að líta á $v=Au$,
$v_j=a_{j1}u_1+\cdots+a_{jm}u_m$ og notfæra okkur að Laplace-ummyndunin
er línuleg vörpun.  Það gefur okkur $\L v_j(s)=a_{j1}\L u_1(s)+\cdots+a_{jm}\L
u_m(s)$.  Vinstri hliðin í þessari jöfnu er þáttur númer $j$ í vinstri
hlið jöfnunnar, en hægri hliðin er þáttur númer $j$ í hægri hlið hennar.


Ef hins $A$ er eitthvert $q\times p$ fylki, þá fæst reglan
\begin{equation*}
\L\set{AU}(s)=A\L U(s).
\label{7.1.9}
\end{equation*}



\section{Upphafsgildisverkefni}

\noindent
Nú skulum við snúa okkur að kjarna málsins, en það er að 
taka  fall $f\in C\sp
1(\R_+)$ af veldisvísisgerð og reikna út heildið
\begin{align*}
\int_0\sp b e\sp{-st}f\dash(t)\, dt &=
\left[e\sp{-st}f(t)\right]_0\sp b+
\int_0\sp b se\sp{-st}f(t)\, dt \\
&=
s\int_0\sp b e\sp{-st}f(t)\, dt -f(0)+e\sp{-sb}f(b).
\end{align*}
Ef $\Re s$ er nógu stórt, þá getum við látið $b\to \infty$ og fáum því
 \begin{equation*}\L\set{f\dash}(s)=s\L\set{f}(s)-f(0).
\label{7.3.1}
 \end{equation*}
Ef við gerum ráð fyrir að $f\in C^2(\R_+)$ og að bæði $f$ og $f\dash$
séu af veldisvísisgerð, þá fáum við með því að
beita þessari formúlu tvisvar  að 
 \begin{equation*}\L\set{f\ddash}(s)=s\L\set{f\dash}(s)-f\dash(0)=s\sp 2\L\set{f}(s)
-sf(0)-f\dash(0),
\label{7.3.2}
 \end{equation*}
og með þrepun fáum við síðan:

\begin{se} 
Ef $f\in C\sp m(\R_+)$ og  $f, f\dash, f\ddash,
\dots, f\sp{(m-1)}$,  eru af veldisvísisgerð, þá er $\L\set{f\sp{(m)}}(s)$
skilgreint fyrir öll $s\in \C$ með  $\Re s$ nógu stórt og
 \begin{equation*}\L\set{f\sp{(m)}}(s)=s\sp
m\L\set{f}(s)-s\sp{m-1}f(0)-\cdots-sf\sp{(m-2)}(0)-f\sp{(m-1)}(0).\label{7.3.3}
 \end{equation*}
\end{se}



Áður en við snúum okkur að því að leysa afleiðujöfnuhneppi með
Laplace-ummyndun, skulum við líta á veldisvísisfylkið:


\begin{se} 
Um sérhvert $m\times m$ fylki  $A$ gildir
\begin{equation*}
\L\set{e^{tA}}(s) = (sI-A)^{-1}.
\label{7.3.4}
\end{equation*}
\end{se}  

\section{Green--fallið og  földun\index{Green-fall}}


\noindent
Lítum nú á afleiðujöfnu með fastastuðla
 \begin{equation*}
P(D)u=(a_mD^m+\cdots+a_1D+a_0)u=f(t),\label{7.4.1}
 \end{equation*}
með upphafsskilyrðunum
 \begin{equation*}
u(a)=b_0, u\dash(a)=b_1,\  \dots,  \  u^{(m-1)}(a)=b_{m-1}.\label{7.4.2}
 \end{equation*}
Með því að hliðra til tímaásnum, þ.e.~skipta á fallinu $u(t)$ og
$u(t-a)$, þá getum við gert ráð fyrir að $a=0$.  


Við höfum
sýnt fram á að fallið $u_p$ sem uppfyllir $P(D)u=f(t)$, með
óhliðruðu upphafsskilyrðunum $b_0=\cdots=b_{m-1}=0$ er gefið með
formúlunni
 \begin{equation*}u_p(t)=\int_0^tG(t,\tau) f(\tau)\, d\tau,\label{7.4.3}
 \end{equation*}
þar sem $G$ er Green--fall virkjans $P(D)$.
Við skulum nú reikna út $U_p(s)=\L\set{u_p}(s)$.  Vegna
þess að upphafsgildin\index{Green-fall!fyrir upphafsgildisverkefni}
eru öll 0, þá er
\begin{equation*}
\L\set{u_p\dash}(s)=sU_p(s), \quad 
\L\set{u_p\ddash}(s)=s^2U_p(s),\dots,
\L\set{u_p^{(m)}}(s)=s^mU_p(s).
\end{equation*}
Þetta gefur okkur að 
$$ \L\set{P(D)u_p}(s)=(a_ms^m+\cdots+a_1s+a_0)U_p(s)=\L f(s), $$
sem er greinilega jafnan
$$ P(s)U_p(s)=\L f(s), $$
og við fáum 
\begin{equation*}
\L\set{u_p}(s)=\dfrac {\L f(s)}{P(s)}.\label{7.4.4}
\end{equation*}
Nú er Green--fallið $G(t,\tau)=g(t-\tau)$, þar sem $g$ uppfyllir
$$
P(D)g=0, \  g(0)=g\dash(0)=\cdots=g^{(m-2)}(0)=0, \ 
g^{(m-1)}(0)=\dfrac 1{a_m}.  
$$ 
Ef við setjum  $U(s)=\L g(s)$, þá fáum við
\begin{align*}
\L\set{g\dash}(s) &= s\L\set{g}(s)-g(0)=sU(s),\\
\L\set{g\ddash}(s) &= s^2\L\set{g}(s)-sg(0)-g\dash(0)\\
&=s^2U(s),\\
&\qquad \vdots\qquad\qquad\vdots\qquad\qquad \vdots\\
\L\set{g^{(m-1)}}(s) &=
s^{m-1}\L\set{g}(s)-s^{m-2}g(0)-\cdots-g^{(m-2)}(0)\\
&=s^{m-1}U(s),\\
\L\set{g^{(m)}}(s) &=
s^m\L\set{g}(s)-s^{m-1}g(0)-\cdots-g^{(m-1)}(0)\\
&=s^mU(s)-\dfrac 1{a_m}.
\end{align*}
Við tökum nú Laplace-myndina af báðum hliðum jöfnunnar $P(D)g=0$ og fáum
$$ (a_ms^mU(s)-1)+a_{m-1}s^{m-1}U(s)+\cdots+a_1sU(s)+a_0U(s)=0, $$
og við fáum $P(s)U(s)=1$, sem jafngildir
\begin{equation*}
\L g(s)=\dfrac 1{P(s)}.\label{7.4.5}
\end{equation*}
Við höfum því sýnt fram á að 
$$
\L\left\{\int_0^tg(t-\tau)f(\tau)\, d\tau\right\}(s)= \L\set{u_p}(s)=
\L\set{g}(s)\L\set{f}(s).
$$
Þessi formúla er engin tilviljun, því við höfum:

\begin{se}\label{set7.4.1}  Ef $f$ og $g$ eru föll af veldisvísisgerð
og heildanleg á sérhverju bili $[0,b]$, þá er
 $$\L\left\{\int_0^tf(t-\tau)g(\tau)\, d\tau\right\}(s)=
\L\set{f}(s)\L\set{g}(s).
 $$
\end{se}

Athugið að 
 $$\int_0^t f(t-\tau)g(\tau) \, d\tau=
\int_0^t f(\tau)g(t-\tau) \, d\tau.
 $$
Með því að velja $g(t)=1$ og nota að $\L\set{1}=1/s$, þá fæst:

\begin{fs} Ef $f$ er  af veldisvísisgerð og heildanlegt á sérhverju
bili $[0,b]$, þá er
 \begin{equation*}\L\left\{\int_0^t f(\tau) \, d\tau\right\}(s) = \dfrac 1s
\L\set{f}(s).
\label{7.4.6}
 \end{equation*}
\end{fs}

Földun\index{földun} tveggja falla $f, g: \R\to \C$  er skilgreind
með formúlunni 
 $$f*g(t)=\int_{-\infty}^{+\infty}f(t-\tau)g(\tau) \, d\tau,
 $$
og talan $t$ liggur í skilgreiningarmengi $f*g$ ef heildið er
samleitið.  Ef $f$ er til dæmis heildanlegt  á $\R$ og $g$ er
takmarkað, þá er földunin vel skilgreind fyrir öll $t\in \R$.
Ef föllin $f$ og $g$ eru bæði skilgreind og heildanleg á  $\R_+$, þá
getum við framlengt  skilgreiningarsvæði þeirra yfir í allt $\R$ með
því að setja 
$f(t)=g(t)=0$ fyrir öll $t<0$.  Þá er $f*g(t)$  skilgreint fyrir öll
$t\in \R$ og $$ f*g(t)= \int_0^tf(t-\tau)g(\tau)\, d\tau.$$
Við getum því umritað síðustu  setningu í 
\begin{equation*}
\L\set{f*g}=\L\set{f}\L\set{g}.\label{7.4.7}
\end{equation*}


