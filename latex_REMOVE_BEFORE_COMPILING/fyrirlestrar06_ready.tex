
%
%Allir pakkar sem þarf að nota.
%
\usepackage[utf8]{inputenc}
\usepackage[T1]{fontenc}
\usepackage[icelandic]{babel}
\usepackage{amsmath}
\usepackage{amssymb}
\usepackage{pictex}
\usepackage{epsfig,psfrag}
\usepackage{makeidx}
%\selectlanguage{icelandic}
%----------------------------

%
\hoffset=-0.4truecm
\voffset=-1truecm
\textwidth=16truecm 
%\textwidth=12truecm 
\textheight=23truecm
\evensidemargin=0truecm
%
% Gömlu gildin á bókinni 
%
%\voffset 1.4truecm
%\hoffset .25truecm
%\vsize  16.0truecm
%\hsize  15truecm
%
%
% Skilgreiningar á ýmsum skipunum.
%
%\newcommand{\Sb}{
%$$
%\sum_{\footnotesize\begin{array}{l} j=1 \\ j\neq k \end{array}}
%$$
%}
\newcommand{\bolddot}{{\mathbf \cdot}}
\newcommand{\C}{{\mathbb  C}}
\newcommand{\Cn}{{\mathbb  C\sp n}}
\newcommand{\crn}{{{\mathbb  C\mathbb  R^n}}}
\newcommand{\R}{{\mathbb  R}}
\newcommand{\Rn}{{\mathbb  R\sp n}}
\newcommand{\Rnn}{{\mathbb  R\sp{n\times n}}}
\newcommand{\Z}{{\mathbb  Z}}
\newcommand{\N}{{\mathbb  N}}
\renewcommand{\P}{{\mathbb  P}}
\newcommand{\Q}{{\mathbb  Q}}
\newcommand{\K}{{\mathbb  K}}
\newcommand{\U}{{\mathbb  U}}
\newcommand{\D}{{\mathbb  D}}
\newcommand{\T}{{\mathbb  T}}
\newcommand{\A}{{\cal A}}
\newcommand{\E}{{\cal E}}
\newcommand{\F}{{\cal F}}
\renewcommand{\H}{{\cal H}}
\renewcommand{\L}{{\cal L}}
\newcommand{\M}{{\cal M}}
\renewcommand{\O}{{\cal O}}
\renewcommand{\S}{{\cal S}}
\newcommand{\dash}{{\sp{\prime}}}
\newcommand{\ddash}{{\sp{\prime\prime}}}
\newcommand{\tdash}{{\sp{\prime\prime\prime}}}
\newcommand{\set }[1]{{\{#1\}}}
\newcommand{\scalar}[2]{{\langle#1,#2\rangle}}
\newcommand{\arccot}{{\operatorname{arccot}}}
\newcommand{\arccoth}{{\operatorname{arccoth}}}
\newcommand{\arccosh}{{\operatorname{arccosh}}}
\newcommand{\arcsinh}{{\operatorname{arcsinh}}}
\newcommand{\arctanh}{{\operatorname{arctanh}}}
\newcommand{\Log}{{\operatorname{Log}}}
\newcommand{\Arg}{{\operatorname{Arg}}}
\newcommand{\grad}{{\operatorname{grad}}}
\newcommand{\graf}{{\operatorname{graf}}}
\renewcommand{\div}{{\operatorname{div}}}
\newcommand{\rot}{{\operatorname{rot}}}
\newcommand{\curl}{{\operatorname{curl}}}
\renewcommand{\Im}{{\operatorname{Im\, }}}
\renewcommand{\Re}{{\operatorname{Re\, }}}
\newcommand{\Res}{{\operatorname{Res}}}
\newcommand{\vp}{{\operatorname{vp}}}
\newcommand{\mynd}[1]{{{\operatorname{mynd}(#1)}}}
\newcommand{\dbar}{{{\overline\partial}}}
\newcommand{\inv}{{\operatorname{inv}}}
\newcommand{\sign}{{\operatorname{sign}}}
\newcommand{\trace}{{\operatorname{trace}}}
\newcommand{\conv}{{\operatorname{conv}}}
\newcommand{\Span}{{\operatorname{Sp}}}
\newcommand{\stig}{{\operatorname{stig}}}
\newcommand{\Exp}{{\operatorname{Exp}}}
\newcommand{\diag}{{\operatorname{diag}}}
\newcommand{\adj}{{\operatorname{adj}}}
\newcommand{\erf}{{\operatorname{erf}}}
\newcommand{\erfc}{{\operatorname{erfc}}}
\newcommand{\Lloc}{{L_{\text{loc}}\sp 1}}
\newcommand{\boldcdot}{{\mathbb \cdot}}
%\newcommand{\Cinf0}[1]{{C_0\sp{\infty}(#1)}}
\newcommand{\supp}{{\text{supp}\, }}
\newcommand{\chsupp}{{\text{ch supp}\, }}
\newcommand{\singsupp}{{\text{sing supp}\, }}
\newcommand{\SL}[1]{{\dfrac {1}{\varrho} 
\bigg(-\dfrac d{dx}\bigg(p\dfrac {d#1}{dx}\bigg)+q#1\bigg)}}
\newcommand{\SLL}[1]{-\dfrac d{dx}\bigg(p\dfrac {d#1}{dx}\bigg)+q#1}
\newcommand{\Laplace}[1]{\dfrac{\partial^2 #1}{\partial x^2}+\dfrac{\partial^2 #1}{\partial y^2}}
\newcommand{\polh}[1]{{\widehat #1_{\C^n}}}
\newcommand{\tilv}{{}}
%
\renewcommand{\chaptername}{Kafli}
%
% Númering á formæulum.
%
\numberwithin{equation}{section}
%
%  Innsetning á myndum.
%
\def\figura#1#2{
\vbox{\centerline{
\input #1
}
\centerline{#2}
}\medskip}
\def\vfigura#1#2{
\setbox0\vbox{{
\input #1
}}
\setbox1\vbox{\hbox{\box0}\hbox{{\obeylines #2}}}
\dimen0 = -\ht1
\advance\dimen0 by-\dp1
\dimen1 = \wd1
\dimen2 = -\dimen0
\divide\dimen2 by\baselineskip
\count100 = 1
\advance\count100 by\dimen2
\advance\count100 by1
\box1
\hangindent\dimen1
\hangafter=-\count100
\vskip\dimen0
}
%
%  Setningar, skilgreiningar, o.s.frv. 
%
\newtheorem{setning+}           {Setning}      [section]
\newtheorem{skilgreining+}  [setning+]  {Skilgreining}
\newtheorem{setningogskilgreining+}  [setning+]  {Setning og
skilgreining}
\newtheorem{hjalparsetning+}  [setning+]  {Hjálparsetning}
\newtheorem{fylgisetning+}  [setning+]  {Fylgisetning}
\newtheorem{synidaemi+}  [setning+]  {Sýnidæmi}
\newtheorem{forrit+}  [setning+]  {Forrit}

\newcommand{\tx}[1]{{\rm({\it #1}). \ }}

\newenvironment{se}{\begin{setning+}\sl}{\hfill$\square$\end{setning+}\rm}
\newenvironment{sex}{\begin{setning+}\sl}{\hfill$\blacksquare$\end{setning+}\rm}
\newenvironment{sk}{\begin{skilgreining+}\rm}{\hfill$\square$\end{skilgreining+}\rm}
\newenvironment{sesk}{\begin{setningogskilgreining+}\rm}{\hfill$\square$\end{setningogskilgreining+}\rm}
\newenvironment{hs}{\begin{hjalparsetning+}\sl}{\hfill$\square$\end{hjalparsetning+}\rm}
\newenvironment{fs}{\begin{fylgisetning+}\sl}{\hfill$\square$\end{fylgisetning+}\rm}
\newenvironment{sy}{\begin{synidaemi+}\rm}{\hfill$\square$\end{synidaemi+}\rm}
\newenvironment{fo}{\begin{forrit+}\rm}{\hfill\end{forrit+}\rm}
\newenvironment{so}{\medbreak\noindent{\it Sönnun:}\rm}{\hfill$\blacksquare$\rm}
\newenvironment{sotx}[1]{\medbreak\noindent{\it #1:}\rm}{\hfill$\blacksquare$\rm}
\newcounter{daemateljari}
\newcommand{\aefing}{\section{Æfingardæmi} \setcounter{daemateljari}{1}}
\newcommand{\daemi}{
{\medskip\noindent{\bf \thedaemateljari.}}
\addtocounter{daemateljari}{1}
}

%\def\aefing{{\large\bf\bigskip\bigskip\noindent Æfingardæmi}}
%\def\daemi#1{\medskip\noindent{\bf #1.}}
\def\svar#1{\smallskip\noindent{\bf #1.} \ }
\def\lausn#1{\smallskip\noindent{\bf #1.} \ }
\def\ugrein#1{\medbreak\noindent{\bf #1.} }
\newcommand{\samantekt}{\noindent{\bf Samantekt.} }
%\newcommand{\proclaimbox}{\hfill$\square$}

\chapter
{UNDIRSTÖÐUATRIÐI UM AFLEIÐUJÖFNUR}




\section{Skilgreiningar á nokkrum hugtökum}

\subsection*{Venjulegar afleiðujöfnur}

{\it Afleiðujafna\index{afleiðujafna}} er  jafna sem lýsir
sambandi milli fallgilda óþekkts falls og gilda á einstökum 
afleiðum þess.  Ef óþekkta fallið er háð einni breytistærð, 
þá kallast jafnan {\it venjuleg afleiðujafna,
\index{afleiðujafna!venjuleg}\index{venjuleg afleiðujafna}} 
en ef það er háð fleiri en einni breytistærð, þá
kallast hún {\it hlutafleiðujafna\index{hlutafleiðujafna}}.  
Venjulega afleiðujöfnu er alltaf hægt að umrita yfir í jafngilda jöfnu
af gerðinni
\begin{equation*}
F(t,u,u',u\ddash,\dots,u\sp{(m)})=0 \label{1.1.1}
\end{equation*}
þar sem við hugsum okkur að $t$ sé breytistærð, sem tekur gildi í
einhverju hlutmengi $A$ af $\R$ og að $u$ sé óþekkt fall sem skilgreint er
á $A$ og tekur gildi í $\R$, $\C$ eða jafnvel $\R^m$.  
Úrlausn jöfnunnar felst í því að finna opið bil $I\subset A$ og
öll föll $u$ þannig að vigurinn
 \begin{equation*}(t,u(t),u\dash(t),\dots,u\sp{(m)}(t))\label{1.1.2}
 \end{equation*}
sé í skilgreiningarmengi fallsins $F$ og uppfylli jöfnuna
 \begin{equation*}F(t,u(t),u'(t),u\ddash(t),\dots,u\sp{(m)}(t))=0,
 \qquad t\in I.\label{1.1.3}
 \end{equation*}
Við  segjum  þá að fallið $u$ sé lausn á jöfnunni (\ref{1.1.1}).
{\it Stig\index{stig}\index{afleiðujafna!stig}}
afleiðujöfnu\index{stig!afleiðujöfnu} er hæsta stig á afleiðu, sem
kemur fyrir í jöfnunni.  
Við segjum að $m$-ta stigs afleiðujafnan (\ref{1.1.1}) sé á 
{\it staðalformi\index{staðalform}\index{afleiðujafna!staðalform}}
þegar hún hefur verið umrituð yfir í jafngilda jöfnu af taginu
 \begin{equation*}u\sp{(m)}=G(t,u,u',\dots,u\sp{(m-1)}).\label{1.1.4}
 \end{equation*}

\subsection*{Línulegar afleiðujöfnur}

Afleiðujafna af gerðinni 
 \begin{equation*}a_m(t)u\sp{(m)}+a_{m-1}(t)u\sp{(m-1)}+\cdots+a_1(t)u'+a_0(t)u=f(t),
\label{1.1.5}
 \end{equation*}
þar sem föllin $a_0,\dots,a_m,f$ eru skilgreind á bili $I\subset \R$,
er sögð vera {\it línuleg\index{afleiðujafna!línuleg}\index{línuleg
afleiðujafna}}.  Ástæðan
fyrir nafngiftinni er, að vinstri hliðin skilgreinir línulega vörpun
\begin{gather*}
L:C\sp m(I)\to C(I),\\
Lu(t)=
a_m(t)u\sp{(m)}(t)+a_{m-1}(t)u\sp{(m-1)}(t)+
\cdots+a_1(t)u'(t)+a_0(t)u(t),
\end{gather*}
ef $a_0,\dots,a_m\in C(I)$.
Hér táknar $C^m(I)$ línulegt rúm allra $m$ sinnum samfellt deildanlegra
falla á $I$ og $C(I)$ táknar rúm allra samfelldra falla á $I$.  
Við segjum að línulega jafna  sé {\it
óhliðruð\index{afleiðujafna!óhliðruð}\index{óhliðraður}} ef $f$ er núllfallið.
Annars segjum við að hún sé {\it
hliðruð\index{afleiðujafna!hliðruð}\index{hliðraður}}.  

\subsection*{Hlutafleiðujöfnur}

Erfitt er að lýsa hlutafleiðujöfnum 
með almennum hætti eins, en sem dæmi um hlutafleiðujöfnur
getum við tekið
\begin{alignat*}{2}
&\partial_xu+i\partial_yu=0,& \qquad
&\text{({\it
Cauchy--Riemann--jafna\index{Cauchy--Riemann!jafna}
\index{jafna!Cauchy--Riemann}})},\\
&\partial_x\sp 2u+\partial_y\sp 2u=0,&\qquad
&\text{({\it Laplace--jafna\index{Laplace!jafna}\index{jafna!Laplace}})},\\
&\partial_tu-\kappa(\partial_x\sp 2u+\partial_y\sp 2u+\partial_z\sp
2u)=f(x,y,z,t),& \qquad
&\text{({\it varmaleiðnijafna\index{varmaleiðnijafna}})},\\
&\partial_t^2u-c^2(\partial_x\sp 2u+\partial_y\sp 2u+\partial_z\sp
2u)=f(x,y,z,t),& \qquad
&\text{({\it bylgjujafna\index{bylgjujafna}})}.
\end{alignat*}

\subsection*{Tilvist og ótvíræðni lausna}

Það eru margvíslegar spurningar sem menn leita svara við þegar
afleiðujöfnur eru leystar.  Eðlilega fjallar fyrsta spurningin um
tilvist á lausn.  Ef henni er svarað játandi er eðlilegt að spyrja
næst með hvaða skilyrðum lausn sé ótvírætt ákvörðuð og síðan
hvernig ákvarða megi lausnir  og finna nálganir á þeim.
Til þess að útskýra þetta skulum við líta á
einföldustu afleiðujöfnu sem hugsast getur
$$u'=0.$$
Við vitum að öll fastaföll, $u(t)=c$, $t\in\R$, uppfylla þessa jöfnu
og að sérhver lausn er fastafall.
Spurningunni um tilvist er því svarað játandi, en spurningunni um ótvíræðni
er svarað neitandi, því  við höfum óendanlega margar lausnir.  Lítum á
aðeins flóknara dæmi, nefnilega jöfnuna
\begin{equation*}u'=f, \label{1.1.6}
\end{equation*}
þar sem við hugsum okkur að fallið $f$ sé samfellt á bilinu $I\subset
\R$.  Undirstöðusetning stærðfræðigreiningarinnar segir okkur að
sérhvert stofnfall $f$ sé lausn.  Jafnframt vitum við að mismunur
tveggja stofnfalla er fastafall og því er sérhver lausn af gerðinni
 \begin{equation*}
 u(t)=b+\int_a\sp t f(\tau) \, d\tau, \qquad t,a\in I.\label{1.1.7}
 \end{equation*}
Ef við setjum nú það skilyrði að lausnin eigi að taka ákveðið gildi $b$ 
í punktinum $a\in I$, 
 \begin{equation*}u'=f(t), \qquad u(a)=b,\label{1.1.8}
 \end{equation*}
þá gefur undirstöðusetning stærðfræðigreiningarinnar að til er 
ótvírætt ákvörðuð lausn og hún er sett fram með formúlunni hér að framan.

\section{Fyrsta stigs jöfnur}  


\subsection*{Línulegar jöfnur}

Fyrsta stigs línuleg afleiðujafna er af gerðinni
 \begin{equation*}a_1(t)u'+a_0(t)u=f(t).\label{1.2.1}
 \end{equation*}\index{afleiðujafna!línuleg fyrsta stigs}
\index{línuleg fyrsta stigs afleiðujafna}
Við skulum rifja upp aðferðina til að leysa þessa jöfnu í  því tilfelli
að stuðlarnir eru samfelld föll á einhverju bili $I$ og að $a_1(t)\neq
0$ fyrir öll $t\in I$. Með því að deila í gegnum jöfnuna með $a_1(t)$,
þá getum við gert ráð fyrir því að $a_1$ sé fastafallið $1$ og við ætlum
því að leysa
 $$u'+a_0(t)u=f(t).
 $$
Aðferðin gengur út á
að skilgreina $A$ sem eitthvert stofnfall $a_0$,
 $$A(t)=c+\int_a^t a_0({\tau})\, d{\tau}, \qquad t,a\in I, 
 $$ 
og athuga að ef $u$ er lausn, þá gildir
 $$\dfrac d{dt} (e\sp{A(t)}u(t))=e\sp{A(t)}(u'(t)+a_0(t)u(t))=e\sp{A(t)}f(t).
 $$
Af þessari jöfnu leiðir síðan að 
 $$e\sp{A(t)}u=C+\int_a^t e\sp{A({\tau})}f({\tau}) \, d{\tau},
 $$
og þar með fæst almenna lausnarformúlan
 $$u(t)=e\sp{-A(t)}(C+\int_a^t e\sp{A({\tau})}f({\tau}) \, d{\tau}),
 $$
þar sem $C$ er einhver fasti.  Þessi útreikningur okkar sýnir að
sérhver lausn á jöfnunni hlýtur að vera af þessari gerð. Nú er hins
vegar lauflétt að sýna að þetta er lausn, með því að stinga þessari 
formúlu inn í
afleiðujöfnuna.  Verkefnið 
 $$u'+a_0(t)u=f(t), \qquad u(a)=b,
 $$
hefur ótvírætt ákvarðaða lausn og hún er fundin  með því að velja
stofnfallið $A$ þannig að $A(a)=0$ og $C=b$,
 \begin{equation*}
u(t)=e\sp{-A(t)}(b+\int_a\sp t e\sp{A(\tau)}f(\tau) \, d\tau), 
\qquad A(t)=\int_a\sp t a_0(\tau) \, d\tau.\label{1.2.2}
 \end{equation*}

\subsection*{Aðskiljanlegar jöfnur}

Við segjum að fyrsta stigs afleiðujafna $u'=f(t,u)$ sé
{\it aðskiljanleg\index{afleiðujafna!aðskiljanleg}\index{aðskiljanleg
afleiðujafna}} ef hægt er að
rita fallið $f$ sem kvóta af gerðinni $f(t,x)=g(t)/h(x)$.
Til þess að leysa jöfnuna, þá skrifum við hana
sem  $h(u)u'=g(t)$ og heildum síðan
$$ \int h(u(t))u'(t) \, dt = c+\int g(t)\, dt, $$
þar sem $c$ er heildunarfasti.  Ef við viljum síðan leysa verkefnið
$$ u'=f(t,u), \qquad u(a)=b, $$
þá  veljum við stofnfall $H$ fyrir
$h$ og heildum
 \begin{equation*}H(u(t))-H(b)= \int_b\sp{u(t)} h(x) \, dx =
\int_a\sp t h(u({\tau}))u'({\tau}) \, d{\tau} = 
\int_a\sp t g(\tau) \, d\tau.\label{1.2.3}
 \end{equation*}
Ef til er grennd um punktinn $b$ þar sem fallið $H$ hefur andhverfu,
þá getum við skrifað lausnina sem
 \begin{equation*}u(t) = H\sp{[-1]}\left( H(b)+G(t)\right), \qquad G(t)=\int_a\sp t
g(\tau)\, d\tau. \label{1.2.4}
 \end{equation*}
Í útreikningum á venjulegum dæmum borgar sig yfirleitt ekki að reikna
út formúlu fyrir $H\sp {[-1]}$ og stinga síðan gildinu $H(b)+G(t)$
inn í þá formúlu eins og lýst er hér.  Þess í stað er betra að
leysa $u(t)$ úr jöfnunni $H(u(t))-H(b)=G(t)$.

\section{Afleiðujöfnuhneppi}\label{grein1.3}

\noindent
{\it Afleiðujöfnuhneppi\index{afleiðujöfnuhneppi}} er safn af jöfnum sem
lýsa sambandi milli gilda
óþekktra falla og gilda á einstökum afleiðum þeirra.  Ef óþekktu föllin eru
háð einni breytistærð, þá kallast það {\it
venjulegt\index{afleiðujöfnuhneppi!venjulegt} afleiðujöfnuhneppi},
en það kallast {\it
hlutafleiðujöfnuhneppi\index{hlutafleiðujöfnuhneppi}} ef þau eru háð fleiri en
einni breytistærð.  Venjulegt afleiðujöfnuhneppi er alltaf hægt að
umrita yfir í jöfnur af gerðinni
\begin{equation*}
F_j(t,u_1,\dots,u_k,u_1\dash,\dots,u_k\dash,\dots,
u_1^{(m)},\dots,u_k^{(m)})=0, \qquad j=1,\dots,l,
\label{1.3.1}
\end{equation*}
þar sem $t$ táknar breytistærðina, $u_1,\dots,u_k$ eru óþekktu föllin og
föllin $F_1,\dots,F_l$  taka gildi í $\R$ eða $\C$.  Til þess að einfalda
ritháttinn, þá skilgreinum við vigurgildu föllin $u=(u_1,\dots,u_k)$ og
$F=(F_1,\dots,F_l)$. Þá eru jöfnurnar jafngildar vigurjöfnunni
$F(t,u,u\dash,\dots,u^{(m)})=0$
sem hefur sama útlit.


\subsection*{Staðalform hneppa}

Við segjum að hneppið sé á {\it
staðalformi\index{staðalform}}\index{afleiðujöfnuhneppi!staðalform}
, ef fjöldi jafna og fjöldi
óþekktra falla er sá sami og það er af gerðinni
$$
u^{(m)}=G(t,u,u\dash,\dots,u^{(m-1)}).
$$
Mikilvægustu hneppin sem við fáumst við eru
fyrsta stigs venjuleg afleiðujöfnuhneppi á staðalformi
$$
u\dash =G(t,u).
$$
Ef við skrifum upp hnitaföllin fyrir þetta hneppi, þá  fáum við jöfnurnar
\begin{align*}
u_1\dash&= G_1(t, u_1,\dots, u_m),\\
u_2\dash&= G_2(t, u_1,\dots, u_m),\\
&\quad \vdots\\
u_m\dash&= G_m(t, u_1,\dots, u_m),
\end{align*}
þar sem $G_j:\Omega\to\R$,  $\Omega\subset \R\times\R^m$ eða
$G_j:\Omega\to\C$,  $\Omega\subset \R\times\C^m$ eftir því hvort við
viljum að lausnin taki rauntölugildi eða tvinntölugildi.
Föllin $u=(u_1,\dots,u_m)$ og $G=(G_1,\dots,G_m)$
taka gildi í vigurrúminu $\R\sp m$ eða $\C\sp m$, eftir því hvort við
hugsum okkur að lausnirnar eigi að taka rauntölugildi eða
tvinntölugildi.


\subsection*{Línuleg afleiðujöfnuhneppi}

Við segjum að fyrsta stigs jöfnuhneppi sé {\it
línulegt\index{afleiðujöfnuhneppi!línulegt}\index{línulegt
afleiðujöfnuhneppi}} ef fallið
$G$ er af gerðinni 
 $$G(t,x)=A(t)x+f(t),
 $$
þar sem $A(t)$ er $m\times m$ fylki og $f(t)$ er $m$--vigur.
Ef við skrifum upp hnitin þá verður hneppið
\begin{align*}
u_1\dash&=a_{11}(t)u_1+\cdots+a_{1m}(t)u_m+f_1(t),\\
u_2\dash&=a_{21}(t)u_1+\cdots+a_{2m}(t)u_m+f_2(t),\\
&\qquad \qquad \vdots\qquad \qquad \qquad \qquad \vdots\\
u_m\dash&=a_{m1}(t)u_1+\cdots+a_{mm}(t)u_m+f_m(t).
\end{align*}
Hér eru föllin $a_{jk}(t)$ stökin í fylkinu $A(t)$.
Við segjum að hneppið sé {\it
óhliðrað\index{afleiðujöfnuhneppi!óhliðrað}\index{línulegt
afleiðujöfnuhneppi!óhliðrað}}
ef $f$ er núllfallið og við segjum  að það sé {\it
hliðrað\index{afleiðujöfnuhneppi!hliðrað}\index{línulegt
afleiðujöfnuhneppi!hliðrað}}  annars.


\subsection*{Jöfnur af hærri stigum og jafngild hneppi}

Lítum nú á venjulega $m$--ta stigs afleiðujöfnu á staðalformi
\begin{equation*}v\sp{(m)}=G(t,v,v\dash,\dots,v\sp{(m-1)}).\label{1.3.2}
\end{equation*}
Ef við skilgreinum vigurfallið $u=(u_1,\dots,u_m)$ með
$$u_1=v, \quad u_2=v\dash,\dots, \quad  u_m=v\sp{(m-1)},
$$ þá uppfyllir $u$
jöfnuhneppið
 \begin{equation*}
u_1\dash = u_2, \quad
u_2\dash = u_3, \quad\dots \quad
u_{m-1}\dash = u_m, \quad
u_m\dash =G(t, u_1,\dots,u_m). 
\label{1.3.3}
 \end{equation*}
Jafnan og jöfnuhneppið eru jafngild í þeim skilningi
að sérhver lausn $v$ á  gefur lausn
$u=(v,v\dash,\dots,v\sp{(m-1)})$ á hneppinu og sérhver lausn $u$ á
hneppinu  gefur lausnina $v=u_1$ á jöfnunni.  Þessi einfalda staðreynd
er mikilvæg, því einfalt reynist  að sanna tilvist á lausnum á fyrsta
stigs jöfnuhneppum á staðalformi.  Þá niðurstöðu er síðan hægt að nota
til að sanna tilvist á lausnum á jöfnum af stigi stærra en $1$.  

Línulega afleiðujafnan
 $$a_m(t)v^{(m)}+\cdots+a_1(t)v\dash+ a_0(t)v=g(t)
 $$
er greinilega jafngild línulega hneppinu
\begin{gather*}\label{1.3.4}
u_1\dash = u_2,\qquad  u_2\dash = u_3, \qquad \dots, \quad
u_{m-1}\dash = u_m\\
u_m\dash
=-(a_0(t)/a_m(t))u_1-\cdots-(a_{m-1}(t)/a_m(t))u_m+g(t)/a_m(t),\nonumber 
\end{gather*}
ef $a_m(t)\neq 0$ fyrir öll $t\in I$.
Fylkið $A$ og vigurinn $f$ verða þá 
\begin{equation*}A=\left[\begin{matrix}
0&1&\dots&0\\
0&0&\dots&0\\
\vdots&\vdots&\ddots&\vdots\\
0&0&\dots&1\\
-a_0/a_m&-a_1/a_m&\dots&-a_{m-1}/a_m
\end{matrix}\right],
\qquad
f=\left[\begin{matrix}
0\\
0\\
\vdots\\
0\\
g/a_m
\end{matrix}\right].\label{1.3.5}
\end{equation*}
 
\section{Upphafsgildisverkefni\index{upphafsgildisverkefni}}

\noindent
Oft hafa menn áhuga á að finna lausnir á afleiðujöfnum og
afleiðujöfnuhneppum sem uppfylla einhverja ákveðna eiginleika.  {\it
Upphafsgildisverkefni}  snúast um að leysa afleiðujöfnuhneppi með því
hliðarskilyrði að lausnin og einhverjar afleiður hennar taki fyrirfram
gefin gildi í ákveðnum punkti.   Upphafsgildisverkefni fyrir fyrsta
stigs hneppi af staðalformi er til dæmis verkefnið
\begin{equation*}
u\dash=f(t,u), \quad t\in I, \qquad u(a)=b.
\label{1.4.1}
\end{equation*}
Hér er átt við að finna eigi lausn $u=(u_1,\dots,u_m)$ á jöfnunni á
bilinu $I$, sem tekur gildið $b=(b_1,\dots,b_m)$ í punktinum $a\in I$.
Upphafsgildisverkefni fyrir $m$-ta stigs línulega jöfnu er af gerðinni
\begin{equation*}
\begin{cases} a_m(t)v^{(m)}+\cdots+a_1(t)v\dash+a_0(t)v=g(t), & t\in I,\\
v(a)=b_0, \quad v\dash(a)=b_1, \quad \dots \quad  v^{(m-1)}(a)=b_{m-1}.&
\end{cases}
\label{1.4.2}
\end{equation*}
Ef $a_m(t)\neq 0$ fyrir öll $t\in I$, þá getum við deilt í gegnum
jöfnuna með $a_m(t)$ og umskrifað hana síðan yfir í jafngilt $m\times m$
línulegt jöfnuhneppi með óþekkta vigurfallið
$u=(v,v\dash,\dots,v^{(m-1)})$. 


\section{Jaðargildisverkefni}


\noindent
{\it Jaðargildisverkefni\index{jaðargildisverkefni}}  snúast um að leysa jöfnu
$$u^{(m)}=f(t,u,u\dash,\dots,u^{(m-1)})$$  
af stigi $m$ á takmörkuðu bili
$I=[a,b]$ með skilyrðum á 
$$
u(a), \ u'(a),\dots,  \ u^{(m-1)}(a)\qquad \text{ og } 
\qquad  u(b), \ u(b),\dots, \ u^{(m-1)}(b).
$$  
Þessi skilyrði eru venjulega sett fram þannig að ákveðnar línulegar
samantektir af þessum fallgildum eigi að taka fyrirfram gefin gildi.
Fyrir annars stigs jöfnu geta {\it
jaðarskilyrðin\index{jaðargildisskilyrði}} til dæmis verið
$$
u(a)=0, \qquad u\dash(b)=0.
$$
{\it Lotubundin\index{jaðargildisskilyrði!lotubundin}\index{lotubundin
jaðarskilyrði}} jaðarskilyrði eru af gerðinni
$$
u(a)=u(b), \qquad u\dash(a)=u\dash(b).
$$
Almenn línuleg jaðarskilyrði fyrir annars stigs jöfnu eru 
\begin{align*}
B_1u&={\alpha}_{11}u(a)+{\alpha}_{12}u\dash(a)
    +{\beta}_{11}u(b)+{\beta}_{12}u\dash(b)=c_1\\
B_2u&={\alpha}_{21}u(a)+{\alpha}_{22}u\dash(a)
    +{\beta}_{21}u(b)+{\beta}_{22}u\dash(b)=c_2,
\end{align*}
þar sem stuðlarnir ${\alpha}_{jk}$, ${\beta}_{jk}$, $c_{j}$ eru gefnir
fyrir $j,k=1,2$.  Almenn línuleg jaðarskilyrði fyrir $m$-ta stigs  jöfnu
eru af gerðinni
$$
B_ju=\sum\limits_{l=1}^m \big({\alpha}_{jl}u^{(l-1)}(a)
+{\beta}_{jl}u^{(l-1)}(b)\big)=c_j, \qquad j=1,2,\dots,m.
$$
Við lítum á $B_j$ sem línulega vörpun $C^{m-1}[a,b]\to \C$ 
og skilgreinum {\it
jaðargildisvirkja\index{jaðargildisvirki}\index{virki}\index{virki!jaðargildis}}  
$B:C^{m-1}[a,b]\to \C^m$ með formúlunni
$Bu=(B_1u,\dots,B_mu)$.  Almennt jaðargildisverkefni fyrir $m$-ta stigs
línulega jöfnu er að leysa
\begin{equation*}
\begin{cases}
a_m(t)u^{(m)}+\cdots+a_1(t)u\dash+a_0(t)u=f(t),  &t\in ]a,b[\\
Bu=c, \qquad B_ju=\sum\limits_{l=1}^m \big({\alpha}_{jl}u^{(l-1)}(a)
+{\beta}_{jl}u^{(l-1)}(b)\big), 
\end{cases} \label{1.5.1}
\end{equation*} 
fyrir gefið fall $f\in C[a,b]$ og gefinn vigur $c\in \C^m$.  Athugið að
upphafsskilyrðin í (\ref{1.4.2}) eru dæmi um almenn línuleg 
jaðarskilyrði, þar
sem við setjum ${\beta}_{jl}=0$ fyrir öll $j$ og $l$, ${\alpha}_{jl}=1$
ef $j=l$ og ${\alpha}_{jl}=0$ ef $j\neq l$. 
Ef bilið $I$
er ótakmarkað geta verið skilyrði á markgildin 
$$
\lim_{x\to\pm\infty}u(x), \qquad 
\lim_{x\to \pm\infty}u\dash(x),\quad \dots 
$$
eftir því sem við á.  Þessi skilyrði geta verið sams konar línulegar 
samantektir og við höfum verið að lýsa.



\section{Tilvist og ótvíræðni lausna á afleiðujöfnum}

\noindent
Í þessari grein ætlum við að fjalla um
tilvist á lausn á upphafsgildisverkefninu
 \begin{equation*}u\dash=f(t,u),  \qquad u(a)=b,\label{1.7.1}
 \end{equation*}
þar sem fallið $f\in C(\Omega,\R^m)$ er skilgreint á einhverju
hlutmengi $\Omega$ í 
$\R\times \R^m$, $a$ er gefin rauntala, $b$ er gefinn vigur og
$(a,b)\in \Omega$.  Tilfellið að $f$ taki gildi í tvinntölurúminu
$\C^m$ og  að $\Omega$ sé hlutmengi í  $\R\times \C^m$ fæst síðan með
því að líta á $\C^m$ sem vigurrúmið $\R^{2m}$.  Ef við ætlumst til þess að
lausnin $u$ hafi samfellda afleiðu, þá þurfum við auðvitað að gera
ráð fyrir því að fallið $f$ sé samfellt.

\begin{se}\tx{Peano\index{Peano}\index{setning!Peano}}  
Gerum ráð fyrir að $\Omega$ sé grennd um punktinn $(a,b)\in
\R\times\R^m$ og að $f\in C(\Omega,\R^m)$.  Þá
er til opið bil $I$ sem inniheldur punktinn $a$ og fall $u:I\to \R^m$,
þannig að $(t,u(t))\in \Omega$,  
$u\dash(t)=f(t,u(t))$ fyrir öll $t\in I$ og $u(a)=b$. 
\end{se}



Setning Peano er of erfið til þess að við getum átt við að sanna hana
hér, en fróðlegt er að vita hvað hún segir. Við munum hins vegar
sanna tvær tilvistarsetningar, sem kenndar eru við Picard.  Í þeim
gefum við okkur meiri forsendur um fallið $f$, en að það sé bara
samfellt, og þær tryggja að lausnin
verði ótvírætt ákvörðuð.  Setning Peano segir okkur
einungis að til sé lausn en hún segir ekkert um það hvort lausnin er
ótvírætt ákvörðuð.  


\begin{sy}\label{syn1.7.2}
\hfill Athugum upphafsgildisverkefnið
$u\dash=3u^{2/3}$, $u(0)=0$.  Fyrir 


\vfigura{fig013} {}
\noindent sérhvert $\alpha>0$ fáum við
lausnina $u_\alpha$, sem skilgreind er með
 $$
 u_\alpha(t)=\begin{cases}
(t+\alpha)^3, &t<-\alpha,\\
0, &-\alpha\leq t<\alpha,\\
(t-\alpha)^3, &\alpha\leq t.
\end{cases}
 $$
Þetta dæmi sýnir okkur að til
þess að fá ótvírætt ákvarðaða lausn þurfum við að setja einhver
strangari skilyrði á $f$ en samfelldni.
\end{sy}

\begin{sk}{\tx{Lipschitz--skilyrði}}
Látum  $f:\Omega\to\R^m$ vera fall, þar sem  $\Omega\subset \R\times
\R^m$  og $A\subset \Omega$.
Ef til er fasti $C$ þannig að 
 \begin{equation*}|f(t,x)-f(t,y)|\leq C|x-y|,\qquad (t,x), (t,y)\in
 A,\label{1.7.2} 
 \end{equation*}
þá segjum við að $f$ uppfylli {\it
Lipschitz--skilyrði\index{Lipschitz--skilyrði}} í menginu $A$.
\end{sk}

\begin{sy}\label{syn1.7.4} (i) Ef jöfnuhneppið er línulegt, $f(t,x)=A(t)x+g(t)$,
$A\in C(I,\C^{m\times m})$ og $g\in C(I,\C^m)$, þá uppfyllir $f$
Lipschitz--skilyrði í $J\times \C^m$ fyrir sérhvert lokað og
takmarkað hlutbil $J\subset I$. Þetta sést á því að 
 $$|f(t,x)-f(t,y)|=|A(t)(x-y)|
\leq \sum\limits_{j,k=1}^m |a_{jk}(t)||x-y|\leq C|x-y|,
 $$
þar sem $C=\sup\sum\limits_{j,k=1}^m |a_{jk}(t)|$ 
og efra markið er tekið yfir öll $t\in J$.

\smallskip
(ii)  Látum $f\in C^{1}(\Omega,\R^m)$ og gerum ráð fyrir að $\Omega$ sé
þannig að fyrir sérhvert par af 
punktum $(t,x), (t,y)$ í $\Omega$  liggi línustrikið milli þeirra í $\Omega$.
Línustrikið  samanstendur af öllum punktum $(t,\tau x+(1-\tau)y)$,
$\tau\in [0,1]$. Látum nú $A$ vera lokað og takmarkað hlutmengi af
$\Omega$, sem hefur þann eiginleika að fyrir sérhvert par af punktum
$(t,x), (t,y)$ í $A$ liggur línustrikið á milli þeirra í $A$.  Þá er
\begin{align*}
|f(t,x)-f(t,y)|&=|\int_0\sp 1\dfrac d{d\tau}f(t,(1-\tau)y+\tau x) \,
d\tau|\\
&=|\int_0\sp 1 \sum\limits_{j=1}\sp m
\partial_{x_j}f(t,(1-\tau)y+\tau x)
(x_j-y_j) \, d\tau|\\
&\leq \sup\limits_{(\tau,\xi)\in A} 
\sum\limits_{j=1}\sp m |\partial_{x_j}f(\tau,\xi)||x-y|,
\end{align*}

\noindent
og þar með uppfyllir $f$ Lipschitz--skilyrði í $A$.

\smallskip
(iii)  Lítum nú á fallið $f(t,x)=x\sp 2$, með $\Omega=\R\times \R$.
Það uppfyllir  $$|f(t,x)-f(t,y)|=|x+y||x-y|, 
 $$
en þetta gefur okkur að $f$ uppfylli ekki Lipschitz--skilyrði í $\Omega$, því
þátturinn $x+y$ er ekki takmarkaður.  Ef við látum hins vegar
$[\alpha,\beta]$ 
vera takmarkað bil og veljum $A=\R\times [\alpha,\beta]$, þá
uppfyllir fallið $f$ Lipschitz--skilyrði í $A$ og við getum valið
fastann $C$ sem 
$C=2(|\alpha|+|\beta|)$.  

\smallskip
(iv) Fallið $f(t,x)=3x\sp{2/3}$, í sýnidæmi \ref{syn1.7.2}, er samfellt, en
uppfyllir ekki
Lipschitz--skilyrði í neinni grennd um $0$, því
$|f(t,x)-f(t,0)|=x\sp{2/3}=x\sp{-1/3}|x-0|$ og $x\sp{-1/3}\to \infty$ ef
$x\to 0$.  
\end{sy}

Nú kemur í ljós að Lipschitz--skilyrði tryggir að lausnin verður
ótvírætt ákvörð\-uð:

\begin{se}\label{set1.7.5}{\tx{Picard; víðfeðm
útgáfa}\index{Picard--setningin}\index{Picard--setningin!víðfeðm
útgáfa}\index{setning!Picard}}
Látum $I\subset \R$ vera opið bil, $a\in I$, $b\in \R\sp m$,
$f\in C(I\times \R\sp m,\R\sp m)$ og gerum ráð fyrir að $f$ uppfylli
Lipschitz--skilyrði í $J\times \R\sp m$ fyrir sérhvert lokað og
takmarkað hlutbil $J$ í $I$. Þá er til ótvírætt ákvörðuð lausn 
$u\in C\sp 1(I,\R\sp m)$ á upphafsgildisverkefninu
 $$u\dash=f(t,u), \qquad u(a)=b.
 $$
\end{se}

Þessi setning er önnur tveggja tilvistarsetninga sem við sönnum í
næstu grein.  Eins og fram hefur komið kallast hún
venjulega {\it víðfeðm} útgáfa af tilvistarsetningu fyrir
fyrsta stigs hneppi. Ástæðan fyrir nafngiftinni er, að við fáum
lausn á bili sem inniheldur öll $t$--gildi þar sem hægri hlið
jöfnunnar er skilgreind.
Tökum nú fyrir 
tvær mikilvægustu afleiðingar setningarinnar.  Í sýnidæmi
\ref{syn1.7.4}
sáum við að forsendurnar í setningu \ref{set1.7.5} eru uppfylltar fyrir línuleg
jöfnuhneppi með samfellda stuðla.  Við lítum á vigurrúmið $\C^m$ yfir
tvinntölurnar sem $2m$ víða rúmið $\R^{2m}$ yfir rauntölurnar og fáum: 

\begin{fs}
Látum $I\subset \R$ vera opið bil, $a\in I$, $b\in \C\sp m$,
$A\in C(I,\C\sp{m\times m})$ og $f\in C(I,\C\sp m)$. Þá er til
ótvírætt ákvörðuð lausn  
$u\in C\sp 1(I,\C\sp m)$ á upphafsgildisverkefninu
 \begin{equation*}u\dash=A(t)u+f(t) \qquad u(a)=b.\label{1.7.3}
 \end{equation*}
\end{fs}

Með umskrift á upphafsgildisverkefni fyrir $m$-ta stigs afleiðujöfnu
yfir í jafngilt hneppi fáum við:

\begin{fs}
Látum $I\subset \R$ vera opið bil, $a\in I$,
$b_0,\dots,b_{m-1} \in \C$, $a_0,\dots,a_m, g\in C(I)$ og $a_m(t)\neq 0$
fyrir öll $t\in I$.
Þá er til ótvírætt ákvörðuð lausn  
$u\in C\sp m(I)$ á upphafsgildisverkefninu
\begin{gather*}
a_m(t)u\sp {(m)}+\cdots+a_1(t)u\dash+a_0(t)u=g(t),\\
u(a)=b_0, u\dash(a)=b_1,\dots, u\sp{(m-1)}(a)=b_{m-1}.
\end{gather*}
\end{fs}

Nú setjum við fram aðra útgáfu sem venjulega
kallast {\it staðbundin} útgáfa af tilvistarsetningu fyrir fyrsta
stigs hneppi:


\begin{se}\label{set1.7.8}{\tx{Picard; staðbundin
útgáfa\index{Picard--setningin}\index{Picard--setningin!staðbundin
útgáfa}}\index{setning!Picard}}
Látum $\Omega$ vera opið hlutmengi í $\R\times \R\sp{m}$,
$a\in \R$, $b\in \R\sp m$,
$(a,b)\in \Omega$  og $f\in C(\Omega,\R\sp m)$. Gerum ráð fyrir að
til sé grennd $U$ um punktinn $(a,b)$ innihaldin í $\Omega$ og að
fallið $f$ uppfylli Lipschitz--skilyrði í $U$.  Þá
er til opið bil $I$ á $\R$ sem inniheldur $a$ og ótvírætt ákvörðuð lausn
$u\in C\sp 1(I, \R^m)$ á upphafsgildisverkefninu 
 $$u\dash=f(t,u), \qquad u(a)=b.
 $$
\end{se}

Ástæðan fyrir því að þessi setning kallast {\it staðbundin}  útgáfa
af tilvistarsetningunni fyrir fyrsta stigs afleiðujöfnuhneppi er sú,
að hún segir okkur einungis að til sé bil $I$ þar sem lausnin er til.
Í sönnuninni, sem við tökum fyrir í næstu grein, 
kemur fram hvernig bilið $I$ er háð
$U$,  Lipschitz--fasta fallsins $f$ og upphafsgildinu $b$.  

\begin{sy}  Við skulum taka eitt dæmi til þess að sjá hvernig 
skilgreiningarsvæði
lausnarinnar er háð upphafsgildinu $b$ og líta á verkefnið $u'=u\sp
2$, $u(a)=b$, þar sem $b$ er jákvæð rauntala.  Lausnin er fallið
 $$u(t)=\dfrac b{1-b(t-a)}, \qquad t\in I=]-\infty,a+1/b[.
 $$
Maður skyldi ætla að óreyndu, að svona einföld jafna hefði lausn, sem
skilgreind er á öllum rauntalnaásnum, en svo er greinilega ekki.
Skilgreiningarsvæðið minnkar eftir því sem upphafsgildið 
stækkar.   Athugið að engu að síður hefur verkefnið lausn í grennd
um $a$ fyrir sérhvert val á $(a,b)$.  Við sáum í sýnidæmi
\ref{syn1.7.4} (iii) uppfyllir skilyrðin í staðbundnu útgáfu Picard
setningarinnar, en ekki þeirrar víðfeðmu.  
\end{sy}

Aðferðin sem beitt er í sönnuninni á  þessum setningum er kennd
við franska stærðfræðinginn Émile Picard.  Eins og áður hefur verið
sagt framkvæmum við hana í smáatriðum í næstu grein.
Auðvelt er að skilja meginhugmyndina í sönnuninni á víðfeðmu útgáfunni
af Picard--setningunni og skulum við líta á hana núna.

 Við athugum fyrst, að 
\begin{equation*}u\in C\sp 1(I,\R\sp m), \quad u\dash=f(t,u),\quad t\in I, \quad
u(a)=b \label{1.7.4}
 \end{equation*}
er jafngilt því að
 \begin{equation*}
u\in C(I,\R\sp m),\quad 
u(t)=b+\int_a\sp t f(\tau,u(\tau))\, d\tau, \qquad t\in I.\label{1.7.5} 
 \end{equation*}
Okkur dugir því að sanna að til sé ótvírætt ákvarðað fall $u\in
C(I,\R\sp m)$ sem uppfyllir heildisjöfnuna (\ref{1.7.5}). 
Tilvistin er fengin með því að skilgreina runu $\set{ u_n}$ af föllum í
$C(I,\R\sp m)$ með formúlunni
 \begin{equation*}u_0(t)=b, \qquad 
u_n(t)=b+\int_a\sp t f(\tau,u_{n-1}(\tau))\, d\tau, \qquad t\in
I,\label{1.7.6} 
 \end{equation*}
og sýna síðan að þessi fallaruna sé samleitin að markfalli $u$.
Ekki er nóg að sýna að runan $\set{u_n(t)}$ stefni á $u(t)$  í sérhverjum
punkti heldur þurfum við að sanna að $\set{u_n}$ sé samleitin í
{\it jöfnum mæli\index{samleitni!í jöfnum mæli}} 
á sérhverju lokuðu og takmörkuðu hlutbili $J$ af
$I$.  Að því fengnu gefa niðurstöðurnar í grein 3.5 að 
markfallið $u$ er í $C(I,\R\sp m)$. 
Lipschitz skilyrðið gefur að 
 $$|f(t,u_n(t))-f(t,u(t))|\leq C|u_n(t)-u(t)|, \qquad t\in J,
 $$
og þar með að runan $f(t,u_n(t))$ stefnir á markfallið $f(t,u(t))$
í jöfnum mæli á $J$.  Þá  megum við skipta á heildi og markgildi og
við fáum það sem sanna á,
\begin{multline*}
u(t)= \lim\limits_{n\to +\infty} u_n(t) =
b+\lim\limits_{n\to +\infty} \int_a\sp t f(\tau,u_{n-1}(\tau)) \, d\tau =\\
=
b+\int_a\sp t \lim\limits_{n\to +\infty} f(\tau,u_{n-1}(\tau)) \, d\tau =
b+ \int_a\sp t f(\tau,u(\tau)) \, d\tau.
\end{multline*}
Tökum nú tvö dæmi, sem sýna hvers er að vænta um
samleitni rununnar $\set{u_n}$. 