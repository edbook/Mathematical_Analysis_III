
%
%Allir pakkar sem þarf að nota.
%
\usepackage[utf8]{inputenc}
\usepackage[T1]{fontenc}
\usepackage[icelandic]{babel}
\usepackage{amsmath}
\usepackage{amssymb}
\usepackage{pictex}
\usepackage{epsfig,psfrag}
\usepackage{makeidx}
%\selectlanguage{icelandic}
%----------------------------

%
\hoffset=-0.4truecm
\voffset=-1truecm
\textwidth=16truecm 
%\textwidth=12truecm 
\textheight=23truecm
\evensidemargin=0truecm
%
% Gömlu gildin á bókinni 
%
%\voffset 1.4truecm
%\hoffset .25truecm
%\vsize  16.0truecm
%\hsize  15truecm
%
%
% Skilgreiningar á ýmsum skipunum.
%
%\newcommand{\Sb}{
%$$
%\sum_{\footnotesize\begin{array}{l} j=1 \\ j\neq k \end{array}}
%$$
%}
\newcommand{\bolddot}{{\mathbf \cdot}}
\newcommand{\C}{{\mathbb  C}}
\newcommand{\Cn}{{\mathbb  C\sp n}}
\newcommand{\crn}{{{\mathbb  C\mathbb  R^n}}}
\newcommand{\R}{{\mathbb  R}}
\newcommand{\Rn}{{\mathbb  R\sp n}}
\newcommand{\Rnn}{{\mathbb  R\sp{n\times n}}}
\newcommand{\Z}{{\mathbb  Z}}
\newcommand{\N}{{\mathbb  N}}
\renewcommand{\P}{{\mathbb  P}}
\newcommand{\Q}{{\mathbb  Q}}
\newcommand{\K}{{\mathbb  K}}
\newcommand{\U}{{\mathbb  U}}
\newcommand{\D}{{\mathbb  D}}
\newcommand{\T}{{\mathbb  T}}
\newcommand{\A}{{\cal A}}
\newcommand{\E}{{\cal E}}
\newcommand{\F}{{\cal F}}
\renewcommand{\H}{{\cal H}}
\renewcommand{\L}{{\cal L}}
\newcommand{\M}{{\cal M}}
\renewcommand{\O}{{\cal O}}
\renewcommand{\S}{{\cal S}}
\newcommand{\dash}{{\sp{\prime}}}
\newcommand{\ddash}{{\sp{\prime\prime}}}
\newcommand{\tdash}{{\sp{\prime\prime\prime}}}
\newcommand{\set }[1]{{\{#1\}}}
\newcommand{\scalar}[2]{{\langle#1,#2\rangle}}
\newcommand{\arccot}{{\operatorname{arccot}}}
\newcommand{\arccoth}{{\operatorname{arccoth}}}
\newcommand{\arccosh}{{\operatorname{arccosh}}}
\newcommand{\arcsinh}{{\operatorname{arcsinh}}}
\newcommand{\arctanh}{{\operatorname{arctanh}}}
\newcommand{\Log}{{\operatorname{Log}}}
\newcommand{\Arg}{{\operatorname{Arg}}}
\newcommand{\grad}{{\operatorname{grad}}}
\newcommand{\graf}{{\operatorname{graf}}}
\renewcommand{\div}{{\operatorname{div}}}
\newcommand{\rot}{{\operatorname{rot}}}
\newcommand{\curl}{{\operatorname{curl}}}
\renewcommand{\Im}{{\operatorname{Im\, }}}
\renewcommand{\Re}{{\operatorname{Re\, }}}
\newcommand{\Res}{{\operatorname{Res}}}
\newcommand{\vp}{{\operatorname{vp}}}
\newcommand{\mynd}[1]{{{\operatorname{mynd}(#1)}}}
\newcommand{\dbar}{{{\overline\partial}}}
\newcommand{\inv}{{\operatorname{inv}}}
\newcommand{\sign}{{\operatorname{sign}}}
\newcommand{\trace}{{\operatorname{trace}}}
\newcommand{\conv}{{\operatorname{conv}}}
\newcommand{\Span}{{\operatorname{Sp}}}
\newcommand{\stig}{{\operatorname{stig}}}
\newcommand{\Exp}{{\operatorname{Exp}}}
\newcommand{\diag}{{\operatorname{diag}}}
\newcommand{\adj}{{\operatorname{adj}}}
\newcommand{\erf}{{\operatorname{erf}}}
\newcommand{\erfc}{{\operatorname{erfc}}}
\newcommand{\Lloc}{{L_{\text{loc}}\sp 1}}
\newcommand{\boldcdot}{{\mathbb \cdot}}
%\newcommand{\Cinf0}[1]{{C_0\sp{\infty}(#1)}}
\newcommand{\supp}{{\text{supp}\, }}
\newcommand{\chsupp}{{\text{ch supp}\, }}
\newcommand{\singsupp}{{\text{sing supp}\, }}
\newcommand{\SL}[1]{{\dfrac {1}{\varrho} 
\bigg(-\dfrac d{dx}\bigg(p\dfrac {d#1}{dx}\bigg)+q#1\bigg)}}
\newcommand{\SLL}[1]{-\dfrac d{dx}\bigg(p\dfrac {d#1}{dx}\bigg)+q#1}
\newcommand{\Laplace}[1]{\dfrac{\partial^2 #1}{\partial x^2}+\dfrac{\partial^2 #1}{\partial y^2}}
\newcommand{\polh}[1]{{\widehat #1_{\C^n}}}
\newcommand{\tilv}{{}}
%
\renewcommand{\chaptername}{Kafli}
%
% Númering á formæulum.
%
\numberwithin{equation}{section}
%
%  Innsetning á myndum.
%
\def\figura#1#2{
\vbox{\centerline{
\input #1
}
\centerline{#2}
}\medskip}
\def\vfigura#1#2{
\setbox0\vbox{{
\input #1
}}
\setbox1\vbox{\hbox{\box0}\hbox{{\obeylines #2}}}
\dimen0 = -\ht1
\advance\dimen0 by-\dp1
\dimen1 = \wd1
\dimen2 = -\dimen0
\divide\dimen2 by\baselineskip
\count100 = 1
\advance\count100 by\dimen2
\advance\count100 by1
\box1
\hangindent\dimen1
\hangafter=-\count100
\vskip\dimen0
}
%
%  Setningar, skilgreiningar, o.s.frv. 
%
\newtheorem{setning+}           {Setning}      [section]
\newtheorem{skilgreining+}  [setning+]  {Skilgreining}
\newtheorem{setningogskilgreining+}  [setning+]  {Setning og
skilgreining}
\newtheorem{hjalparsetning+}  [setning+]  {Hjálparsetning}
\newtheorem{fylgisetning+}  [setning+]  {Fylgisetning}
\newtheorem{synidaemi+}  [setning+]  {Sýnidæmi}
\newtheorem{forrit+}  [setning+]  {Forrit}

\newcommand{\tx}[1]{{\rm({\it #1}). \ }}

\newenvironment{se}{\begin{setning+}\sl}{\hfill$\square$\end{setning+}\rm}
\newenvironment{sex}{\begin{setning+}\sl}{\hfill$\blacksquare$\end{setning+}\rm}
\newenvironment{sk}{\begin{skilgreining+}\rm}{\hfill$\square$\end{skilgreining+}\rm}
\newenvironment{sesk}{\begin{setningogskilgreining+}\rm}{\hfill$\square$\end{setningogskilgreining+}\rm}
\newenvironment{hs}{\begin{hjalparsetning+}\sl}{\hfill$\square$\end{hjalparsetning+}\rm}
\newenvironment{fs}{\begin{fylgisetning+}\sl}{\hfill$\square$\end{fylgisetning+}\rm}
\newenvironment{sy}{\begin{synidaemi+}\rm}{\hfill$\square$\end{synidaemi+}\rm}
\newenvironment{fo}{\begin{forrit+}\rm}{\hfill\end{forrit+}\rm}
\newenvironment{so}{\medbreak\noindent{\it Sönnun:}\rm}{\hfill$\blacksquare$\rm}
\newenvironment{sotx}[1]{\medbreak\noindent{\it #1:}\rm}{\hfill$\blacksquare$\rm}
\newcounter{daemateljari}
\newcommand{\aefing}{\section{Æfingardæmi} \setcounter{daemateljari}{1}}
\newcommand{\daemi}{
{\medskip\noindent{\bf \thedaemateljari.}}
\addtocounter{daemateljari}{1}
}

%\def\aefing{{\large\bf\bigskip\bigskip\noindent Æfingardæmi}}
%\def\daemi#1{\medskip\noindent{\bf #1.}}
\def\svar#1{\smallskip\noindent{\bf #1.} \ }
\def\lausn#1{\smallskip\noindent{\bf #1.} \ }
\def\ugrein#1{\medbreak\noindent{\bf #1.} }
\newcommand{\samantekt}{\noindent{\bf Samantekt.} }
%\newcommand{\proclaimbox}{\hfill$\square$}

\chapter
{LÍNULEGAR AFLEIÐUJÖFNUR}
 
%\kaflahaus{Línulegar afleiðujöfnur}


\section{Línulegir afleiðuvirkjar}


\subsection*{Kennimargliðan}

Afleiðujafna af gerðinni
 $$a_m(t)u\sp{(m)}+a_{m-1}(t)u\sp{(m-1)}+\cdots+a_1(t)u'+a_0(t)u=f(t),
 $$
þar sem föllin $a_0,\dots,a_m,f$ eru skilgreind á bili $I\subset \R$,
er sögð vera {\it línuleg :hover:`afleiðujafna!línuleg`}, því vinstri
hliðin skilgreinir línulega vörpun
\begin{gather*}
L:C\sp m(I)\to C(I),\\
Lu(t)=
a_m(t)u\sp{(m)}(t)+a_{m-1}(t)u\sp{(m-1)}(t)+
\cdots+a_1(t)u'(t)+a_0(t)u(t),
\end{gather*}
ef $a_0,\dots,a_m\in C(I)$.
Línuleg vörpun af þessari gerð  kallast {\it
afleiðuvirki :hover:`afleiðuvirki` :hover:`virki!afleiðuvirki`}.  
Við segjum að jafnan sé {\it
óhliðruð :hover:`afleiðujafna!óhliðruð` :hover:`óhliðraður`} ef $f$ er núllfallið.
Annars segjum við að hún sé {\it
hliðruð :hover:`afleiðujafna!hliðruð` :hover:`hliðraður`}.
Fyrir sérhvern punkt $t\in I$ fáum við margliðu af einni breytistærð
$\lambda$,
 \begin{equation*}
P(t,\lambda)= a_m(t)\lambda\sp{m}+a_{m-1}(t)\lambda\sp{m-1}+

.. _2.1.1:

\cdots+a_1(t)\lambda+a_0(t).
 \end{equation*}
Þessa margliðu köllum við {\it
kennimargliðu :hover:`kennimargliða` :hover:`margliða!kennimargliða`
 :hover:`afleiðuvirki!kennimargliða`} afleiðuvirkjans $L$.


\subsection*{Afleiðuvirkinn $D$}

Til þess að geta reiknað á auðveldan hátt með afleiðuvirkjum er venja að
skilgreina virkjann $D$ sem $Du=u'$ og síðan veldi $D\sp k$ af $D$ með
$$
D\sp 0u=u, \quad D\sp 1u=u', \quad
D\sp 2u=DDu=u\ddash, \quad \dots \quad D\sp ku= D D\sp
{k-1}u=u\sp{(k)}. 
$$
Afleiðuvirkinn $L$ er síðan  skrifaður með formúlunni
 \begin{equation*}P(t,D)=a_m(t)D\sp m+\cdots+a_1(t)D+a_0(t).

.. _2.1.2:

 \end{equation*}
Athugið að í síðasta liðnum höfum við sleppt því að skrifa $D\sp 0$,
en þetta á að lesa þannig að þegar virkinn $L$ er látinn verka á fallið $u$,
er margfaldað með $a_0(t)$ í síðasta liðnum.  

Ef allir stuðlarnir $a_j$ eru fastaföll, þá
segjum við að virkinn hafi {\it fastastuðla :hover:`fastastuðlar`} og við
skrifum  þá einungis $P(D)$ í stað $P(t,D)$.  
Þegar ekki er ljóst í formúlum með tilliti til hvaða breytistærðar er
verið að deilda, þá tilgreinum  við það með 
$D_t$, $D_x$, $D_s$, \dots, í stað $D$
í tákninu fyrir virkjann, þar sem lágvísirinn er táknið fyrir
breytistærðina. 
Línulega samantekt tveggja
afleiðuvirkja $P(t,D)$ og $Q(t,D)$ með tvinntölunum $\alpha$ og
$\beta$ táknum við með $\alpha P(t,D)+\beta Q(t,D)$.  Þetta er
virkinn sem skilgreindur er með formúlunni
$$
\big(\alpha P(t,D) + \beta Q(t,D)\big)u=
\alpha P(t,D)u + \beta Q(t,D)u.
$$
Samsetningu virkjanna $P(t,D)$ og $Q(t,D)$ táknum við með
$P(t,D)Q(t,D)$ og er hún skilgreind með
$$
\big(P(t,D)Q(t,D)\big)u=
P(t,D)\big(Q(t,D)u\big).
$$


Sýnt er með dæmum sýnir að almennt
er $P(t,D)Q(t,D)\neq Q(t,D)P(t,D)$, með öðrum orðum, víxlreglan
gildir ekki við samsetningu afleiðuvirkja.
Hins vegar gildir hún ef virkjarnir hafa fastastuðla:

\subsubsection{Setning}
Ef $P(D)$ og $Q(D)$ eru línulegir
afleiðuvirkjar :hover:`afleiðuvirki!línulegur` með
fastastuðla :hover:`afleiðuvirki!með fastastuðla`,
þá er
 $$P(D)Q(D)=Q(D)P(D).
 $$


--------------





Nú skulum við líta á tilvist á lausnum á jaðargildisverkefnum.  Í grein
6.5 skilgreindum við almennan jaðargildisvirkja með formúlunni
\begin{equation*}
\begin{cases}
B:C^{m-1}[a,b]\to \C^m, \qquad Bu=(B_1u,\dots,B_mu),\\
B_ju=\sum\limits_{l=1}^m {\alpha}_{jl}u^{(l-1)}(a)
+{\beta}_{jl}u^{(l-1)}(b)=c_j,  &j=1,2,\dots,m.

.. _2.1.9:

\end{cases}
\end{equation*}
Við höfum fullkomna lýsingu á því hvenær lausn fæst:


.. _set2.1.6:

\subsubsection{Setning} Látum $I$ vera opið bil sem inniheldur $[a,b]$, $P(t,D)$ vera
línulegan afleiðuvirkja, $a_m(t)\neq 0$ fyrir
öll $t\in I$ og $B$ vera almennan jaðargildisvirkja.
Þá eru eftirfarandi skilyrði jafngild

\item{(i)} Jaðargildisverkefnið $P(t,D)u=f(t)$, $Bu=c$,
hefur ótvírætt ákvarðaða lausn $u\in C^m(I)$ fyrir sérhvert $f\in C(I)$
og sérhvert $c\in \C^m$.
\item{(ii)} Jaðargildisverkefnið $P(t,D)u=0$, $Bu=0$,
hefur einungis núllfallið sem lausn.
\item{(iii)}  Ef $u_1,\dots,u_m$ er grunnur í ${\cal N}(P(t,D))$, þá er
$$
\left|\begin{matrix} B_1u_1 & B_1u_2 & \cdots & B_1u_m\\
B_2u_1 & B_2u_2 & \cdots & B_2u_m\\
\vdots & \vdots &\ddots & \vdots \\
B_mu_1 & B_mu_2 & \cdots & B_mu_m
\end{matrix}\right|\neq 0.
$$



--------------



\subsubsection{Sönnun}
Byrjum á því að athuga að (i) er jafngilt:

\item{(i)$^\prime$} Jaðargildisverkefnið 
$P(t,D)u=0$, $Bu=c$,
hefur ótvírætt ákvarðaða lausn fyrir sérhvert $c\in \C^m$.

Við sjáum að (i)$\dash$ er sértilfelli af (i).  Til þess að sjá að
(i)$\dash$ hafi (i) í för með sér,  þá notfærum við okkur að
fylgisetning \tilv 6.7.7 gefur okkur fall $v$ sem uppfyllir
$P(t,D)v=f(t)$ án nokkurra hliðarskilyrða.  Samkvæmt (i)$\dash$ er til
fall $w$ sem uppfyllir $P(t,D)w=0$ og $Bw=c-Bv$.  Nú er ljóst að $u=v+w$
uppfyllir (i), því 
$$
P(t,D)u=P(t,D)v+P(t,D)w=f(t), \qquad Bu=Bv+Bw=c.
$$
Lausnin er ótvírætt ákvörðuð, því mismunur tveggja lausna er núllfallið
samkvæmt (i)$\dash$.
Nú snúum við okkur að því að sanna að (i)$\dash$, (ii) og (iii) séu
jafngild. Til þess látum við
$u_1,\dots,u_m$ vera grunn í núllrúminu og skrifum lausn $u$ á
$P(t,D)u=0$ sem $u=d_1u_1+\cdots+d_mu_m$, þar sem $d_1,\dots,d_m\in \C$.
Skilyrðið $Bu=c$ er þá jafngilt jöfnuhneppinu
\begin{align*}
\big(B_1u_1)d_1+\cdots+\big(B_1u_m\big)d_m &=c_1,\\
\big(B_2u_1)d_1+\cdots+\big(B_2u_m\big)d_m &=c_2,\\
\vdots\qquad\qquad\vdots\qquad\qquad & \vdots \\
\big(B_mu_1)d_1+\cdots+\big(B_mu_m\big)d_m &=c_m.\\
\end{align*}
Við vitum úr línulegri algebru að það er jafngilt að hliðraða jafnan
hafi lausn fyrir sérhverja hægri hlið, 
að óhliðraða jafnan hafi aðeins núlllausnina og að
ákveða stuðlafylkisins sé frábrugðin núlli.  Þetta er nákvæmlega það
sem skilyrðin (i)$\dash$, (ii) og (iii) segja.


--------------




Hugsum okkur nú að við þekkjum grunn $u_1,\dots,u_m$ fyrir núllrúm
virkjans $P(t,D)$ og eina sérlausn $u_p$  á $P(t,D)u=f$.  Þá er lausnin $u$
á  (i) af gerðinni $u=d_1u_1+\cdots+d_mu_m+u_p$ þar sem stuðlarnir 
$d_1,\dots,d_m$ eru lausnir jöfnuhneppisins
\begin{equation*}
\left[\begin{matrix} B_1u_1 & B_1u_2 & \cdots & B_1u_m\\
B_2u_1 & B_2u_2 & \cdots & B_2u_m\\
\vdots & \vdots &\ddots & \vdots \\
B_mu_1 & B_mu_2 & \cdots & B_mu_m
\end{matrix}\right]
\left[\begin{matrix} d_1\\ d_2\\ \vdots \\ d_m\end{matrix}\right]
=\left[\begin{matrix} c_1-B_1u_p\\ c_2-B_2u_p\\ \vdots \\ c_m-B_mu_p
\end{matrix}\right].


.. _2.1.10:

\end{equation*}


\section{Línulegar jöfnur með fastastuðla :hover:`afleiðuvirki!með
fastastuðla`}

\subsection{Línulegar jöfnur með fastastuðla :hover:`afleiðuvirki!með
fastastuðla`}

\noindent
Við skulum nú líta á línulega afleiðujöfnu :hover:`afleiðujafna!línuleg`
 \begin{equation*}P(D)u = (a_mD^m+\cdots+a_1D+a_0)u
=f(t), \qquad t\in I,

.. _2.2.1:

 \end{equation*}
þar sem við gerum ráð fyrir því að stuðlarnir $a_j$ í virkjanum séu
fastaföll, $a_j\in \C$, og $a_m\neq 0$.
Kennimargliðan :hover:`kennimargliða`
 :hover:`kennimargliða!virkja` :hover:`margliða!kennimargliða` er þá 
\begin{equation*}
P(\lambda)=a_m\lambda^m+\cdots+a_1\lambda+a_0.

.. _2.2.2:

\end{equation*}
Fyrsta viðfangsefni okkar  er að finna grunn fyrir núllrúmið 
${\cal N}(P(D))$ og fá 
þannig framsetningu á almennri lausn :hover:`afleiðujafna!almenn lausn`
 :hover:`almenn lausn`
óhliðruðu :hover:`afleiðujafna!óhliðruð` jöfnunnar $P(D)u=0$. Við
byrjum á því að láta afleiðuvirkjana
$D\sp k$ verka á veldisvísisfallið $e\sp{\alpha t}$, þar sem $\alpha$
er einhver tvinntala. Þá fæst
$$
De^{\alpha t}=\alpha e^{\alpha t},\quad
D^2e^{\alpha t}=\alpha^2 e^{\alpha t},\quad
\dots , \quad 
D^me^{\alpha t}=\alpha^m e^{\alpha t}.
$$ 
Þetta gefur okkur síðan 

.. _2.2.3:

\begin{align*}
P(D)e^{\alpha t}&=(a_mD^m+\cdots+a_1D+a_0)e^{\alpha t} \\
&=(a_m{\alpha}^m+\cdots+a_1{\alpha}+a_0)e^{\alpha
t}=P(\alpha)e^{\alpha t}.\nonumber
\end{align*}
Ef við veljum $\alpha$ sem eina af núllstöðvum kennimargliðunnar $P$,
þá sjáum við að $e\sp{\alpha t}$ er lausn á óhliðruðu jöfnunni.
Undirstöðusetning algebrunnar gefur okkur, að við getum þáttað
margliðuna $P$ fullkomlega yfir tvinntölurnar og skrifað hana sem
 \begin{equation*}P(\lambda)=a_m(\lambda-\lambda_1)\sp{m_1}\cdots
(\lambda-\lambda_\ell)\sp{m_\ell},

.. _2.2.4:

 \end{equation*}
þar sem $\lambda_1,\dots,\lambda_\ell\in \C$ eru {\it
núllstöðvarnar :hover:`núllstöð` :hover:`margfeldni!núllstöðvar`}
og $m_1,\dots,m_\ell$ er {\it
margfeldni :hover:`margfeldni` :hover:`núllrúm!margfeldni`} þeirra, $m_1+\cdots+m_\ell=m$.
Með því að nota þessa framsetningu á kennimargliðunni getum við skrifað
afleiðuvirkjann sem 
 \begin{equation*}P(D)=a_m(D-\lambda_1)\sp{m_1}\cdots(D-\lambda_\ell)\sp{m_\ell}.

.. _2.2.5:

 \end{equation*}
Við fáum nú fullkomna lýsingu á núllrúmi :hover:`núllrúm` afleiðuvirkja með fastastuðla:

\subsubsection{Setning}
Gerum ráð fyrir að $P(D)$ sé línulegur afleiðuvirki af stigi $m$
með fasta\-stuðla og  að kennimargliðan $P(\lambda)$  hafi $\ell$ ólíkar
núllstöðvar  
$\lambda_1,\dots,\lambda_\ell\in \C$ með margfeldnina
$m_1,\dots,m_\ell$.  Þá mynda 
föllin
\begin{gather*}
e^{\lambda_1t}, te^{\lambda_1t},\dots, t^{m_1-1}e^{\lambda_1t},\\
e^{\lambda_2t}, te^{\lambda_2t},\dots, t^{m_2-1}e^{\lambda_2t},\\
\quad \vdots\qquad \vdots \qquad \qquad \vdots\\
e^{\lambda_\ell t}, te^{\lambda_\ell t},\dots, t^{m_\ell-1}e^{\lambda_\ell t},
\end{gather*}
grunn í núllrúmi virkjans $P(D)$ og þar með má skrifa 
sérhvert stak í núllrúminu sem 
$$
q_1(t)e^{\lambda_1t}+\cdots+q_\ell(t)e^{\lambda_\ell t},
$$
þar sem $q_j$ eru margliður af stigi $<m_j$, $1\leq j\leq \ell$.


--------------




 
\section{Euler-jöfnur :hover:`Euler` :hover:`Euler!jafna` :hover:`jafna!Euler`}

\subsection{Euler-jöfnur :hover:`Euler` :hover:`Euler!jafna` :hover:`jafna!Euler`}

\noindent
Afleiðujafna af gerðinni
\begin{equation*}
P(x,D_x)u=
a_mx^mu^{(m)}+\cdots+a_1xu\dash+a_0u=0,

.. _2.3.1:

\end{equation*}
þar sem stuðlarnir $a_j$ eru tvinntölur, $a_m\neq 0$ og $u$ er óþekkt
fall af $x$, nefnist {\it Euler-jafna}.  
Til þess að fá almenna lýsingu á lausnum jöfnunnar á $\R\setminus\set 0$
dugir okkur að finna almenna lausn á jákvæða raunásnum, því auðvelt er
að sannfæra sig um að $v(x)=u(|x|)$ er lausn á $\R\setminus\set 0$ þá og
því aðeins að $u$ sé lausn á $\{x\in \R; x>0\}$.
Athugið að veldið á $x$ í
hverjum lið er það sama og stigið á afleiðunni.  Ef við stingum 
$u(x)=x^r$ inn í afleiðuvirkjann, þá fæst
\begin{align*}
P(x,D_x)u
&=a_mx^m r(r-1)\cdots(r-m+1)x^{r-m}
+\cdots+a_1xrx^{r-1}+a_0x^r\\
&=\big(a_m r(r-1)\cdots(r-m+1)+
\cdots+a_1r+a_0\big)x^r.
\end{align*}
Þar með er $u$ lausn þá og því aðeins að $r$ sé núllstöð $m$-ta stigs
margliðunnar $Q$, sem skilgreind er með formúlunni
\begin{equation*}
Q(r)=a_m r(r-1)\cdots(r-m+1)+

.. _2.3.2:

\cdots+a_1r+a_0.
\end{equation*}
Lítum fyrst á tilfellið að þessi jafna hafi ólíkar núllstöðvar
$r_1,\dots, r_m$. 
Þá er auðvelt að sannfæra sig  um að
föllin $x^{r_1},\dots,x^{r_m}$ eru línulega óháð og þar
með er almenn lausn á Euler jöfnu af gerðinni
\begin{equation*}
u(x)=c_1x^{r_1}+\cdots+c_mx^{r_m}.

.. _2.3.3:

\end{equation*}
Nú skulum við athuga tilfellið þegar $Q(r)$ hefur margfaldar
núllstöðvar.  Þá skilgreinum við fallið $v(t)=u(e^t)$ og sýnum fram á að
$v$ uppfylli  $Q(D)v=0$.  Við 
þurfum þá að þekkja sambandið milli afleiða fallanna $u$ og $v$.
Við höfum
\begin{align*}
u(x)&=v(\ln x),\\
u\dash(x)&=v\dash(\ln x)\cdot \dfrac 1x,\\
u\ddash(x)&=v\ddash(\ln x)\cdot \dfrac 1{x^2}
-v\dash(\ln x)\cdot \dfrac 1{x^2} = D(D-1)v(\ln x)\cdot \dfrac 1{x^2}.
\end{align*}
Með þrepun fæst síðan að 
\begin{equation*}
u^{(k)}(x)=D(D-1)\cdots(D-k+1)v(\ln x)\cdot \dfrac 1{x^k}.


.. _2.3.4:

\end{equation*}
Þetta gefur 
\begin{align*}
P(x,D)u(x)&=\sum\limits_{k=0}^m a_kx^ku^{(k)}(x)\\
&=\sum\limits_{k=0}^m a_kD(D-1)\cdots(D-k+1)v(\ln x)\\
&=Q(D)v(\ln x).
\end{align*}
Þar með er $u$ lausn
á Euler-jöfnunni þá og því aðeins að $v$ sé lausn á jöfnunni 
$Q(D)v=0$.  Nú hefur virkinn $Q$ fastastuðla svo við getum beitt 
setningu 7.2.1:

\subsubsection{Setning}  Almenn lausn Euler-jöfnunnar á jákvæða raunásnum
er línuleg samatekt fallanna
\begin{gather*}
x^{r_1}, \big(\ln x \big) x^{r_1}, \dots,
\big(\ln x\big)^{m_1-1}x^{r_1},\\
x^{r_2}, \big(\ln x\big)x^{r_2}, \dots,
\big(\ln x \big)^{m_2-1} x^{r_2},\\
\vdots \qquad \qquad \qquad \vdots \qquad \qquad \qquad \vdots\\ 
x^{r_\ell}, \big(\ln x \big)x^{r_\ell}, \dots,
\big(\ln x\big)^{m_\ell-1} x^{r_\ell},
\end{gather*}
þar sem $r_1,\dots,r_\ell$ eru ólíkar núllstöðvar  margliðunnar
$Q$, sem gefin er með (:ref:`2.3.2`), og margfeldni þeirra er 
$m_1,\dots,m_\ell$.


--------------





\section{Sérlausnir :hover:`sérlausn` :hover:`afleiðujafna!sérlausn`}

\noindent
Algengt er að ástandsjöfnur eðlisfræðilegra kerfa séu af gerðinni
 $$P(D)u=f
 $$
þar sem $P(D)$ er línulegur afleiðuvirki með fastastuðla og $f$ er gefið
fall á einhverju bili.  Fallið $f$ stendur oft fyrir ytra álag, örvun
eða krafta, sem á kerfið verka, en lausnin er svörun kerfisins við
þessu ytra álagi.   Til þess að skilja kerfið er nauðsynlegt að ráða
yfir fjölbreytilegum aðferðum til þess að reikna út svörunina $u$
þegar ytra álagið $f$ er gefið.  Í þessari grein ætlum við að líta á
tilfellið að $f$ sé veldisvísisfall eða hornafall og athuga hvort hægt
sé að finna sérlausn af sömu gerð.  Í næstu grein munum við hins vegar
fjalla um almenna aðferð til þess að finna sérlausn fyrir hvaða hægri
hlið sem er.
við höfum séð að $P(D)e^{\alpha t}=P(\alpha)e^{\alpha t}$.  Ef
$\alpha$ er núllstöð
kennimargliðunnar $P$, þá er veldisvísisfallið $e^{\alpha t}$ lausn á
óhliðruðu jöfnunni.  Ef aftur á móti $P(\alpha) \neq 0$, þá er
\begin{equation*}
P(D)u_p=e^{\alpha t} \qquad\text{ ef } \qquad 
u_p(t)=\dfrac{e^{\alpha t}}{P(\alpha)}.

.. _2.4.1:

\end{equation*}
Ef $\alpha\in \R$, $P(i\alpha)\neq 0$ og $P(-i\alpha)\neq 0$, þá fáum
við með því að nota jöfnur Eulers að
\begin{equation*}
P(D)u_p=\cos \alpha t \qquad\text{ ef } \qquad 
u_p(t)=

.. _2.4.2:

\dfrac{e^{i\alpha t}}{2P(i\alpha)}+
\dfrac{e^{-i\alpha t}}{2P(-i\alpha)},
\end{equation*}
og
\begin{equation*}
P(D)u_p=\sin \alpha t \qquad\text{ ef } \qquad 
u_p(t)=\dfrac{e^{i\alpha t}}{2iP(i\alpha)}
-\dfrac{e^{-i\alpha t}}{2iP(-i\alpha)}.

.. _2.4.3:

\end{equation*}
Í því tilfelli að kennimargliðan hefur eingöngu rauntalnastuðla, þá
verða lausnirnar í þessum tveimur dæmum
\begin{equation*}
u_p(t)=\Re \bigg(\dfrac{e^{i{\alpha}t}}{P(i{\alpha})}\bigg), \qquad
\text{ og } \qquad
u_p(t)=\Im \bigg(\dfrac{e^{i{\alpha}t}}{P(i{\alpha})}\bigg).

.. _2.4.4:

\end{equation*}
Ef $\alpha\in \R$, $P(\alpha)\neq 0$ og $P(-\alpha)\neq 0$, þá fáum
við að 
\begin{equation*}
P(D)u_p=\cosh \alpha t \qquad\text{ ef }\qquad
u_p(t)=\dfrac{e^{\alpha t}}{2P(\alpha)}+\dfrac{e^{-\alpha
t}}{2P(-\alpha)},

.. _2.4.5:

\end{equation*}
og
\begin{equation*}
P(D)u_p=\sinh \alpha t \qquad\text{ ef }\qquad
u_p(t)=\dfrac{e^{\alpha t}}{2P(\alpha)}-\dfrac{e^{-\alpha
t}}{2P(-\alpha)}.

.. _2.4.6:

\end{equation*}


\subsection*{Sérlausnir fundnar með virkjareikningi}

Nú skulum við láta afleiðuvirkjann $D-{\alpha}$ verka á margfeldi
fallanna $v$ og $e^{{\alpha} t}$.  Við fáum þá 
 \begin{equation*}(D-\alpha)(ve\sp{\alpha t})
=D(ve\sp{\alpha t})-\alpha ve\sp{\alpha t} = v\dash e\sp{\alpha t}.


.. _2.4.10:

 \end{equation*}
Af þessari formúlu fæst síðan með þrepun
 \begin{equation*}(D-\alpha)\sp m(ve\sp{\alpha t})= v\sp{(m)} e\sp{\alpha
t}\qquad m\geq 1.

.. _2.4.11:

 \end{equation*}
Ef við veljum nú $v(t)=t\sp k$, þá fáum við 
 \begin{equation*}

.. _2.4.12:
(D-\alpha)\sp m(t\sp ke\sp{\alpha t})= 
\begin{cases}
0, &k<m,\\
k!e^{\alpha t},& k=m,\\
k(k-1)\cdots(k-m+1)t^{k-m}e^{\alpha t},& k>m.
\end{cases}
 \end{equation*}
Hugsum okkur nú að $\alpha$ sé núllstöð $P$ af stigi $k$.  Þá er
unnt að þátta margliðuna $P$ í
$P(\lambda)=(\lambda-\alpha)^kQ(\lambda)$, þar sem $Q(\lambda)$ er
margliða af stigi $m-k$ og $Q(\alpha)\neq 0$.  Samkvæmt jöfnunni hér
að framan  er
$$
P(D)(t^ke^{\alpha t}) = Q(D)(D-\alpha)^k(t^ke^{\alpha t})=
Q(D)(k!e^{\alpha t})=k!Q(\alpha)e^{\alpha t}.
$$
Þetta gefur okkur að
\begin{equation*}
P(D)u_p=e\sp{\alpha t} \qquad \text{ ef } \qquad
u_p(t) = \dfrac{t^ke^{\alpha t}}{k!Q(\alpha)}.


.. _2.4.13:

\end{equation*}
Nú skulum við gera ráð fyrir því að
$i\alpha$ sé núllstöð $P$ af stigi $k$ og að $-i\alpha$ sé núllstöð
$P$ af stigi $l$.  Þá getum við þáttað $P$ á tvo mismunandi vegu í
$$
P(\lambda)= (\lambda-i\alpha)^kQ(\lambda), \qquad
P(\lambda)= (\lambda+i\alpha)^lR(\lambda),
$$
þar sem $Q$ og $R$ eru margliður af stigi $m-k$ og $m-l$, 
$Q(i\alpha)\neq 0$ og $R(-i\alpha)\neq 0$.   Þetta gefur að
 \begin{equation*}
P(D)u_p=\cos \alpha t \qquad\text{ ef } \qquad
u_p(t)=\dfrac{t^ke^{i\alpha t}}{2(k!)Q(i\alpha)}+
\dfrac{t^le^{-i\alpha t}}{2(l!)R(-i\alpha)},


.. _2.4.14:

 \end{equation*}
og
 \begin{equation*}
P(D)u_p=\sin \alpha t \qquad \text{ ef } \qquad
u_p(t)=\dfrac{t^ke^{i\alpha t}}{2i(k!)Q(i\alpha)}-
\dfrac{t^le^{-i\alpha t}}{2i(l!)R(-i\alpha)}.


.. _2.4.15:

 \end{equation*}

\section{Green-föll :hover:`Green-fall`}

\subsection{Green-föll :hover:`Green-fall`}

\noindent
Í síðustu grein skoðuðum við nokkrar einfaldar aðferðir til að finna
sérlausnir á línulegum jöfnum með fastastuðla, þar sem hægri hlið
jöfnunnar $f(t)$ er veldisvísisfall eða eitthvert skylt fall.  Núna
ætlum við að kynna okkur almenna aðferð til þess að 
finna sérlausn á
 \begin{equation*}P(t,D)u=(a_m(t)D^m+\cdots+a_1(t)D+a_0(t))u=f(t), \qquad
t\in I,


.. _2.5.1:

 \end{equation*}
þar sem $I$ er eitthvert bil á rauntalnaásnum, föllin $a_0,
\dots,a_m,f$ eru í $C(I)$ og $a_m(t)\neq 0$ fyrir öll $t\in I$.  

Ef $\tau\in I$ er einhver ótiltekinn punktur, þá segir fylgisetning \tilv 
6.7.7 að til sé ótvírætt ákvörðuð lausn í $C^m(I)$ á
upphafs\-gildis\-verk\-efninu 
 $$P(t,D_t)u=0, \qquad
u(\tau)=u\dash(\tau)=\cdots=u\sp{(m-2)}(\tau)=0, \quad 
u\sp{(m-1)}(\tau)=1/a_m({\tau}). 
 $$
Við táknum hana með $G(t,\tau)$.  Þar með ákvarðast fallið $G$ af
skilyrðunum 
 
\begin{gather*}
P(t,D_t)G(t,\tau)=0,  \qquad t,\tau\in I,

.. _2.5.2:
\\
G(\tau,\tau)=\partial_tG(\tau,\tau)=\cdots=

.. _2.5.3:

\partial_t\sp{(m-2)}G(\tau,\tau)=0, \quad
\partial_t\sp{(m-1)}G(\tau,\tau)=1/a_m({\tau}). 
\end{gather*}
 
Nú tökum við $a\in I$ og sýnum fram á
að fallið 
\begin{equation*}
u_p(t) = \int_a\sp t G(t,\tau)f(\tau) \, d\tau, \qquad t\in I,

.. _2.5.4:

\end{equation*}
uppfylli jöfnuna $P(t,D)u=f(t)$, $t\in I$.
Til þess að ráða við þetta þurfum við að vita að fallið $G(t,\tau)$
sé heildanlegt með tilliti til $\tau$ og jafnframt hvernig á að deilda
fall sem gefið er með svona formúlu:


.. _hs2.5.1:

\subsubsection{Hjálparsetning}
Ef $I$ er bil á raunásnum, $a\in I$, $f\in C(I)$ og $g\in
C(I\times I)$, er samfellt deildanlegt fall af fyrri breytistærðinni,
þ.e.~${\partial}_tg\in C(I\times I)$,  þá
er fallið $h$, sem gefið er með formúlunni  
 $$h(t)=\int_a\sp t g(t, \tau)f(\tau) \, d\tau, \qquad t\in I,
 $$
í $C\sp 1(I)$ og afleiða þess er 
 $$h\dash(t)=g(t,t)f(t)+\int_a\sp t \partial_tg(t,\tau)f(\tau) \, d\tau,
\qquad t\in I.
 $$


--------------



Nú skulum við ganga út frá því að $\partial_t^{j}G\in C(I\times I)$ 
fyrir $j=0,\dots,m$ og líta
aftur á fallið $u_p$ sem skilgreint var með (:ref:`2.5.4`). 
Með því að beita hjálparsetningu :ref:`hs2.5.1`, fáum við að $u_p\in C\sp
1(I)$  og
 $$u_p\dash(t) = G(t,t)f(t)+\int_a\sp t \partial_t G(t,\tau)f(\tau) \, d\tau.
 $$
Nú er $G(t,t)=0$ fyrir öll $t\in I$ samkvæmt fyrsta upphafsskilyrðinu
á $G$, svo við fáum að $u_p\in C\sp 2(I)$ og
 $$u_p\ddash(t) = \partial_tG(t,t)f(t)
+\int_a\sp t \partial_t^2G(t,\tau)f(\tau) \, d\tau.
 $$
Ef $m > 2$ þá er $\partial_tG(t,t)=0$ fyrir öll $t\in I$
og við getum haldið áfram að deilda fallið $u_p$, þar til við fáum að
$u_p\in C\sp m(I)$ og  
 $$
u_p\sp{(m)}(t) = \partial_t^{m-1}G(t,t)f(t)+\int_a\sp t
\partial_t^mG(t,\tau)f(\tau) \, d\tau. 
 $$
Nú er $\partial_t\sp{m-1}G(t,t)=1/a_m(t)$ fyrir öll $t\in I$.  
Við tökum saman liði og fáum
\begin{align*}
P(t,D_t)u_p(t)&=
a_m(t)f(t)/a_m(t) +\sum\limits_{j=0}\sp m
a_j(t)\int_a\sp t \partial_t^jG(t,\tau)f(\tau)\, d\tau=\\
&=f(t)+\int_a\sp t P(t,D_t)G(t,\tau)f(\tau)\, d\tau=f(t),
\end{align*}
því $P(t,D_t)G(t,\tau)=0$ fyrir öll $\tau\in I$. Á jöfnunum fyrir
afleiður $u_p$ sjáum við að  
$$u_p(a)=u_p\dash(a)=\cdots=u_p\sp{(m-1)}(a)=0.$$
 Við getum því tekið saman útreikninga okkar:

\subsubsection{Setning}
Látum $I$ vera bil á rauntöluásnum,  $a\in I$ og $P(t,D)$ vera
línu\-legan afleiðuvirkja á forminu (:ref:`2.5.3`) með samfellda stuðla
og $a_m(t)\neq 0$ fyrir öll $t\in I$. 
Fyrir sérhvert $f\in C(I)$ er til ótvírætt ákvörðuð lausn $u_p\in
C\sp m(I)$ á
upphafsgildisverkefninu 
 \begin{equation*}P(t,D)u=f(t), \qquad u(a)=u\dash(a)=\cdots=u\sp{(m-1)}(a)=0,

.. _2.5.5:

 \end{equation*}
og er hún gefin með formúlunni

.. _2.5.6:

\begin{equation*}u_p(t) = \int_a\sp t G(t,\tau)f(\tau) \, d\tau, \qquad t\in I,
\end{equation*}
þar sem $G$, er lausnin á  upphafsgildisverkefninu
 \begin{gather*}
P(t,D_t)G(t,\tau)=0,  \qquad t,\tau\in I,

.. _2.5.7:
\\
G(\tau,\tau)=\partial_tG(\tau,\tau)=\cdots=

.. _2.5.8:

\partial_t\sp{(m-2)}G(\tau,\tau)=0, \quad
\partial_t\sp{(m-1)}G(\tau,\tau)=1/a_m({\tau}). 
\end{gather*}
 Fallið $G(t,\tau)$ er $m$-sinnum samfellt deildanlegt fall af $t$
fyrir sérhvert $\tau\in I$ og 
$\partial_t^jG\in C(I\times I)$ fyrir $j=0,\dots,m$. 


--------------





\subsubsection{Skilgreining} Fallið $G(t,\tau)$ í síðustu setningu kallast {\it
Green-fall :hover:`Green-fall` :hover:`Green-fall!fyrir
upphafsgildisverkefni`}
virkjans $P(t,D)$.
 Við tölum einnig um {\it fall Greens}.


--------------



Mjög auðvelt er að ákvarða Green-fallið fyrir línulegan 
afleiðuvirkja með fasta\-stuðla:

\subsubsection{Fylgisetning}  Gerum ráð fyrir að 
$P(D)=a_mD\sp m+\cdots+a_1D+a_0$ sé línulegur afleiðuvirki með
fastastuðla.  Látum $g\in C\sp{\infty}(\R)$ vera fallið sem uppfyllir
 \begin{equation*}P(D)g=0, \quad g(0)=g\dash(0)=\cdots=g^{(m-2)}(0)=0, \quad
g^{(m-1)}(0)=1/a_m.

.. _2.5.9:

 \end{equation*}
Þá er $G(t,\tau)=g(t-\tau)$  Green-fall virkjans $P(D)$.


--------------




\section{Wronski-fylkið :hover:`Wronski-fylki` og
Wronski-ákveðan :hover:`Wronski-ákveða`}

\subsection{Wronski-fylkið :hover:`Wronski-fylki` og
Wronski-ákveðan :hover:`Wronski-ákveða`}

\noindent
Nú skulum við láta $G(t,\tau)$ tákna Green-fallið sem lýst er í
setningu 7.5.2 og jafnframt gera ráð fyrir því að $u_1,\dots, u_m$ sé
grunnur í ${\cal N}(P(t,D))$.  Fyrst $G(t,\tau)$ er lausn á óhliðruðu
jöfnunni   $P(t,D_t)G(t,\tau)=0$
fyrir sérhvert  $\tau\in I$, þá getum við skrifað
$G(t,\tau)$ sem línulega samantekt af grunnföllunum með stuðlum sem
eru háðir $\tau$,
$$
G(t,\tau)=c_1(\tau)u_1(t)+\cdots+c_m(\tau)u_m(t), \qquad t,\tau\in I.
$$
Stuðlaföllin $c_1,\dots,c_m$ ákvarðast síðan af upphafsskilyrðunum,
\begin{align*}
G(\tau,\tau) &= c_1(\tau)u_1(\tau)+\cdots+c_m(\tau)u_m(\tau)=0,\\
\partial_tG(\tau,\tau) &= c_1(\tau)u_1\dash(\tau)+
\cdots+c_m(\tau)u_m\dash(\tau)=0,\\
&\qquad\vdots\qquad\qquad\vdots\qquad\qquad\vdots\\
\partial_t\sp{m-2}G(\tau,\tau) &= c_1(\tau)u_1\sp{(m-2)}(\tau)+
\cdots+c_m(\tau)u_m\sp{(m-2)}(\tau)=0,\\
\partial_t\sp{m-1}G(\tau,\tau) &= c_1(\tau)u_1\sp{(m-1)}(\tau)+
\cdots+c_m(\tau)u_m\sp{(m-1)}(\tau)=1/a_m({\tau}).
\end{align*}
Á fylkjaformi verður þetta jöfnuhneppi
 \begin{equation*}V(\tau)c(\tau)=a_m({\tau})^{-1}e_m,

.. _2.6.1:

 \end{equation*}
þar sem $V\in C(I,\C\sp{m\times m})$ er fylkjafallið
 \begin{equation*}V(\tau)=V(u_1,\dots,u_m)(\tau)=
\left[\begin{matrix}
u_1(\tau)&\dots&u_m(\tau)\\
u_1\dash(\tau)&\dots&u_m\dash(\tau)\\
\vdots&\ddots&\vdots\\
u_1\sp{(m-1)}(\tau)&\dots&u_m\sp{(m-1)}(\tau)
\end{matrix}\right]


.. _2.6.2:

 \end{equation*}
en $c(\tau)=[c_1(\tau),\dots,c_m(\tau)]^t$ og $e_m=[0,\dots,0,1]^t$.

\subsubsection{Skilgreining}
Látum $I$ vera bil á $\R$ og $u_1,\dots,u_m$ vera $m-1$ sinnum
deildanleg föll á $I$.  Þá nefnist fylkjagilda fallið
$V=V(u_1,\dots,u_m)$, sem skilgreint er með (:ref:`2.6.2`),  {\it Wronski-fylki}
fallanna $u_1,\dots, u_m$.  Ákveða þess kallast {\it Wronski-ákveða}
fallanna $u_1,\dots, u_m$ og hana táknum við með 
$W=W(u_1,\dots,u_m)$.


--------------



Ef við þekkjum Wronski-ákveðuna af $m$ lausnum á afleiðujöfnu í einum
punkti, þá getum við reiknað hana út með því að leysa fyrsta stigs
afleiðujöfnu: 


.. _set2.6.2:

\subsubsection{Setning}
Látum
$P(t,D)=a_m(t)D\sp m+\cdots+a_1(t)D+a_0(t)$ vera afleiðuvirkja með
samfellda stuðla, $u_1,\dots,u_m$ vera lausnir á óhliðruðu jöfnunni
$P(t,D)u=0$ og táknum  Wronski-ákveðu þeirra með $W(t)$.  Þá
uppfyllir $W$ fyrsta stigs afleiðujöfnuna

.. _2.6.3:

\begin{equation*}a_m(t) W\dash+a_{m-1}(t)W=0 
\end{equation*}
og þar með gildir formúlan
 \begin{equation*}W(t)=W(a)\exp\bigg(-\int_a\sp t\dfrac{a_{m-1}(\tau)}{a_m(\tau)}\,
d\tau\bigg) 

.. _2.6.4:

 \end{equation*}
fyrir öll $a$ og $t$ á bili $J$ þar sem $a_m$ er núllstöðvalaust.


--------------



Sönnunin er tekin fyrir í grein \tilv 7.7.
Formúluna fyrir Wronski-ákveðuna má nota á ýmsa vegu:

\subsubsection{Setning}
  Látum $u_1,\dots,u_m$ vera lausnir á óhliðruðu jöfnunni
$P(t,D)u=0$, þar sem $P(t,D)$ er sami virkinn og í setningu
:ref:`set2.6.2`, og gerum ráð fyrir að $a_m$ sé núllstöðvalaust á opnu bili
$J\subset I$.  Þá eru eftirfarandi skilyrði jafngild:

\item{(i)}  Föllin $u_1,\dots,u_m$ eru línulega óháð á bilinu $J$.

\item{(ii)} $W(u_1,\dots,u_m)(t)\neq 0$ fyrir sérhvert $t\in J$.

\item{(iii)} $W(u_1,\dots,u_m)(a)\neq 0$ fyrir eitthvert $a\in J$.

\item{(iv)}  Dálkvigrarnir í Wronski-fylkinu $V(u_1,\dots,u_m)(t)$  eru
línulega óháðir fyrir sérhvert $t\in J$.

\item{(v)}  Dálkvigrarnir í Wronski-fylkinu $V(u_1,\dots,u_m)(a)$  eru
línulega óháðir fyrir eitthvert $a\in J$.


--------------



Nú skulum við rifja það upp að $n\times n$ fylki $A$ hefur andhverfu
þá og því aðeins að $\det A\neq 0$.  Andhverfuna er hægt að reikna út
á ýmsa vegu, en til er formúla fyrir henni,
 \begin{equation*}A\sp{[-1]}=\dfrac 1{\det A}B\sp t,

.. _2.6.7:

 \end{equation*}
þar sem $B=(b_{jk})_{j,k=1}\sp n$ táknar fylgiþáttafylki $A$, sem 
er $n\times n$ fylkið með stökin
 \begin{equation*}b_{jk}=(-1)\sp{j+k}\det A_{jk},

.. _2.6.8:

 \end{equation*}
þar sem $A_{jk}$ er $(n-1)\times (n-1)$ fylkið, sem fæst með því að
fella niður línu númer $j$ og dálk númer $k$ í fylkinu $A$, og $B\sp
t$ er fylkið $B$ bylt, þar sem víxlað er á línum og dálkum í $B$.
Við höfum nú bætt miklu við þekkingu okkar á Green-föllum:



.. _set2.6.4:

\subsubsection{Setning}
Látum $I$ vera bil á $\R$ og $P(t,D)=a_m(t)D\sp m+\cdots+a_1(t)D+a_0(t)$
vera afleiðuvirkja með
samfellda stuðla á $I$ og $u_1,\dots,u_m$ vera grunn í ${\cal N}(P(t,D))$. 
Green-fallið sem lýst er í setningu 7.5.2
er gefið með formúlunni
 \begin{equation*}G(t,\tau)=c_1(\tau)u_1(t)+\cdots+c_m(\tau)u_m(t), \qquad t,\tau\in I,


.. _2.6.9:

 \end{equation*}
þar sem vigurinn $a_m({\tau})(c_1(\tau),\dots,c_m(\tau))$ myndar aftasta dálkinn
í andhverfu Wronski-fylkisins $V(u_1,\dots,u_m)(\tau)$,
 \begin{equation*}c_j(\tau)=(-1)^{m+j} \dfrac{\det V_{mj}(u_1,\dots,u_m)(\tau)}
{a_m({\tau})W(u_1,\dots, u_m)(\tau)},


.. _2.6.10:

 \end{equation*}
þar sem $V_{mj}(u_1,\dots,u_m)(\tau)$ táknar $(m-1)\times (m-1)$
fylkið sem fæst með því að fella niður neðstu línuna og dálk númer
$j$ í $V(u_1,\dots,u_m)(\tau)$.    Ef $f\in
C(I)$, þá hefur upphafsgildisverkefnið (7.5.5) lausnina $u_p\in C\sp
m(I)$  sem
gefin er með
 \begin{equation*}u_p(t)=v_1(t)u_1(t)+\cdots+v_m(t)u_m(t), \qquad t\in I,

.. _2.6.11:

 \end{equation*}
þar sem stuðlaföllin $v_j$ eru gefin með formúlunni
 \begin{equation*}v_j(t)=\int_a\sp t c_j(\tau)f(\tau) \, d\tau.

.. _2.6.12:

 \end{equation*}


--------------




Við fáum nú beina formúlu fyrir Green-falli annars stigs virkja:

\subsubsection{Fylgisetning} Látum $P(t,D)=a_2(t)D^2+a_1(t)D+a_0(t)$ vera annars stigs
afleiðuvirkja á bilinu $I$ með samfellda stuðla og $a_2(t)\neq 0$ fyrir
öll $t\in I$. Gerum nú ráð fyrir að $u_1$ og $u_2$ séu línulega óháðar 
lausnir á óhliðruðu jöfnunni $P(t,D)u=0$.  Þá er 
\begin{equation*}
G(t,\tau) 
=a_2(\tau)^{-1}
\left|\begin{matrix}
u_1(\tau) & u_1(t)\\
u_2(\tau) & u_2(t)
\end{matrix}\right|\bigg /
\left|\begin{matrix}
u_1(\tau) & u_2({\tau})\\
u_1\dash(\tau) & u_2\dash({\tau})
\end{matrix}\right|.


.. _2.6.13:

\end{equation*}


--------------


