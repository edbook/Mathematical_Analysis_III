\chapter{ÞÝÐ FÖLL OG FÁGAÐAR VARPANIR}


\section{Þýð föll}


\subsection*{Laplace-virki, Laplace-jafna og þýð föll }

Látum nú $X$ tákna opið mengi í $\C=\R^2$ og látum $\varphi:X\to \R$
vera deildanlegt fall á $X$. Munum að {\it stigull} fallsins
$\varphi$  er vigursviðið 
$$
\nabla \varphi=\big(\dfrac{\partial \varphi}{\partial x},
\dfrac{\partial \varphi}{\partial y}\big).
$$
Munum einnig að fyrir deildanlegt vigursvið  $\vec V=(p,q):X\to \R^ 2$ er 
{\it sundurleitni} þess skilgreind sem fallið
$$
\nabla\cdot \vec V=\dfrac{\partial p}{\partial x}+\dfrac{\partial
q}{\partial y}.
$$
Ef við tengjum saman stigul og sundurleitni, þá fáum við  
$$
\nabla^2\varphi=\nabla\cdot (\nabla \varphi)= \dfrac {\partial^2 \varphi}{\partial x^2}+  
\dfrac {\partial^2 \varphi}{\partial y^2}.  
$$

\begin{sk} Látum  $\varphi:X\to \R$ vera tvisvar deildanlegt fall 
á opnu hlutmengi $X$ í $\C$.  Hlutafleiðuvirkinn
$$
{\Delta}=\nabla^2=\dfrac {\partial^2 }{\partial x^2}+  
\dfrac {\partial^2 }{\partial y^2}
$$
nefnist {\it Laplace-virki\index{Laplace!virki}\index{virki!Laplace}},
óhliðraða hlutafleiðujafnan
${\Delta}\varphi=0$ nefnist {\it
Laplace-jafna\index{Laplace!jafna}\index{jafna!Laplace}}
og lausn $\varphi:X\to \R$  á  henni er sögð vera {\it þýtt fall\index{þýtt
fall}} á $X$.
\end{sk}

\subsection*{Wirtinger-afleiðuvirkjarnir}

Rifjum nú upp skilgreininguna á Wirtinger-afleiðuvirkjunum:
$$
\dfrac{\partial}{\partial z}=\dfrac 12\bigg(\dfrac{\partial }{\partial x}-i
\dfrac{\partial}{\partial y}\bigg)
\qquad \text{ og } \qquad 
\dfrac{\partial}{\partial \bar z}=\dfrac 12\bigg(\dfrac{\partial
}{\partial x}+i \dfrac{\partial}{\partial y}\bigg)
$$
Með smá útreikningi sjáum við að
$$
\Delta u=4\dfrac{\partial^2 u}{\partial z\partial \bar z}
=4\dfrac{\partial^2 u}{\partial \bar z\partial z}
$$
og þar með er fallið $u$ er þýtt þá og því aðeins að
$\partial^2 u/\partial z\partial\bar z =0$.  Munum einnig að fall $f$
er fágað þá og því aðeins að $\partial f/\partial \bar z=0$.


\subsection*{Tengsl við fáguð föll}

Látum $f: X\to \C$, $f=u+iv$ vera fágað fall þar sem $u=\Re f$
og $v=\Im f$ tákna raun- og þverhluta.  Þá eru bæði $u$ og $v$
óendanlega oft deildanleg föll og þau uppfylla Cauchy-Riemann 
jöfnurnar
$$
\dfrac{\partial u}{\partial x}
=\dfrac{\partial v}{\partial y} \qquad \text{ og } \qquad
\dfrac{\partial u}{\partial y}
=-\dfrac{\partial v}{\partial x}.
$$
Við getum nú skrifað Cauchy-Riemann jöfnurnar sem
$$
\nabla u=\big(\dfrac{\partial v}{\partial y},-\dfrac{\partial
v}{\partial x}\big) \qquad \text{ og } \qquad
\nabla v=\big(-\dfrac{\partial u}{\partial y},\dfrac{\partial
u}{\partial x}\big). 
$$
Af þessu leiðir að 
$$
\nabla u\cdot \nabla v=0,
$$
sem segir okkur að stiglar $u$ og $v$ eru hornréttir.

Munum að raungilt á fall á svæði $X$ er fastafall þá og því aðeins að
stigull þess sé núll í sérhverjum punkti.  Cauchy-Riemann jöfnurnar
segja okkur að $u$ sé fastafall þá og því aðeins að $v$ sé fastafall.

Af Cauchy-Riemann jöfnunum leiðir einnig
$$
\dfrac {\partial^2 u}{\partial x^2}+  
\dfrac {\partial^2 u}{\partial y^2}
=\dfrac{\partial^2 v}{\partial x\partial y}  
-\dfrac{\partial^2 v}{\partial y\partial x}=0,  
$$
af því að $v$ er óendanlega oft deildanlegt og
$\partial^2 v/\partial x\partial y=\partial^2 v/\partial y\partial x$,
og einnig fæst að
$$
\dfrac {\partial^2 v}{\partial x^2}+  
\dfrac {\partial^2 v}{\partial y^2}
=-\dfrac{\partial^2 u}{\partial x\partial y}  
+\dfrac{\partial^2 u}{\partial y\partial x}=0.  
$$
Við höfum því sannað:

\begin{sex}  Ef $f$ er fágað fall á opnu mengi $X$ í $\C$, þá eru 
$u=\Re f$ og $v=\Im f$ þýð föll og stiglar þeirra eru hornréttir í
sérhverjum punkti í $X$.
Ef $X$ er svæði og annað hvort $u$ eða $v$ er fastafall, þá er hitt
fallið það líka.
\end{sex}


Gerum nú aftur ráð fyrir að $u$ sé þýtt fall á svæði $X$ í $\C$
og athugum hvernig hægt er að finna $v$ þannig að $u+iv$ verði fágað
fall.  Gerum ráð fyrir að slíkt $v$ sé til og setjum $f=u+iv$.  Þá
er 
$$
f'(z)=\dfrac{\partial u}{\partial x}+i\dfrac{\partial v}{\partial x}
$$
og fyrri Cauchy-Riemann-jafnan gefur að 
$$
f'(z)=\dfrac{\partial u}{\partial x}-i\dfrac{\partial u}{\partial y}
=2\dfrac{\partial u}{\partial z}.
$$ 
Það er því nauðsynlegt skilyrði að afleiðan af $f$ sé gefin með
þessari formúlu. Athugum að fallið sem stendur í hægri hliðinni
uppfyllir Cauchy-Riemann-jöfnurnar og er þar með fágað, 
því ef við látum virkjann
$\partial/\partial \bar z$ verka á hægri hliðina þá fáum við
$\partial^2u/\partial\bar z \partial z=0$.


Nú sjáum við að sérhvert þýtt fall á $X$ er raunhluti af fáguðu falli
þá og því aðeins að sérhvert fágað fall á $X$ hafi stofnfall. Í
3. kafla  sáum við að  þetta einkennir einfaldlega samanhangandi
svæði:

\begin{se}  Látum $X$ vera svæði í $\C$.  Þá er sérhvert þýtt fall á
$X$ raunhluti af fáguðu falli þá og því aðeins að $X$ sé einfaldlega
samanhangandi.   Ef $a\in X$ er fastur punktur þá er fallið
$f$ gefið með formúlunni
$$
f(z)=u(a)+ic+2\int_{\gamma_z} \dfrac{\partial u}{\partial
\zeta}(\zeta) \, d\zeta,
$$
þar sem $\gamma_z$ er einhver vegur í $X$ með upphafspunkt $a$ og
lokapunkt $z$ og $c\in \R$ er fasti.
\end{se}


Athugið að veginn í setningunni má velja sem línustrik, ef 
$X$ er stjörnusvæði með tilliti til $a$.

\medskip
Gerum nú ráð fyrir að $u$ sé þýtt fall á svæði $Y$ og að 
$g:X\to \C$ sé fágað fall á svæði $X\subset \C$ þannig að 
$g(X)\subset Y$.  Ef $a\in  X$ þá er til opin skífa með miðju í 
$g(a)$ í $Y$ þannig að $u$ er raunhluti fágaðs falls á 
$f$ á skífunni.  Þá verður samskeytingin $u\circ g$ raunhluti 
$f\circ g$ sem er fágað fall í grennd um $a$.  Þetta segir okkur að
samskeyting af þýðu falli við fágað fall er þýtt fall.
 

\section{Hagnýtingar í straumfræði}  

\noindent
Látum nú $\vec V$ vera vigursvið á opnu mengi $X$ í $\R^2$.  Við ætlum
að líta á $\vec V$ sem hraðasvið\index{hraðasvið}, sem er háð tveimur breytistærðum
$$
\vec V(x,y)= (p(x,y), q(x,y)), \qquad (x,y)\in X.
$$
{\it Straumlína\index{straumlína}} vigursviðsins $\vec V$ er
ferill\index{ferill}\index{ferill!einfaldur}\index{ferill!lokaður}
í $X$ sem stikaður er með lausn $\vec z:I\to \R^2$ á
\begin{equation*}
\vec z\, \dash(t)=\vec V(\vec z(t)), \qquad t\in I,
\label{4.7.1}
\end{equation*}
á einhverju bili $I$ á $\R$.  Þessi jafna jafngildir afleiðujöfnuhneppinu
\begin{equation*}
x\dash=p(x,y), \qquad y\dash=q(x,y).
\label{4.7.2}
\end{equation*}
Vigursviðið getur átt sér eðlisfræðilega túlkun.  Við getum til dæmis
litið á $\vec V$ sem hraðasvið fyrir streymi vökva eða lofts.  Gengið
er út frá því að streymið sé óháð tíma og einni rúmbreytistærð og að
það sé samsíða einhverju plani, sem við höfum valið sem $(x,y)$-plan. 
Straumlínurnar eru þá brautir agnanna í vökvanum eða loftinu.  $\vec V$
getur einnig verið hraðasvið rafstraums  í þunnri plötu og þá er
$\vec V$ samsíða straumsviðinu í sérhverjum punkti.



Hugsum okkur nú að ${\Omega}$ sé hlutsvæði í $X$ með jaðar
${\partial} {\Omega}$ í $X$ og gerum ráð fyrir að hægt sé að stika
${\partial}{\Omega}$ með einföldum lokuðum ferli ${\gamma}$, sem er samfellt
deildanlegur á köflum og ${\gamma}$ stikar ${\partial}{\Omega}$ í
jákvæða stefnu, en það þýðir að svæðið ${\Omega}$ er vinstra megin við
snertilínuna í ${\gamma}(t)$, ef horft er í stefnu snertilsins
${\gamma}\dash(t)$. Ef $(x,y)={\gamma}(t)\in {\partial}{\Omega}$ er
punktur, þar sem ${\gamma}$ er deildanlegt fall, þá
skilgreinum við {\it einingarsnertil\index{einingarsnertill}}
$\vec T(x,y)$ í $(x,y)$, sem einingarvigurinn í stefnu
${\gamma}\dash(t)$, $\vec T(x,y)={\gamma}\dash(t)/|{\gamma}'(t)|$,
og {\it ytri
einingarþvervigur\index{einingarþverhringur}\index{ytri
einingarþverhringur} á} ${\partial}{\Omega}$ sem
einingarvigurinn $\vec n(x,y)$ sem er hornréttur á
${\gamma}\dash(t)$ og vísar út úr ${\Omega}$. Við
látum $ds$ tákna {\it bogalengdarfrymið\index{bogalengdarfrymi}}.
Með ${\gamma}$ sem stikun á ${\partial}{\Omega}$ er það gefið sem
$ds=|{\gamma}\dash(t)|\, dt$.



\figura {fig0328}{{\small Mynd: Jaðar á svæði, snertill og þvervigur}}

\noindent
Gauss-setningin\index{Gauss-setningin}\index{setning!Gauss} gefur nú
\begin{align*}
\int_{\partial\Omega}(\vec V\cdot\vec n)\, ds
&=\int_{{\gamma}}(\vec V\cdot\vec n)\, ds
=\iint\limits_{{\Omega}} \div \vec V\, dxdy\\
&=\iint\limits_{{\Omega}}
\big({\partial}_xp(x,y)+{\partial}_yq(x,y)\big)\, dxdy.\nonumber
\end{align*}
Heildið í vinstri hliðinne nefnist {\it flæði\index{flæði}
vigursviðsins} $\vec
V$ yfir jaðarinn ${\partial}{\Omega}$.
Green-setningin\index{Green-setningin}\index{setning!Green} gefur
\begin{align*}
\int_{\partial\Omega}(\vec V\cdot\vec T)\, ds
&=\int_{{\gamma}}(\vec V\cdot\vec T)\, ds
=\iint\limits_{{\Omega}} \rot \vec V\, dxdy\\
&=\iint\limits_{{\Omega}}
\big({\partial}_xq(x,y)-{\partial}_yp(x,y)\big)\, dxdy.\nonumber
\end{align*}
Heildið í vinstri hliðinni  nefnist 
{\it hringstreymi\index{hringstreymi}} vigursviðsins $\vec V$ eftir jaðrinum
${\partial}{\Omega}$.  
Við gefum okkur nú tvær forsendur um hraðasviðið $\vec V$:

\smallskip\noindent
(i) {\it Streymið er geymið}:  Fyrir sérhvert ${\Omega}\subset X$ er
flæðið yfir ${\partial}{\Omega}$ jafnt $0$.  Þetta hefur
í för með sér að
\begin{equation*}
\dfrac{\partial p}{\partial x}(x,y)+
\dfrac{\partial q}{\partial y}(x,y)=0, \qquad (x,y)\in X.
\label{4.7.5}
\end{equation*}
Þessi jafna er oft nefnd {\it
samfelldnijafna\index{samfelldnijafna}}. 
Þetta er lögmálið um varðveislu massans, ef $\vec V$ er hraðasvið fyrir
vökvastreymi, en lögmálið um varðveislu hleðslunnar, ef $\vec V$ er
hraðasvið rafstraums.  

\smallskip\noindent
(ii) {\it Streymið er án hvirfla}:  Fyrir sérhvert ${\Omega}$ er
hringstreymi $\vec V$ eftir jaðrinum ${\partial}{\Omega}$ jafnt $0$.  
Þetta  hefur í för með sér að 
\begin{equation*}
\dfrac{\partial q}{\partial x}(x,y)-
\dfrac{\partial p}{\partial y}(x,y)=0, \qquad (x,y)\in X.
\label{4.7.6}
\end{equation*}
Ein mikilvæg afleiðing þessa skilyrðis er að í streyminu geta ekki verið
{\it hvirflar\index{hvirfill}}, en það eru lokaðar straumlínur, sem
mynda jaðar á svæði ${\Omega}\subset X$.  Hugsum okkur að
$\vec z:[a,b]\to \R^2$ væri slík straumlína. Þá er
$\vec T(\vec z(t))=\pm z\dash(t)/|z\dash(t)|$, 
$\vec V(\vec z(t))= z\dash(t)$, $ds=|z\dash(t)|\, dt$ og þar með
$$
\int_{{\partial}{\Omega}} \vec V\cdot \vec T\, ds =
\pm\int_a^b |z\dash(t)|^2\, dt \neq 0.
$$

\smallskip 
Nú skulum við skrifa $\vec V$ sem tvinnfall,
$V(z)=p(z)+iq(z)$.  Hlutafleiðujöfnurnar hér að framan
segja að $\overline  V=p-iq$
uppfylli Cauchy-Riemann-jöfnurnar og þar með er
fallið  $\overline V$ fágað.  Hugsum okkur að $\overline V$
hafi stofnfall\index{stofnfall}, sem við táknum með $f$. 
Ef ${\varphi}=\Re f$ og
${\psi}=\Im f$, þá leiðir af Cauchy-Riemann-jöfnunum að 
 $$
f\dash(z)=\partial_x\varphi(z)+i\partial_x\psi(z)
=\partial_x\varphi(z)-i\partial_y\varphi(z)
=p(z)-iq(z).
 $$
Við höfum því $\grad\varphi=\vec V=(p,q)$, svo straumlínurnar eru
hornréttar á jafnhæðarlínurnar\index{jafnhæðarlína}
$\set{z; \varphi(z)=c}$, þar sem $c$ er fasti.  Nú gefa
Cauchy-Riemann-jöfnurnar hins vegar að $\grad \psi=(\partial_x\psi,
\partial_y\psi)$ er hornréttur á $\grad\varphi=(\partial_x\varphi,
\partial_y\varphi)$ og þar með eru staumlínurnar fyrir vigursviðið
$\vec V$ gefnar sem jafnhæðarlínurnar $\set{z; \psi(z)=c}$, þar sem
$c$ fasti.

Fallið $f$ kallast {\it tvinnmætti\index{tvinnmætti}} fyrir straumfallið $V$, fallið
$\varphi$ kallast {\it raunmætti\index{raunmætti}} fyrir  $V$ og fallið $\psi$
kallast {\it streymisfall\index{streymisfall}}.  Niðurstaða athugana okkar er því að
straumlínur vigursviðsins $\vec V$ eru jafnhæðarlínur
streymisfallsins $\psi$, þar sem $\psi= \Im f$ og $f\dash = \overline
V$.  Ef við þekkjum streymisfallið ${\psi}$ og getum ákvarðað
jafnhæðarlínur þess, þá höfum við ákvarðað brautir lausna
afleiðujöfnuhneppisins 
\begin{equation*}
x\dash=p(x,y), \qquad y\dash=q(x,y).
\end{equation*}
án þess að leysa jöfnurnar.

\begin{sy}  Lítum fyrst á hraðasviðið $V$ sem gefið er með 
 \begin{equation*}V(z)=\dfrac a{\overline z}= a\dfrac {e\sp{i\theta}}r, \qquad
z=re\sp{i\theta}, \quad z\in \C\setminus\set 0,
\label{4.7.7}
 \end{equation*}
þar sem $a\in \R$. Fallið $\overline V$ hefur ekkert stofnfall á öllu 
$\C\setminus \set
0$, en á menginu $X=\C\setminus \R_-$ getum við tekið 
 $$f(z)=a\Log z=a(\ln |z|+i\theta(z)),  \qquad -\pi<\theta(z)<\pi,
 $$
fyrir stofnfall, þar sem $\Log$ táknar höfuðgrein lografallsins.
Straumlínurnar verða þá jafnhæðarlínur fyrir hornið $\set{z;
\theta(z)=c}$, en þær eru geislar út frá $0$.  Heildarflæði
straumfallsins gegnum hring með geislann $r$ er
 $$\int_{|z|=r}\scalar{\vec V}{\vec n} \, ds=
\int_0\sp{2\pi}\dfrac ar \, rd\theta=2\pi a.
 $$
Ef $a>0$ þá stefna straumlínurnar út frá $0$ og 
þetta straumfall er til komið af {\it uppsprettu\index{uppspretta}} í
punktinum $0$ með styrkinn $2\pi a$. Ef $a<0$ þá er 
straumfallið  til komið af {\it svelg\index{svelgur}} í punktinum $0$
með styrkinn $2\pi a$.  
\end{sy}



\figura {fig0319}{{\small Mynd: Punktuppspretta}}



\begin{sy}  Lítum nú á fallið $V$ sem gefið er með 
 \begin{equation*}V(z)=\dfrac {ib}{\overline z}= ib\dfrac {e\sp{i\theta}}r, \qquad
z=re\sp{i\theta}, \quad z\in X=\C\setminus\set 0,
\label{4.7.8}
 \end{equation*}
þar sem $b\in \R$.
Hér er hraðavigurinn í stefnu $ie\sp{i\theta}$ og þar með hornréttur
á stöðuvigurinn.  {Á}  menginu $X=\C\setminus \R_-$ 
höfum við tvinnmættið  
 $$f(z)=-ib\Log z=b(\theta(z)-i\ln |z|).
 $$
Hér verða straumlínurnar $\set{z; \ln|z|=c}$ hringir  með miðju í
$0$.  Hringstreymi vigursviðsins $\vec V$ eftir hring með geisla $r$
er
 $$\int_{|z|=r}\scalar{\vec V}{\vec T} \, ds=
\int_0\sp{2\pi}\dfrac br \, rd\theta=2\pi b.
 $$
Þetta mætti er sagt lýsa {\it hringstreymi\index{hringstreymi}}
umhverfis {\it hvirfilpunkt} með styrk $2\pi b$ í $0$.

{}
\end{sy}


\figura {fig0320}{{\small Mynd: Hringstreymi}}


\begin{sy} Lítum á enn eitt afbrigðið, 
 \begin{equation*}V(z)=\dfrac {(a+ib)}{\overline z}= (a+ib)\dfrac {e\sp{i\theta}}r, \qquad
z=re\sp{i\theta}, \quad z\in \C\setminus\set 0,
\label{4.7.9}
 \end{equation*}
þar sem $a,b\in \R$.
Hér tvinnmætti á menginu $\C\setminus \R_-$ gefið með
 $$f(z)=(a-ib)\Log z=(a\ln |z| + b\theta(z))+i(a\theta(z)-b\ln |z|).
 $$
Straumlínurnar eru $\set{z;a\theta(z)-b\ln |z|=c}$.  Í pólhnitum
eru þær gefnar með jöfnunni $r=e\sp{(a\theta-c)/b}$, en þetta eru
{\it skrúflínur\index{skrúflína}}  eða {\it iðustreymi\index{iðustrymi}} 
út frá $0$.  Þetta mætti er myndað af straumuppsprettu með
styrkinn $2\pi a$ og  hvirfilpunkti með styrkinn $2\pi b$ í $0$.
\end{sy}



\figura {fig0321}{{\small Mynd: Iðustreymi}}


\begin{sy}  Lítum nú á dæmið þar sem tvær uppsprettur með styrk $2\pi a$ eru
í punktunum $\alpha$ og $-\alpha$ á raunásnum.  Straumfallið verður
þá
 $$V(z)= \dfrac a{\overline z+\alpha}+\dfrac a{\overline z-\alpha},
 $$
og sem tvinnmætti á $\C\setminus\set{x\in \R; x\leq \alpha}$ getum
við tekið 
\begin{align*}
f(z)&= a\Log(z+\alpha)+a\Log(z-\alpha) \\
&=
a(\ln|z+\alpha|+\ln|z-\alpha|)+ia(\theta(z+\alpha)+\theta(z-\alpha)).
\end{align*}

Við sjáum vð þverásinn er straumlína, því þar er
$\theta(iy+\alpha)+\theta(iy-\alpha)={\pi}$, ef $y>0$ og
$\theta(iy+\alpha)+\theta(iy-\alpha)=-{\pi}$, ef $y<0$.
Straumvigurinn er í stefnu þverássins, upp ef $y>0$ og niður
ef $y<0$, því
 $$V(z)=\dfrac{2a\overline z}{\overline z\sp 2-\alpha\sp 2},\qquad
V(iy)=\dfrac{2ayi}{y\sp 2+\alpha\sp 2}.
 $$
Við getum einnig notað þetta fall til þess að lýsa streymi út frá
uppsprettu í punktinum $\alpha$ af styrk $2\pi a$ í hálfplaninu
$\set{z\in \C; \Re z>0}$, þar sem litið er á þverásinn sem vegg.
\end{sy}


\figura {fig0322}{{\small Mynd: Straumuppspretta við vegg}}


\begin{sy}  Lítum nú á mættið sem til er komið vegna uppsprettu af styrk
$2\pi a$ í punktinum $\alpha$ og svelgs af styrk $2\pi a$ í punktinum
$-\alpha$.  Straumfallið verður
 $$V(z)=\dfrac {a}{\overline z-\alpha}-\dfrac a{\overline z+\alpha}.
 $$
Tvinnmættið á $\C\setminus\set{x\in \R; x\leq \alpha}$ getum við
valið sem 
 $$f(z)=a\Log(z-\alpha)-a\Log(z+\alpha)=
a\ln\bigg|\dfrac{z-\alpha}{z+\alpha}
\bigg | +ia\theta\bigg(\dfrac{z-\alpha}{z+\alpha}\bigg).
 $$
Talan $\theta((z-\alpha)/(z+\alpha))$ er hornið sem bilið
$[-\alpha,\alpha]$ sést undir miðað við punktinn $z$.
Við getum lýst straumlínu $\set{z\in \C; 
\theta((z-\alpha)/(z+\alpha))=c}$ fyrir þetta streymi,  sem mengi allra
punkta sem eru þannig að bilið $[-\alpha,\alpha]$ sést undir 
horninu $c$ frá $z$.  
Við sjáum að  
$$w=\dfrac{z-\alpha}{z+\alpha} \qquad \Leftrightarrow \qquad 
z=\dfrac {\alpha w+\alpha}{-w+1}=-\alpha\dfrac{w+1}{w-1}.
 $$
Straumlínurnar eru gefnar sem $\theta(w)=c$, sem eru hálflínur út frá
$0$ í $w$-planinu með stefnuvigur $e^{ic}$.  
Við sjáum að $w=0 \Leftrightarrow z=\alpha$ og
$w=\infty \Leftrightarrow z=-\alpha$. Straumlínurnar eru því
hringbogar frá $\alpha$ til $-\alpha$.
Jafnmættislínurnar eru síðan gefnar með jöfnum af gerðinni
$$
\bigg| \dfrac{z-{\alpha}}{z+{\alpha}} \bigg|^2=c,
$$
þar sem $c>0$.
Ef $c=1$, þá er þetta þverásinn, en fyrir $c\neq 1$ er þetta hringur.

{}
\end{sy}


\figura {fig0323}{{\small Mynd:  Straumuppspretta í $-\alpha$ og
    svelgur
í $+\alpha$}}

\begin{sy}  Lítum nú á fallið $f:X\to \C$, 
$$
f(z)=\arcsin z, \qquad
z\in X=\C\setminus\set{x\in \R; |x|\geq 1}.
$$
sem tvinnmætti.  Við skrifum $w=\arcsin z$, 
$z=x+iy$ og $w=u+iv$.  Þá er $-{\pi}/2<u<{\pi}/2$ og 
\begin{align*}
z&=x+iy=\sin w=\sin(u+iv)\\
&=\sin u\cos(iv)+\cos u\sin(iv)\\
&=\sin u\cosh v+i\cos u\sinh v.
\end{align*}
Straumlínurnar eru því gefnar sem 
${\psi}(z)=\Im \arcsin z=v=\text{fasti}$ 
og við sjáum að jöfnur þeirra í $z$-planinu eru
$$
\dfrac{x^2}{\cosh^2 v}+\dfrac{y^2}{\sinh^2 v}=
\sin^2u+\cos^2u=1.
$$
Þetta eru sporbaugar með hálfásana $a=\cosh v$ og $b=\sinh v$.
Jafnmættislínurnar eru hins vegar gefnar sem
${\varphi}(z)=\Re \arcsin z=u=\text{fasti}$ 
og  jöfnur þeirra í $z$-planinu eru
$$
\dfrac{x^2}{\sin^2 u}-\dfrac{y^2}{\cos^2 u}=
\cosh^2v-\sinh^2v=1.
$$
Þetta eru jöfnur fyrir breiðboga.


\figura {fig0326}{{\small Mynd:  Tvinnmættið $f(z)=\arcsin z$}} 


Ef við lítum á fallið $g(z)=-i\arcsin z$, þá skipta straumlínur og
jafnmættislínur um hlutverk og breiðbogarnir verða straumlínur.  Við
tökum eftir því að þverásinn er straumlína.  Við getum því túlkað þetta
sem mætti fyrir streymi gegnum hlið\index{streymi!gegnum hlið}.



\figura {fig0327}{{\small Mynd:  Streymi gegnum hlið}}
\end{sy}

