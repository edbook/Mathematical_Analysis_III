
%
%Allir pakkar sem þarf að nota.
%
\usepackage[utf8]{inputenc}
\usepackage[T1]{fontenc}
\usepackage[icelandic]{babel}
\usepackage{amsmath}
\usepackage{amssymb}
\usepackage{pictex}
\usepackage{epsfig,psfrag}
\usepackage{makeidx}
%\selectlanguage{icelandic}
%----------------------------

%
\hoffset=-0.4truecm
\voffset=-1truecm
\textwidth=16truecm 
%\textwidth=12truecm 
\textheight=23truecm
\evensidemargin=0truecm
%
% Gömlu gildin á bókinni 
%
%\voffset 1.4truecm
%\hoffset .25truecm
%\vsize  16.0truecm
%\hsize  15truecm
%
%
% Skilgreiningar á ýmsum skipunum.
%
%\newcommand{\Sb}{
%$$
%\sum_{\footnotesize\begin{array}{l} j=1 \\ j\neq k \end{array}}
%$$
%}
\newcommand{\bolddot}{{\mathbf \cdot}}
\newcommand{\C}{{\mathbb  C}}
\newcommand{\Cn}{{\mathbb  C\sp n}}
\newcommand{\crn}{{{\mathbb  C\mathbb  R^n}}}
\newcommand{\R}{{\mathbb  R}}
\newcommand{\Rn}{{\mathbb  R\sp n}}
\newcommand{\Rnn}{{\mathbb  R\sp{n\times n}}}
\newcommand{\Z}{{\mathbb  Z}}
\newcommand{\N}{{\mathbb  N}}
\renewcommand{\P}{{\mathbb  P}}
\newcommand{\Q}{{\mathbb  Q}}
\newcommand{\K}{{\mathbb  K}}
\newcommand{\U}{{\mathbb  U}}
\newcommand{\D}{{\mathbb  D}}
\newcommand{\T}{{\mathbb  T}}
\newcommand{\A}{{\cal A}}
\newcommand{\E}{{\cal E}}
\newcommand{\F}{{\cal F}}
\renewcommand{\H}{{\cal H}}
\renewcommand{\L}{{\cal L}}
\newcommand{\M}{{\cal M}}
\renewcommand{\O}{{\cal O}}
\renewcommand{\S}{{\cal S}}
\newcommand{\dash}{{\sp{\prime}}}
\newcommand{\ddash}{{\sp{\prime\prime}}}
\newcommand{\tdash}{{\sp{\prime\prime\prime}}}
\newcommand{\set }[1]{{\{#1\}}}
\newcommand{\scalar}[2]{{\langle#1,#2\rangle}}
\newcommand{\arccot}{{\operatorname{arccot}}}
\newcommand{\arccoth}{{\operatorname{arccoth}}}
\newcommand{\arccosh}{{\operatorname{arccosh}}}
\newcommand{\arcsinh}{{\operatorname{arcsinh}}}
\newcommand{\arctanh}{{\operatorname{arctanh}}}
\newcommand{\Log}{{\operatorname{Log}}}
\newcommand{\Arg}{{\operatorname{Arg}}}
\newcommand{\grad}{{\operatorname{grad}}}
\newcommand{\graf}{{\operatorname{graf}}}
\renewcommand{\div}{{\operatorname{div}}}
\newcommand{\rot}{{\operatorname{rot}}}
\newcommand{\curl}{{\operatorname{curl}}}
\renewcommand{\Im}{{\operatorname{Im\, }}}
\renewcommand{\Re}{{\operatorname{Re\, }}}
\newcommand{\Res}{{\operatorname{Res}}}
\newcommand{\vp}{{\operatorname{vp}}}
\newcommand{\mynd}[1]{{{\operatorname{mynd}(#1)}}}
\newcommand{\dbar}{{{\overline\partial}}}
\newcommand{\inv}{{\operatorname{inv}}}
\newcommand{\sign}{{\operatorname{sign}}}
\newcommand{\trace}{{\operatorname{trace}}}
\newcommand{\conv}{{\operatorname{conv}}}
\newcommand{\Span}{{\operatorname{Sp}}}
\newcommand{\stig}{{\operatorname{stig}}}
\newcommand{\Exp}{{\operatorname{Exp}}}
\newcommand{\diag}{{\operatorname{diag}}}
\newcommand{\adj}{{\operatorname{adj}}}
\newcommand{\erf}{{\operatorname{erf}}}
\newcommand{\erfc}{{\operatorname{erfc}}}
\newcommand{\Lloc}{{L_{\text{loc}}\sp 1}}
\newcommand{\boldcdot}{{\mathbb \cdot}}
%\newcommand{\Cinf0}[1]{{C_0\sp{\infty}(#1)}}
\newcommand{\supp}{{\text{supp}\, }}
\newcommand{\chsupp}{{\text{ch supp}\, }}
\newcommand{\singsupp}{{\text{sing supp}\, }}
\newcommand{\SL}[1]{{\dfrac {1}{\varrho} 
\bigg(-\dfrac d{dx}\bigg(p\dfrac {d#1}{dx}\bigg)+q#1\bigg)}}
\newcommand{\SLL}[1]{-\dfrac d{dx}\bigg(p\dfrac {d#1}{dx}\bigg)+q#1}
\newcommand{\Laplace}[1]{\dfrac{\partial^2 #1}{\partial x^2}+\dfrac{\partial^2 #1}{\partial y^2}}
\newcommand{\polh}[1]{{\widehat #1_{\C^n}}}
\newcommand{\tilv}{{}}
%
\renewcommand{\chaptername}{Kafli}
%
% Númering á formæulum.
%
\numberwithin{equation}{section}
%
%  Innsetning á myndum.
%
\def\figura#1#2{
\vbox{\centerline{
\input #1
}
\centerline{#2}
}\medskip}
\def\vfigura#1#2{
\setbox0\vbox{{
\input #1
}}
\setbox1\vbox{\hbox{\box0}\hbox{{\obeylines #2}}}
\dimen0 = -\ht1
\advance\dimen0 by-\dp1
\dimen1 = \wd1
\dimen2 = -\dimen0
\divide\dimen2 by\baselineskip
\count100 = 1
\advance\count100 by\dimen2
\advance\count100 by1
\box1
\hangindent\dimen1
\hangafter=-\count100
\vskip\dimen0
}
%
%  Setningar, skilgreiningar, o.s.frv. 
%
\newtheorem{setning+}           {Setning}      [section]
\newtheorem{skilgreining+}  [setning+]  {Skilgreining}
\newtheorem{setningogskilgreining+}  [setning+]  {Setning og
skilgreining}
\newtheorem{hjalparsetning+}  [setning+]  {Hjálparsetning}
\newtheorem{fylgisetning+}  [setning+]  {Fylgisetning}
\newtheorem{synidaemi+}  [setning+]  {Sýnidæmi}
\newtheorem{forrit+}  [setning+]  {Forrit}

\newcommand{\tx}[1]{{\rm({\it #1}). \ }}

\newenvironment{se}{\begin{setning+}\sl}{\hfill$\square$\end{setning+}\rm}
\newenvironment{sex}{\begin{setning+}\sl}{\hfill$\blacksquare$\end{setning+}\rm}
\newenvironment{sk}{\begin{skilgreining+}\rm}{\hfill$\square$\end{skilgreining+}\rm}
\newenvironment{sesk}{\begin{setningogskilgreining+}\rm}{\hfill$\square$\end{setningogskilgreining+}\rm}
\newenvironment{hs}{\begin{hjalparsetning+}\sl}{\hfill$\square$\end{hjalparsetning+}\rm}
\newenvironment{fs}{\begin{fylgisetning+}\sl}{\hfill$\square$\end{fylgisetning+}\rm}
\newenvironment{sy}{\begin{synidaemi+}\rm}{\hfill$\square$\end{synidaemi+}\rm}
\newenvironment{fo}{\begin{forrit+}\rm}{\hfill\end{forrit+}\rm}
\newenvironment{so}{\medbreak\noindent{\it Sönnun:}\rm}{\hfill$\blacksquare$\rm}
\newenvironment{sotx}[1]{\medbreak\noindent{\it #1:}\rm}{\hfill$\blacksquare$\rm}
\newcounter{daemateljari}
\newcommand{\aefing}{\section{Æfingardæmi} \setcounter{daemateljari}{1}}
\newcommand{\daemi}{
{\medskip\noindent{\bf \thedaemateljari.}}
\addtocounter{daemateljari}{1}
}

%\def\aefing{{\large\bf\bigskip\bigskip\noindent Æfingardæmi}}
%\def\daemi#1{\medskip\noindent{\bf #1.}}
\def\svar#1{\smallskip\noindent{\bf #1.} \ }
\def\lausn#1{\smallskip\noindent{\bf #1.} \ }
\def\ugrein#1{\medbreak\noindent{\bf #1.} }
\newcommand{\samantekt}{\noindent{\bf Samantekt.} }
%\newcommand{\proclaimbox}{\hfill$\square$}

%
%
%

\chapter{LÍNULEG AFLEIÐUJÖFNUHNEPPI}


\section {Tilvist og ótvíræðni lausna}


\noindent
Viðfangsefni þessa kafla eru línuleg afleiðujöfnuhneppi  af gerðinni
 \begin{equation*}u\dash=A(t)u+f(t),\label{5.1.1}
 \end{equation*}
þar sem $A\in C(I,\C^{m\times m})$ er fylkjafall og $f\in C(I,\C^m)$ 
er vigurfall sem er skilgreint á opnu bili $I$ á $\R$.  Ef við skrifum upp
hnitin þá verður hneppið 
\begin{align*}
u_1\dash&=a_{11}(t)u_1+\cdots+a_{1m}(t)u_m+f_1(t),\\
u_2\dash&=a_{21}(t)u_1+\cdots+a_{2m}(t)u_m+f_2(t),\\
&\qquad \qquad \vdots\qquad \qquad \qquad \qquad \vdots\\
u_m\dash&=a_{m1}(t)u_1+\cdots+a_{mm}(t)u_m+f_m(t).
\end{align*}
Hneppið er sagt vera {\it
óhliðrað\index{óhliðraður}\index{afleiðujöfnuhneppi!óhliðrað}} ef
fallið $f$ er núllfallið, en {\it
hliðrað\index{afleiðujöfnuhneppi!hliðrað}} annars.   Samkvæmt
fylgisetningu 6.7.6  hefur upphafsgildisverkefnið
\begin{equation*}
u\dash=A(t)u+f(t), \qquad u(a)=b,\label{5.1.2}
\end{equation*}
ótvírætt ákvarðaða lausn, þar sem $a$ er einhver gefinn punktur í $I$
og $b$ er einhver gefinn vigur í $\C^m$.  

\begin{sk}
Línulega rúmið, sem samanstendur af öllum lausnum á óhliðruðu
jöfnunni $u\dash=A(t)u$, kallast {\it
núllrúm\index{núllrúm}\index{núllrúm!afleiðujöfnuhneppis}
\index{línulegt afleiðujöfnuhneppi!núllrúm}} línulega
jöfnuhneppisins.  Við táknum það með ${\cal N}(A)$.
\end{sk}


\begin{se}
Látum $I\subset \R$ vera bil og $A\in C(I,\C\sp
{m\times m})$. Þá hefur núllrúmið ${\cal N}(A)$ víddina $m$.
\end{se}

Við athugum nú að ef $v$ og $w$ eru tvær lausnir á hliðruðu jöfnunni
$u\dash=A(t)u+f(t)$, þá er mismunurinn $v-w$ í núllrúminu, því
 $$(v-w)\dash=v\dash-w\dash=A(t)v+f(t)-A(t)w-f(t)=A(t)(v-w).
 $$
Þetta gefur okkur því:

\begin{se}  
Látum $I$ vera bil á rauntalnaásnum, $A\in C(I,\C\sp
{m\times m})$ og $f\in C(I,\C\sp m)$.
Þá er sérhver lausn á $u'=A(t)u+f(t)$ af gerðinni
$$ u(t)=c_1u_1(t)+\cdots+c_mu_m(t)+u_p(t),$$
þar sem $u_1,\dots,u_m$ er einhver grunnur í ${\cal N}(A)$,
$c_1,\dots,c_m\in\C$ og $u_p$ er einhver lausn á hliðruðu jöfnunni.
\end{se}

\subsection*{Jöfnur af hærri stigum  og jafngild hneppi}

við vitum að línuleg $m$--ta stigs afleiðujafna
 \begin{equation*}
P(t,D)v= v\sp{(m)}+a_{m-1}(t)v\sp{(m-1)}+\cdots+a_1(t)v\dash
+a_0(t)v=g(t), \qquad t\in I, \label{5.1.3}
\end{equation*} 
er jafngild hneppinu
\begin{align*}
u_1\dash &= u_2,\quad
u_2\dash = u_3,\quad
\dots, \quad u_{m-1}\dash = u_m\label{5.1.4}\\
u_m\dash &=-a_0(t)u_1-a_1(t)u_2-\cdots-a_{m-1}(t)u_m+g(t).\nonumber
\end{align*}
í þeim skilningi að $v$ er lausn á afleiðujöfnunni þá og því aðeins að
$u=[v,v\dash,\dots,v^{(m-1)}]^t$ sé lausn á hneppinu.
Fylkið $A(t)$ og vigurinn $f(t)$ verða í þessu tilfelli 
\begin{equation*}A(t)=\left[\begin{matrix}
0&1&0&\dots&0\\
0&0&1&\dots&0\\
\vdots&\vdots&\vdots&\ddots&\vdots\\
0&0&0&\dots&1\\
-a_0(t)&-a_1(t)&-a_2(t)&\dots&-a_{m-1}(t)
\end{matrix}\right],
\qquad
f(t)=\left[\begin{matrix}
0\\
0\\
\vdots\\
0\\
g(t)
\end{matrix}\right].\label{5.1.5}
\end{equation*}

\begin{se} Látum  $P(t,D)=D\sp m+a_{m-1}(t)D\sp{m-1}+\cdots+a_1(t)D+a_0(t)$
vera línulegan afleiðuvirkja og skilgreinum $A(t)$ sem fylkið í
(\ref{5.1.5}).  Þá er
$$\det(\lambda I-A(t))=P(t,\lambda),$$
þ.e.a.s.~kennimargliða\index{kennimargliða}\index{margliða!kennimargliða}
fylkisins\index{kennimargliða!fylkis} $A(t)$ er kennimargliða
virkjans\index{kennimargliða!virkja} $P(t,D)$.
\end{se}



\section{Hneppi með fastastuðla\index{afleiðujöfnuhneppi!með
fastastuðla}\index{fastastuðlar}}

\noindent
Nú ætlum við að byrja á því að reikna út lausnir á línulegum
afleiðujöfnuhneppum.
Við lítum á óhliðrað\index{afleiðujöfnuhneppi!óhliðrað}
línulegt jöfnuhneppi $u\dash=Au$ og gerum ráð fyrir að stuðlarnir
í fylkinu $A$  séu fastaföll.

\begin{hs}
Látum $A$ vera $m\times m$ fylki og $\varepsilon$ vera
eiginvigur\index{eiginvigur} þess með
tilliti til eigingildisins\index{eigingildi}\index{eigingildi!fylkis}
$\lambda$. Þá uppfyllir vigurfallið
$u(t)=e\sp{\lambda t}\varepsilon$ jöfnuna $u\dash=Au$.

{}
\end{hs}

Þessi einfalda hjálparsetning gefur okkur að í því tilfelli að hægt er
að liða $b$ og $f(t)$ í línulegar samantektir af eiginvigrunum, þá
leysist jöfnuhneppið  upp í óháðar jöfnur sem við getum leyst hverja
fyrir sig:

\begin{se}\label{set5.2.2}
Látum $A$ vera $m\times m$ fylki og gerum ráð fyrir að 
$\varepsilon_1,\dots,\varepsilon_\ell$   séu eiginvigrar
þess með tilliti til eigingildanna $\lambda_1,\dots,\lambda_\ell$.  
Ef $a \in I$, $b\in \C^m$ og unnt er að skrifa
$b=\beta_1\varepsilon_1+\cdots+\beta_\ell\varepsilon_\ell$ og 
$f(t)=g_1(t)\varepsilon_1+\cdots+g_\ell(t)\varepsilon_\ell$, þá er
lausnin á upphafsgildisverkefninu\index{upphafsgildisverkefni} 
$$u\dash=Au+f(t), \qquad \qquad u(a)=b,
$$ 
gefin með  $u(t)=v_1(t)\varepsilon_1+\cdots+v_\ell(t)\varepsilon_\ell$, þar sem
stuðullinn $v_j$  uppfyllir
 \begin{equation*}v_j\dash(t)=\lambda_jv_j(t)+g_j(t), \qquad v_j(a)=\beta_j,\label{5.2.1}
 \end{equation*}
og er þar með 
 \begin{equation*}v_j(t)=\beta_je^{\lambda_j(t-a)}+e^{\lambda_jt}\int_a^t e^{-\lambda_j
\tau}g_j(\tau) \, d\tau.\label{5.2.2}
 \end{equation*}
\end{se}


\subsection*{Úrlausn með gefinn eiginvigragrunn}

Nú skulum við  gera ráð fyrir því að
fylkið $A$ hafi eiginvigragrunn\index{eiginvigragrunnur}
$\varepsilon_1,\dots, \varepsilon_m$ með
eigingildin $\lambda_1,\dots,\lambda_m$.  Þá getum við þáttað fylkið
$A$ í
 \begin{equation*}A=T\Lambda T\sp{-1},\label{5.2.3}
 \end{equation*}
þar sem eiginvigrarnir eru dálkar fylkisins $T$  og
$\Lambda=\diag(\lambda_1,\dots,\lambda_m)  $ er
hornalínufylki með tilsvarandi  eigingildi á hornalínunni,
 $$T=\left[\begin{matrix}
\varepsilon_{11}&\varepsilon_{12}&\dots&\varepsilon_{1m}\\
\varepsilon_{21}&\varepsilon_{22}&\dots&\varepsilon_{2m}\\
\vdots&\vdots&\ddots&\vdots\\
\varepsilon_{m1}&\varepsilon_{m2}&\dots&\varepsilon_{mm}
\end{matrix}\right],\qquad
\Lambda =\left[\begin{matrix}
\lambda_1&0&\dots&0\\
0&\lambda_2&\dots&0\\
\vdots&\vdots&\ddots&\vdots\\
0&0&\dots&\lambda_m
\end{matrix}\right].
 $$
Hér skrifum við $\varepsilon_j=[\varepsilon_{1j},\dots,\varepsilon_{mj}]^t$.  
Hér mikilvægt að minnast þess að ef $b$ er vigur í $\C^m$, þá eru
hnit hans $\beta=[\beta_1,\dots,\beta_m]^t$ miðað við grunninn
$\set{\varepsilon_1,\dots,\varepsilon_m}$ gefin með  jöfnunni ${\beta}=T^{-1}b$.

Nú skulum við skoða aftur lausnina á upphafsgildisverkefninu \tilv 9.1.2.
Við látum $v(t)=[v_1(t),\dots,v_m(t)]^t$ vera hnit $u(t)$, 
$g(t)=[g_1(t),\dots,g_m(t)]^t$ vera hnit $f(t)$ og
$\beta=[\beta_1,\dots,\beta_m]^t$ 
vera hnit $b$ miðað við grunninn
$\set{\varepsilon_1,\dots,\varepsilon_m}$, þ.e.~$v=T^{-1}u$,
$g=T^{-1}f$ og $\beta=T^{-1}b$.
Við reiknum nú afleiðuna af $v$ og notum (\ref{5.2.3})
\begin{gather*}
v\dash=T^{-1}u\dash=T^{-1}(Au+f(t))=
T^{-1}T\Lambda T^{-1}u+T^{-1}f(t)=\Lambda v+g(t), \qquad t\in I,\\
v(a)=T^{-1}u(a)=T^{-1}b=\beta 
\end{gather*}
Nú er $\Lambda v=(\lambda_1v_1,\dots,\lambda_mv_m)$, svo við höfum fengið
upphafsgildisverkefni fyrir $v$, sem sett
er fram með jöfnunum (\ref{5.2.1}) og þar með er lausnin gefin með 
(\ref{5.2.2}).


Nú skulum við líta á þessa formúlu ögn nánar.  Við skilgreinum  fylkjafallið
$$\diag(e^{\lambda_1t},\dots,e^{\lambda_mt})=
\left[\begin{matrix}
e\sp{\lambda_1t}&0&\dots&0\\
0&e\sp{\lambda_2t}&\dots&0\\
\vdots&\vdots&\ddots&\vdots\\
0&0&\dots&e^{\lambda_mt}
\end{matrix}\right],
$$
og athugum síðan að $T\diag(e^{\lambda_1t},\dots,e^{\lambda_mt})$ 
hefur dálkana
$e\sp{\lambda_1t}\varepsilon_1,\dots,e\sp{\lambda_mt}\varepsilon_m$ og því
er 
\begin{gather*}
\beta_1e^{\lambda_1(t-a)}\varepsilon_1+
\cdots+\beta_me^{\lambda_m(t-a)}\varepsilon_m=
T\diag(e^{\lambda_1(t-a)},\dots,e^{\lambda_m(t-a)})\beta,\\
e^{\lambda_1(t-\tau)}g_1(\tau)\varepsilon_1
+\cdots+
e^{\lambda_m(t-\tau)}g_m(\tau)\varepsilon_m=
T\diag(e^{\lambda_1(t-\tau)},\dots,e^{\lambda_m(t-\tau)})g(\tau).
\end{gather*}
Nú er $\beta=T^{-1}b$ og $g(\tau)=T^{-1}f(\tau)$, svo við fáum umritaða
framsetningu á setningu \ref{set5.2.2}:

\begin{se}
Látum $A$ vera $m\times m$ fylki og gerum ráð fyrir að hægt sé að
þátta $A$ í $A=T\Lambda T^{-1}$ þar sem $\Lambda $ er hornalínufylki með
hornalínustökin $\lambda_1,\dots,\lambda_m$.  Látum $I$ vera bil á
$\R$, $a\in I$, $f\in C(I,\C^m)$ og $b\in \C^m$.  Þá hefur
upphafsgildisverkefnið  $$u\dash=Au+f(t), \qquad u(a)=b
 $$
ótvírætt ákvarðaða lausn á $I$, sem gefin er með formúlunni
 
\begin{align*}
u(t)&=T\diag(e^{\lambda_1(t-a)},\dots,e^{\lambda_m(t-a)})T^{-1}b\\
&+\int_a^t T\diag(e^{\lambda_1(t-\tau)},\dots,e^{\lambda_m(t-\tau)})
T^{-1}f(\tau)\, d\tau.
\end{align*}
 
\end{se}

Þessi setning segir ekkert meira en setning \ref{set5.2.2} og hana höfum
við sannað.  Við skulum engu að síður  staðfesta að þetta sé lausn á
upphafsgildisverkefninu.  Athugum fyrst að
\begin{align*}
\dfrac d{dt}\diag(e^{\lambda_1 t},\dots,e^{\lambda_mt})
&=
\left[\begin{matrix}
\lambda_1e\sp{\lambda_1t}&0&\dots&0\\
0&\lambda_2e\sp{\lambda_2t}&\dots&0\\
\vdots&\vdots&\ddots&\vdots\\
0&0&\dots&\lambda_me^{\lambda_mt}
\end{matrix}\right]\\
&=\Lambda\diag(e^{\lambda_1 t},\dots,e^{\lambda_mt}).
\end{align*}
Ef við notum þessa formúlu, þá fáum við
 \begin{multline*}
u\dash(t)=T\Lambda \diag(e^{\lambda_1(t-a)},\dots,e^{\lambda_m(t-a)})T^{-1}b
\\
+\int_a^t T\Lambda 
\diag(e^{\lambda_1(t-\tau)},\dots,e^{\lambda_m(t-\tau)})T^{-1}f(\tau)\,
d\tau + TT^{-1}f(t).
\end{multline*}
Nú notum við formúluna $T\Lambda =T\Lambda T\sp{-1}T=AT$ og fáum
\begin{multline*}
u\dash(t)
=A\bigg(T\diag(e^{\lambda_1(t-a)},\dots,e^{\lambda_m(t-a)})T^{-1}b \\
+\int_a^t
T\diag(e^{\lambda_1(t-\tau)},\dots,e^{\lambda_m(t-\tau)})T^{-1}f(\tau)\, 
d\tau\bigg)+ f(t)=Au(t)+f(t).
\end{multline*}
 

\subsection*{Annars stigs hneppi}

Aðferðinni sem við höfum verið að lýsa er oft hægt að beita á annars
stigs hneppi, til að leysa upphafsgildisverkefni af gerðinni
 \begin{equation*}u\ddash=Au+f(t), \qquad u(a)=b, \quad u\dash(a)=c,
\label{5.2.4}
 \end{equation*}
í því tilfelli að hægt er að skrifa 
 $$b=\beta_1\varepsilon_1+\cdots+\beta_\ell\varepsilon_\ell, \quad
c=\gamma_1\varepsilon_1+\cdots+\gamma_\ell\varepsilon_\ell,\quad
f(t)=g_1(t)\varepsilon_1+\cdots+g_\ell(t)\varepsilon_\ell.
 $$
Lausnin verður þá einfaldlega af gerðinni
 \begin{equation*}u(t)=v_1(t)\varepsilon_1+\cdots+v_\ell(t)\varepsilon_\ell,
\label{5.2.5}
 \end{equation*}
þar sem $v_j$ er lausnin á upphafsgildisverkefninu
 \begin{equation*}v_j\ddash=\lambda_j v_j +g_j(t), \qquad v_j(a)=\beta_j, \quad
v_j\dash(a)=\gamma_j. 
\label{5.2.6}
 \end{equation*}
Þessi formúla er staðfest með beinum útreikningi.
Ef við gerum ráð fyrir því að öll eigingildin séu neikvæð
$\lambda_j=-\omega_j^2$, þá notfærum við okkur að
$\cos {\omega}_j t$ og $\sin {\omega}_jt$ er lausnargrunnur fyrir 
óhliðruðu jöfnuna og $\sin({\omega}_j(t-{\tau}))/{\omega}_j$ er
Green--fall virkjans.  Þar með er lausnin
 \begin{equation*}v_j(t)=\beta_j \cos(\omega_j(t-a))+
(\gamma_j/\omega_j)\sin (\omega_j(t-a)) +
\int_a^t\dfrac{\sin (\omega_j(t-\tau))}{\omega_j}g_j(\tau) \, d\tau. 
\label{5.2.7}
 \end{equation*}
Í því tilfelli að hneppið er hreyfijöfnur einhvers
eðlisfræðilegs kerfis, þá kallast liðirnir $v_j(t)\varepsilon_j$ í
lausnarformúlunni  {\it sveifluhættir\index{sveifluháttur}} kerfisins.  Þeir eru
innbyrðis óháðir eins og jöfnurnar.  Stærðin 
${\omega}_j$ nefnist {\it tíðni
sveifluháttarins\index{tíðni!sveifluháttar}} $v_j(t)\varepsilon_j$.

 
\section{Grunnfylki}

\noindent
Lítum á óhliðrað línulegt afleiðujöfnuhneppi 
 $$u\dash=A(t)u, \qquad t\in I,
 $$
þar sem  $A\in C(I,\C\sp{m\times m})$, 
$A(t)=(a_{jk}(t))_{j,k=1}\sp m$.
Mengi allra lausna myndar línulegt rúm af vídd $m$.




\begin{hs}\label{hs5.3.1} Látum $u_1,\dots,u_m$ vera föll í ${\cal N}(A)$.  Þá eru
eftirfarandi skilyrði jafngild:

\noindent
(i) Vigurföllin $u_1,\dots,u_m$ eru línulega óháð á bilinu $I$.

\noindent
(ii) Vigrarnir $u_1(t),\dots,u_m(t)$ eru línulega óháðir í $\R\sp m$
(eða $\C\sp m$) fyrir sérhvert $t\in I$.

\noindent
(iii) Vigrarnir $u_1(a),\dots,u_m(a)$ eru línulega óháðir í $\R\sp m$
(eða $\C\sp m$) fyrir eitthvert  $a\in I$.
 \end{hs}

\begin{sk} Fylki af gerðinni
$$ \Phi(t)=[u_1(t),\dots,u_m(t)], \qquad t\in I, $$ 
þar sem dálkavigrarnir $u_1,\dots,u_m$ mynda grunn í núllrúminu ${\cal
N}(A)$ fyrir afleiðujöfnuhneppið $u\dash=A(t)u$, kallast
{\it grunnfylki} fyrir afleiðujöfnuhneppið.
\end{sk}

Samkvæmt hjálparsetningu \ref{hs5.3.1}, þá eru dálkarnir í $\Phi(t)$
línulega óháðir fyrir öll $t\in I$ og þar með er andhverfan
$\Phi(t)\sp{-1}$ til í sérhverjum punkti $t\in I$.  Við sjáum
jafnframt að
\begin{align*}
\Phi\dash(t)&= [u_1\dash(t),\dots,u_m\dash(t)]=\nonumber\\
&=[A(t)u_1(t),\dots,A(t)u_m(t)]=\label{5.3.1}\\
&=A(t)\Phi(t).\nonumber
\end{align*}
Af hjálparsetningunni leiðir einnig að ef $m\times
m$ fylkjafallið $\Phi$ uppfyllir $\Phi\dash=A(t)\Phi$ og $\Phi(a)$
hefur andhverfu fyrir eitthvert $a\in I$, þá er $\Phi(t)$ grunnfylki
fyrir afleiðujöfnuhneppið $u\dash =A(t)u$.

\begin{se}
Látum $\Phi$ og $\Psi$ vera tvö grunnfylki fyrir jöfnuhneppið
$u\dash=A(t)u$. Þá er til andhverfanlegt fylki $B$ þannig að 
 \begin{equation*}\Psi(t)=\Phi(t)B.\label{5.3.2}
 \end{equation*}
\end{se}


\subsection*{Upphafsgildisverkefni fyrir grunnfylki}

\noindent
Við fáum nú lýsingu á lausn upphafsgildisverkefnisins með
grunnfylkjum:

\begin{se}\label{set5.3.4}
 Látum $\Phi(t)$ vera grunnfylki fyrir jöfnuhneppið $u\dash
=A(t)u$. 

\noindent
(i)  Sérhvert stak í ${\cal N}(A)$ er af gerðinni $u(t)=\Phi(t)c$, þar
sem $c$ er vigur í $\C\sp m$.

\noindent
(ii) Vigurfallið  $u_p$, sem gefið er með formúlunni
 $$u_p(t)=\Phi(t)\int_a\sp t \Phi(\tau)\sp{-1}f(\tau)\, d\tau,
 $$
uppfyllir $u\dash=A(t)u+f(t)$ og $u(a)=0$. 

\noindent
(iii)  Lausnin á upphafsgildisverkefninu $u\dash=A(t)u+f(t)$, $u(a)=b$
er gefin með formúlunni
 $$u(t)=\Phi(t)\Phi(a)\sp{-1}b+
\Phi(t)\int_a\sp t \Phi(\tau)\sp{-1}f(\tau)\, d\tau.
 $$
\end{se}

\medskip
Nú getum við beitt setningunni á dálkana í $m\times m$ fylkinu $U(t)$ og
fengið eftirfarandi
tilvistarsetningu\index{fylkjaafleiðujafna!tilvistarsetning}:

\begin{se}
Látum $A, F\in C(I,\C\sp {m\times m})$ og látum $\Phi$ vera grunnfylki
fyrir $A$.  Þá hefur $m\times m$
fylkjaafleiðujafnan\index{fylkjaafleiðujafna}
 $$U\dash=A(t)U+F(t), \qquad U(a)=B, 
 $$
ótvírætt ákvarðaða lausn $U(t)$,  sem gefin er með formúlunni
 $$U(t)=\Phi(t)\Phi(a)\sp{-1}B + \Phi(t)\int_a\sp t \Phi(\tau)\sp
{-1}F(\tau) \, d\tau.
 $$
\end{se}

\subsection*{Hneppi með fastastuðla}

Gerum nú ráð fyrir því að $A$ hafi fastastuðla og að eiginvigrar þess
myndi grunn í $\C\sp m$.  Eins og við höfum áður sannfært okkur um,
þá er það jafngilt því að unnt sé að þátta fylkið $A$ í 
 $$A=T\Lambda T\sp{-1},
 $$
þar sem $\Lambda$ er hornalínufylki með eigingildin á hornalínunni,
 $$\Lambda=\diag(\lambda_1,\dots,\lambda_m)=\left[\begin{matrix} 
\lambda_1&0&\dots&0\\
0&\lambda_2&\dots&0\\
\vdots&\vdots&\ddots&\vdots\\
0&0&\dots&\lambda_m\end{matrix}\right].
 $$
Lítum á fylkið 
 $$\Phi(t)=T\diag(e\sp{t\lambda_1},\dots,e\sp{t\lambda_m})T\sp{-1}.
 $$
Það uppfyllir
\begin{align*}
\Phi\dash(t)
&=T\diag(\lambda_1e\sp{t\lambda_1},\dots,\lambda_me\sp{t\lambda_m})T\sp{-1}=\\
&=T\diag(\lambda_1,\dots,\lambda_m)
\diag(e\sp{t\lambda_1},\dots,e\sp{t\lambda_m})T\sp{-1}=\\
&=T\Lambda T\sp{-1} T
\diag(e\sp{t\lambda_1},\dots,e\sp{t\lambda_m})T\sp{-1}=\\
&=A\Phi(t), 
\end{align*}
með upphafsskilyrðinu
$$
\Phi(0)=I.
$$
Þar með er $\Phi$ grunnfylki fyrir hneppið $u\dash =Au$.  Hér er
komin grunnlausnin sem við notuðum í útleiðslu okkar í grein 9.2.


\section{Fylkjamargliður\index{fylkjamargliða}\index{margliða!fylkjamargliða} og
fylkjaveldaraðir\index{fylkjaveldaröð}\index{veldaröð!fylkjaveldaröð}}

\noindent
Ef $A$ er $m\times m$ fylki og $p(\lambda)$ er margliða af
tvinnbreytistærðinni $\lambda$, 
 $$p(\lambda)=a_0+a_1\lambda+\cdots+a_n\lambda^n, 
 $$
þá getum við skilgreint  fylkjamargliðuna $p(A)$ með formúlunni
 $$p(A)=a_0 I+a_1A+\cdots+a_n A^n, 
 $$
þar sem $I$ táknar $m\times m$--einingarfylkið.  Hér höfum við
einfaldlega skipt á veldum $\lambda^k$ af $\lambda$ og veldum $A^k$
af $A$ og jafnframt margfaldað fastaliðinn með einingarfylkinu $I$. 
Til þess að geta stungið $A$ inn í  óendanlegar veldaraðir, þá þurfum
við að skilgreina samleitni:

\subsection*{Samleitnar fylkjarunur}

\begin{sk}  Runa $\set{C_n}_{n=0}^\infty$,  af
$\ell\times m$ fylkjum $C_n=\big(c_{jkn}\big)_{j=1,k=1}^{\ell, m}$
er sögð vera samleitin ef allar stuðlarunurnar
 $$\set{c_{jkn}}_{n=0}^\infty, \qquad j=1,\dots,\ell, \quad k=1,\dots, m.
 $$
eru samleitnar.  Fylkið $C=\big(c_{jk}\big)_{j=1,k=1}^{\ell, m}$ sem
hefur stuðlana
 $$c_{jk}=\lim\limits_{n\to\infty}c_{jkn}, \qquad j=1,\dots,\ell, \quad
k=1,\dots, m,
 $$
kallast markgildi rununnar $\set{C_n}_{n=0}^\infty$ og við táknum það
með
 $$C=\lim\limits_{n\to \infty}C_n.
 $$
Óendanleg summa $\sum_{n=0}^\infty C_n$ af $\ell\times m$ fylkjum er
sögð vera samleitin, ef runan af hlutsummum  $\set{\sum_{n=0}^N
C_n}_{N=0}^\infty$  er samleitin.  Við táknum markgildið einnig með
$\sum_{n=0}^\infty C_n$,
 $$\sum_{n=0}^\infty C_n= \lim\limits_{N\to \infty}
\sum_{n=0}^N C_n.
 $$
\end{sk}

Ef $C_n=a_n A^n$ og $A^0=I$, þá er 
$\sum_{n=0}^\infty C_n=\sum_{n=0}^\infty a_nA^n$ 
veldaröð.  


\subsection*{Fylkjastaðall}

Til þess að geta
skorið úr um samleitni veldaraða þá þurfum við að tengja fylkið við
samleitnigeisla raðarinnar.  Til þess innleiðum við:

\begin{sk}\tx{Fylkjastaðall\index{fylkjastaðall}\index{staðall}
\index{staðall!fylkjastaðall}}  Ef $A$ er $\ell\times m$ fylki, $A=(a_{jk})$,
með tvinntölustök, þá skilgreinum við {\it staðalinn\index{staðall}} $\|A\|$ af $A$ með
formúlunni 
 $$\|A\|=\sum_{j=1}\sp \ell \sum_{k=1}\sp m |a_{jk}|.
 $$
Við köllum töluna $\|A\|$ einnig {\it
lengd\index{lengd}\index{lengd!fylkis}} fylkisins $A$.
\end{sk}

\begin{se}\label{set5.4.3}\tx{Reiknireglur um fylkjastaðal}  (i) Ef $A$ og $B$ eru
$\ell\times m$ fylki með stök í $\C$ og $c\in \C$, þá er
 $$\|A+B\|\leq \|A\|+\|B\| \qquad \text{og} \qquad
\|cA\|=|c|\|A\|.
 $$

\noindent 
(ii) Ef $A$ er $\ell\times m$ fylki og $B$ er $m\times n$ fylki, þá
er
 $$\|AB\|\leq \|A\|\|B\|.
 $$

\noindent
(iii) Ef $A$ er $m\times m$ fylki, þá er
 $$\|A\sp n\|\leq \|A\|\sp n.
 $$
\end{se}

\subsection*{Samleitnar fylkjaraðir}

\begin{se}\label{set5.4.4}\tx{Samleitnipróf fyrir
fylkjaraðir\index{samleitnipróf fyrir
fylkjaraðir}\index{samleitni!fylkjaraða}} Látum $\set{C_n}$ vera runu
af $\ell\times m$ fylkjum  þannig að talnaröðin $\sum_{n=0}\sp
\infty\|C_n\|$ sé samleitin.  Þá er fylkjaröðin $\sum_{n=0}\sp \infty
C_n$ samleitin.
\end{se}


\begin{fs}  Látum $\sum_{n=0}\sp \infty c_nz\sp n$ vera veldaröð með
tvinntölustuðla  og gerum ráð fyrir að samleitnigeisli hennar sé
${\varrho}>0$.  Ef $A$ er $m\times m$ fylki með tvinntölustuðla og 
$\|A\|<{\varrho}$,
þá er fylkjaveldaröðin $\sum_{n=0}\sp \infty c_nA\sp n$
samleitin.
\end{fs}



Hugsum okkur nú að $f:S(0,\varrho)\to \C$ 
sé fágað fall sem gefið er með 
 $$f(z)=\sum_{n=0}\sp \infty c_n z\sp n, \qquad z\in S(0,\varrho).
 $$
Ef $A$ er $m\times m$ fylki og $\|A\|< \varrho$, þá getum við
skilgreint $m\times m$ fylkið $f(A)$ með því að stinga $A$ inn í
veldaröðina fyrir fágaða fallið $f$,   
 $$f(A)=\sum_{n=0}\sp \infty c_nA\sp n,
 $$
því fylkjaveldaröðin í hægri hliðinni er samleitin. Við skilgreinum
$A^0=I$.  Ef við vitum að
$f$ er fágað fall á öllu $\C$ þá þurfum við engar áhyggjur að hafa af
samleitninni og við getum sett hvaða $m\times m$ fylki sem er inn í röðina.
Sem dæmi um fylkjaföll getum við tekið
\begin{align*}
e^A&=\sum\limits_{n=0}^\infty\dfrac 1{n!}{A^n}
=I+A+\dfrac {1}{2!}A^2+\dfrac{1}{3!}A^3+\cdots,\\
\cos A&= \sum\limits_{n=0}^\infty \dfrac{(-1)^n}{(2n)!}A^{2n}
=I-\dfrac{1}{2!}A^2+\dfrac{1}{4!}A^4-\cdots,\\
\sin A &=\sum\limits_{n=0}^\infty\dfrac{(-1)^n}{(2n+1)!}A^{2n+1}
= A-\dfrac {1}{3!}A^3+\dfrac{1}{5!}A^5-\cdots,\\
\cosh A&=\sum\limits_{n=0}^\infty\dfrac{1}{(2n)!}A^{2n}
=I+\dfrac{1}{2!}A^2+\dfrac{1}{4!}A^4+\cdots,\\
\sinh A &=\sum\limits_{n=0}^\infty\dfrac{1}{(2n+1)!}A^{2n+1}
= A+\dfrac {1}{3!}A^3+\dfrac{1}{5!}A^5+\cdots,\\
\ln (I+A) &= \sum\limits_{n=1}^\infty\dfrac{(-1)^{n+1}}{n}A^n
=A-\dfrac{1}{2}A^2+\frac{1}3A^3-\cdots,\\
(I-A)^{-1}&=\sum\limits_{n=0}^\infty A^n
=I+A+A^2+\cdots, \\
(I+A)^\alpha&= I+\alpha A+ \dfrac{\alpha(\alpha-1)}{2!}A^2 + 
\dfrac {\alpha(\alpha-1)(\alpha-2)}{3!}A^3+\cdots.
\end{align*}
Fyrstu fimm raðirnar eru vel skilgreindar fyrir öll $m\times m$
fylki, en hinar þrjár eru vel skilgreindar  ef $\|A\|<1$.  

 


\section{Veldisvísisfylkið\index{veldisvísisfylki}}

\noindent
Nú ætlum við að finna almenna formúlu fyrir grunnfylki fyrir línulegt
jöfnuhneppi með fastastuðla.  Í grein 9.3 sáum við hvernig
grunnfylkið lítur út í því tilfelli að eiginvigrar stuðlafylkisins
myndi grunn í $\C\sp m$.  Við byrjum á því að skoða rununa $\set
{u_n}$ sem skilgreind var í aðferð Picards til að sanna setningu
\tilv 6.7.5.  Hún er 
\begin{gather*}
u_0(t)=b,\\
u_1(t)=b+\int_0\sp t Ab \, d\tau = (I+tA)b,\\
u_2(t)=b+\int_0\sp t A(I+\tau A)b \, d\tau = (I+tA+\dfrac 12(tA)\sp 2)b,\\
u_3(t)=b+\int_0\sp t A(I+\tau A + \dfrac{\tau\sp 2}2A\sp 2)b \, d\tau 
= (I+tA+\dfrac 12(tA)\sp 2+\dfrac 1{3!}(tA)\sp 3)b,\\
u_n(t)= (I+tA+\dots+\dfrac 1{n!}(tA)\sp n)b.
\end{gather*}
Í sönnun okkar á tilvistarsetningunni sýndum við fram á að þessi runa
er samleitin í jöfnum mæli á sérhverju takmörkuðu bili á
rauntalnaásnum $\R$.
Með því að velja vigurinn $b$ sem grunnvigrana 
$$[1,0,\dots,0]^t, \ [0,1,0,\dots,0]^t\ \dots, 
\ [0,\dots,0,1]^t,$$ 
þá fáum við út úr aðferð Picards að fylkjaröðin
$\sum_{n=0}\sp \infty 
\dfrac 1{n!}(tA)\sp n$ er samleitin.  Við sjáum að hér er komin
veldaröðin fyrir veldisvísisfallið og sem grunnfylki fyrir
jöfnuhneppið $u\dash=Au$ fáum við síðan $\Phi(t)=e\sp {tA}$.  

\begin{se}
Fylkjafallið\index{fylkjafall} $\Phi(t)= e^{tA}$ er hin ótvírætt ákvarðaða lausn
upphafsgildisverkefnisins
 $$\Phi\dash(t) = A\Phi(t), \qquad t\in \R, \qquad \Phi(0)=I.
 $$
\end{se} 

\medskip
Nú skulum við sjá hvernig unnt er að nota tilvistarsetninguna fyrir
línuleg hneppi til þess að sanna samlagningarformúluna fyrir
fylkja\-veldis\-vísis\-fallið
\index{samlagnngarformúla!fylkjaveldisvísisfallsins}: 

\begin{se}\label{set5.5.2}
Ef $A$ og $B$ eru $m\times m$ fylki og $AB=BA$, þá er
 \begin{equation*}e\sp{A+B}=e\sp Ae\sp B=e^Be^A.\label{5.5.1}
 \end{equation*}
\end{se}

\begin{fs}
Fylkið $e\sp {tA}$ hefur andhverfuna $e\sp{-tA}$.
\end{fs}

Setningu \ref{set5.5.2} er ekki nokkur vandi að alhæfa:

\begin{se}
Ef $A$ og $B$ eru $m\times m$ fylki og $AB=BA$, $f$ og $g$ eru fáguð
föll á $S(0,\varrho)$, $\|A\|< \varrho$ og $\|B\|<\varrho$, þá er
 $$f(A)g(B)=g(B)f(A).
 $$
\end{se}



\begin{se}\label{set5.5.5}
Ef $A=TBT\sp{-1}$, $f(z)=\sum_{n=0}^\infty a_nz^n$ er fágað fall,
gefið  með veldaröð sem hefur samleitnigeisla $>\|A\|$, þá er
$f(A)=Tf(B)T\sp{-1}$. 
\end{se}

\medskip
Látum nú $A$ vera $m\times m$ fylki og gerum ráð því að
eiginvigrarnir $\varepsilon_1,\dots,\varepsilon_m$ með tilliti til
eigingildanna $\lambda_1,\dots,\lambda_m$ myndi grunn í $\C\sp m$.
Það er jafgilt því að unnt sé að þátta fylkið $A$ í
 $$A=T\Lambda T\sp{-1},
 $$
þar sem $\varepsilon_1,\dots,\varepsilon_m$ mynda dálkana í
$T$ og $\Lambda=\diag(\lambda_1,\dots,\lambda_m)$.
Setning \ref{set5.5.5} gefur nú
 $$e\sp{t A}=Te\sp{t\Lambda} T\sp{-1}.
 $$



\section{Cayley--Hamilton--setningin\index{Cayley-Hamilton-setningin}
\index{setning!Cayley-Hamilton}}

\noindent
Veldisvísisfylkið $e\sp {tA}$ af $m\times m$ fylki $A$,
 er gefið með óendanlegri veldaröð, sem
ekki er árennileg við fyrstu sýn.  Við ætlum nú að sýna fram á að
ætíð sé unnt að skrifa $e\sp{tA}$ á forminu
 $$e\sp{tA}= f_0(t)I+f_1(t)A+\cdots+f_{m-1}(t)A\sp{m-1}, 
 $$
þar sem föllin $f_0,\dots,f_{m-1}$ eru gefin með samleitnum
veldaröðum á $\R$.
Veldisvísisfallið $e\sp{tA}$ er sem sagt margliða í $A$ af stigi 
$\leq (m-1)$  með tvinntölustuðla sem eru háðir $t$.  

\begin{sk} Ef $A$ er $m\times m$ fylki með stuðla í $\C$, þá táknum við
kennimargliðu þess með $p_A(\lambda)$, 
 $$p_A(\lambda)=\det(\lambda I-A).
 $$
\end{sk}

 
Við getum skrifað
 $$
p_A(\lambda)=a_0+a_1\lambda+\cdots+a_{m-1}\lambda\sp{m-1}+\lambda\sp m
 $$
og jafnframt myndað fylkjamargliðuna $p_A(A)$, sem er $m\times m$ fylki,
með
því að setja $A$ inn í þessa formúlu,
 $$
p_A(A)=a_0I+a_1A+\cdots+a_{m-1}A\sp{m-1}+A\sp m.
 $$

\begin{se}\tx{Cayley--Hamilton}\index{Cayley-Hamilton-setningin}
\index{setning!Caylay-Hamilton}
Ef $A$ er $m\times m$ fylki, þá er $p_A(A)=0$.

{}
\end{se}

  Við athugum fyrst að setningin er algerlega augljós ef eiginvigrar
$A$ mynda grunn í $\C\sp m$, því þá er unnt að þátta fylkið $A$ í
$A=T\Lambda T\sp{-1}$, þar sem
$\Lambda=\diag(\lambda_1,\dots,\lambda_m)$ er hornalínufylkið með
eigingildin á hornalínunni og 
 $$p_A(A)=Tp_A(\Lambda)T\sp{-1}=
T\diag(p_A(\lambda_1),\dots,p_A(\lambda_m))T\sp{-1}=0
 $$
því eigingildin $\lambda_1,\dots,\lambda_m$ eru núllstövar
kennimargliðunnar $p_A$.

Nú skulum við athuga hvaða þýðingu setning Cayley--Hamilton hefur.
Ef við skrifum 
 $$p_A(\lambda)=\lambda\sp
m+a_{m-1}\lambda\sp{m-1}+\cdots+a_1\lambda+a_0,
 $$
þá gefur hún að 
 \begin{equation*}A\sp m=-a_0I-a_1A-\cdots-a_{m-1}A\sp {m-1}. \label{5.6.2}
 \end{equation*}
Með þrepun fáum við síðan að fyrir sérhvert $n\geq m$ eru til stuðlar
$c_{jn}$ þannig að 
 $$\dfrac 1{n!}A\sp n=
c_{0n}I+c_{1n}A+\cdots+c_{m-1,n}A\sp{m-1}.
 $$
Þegar við stingum þessu inn í veldaröðina fyrir $e\sp{tA}$, þá fáum við
 $$e\sp {tA}= \sum_{j=0}\sp {m-1}\bigg(
\sum_{n=0}\sp\infty c_{jn}t\sp n\bigg)A\sp j.
 $$
Þessi formúla er alls ekki svo fráleit til útreikninga á tölvu, því
við fáum rakningarformúlur fyrir stuðlana út frá (\ref{5.6.2}) og
\begin{multline*}
\dfrac 1{(n+1)!}A\sp{n+1} =\dfrac 1{n+1}A\cdot\dfrac 1{n!}A\sp n=\\
=\dfrac{c_{0n}}{n+1}A+\dfrac{c_{1n}}{n+1}A\sp 2+\cdots
+\dfrac{c_{m-1,n}}{n+1}A\sp m=\\
=\dfrac{-c_{m-1,n}a_0}{n+1}I+\dfrac{c_{0n}-c_{m-1,n}a_1}{n+1}A+
\cdots+\dfrac{c_{m-2,n}-c_{m-1,n}a_{m-1}}{n+1}A\sp{m-1}.
\end{multline*}
Stuðlarnir með númer $n=0,\dots,m-1$ eru gefnir með 
 $$\begin{matrix}
 & c_{0n}& c_{1n}&\dots&c_{(m-1),n}\\
n=0&1/0!&0&\dots&0\\
n=1&0&1/1!&\dots&0\\
\vdots&\vdots&\vdots&\ddots&\vdots\\
n=m-1&0&0&\dots&1/n!.
\end{matrix}
 $$
Rakningarformúlurnar fyrir stuðlana með númer $n\geq m$ verða síðan
\begin{align*}
c_{0,n+1}&= \dfrac{-c_{m-1,n}a_0}{n+1},\\
c_{j,n+1}&= \dfrac{c_{j-1,n}-c_{m-1,n}a_j}{n+1}, 
\qquad j=1,\dots,m-1.
\end{align*}
Það  er greinilega auðvelt að forrita þetta í tölvu.
Lausnin á upphafsgildisverkefninu $u\dash=Au$, $u(0)=b$ er síðan
 $$u(t) =e\sp{tA}b = 
\bigg( \sum_{n=0}\sp\infty c_{0n}t\sp n\bigg) b_0+\cdots+
\bigg( \sum_{n=0}\sp \infty c_{m-1,n}t\sp n\bigg) b_{m-1},
 $$
þar sem vigrarnir $b_0,\dots, b_{m-1}$ eru reiknaðir út frá
 $$
b_0=b, \qquad b_1=Ab, \qquad b_2=A\sp 2b=Ab_1, \dots,
b_{m-1}=A\sp{m-1}b=Ab_{m-2}.
 $$



\section{Newton-margliður\index{margliða!Newton}}


\subsection*{Brúunarverkefni}

Látum $f\in \O(\C)$ vera gefið fall, látum $\alpha_1,\dots,\alpha_\ell$
vera ólíka punkta í $\C$, látum $m_1,\dots,m_\ell$ vera jákvæðar
heiltölur og setjum $m=m_1+\cdots+m_\ell$.  Nú ætlum við að sýna fram
á að það verkefni að finna margliðu $r$ af
stigi $<m$, sem uppfyllir
 \begin{equation*}f^{(j)}(\alpha_k) = r^{(j)}(\alpha_k), \qquad j=0,\dots,m_k-1, \quad
k=1,\dots, \ell,\label{10.13.1}
 \end{equation*}
hafi ótvírætt ákvarðaða lausn $r$ og við ætlum jafnframt að finna
formúlu fyrir margliðuna $r$.   Verkefni af þessu tagi nefnist
{\it brúunarverkefni}.
Síðan munum við sjá hvernig
þessar formúlur eru notaðar til þess að reikna út veldisvísisfylkið
$e^{tA}$.


\subsection*{Úrlausn á brúunarverkefninu}

Við skilgreinum rununa
$\lambda_1,\dots,\lambda_m$ með því að telja 
$\alpha_1,\dots,\alpha_\ell$ með margfeldni, þannig að fyrstu $m_1$
gildin á $\lambda_j$ séu $\alpha_1$, næstu $m_2$ gildin á $\lambda_j$
séu $\alpha_2$ o.s.frv.  Við skilgreinum síðan
 \begin{equation*}p(z)=(z-\alpha_1)^{m_1}\cdots(z-\alpha_\ell)^{m_\ell}
=(z-\lambda_1)\cdots(z-\lambda_m).\label{10.13.2}
 \end{equation*}

Athugum sértilfellið þegar $\ell=1$.  Þá getum við skrifað lausnina
$r$  beint niður því hún er Taylor-margliða
fallsins $f$ í punktinum $\alpha_1$ númer $m-1$,
$$
r(z)=f(\alpha_1)+f'(\alpha_1)(z-\alpha_1)+\cdots +
\frac {f^{(m-1)}(\alpha_1)}{(m-1)!}(z-\alpha_1)^{m-1}.
$$
Almenna niðurstaðan er:

\begin{se}  Látum $f\in \O(\C)$, $\alpha_1,\dots,\alpha_\ell$ vera ólíka
punkta í $\C$, $m_1,\dots,m_\ell$ vera jákvæðar heiltölur, setjum
$m=m_1+\cdots+m_\ell$ og skilgreinum $p(z)$ með (\ref{10.13.2}).  Þá er til
margliða $r$ af stigi $<m$
og $g\in \O(\C)$ þannig að
\begin{equation*}
f(z)=r(z)+p(z)g(z), \qquad z\in \C.\label{10.13.3}
\end{equation*}
Margliðan $r$ er lausn á  (\ref{10.13.1}).
Bæði $r$ og $g$ eru ótvírætt ákvörðuð og eru gefin með
formúlunum
\begin{align*}
r(z)=f[\lambda_1]&+f[\lambda_1,\lambda_2](z-\lambda_1)+\cdots\\
&+ f[\lambda_1,\dots,\lambda_m](z-\lambda_1)\cdots(z-\lambda_{m-1})
\end{align*}
og 
$$
g(z)=f[\lambda_1,\dots,\lambda_m,z](z-\lambda_1)\cdots(z-\lambda_m),
$$
þar sem mismunakvótarnir\index{mismunakvóti} eru skilgreindir með
\begin{equation*}
f[\lambda_i,\dots,\lambda_{i+j}]=
\begin{cases}\dfrac{f^{(j)}(\lambda_i)}{j!},& 
\lambda_i=\cdots=\lambda_{i+j}, \\
\dfrac{f[\lambda_i,\dots,\lambda_{i+j-1}]-f[\lambda_{i+1},\dots,\lambda_{i+j}]}
{\lambda_i-\lambda_{i+j}},&\lambda_i\neq \lambda_{i+j}, 
\end{cases}
\label{10.13.4}
\end{equation*}
fyrir $i=1,\dots,m$ og $j=0,\dots,m-i$.
 \end{se}


\medskip
Framsetningin  á brúunarmargliðunni
$r$, sem við notum hér, er kennd við
Newton\index{Newton}\index{Newton!margliða}.  Í þessari útleiðslu
höfum við gert ráð fyrir því að $f$ sé fágað á öllu $\C$.  En með
því að huga vel að valinu á veginum sem heildað er yfir, þá er hægt að
sýna fram á að þessar formúlur gildi í hvaða svæði sem er.


\subsection*{Newton-margliður}


\noindent
 Nú segir setning
Cayley--Hamilton okkur að sérhvert veldi $A\sp n$ af 
$m\times m$ fylkinu  $A$ með $n\geq
m$ megi skrifa sem línulega samantekt af $I,A,\dots, A\sp {m-1}$, og
af því leiðir að fylkjafall $f(A)$, sem gefið er
með samleitinni veldaröð, er í raun margliða í $A$ af stigi $\leq
(m-1)$.  Nú viljum við reikna út þessa margliðu og nota til þess
fallgildin $f(z)$.  Í tilfellinu $m=4$ þurfum við fyrst að reikna
út mismuakvótatöfluna
 $$\begin{matrix}
f[\lambda_1]\\
            &f[\lambda_1,\lambda_2]\\
f[\lambda_2]&                       &f[\lambda_1, \lambda_2, \lambda_3]\\
        &f[\lambda_2,\lambda_3]& &f[\lambda_1,\lambda_2,\lambda_3,\lambda_4]\\
f[\lambda_3]&                       &f[\lambda_2, \lambda_3, \lambda_4]\\
            &f[\lambda_3,\lambda_4]\\
f[\lambda_4]
\end{matrix}
 $$
þar sem $\lambda_1,\dots,\lambda_4$ er upptalning með margfeldni
á núllstöðvum kennimargliðu $A$.
Margliðan $r(z)$ er síðan reiknuð út frá hornalínustökunum
\begin{align*}
r(z)&=f[\lambda_1]+f[\lambda_1,\lambda_2](z-\lambda_1)
+f[\lambda_1, \lambda_2, \lambda_3](z-\lambda_1)(z-\lambda_2)\\
&+f[\lambda_1, \lambda_2, \lambda_3,\lambda_4]
(z-\lambda_1)(z-\lambda_2)(z-\lambda_3).
\end{align*}
Fylkið $f(A)$ fæst nú með því að stinga $A$ inn í formúluna í stað 
breytunnar $z$ og setja $I$ inn í stað allra fastaliða í
margliðuþáttum,
\begin{align*}
f(A)&=f[\lambda_1]I+f[\lambda_1,\lambda_2](A-\lambda_1I)
+f[\lambda_1, \lambda_2, \lambda_3](A-\lambda_1I)(A-\lambda_2I)\\
&+f[\lambda_1, \lambda_2, \lambda_3,\lambda_4]
(A-\lambda_1I)(A-\lambda_2I)(A-\lambda_3I).
\end{align*}

\subsection*{Veldisvísisfylkið}

Við fórum út í þetta æfintýri til þes að reikna út margliðuna
$e^{tA}$, sem byggir á fallinu $f(z)=e^{tz}$, þar sem $t$ er
raunbreytistærð.  Afleiðurnar eru
 $$f\dash(z)=te^{tz}, \qquad f\ddash(z)=t^2e^{tz}, \qquad
f\tdash(z)=t^3e^{tz}, \qquad \dots.
 $$
Margliðan $p$ verður síðan kennimargliða
fylkisins\index{kennimargliða!fylkis}\index{margliða!kennimargliða} $A$.

\begin{sy}\label{syn5.7.2} (i) Gerum ráð fyrir að $A$ sé $2\times 2$ fylki með ólík
eigingildi $\alpha_1$ og $\alpha_2$.  Þá er 
kennimargliðan $p_A(z)=(z-\alpha_1)(z-\alpha_2)$ og mismunakvótataflan
 $$\begin{matrix}
e^{t\alpha_1}\\
&\dfrac{e^{t\alpha_1}-e^{t\alpha_2}}{\alpha_1-\alpha_2}\\
e^{t\alpha_2}
\end{matrix}
 $$
og við fáum 
 $$e\sp{tz} = e\sp{t\alpha_1}+ 
\dfrac{e^{t\alpha_1}-e^{t\alpha_2}}{\alpha_1-\alpha_2}
(z-\alpha_1) +(z-\alpha_1)(z-\alpha_2)g(z),
 $$
sem gefur okkur formúluna fyrir $e\sp{tA}$, 
 $$e^{tA}=e^{t\alpha_1}I+\dfrac{e^{t\alpha_1}-e^{t\alpha_2}}
{\alpha_1-\alpha_2}(A-\alpha_1I).
 $$
(ii) Ef hins vegar $A$ er $2\times 2$ fylki með aðeins eitt
eigingildi $\alpha_1$, þá verður mismunakvótataflan
 $$\begin{matrix}
e^{t\alpha_1}\\
&te^{t\alpha_1}\\
e^{t\alpha_1}
\end{matrix}
 $$
og við fáum 
 $$e\sp{tz}=e\sp{t\alpha_1}+te\sp{t\alpha_1}(z-\alpha_1)+(z-\alpha_1)\sp
2g(z). 
 $$
Veldisvísisfylkið verður því
 $$e^{tA}=e^{t\alpha_1}I+te^{t\alpha_1}(A-\alpha_1I).
 $$
(iii)  Ef $A$ er $3\times 3$ fylki með þrjú ólík eigingildi,
${\alpha}_1,{\alpha}_2,{\alpha}_3$ þá verður mismunakvótataflan
 $$\begin{matrix}
e^{t{\alpha}_1}\\
&\dfrac{e^{t\alpha_1}-e^{t\alpha_2}}{\alpha_1-\alpha_2}\\
e^{t\alpha_2}& 
&\dfrac1{\alpha_1-\alpha_3}\left\{
\dfrac{e^{t\alpha_1}-e^{t\alpha_2}}{\alpha_1-\alpha_2}-
\dfrac{e^{t\alpha_2}-e^{t\alpha_3}}{\alpha_2-\alpha_3}
\right\}\\ 
&\dfrac{e^{t\alpha_2}-e^{t\alpha_3}}{\alpha_2-\alpha_3}\\
e^{t\alpha_3}\\
\end{matrix}
 $$
og formúlan fyrir $e^{tA}$ verður
\begin{multline*}
e^{tA}=e^{t\alpha_1}I+\dfrac{e^{t\alpha_1}-e^{t\alpha_2}}
{\alpha_1-\alpha_2}(A-\alpha_1I)+\\
+\dfrac1{\alpha_1-\alpha_3}\left\{
\dfrac{e^{t\alpha_1}-e^{t\alpha_2}}{\alpha_1-\alpha_2}-
\dfrac{e^{t\alpha_2}-e^{t\alpha_3}}{\alpha_2-\alpha_3}
\right\} (A-\alpha_1I)(A-\alpha_2I).
\end{multline*}
(iv)  Ef $A$ er $3\times 3$ fylki með tvö ólík eigingildi,
$\alpha_1$ tvöfalt og $\alpha_2$ einfalt, þá verður
mismunakvótataflan 
 $$\begin{matrix}
e^{t\alpha_1}\\
&te^{t\alpha_1}\\
e^{t\alpha_1}& 
&\dfrac1{\alpha_1-\alpha_2}\left\{te^{t\alpha_1}-
\dfrac{e^{t\alpha_1}-e^{t\alpha_2}}{\alpha_1-\alpha_2}\right\}\\ 
&\dfrac{e^{t\alpha_1}-e^{t\alpha_2}}{\alpha_1-\alpha_2}\\
e^{t\alpha_2}\\
\end{matrix}
 $$
og formúlan verður
 $$e^{tA}=e^{t\alpha_1}I+te^{t\alpha_1}(A-\alpha_1I)+
\dfrac1{\alpha_1-\alpha_2}\left\{te^{t\alpha_1}-
\dfrac{e^{t\alpha_1}-e^{t\alpha_2}}{\alpha_1-\alpha_2}\right\}
(A-\alpha_1I)^2.
 $$
(v) Að lokum skulum við líta á tilfellið að $A$ sé $3\times 3$ fylki
með eitt eigingildi $\alpha_1$.  Mismunakvótataflan verður þá
einfaldlega
 $$\begin{matrix}
e^{t\alpha_1}\\
&te^{t\alpha_1}\\
e^{t\alpha_1}& 
&\dfrac{t^2}{2}e^{t\alpha_1}\\ 
&te^{t\alpha_1}\\
e^{t\alpha_1}\\
\end{matrix}
 $$
og veldisvísisfylkið verður
$$e^{tA}=e^{t\alpha_1}I+te^{t\alpha_1}(A-\alpha_1I)+
\dfrac{t^2}2e^{t\alpha_1}(A-\alpha_1I)^2.
 $$
\end{sy}


Hugsum okkur nú að við séum að finna lausn á
upphafsgildisverkefninu $u\dash=Au$, $u(0)=b$, þar sem $A$ er
$3\times 3$ fylki með eitt eigingildi $\alpha_1$.  Formúlan í
sýnidæmi \ref{syn5.7.2} (v) gefur
 $$e^{tA}b=e^{t\alpha_1}b_0+te^{t\alpha_1}b_1+\dfrac {t^2}2e^{t\alpha_1}b_2
 $$
þar sem 
 $$b_0=b, \qquad b_1=(A-\alpha_1I)b_0, \qquad b_2=(A-\alpha_1I)b_1.
 $$
Athugið að hér væri ákaflega heimskulegt að reikna fyrst út fylkið
$(A-\alpha_1I)^2$ og margfalda það síðan með $b$ til að fá $b_2$, því
það kostar almennt margfalt meiri vinnu en við þurfum að framkvæma í
þeirri aðferð sem hér er lýst.



