
%
%Allir pakkar sem þarf að nota.
%
\usepackage[utf8]{inputenc}
\usepackage[T1]{fontenc}
\usepackage[icelandic]{babel}
\usepackage{amsmath}
\usepackage{amssymb}
\usepackage{pictex}
\usepackage{epsfig,psfrag}
\usepackage{makeidx}
%\selectlanguage{icelandic}
%----------------------------

%
\hoffset=-0.4truecm
\voffset=-1truecm
\textwidth=16truecm 
%\textwidth=12truecm 
\textheight=23truecm
\evensidemargin=0truecm
%
% Gömlu gildin á bókinni 
%
%\voffset 1.4truecm
%\hoffset .25truecm
%\vsize  16.0truecm
%\hsize  15truecm
%
%
% Skilgreiningar á ýmsum skipunum.
%
%\newcommand{\Sb}{
%$$
%\sum_{\footnotesize\begin{array}{l} j=1 \\ j\neq k \end{array}}
%$$
%}
\newcommand{\bolddot}{{\mathbf \cdot}}
\newcommand{\C}{{\mathbb  C}}
\newcommand{\Cn}{{\mathbb  C\sp n}}
\newcommand{\crn}{{{\mathbb  C\mathbb  R^n}}}
\newcommand{\R}{{\mathbb  R}}
\newcommand{\Rn}{{\mathbb  R\sp n}}
\newcommand{\Rnn}{{\mathbb  R\sp{n\times n}}}
\newcommand{\Z}{{\mathbb  Z}}
\newcommand{\N}{{\mathbb  N}}
\renewcommand{\P}{{\mathbb  P}}
\newcommand{\Q}{{\mathbb  Q}}
\newcommand{\K}{{\mathbb  K}}
\newcommand{\U}{{\mathbb  U}}
\newcommand{\D}{{\mathbb  D}}
\newcommand{\T}{{\mathbb  T}}
\newcommand{\A}{{\cal A}}
\newcommand{\E}{{\cal E}}
\newcommand{\F}{{\cal F}}
\renewcommand{\H}{{\cal H}}
\renewcommand{\L}{{\cal L}}
\newcommand{\M}{{\cal M}}
\renewcommand{\O}{{\cal O}}
\renewcommand{\S}{{\cal S}}
\newcommand{\dash}{{\sp{\prime}}}
\newcommand{\ddash}{{\sp{\prime\prime}}}
\newcommand{\tdash}{{\sp{\prime\prime\prime}}}
\newcommand{\set }[1]{{\{#1\}}}
\newcommand{\scalar}[2]{{\langle#1,#2\rangle}}
\newcommand{\arccot}{{\operatorname{arccot}}}
\newcommand{\arccoth}{{\operatorname{arccoth}}}
\newcommand{\arccosh}{{\operatorname{arccosh}}}
\newcommand{\arcsinh}{{\operatorname{arcsinh}}}
\newcommand{\arctanh}{{\operatorname{arctanh}}}
\newcommand{\Log}{{\operatorname{Log}}}
\newcommand{\Arg}{{\operatorname{Arg}}}
\newcommand{\grad}{{\operatorname{grad}}}
\newcommand{\graf}{{\operatorname{graf}}}
\renewcommand{\div}{{\operatorname{div}}}
\newcommand{\rot}{{\operatorname{rot}}}
\newcommand{\curl}{{\operatorname{curl}}}
\renewcommand{\Im}{{\operatorname{Im\, }}}
\renewcommand{\Re}{{\operatorname{Re\, }}}
\newcommand{\Res}{{\operatorname{Res}}}
\newcommand{\vp}{{\operatorname{vp}}}
\newcommand{\mynd}[1]{{{\operatorname{mynd}(#1)}}}
\newcommand{\dbar}{{{\overline\partial}}}
\newcommand{\inv}{{\operatorname{inv}}}
\newcommand{\sign}{{\operatorname{sign}}}
\newcommand{\trace}{{\operatorname{trace}}}
\newcommand{\conv}{{\operatorname{conv}}}
\newcommand{\Span}{{\operatorname{Sp}}}
\newcommand{\stig}{{\operatorname{stig}}}
\newcommand{\Exp}{{\operatorname{Exp}}}
\newcommand{\diag}{{\operatorname{diag}}}
\newcommand{\adj}{{\operatorname{adj}}}
\newcommand{\erf}{{\operatorname{erf}}}
\newcommand{\erfc}{{\operatorname{erfc}}}
\newcommand{\Lloc}{{L_{\text{loc}}\sp 1}}
\newcommand{\boldcdot}{{\mathbb \cdot}}
%\newcommand{\Cinf0}[1]{{C_0\sp{\infty}(#1)}}
\newcommand{\supp}{{\text{supp}\, }}
\newcommand{\chsupp}{{\text{ch supp}\, }}
\newcommand{\singsupp}{{\text{sing supp}\, }}
\newcommand{\SL}[1]{{\dfrac {1}{\varrho} 
\bigg(-\dfrac d{dx}\bigg(p\dfrac {d#1}{dx}\bigg)+q#1\bigg)}}
\newcommand{\SLL}[1]{-\dfrac d{dx}\bigg(p\dfrac {d#1}{dx}\bigg)+q#1}
\newcommand{\Laplace}[1]{\dfrac{\partial^2 #1}{\partial x^2}+\dfrac{\partial^2 #1}{\partial y^2}}
\newcommand{\polh}[1]{{\widehat #1_{\C^n}}}
\newcommand{\tilv}{{}}
%
\renewcommand{\chaptername}{Kafli}
%
% Númering á formæulum.
%
\numberwithin{equation}{section}
%
%  Innsetning á myndum.
%
\def\figura#1#2{
\vbox{\centerline{
\input #1
}
\centerline{#2}
}\medskip}
\def\vfigura#1#2{
\setbox0\vbox{{
\input #1
}}
\setbox1\vbox{\hbox{\box0}\hbox{{\obeylines #2}}}
\dimen0 = -\ht1
\advance\dimen0 by-\dp1
\dimen1 = \wd1
\dimen2 = -\dimen0
\divide\dimen2 by\baselineskip
\count100 = 1
\advance\count100 by\dimen2
\advance\count100 by1
\box1
\hangindent\dimen1
\hangafter=-\count100
\vskip\dimen0
}
%
%  Setningar, skilgreiningar, o.s.frv. 
%
\newtheorem{setning+}           {Setning}      [section]
\newtheorem{skilgreining+}  [setning+]  {Skilgreining}
\newtheorem{setningogskilgreining+}  [setning+]  {Setning og
skilgreining}
\newtheorem{hjalparsetning+}  [setning+]  {Hjálparsetning}
\newtheorem{fylgisetning+}  [setning+]  {Fylgisetning}
\newtheorem{synidaemi+}  [setning+]  {Sýnidæmi}
\newtheorem{forrit+}  [setning+]  {Forrit}

\newcommand{\tx}[1]{{\rm({\it #1}). \ }}

\newenvironment{se}{\begin{setning+}\sl}{\hfill$\square$\end{setning+}\rm}
\newenvironment{sex}{\begin{setning+}\sl}{\hfill$\blacksquare$\end{setning+}\rm}
\newenvironment{sk}{\begin{skilgreining+}\rm}{\hfill$\square$\end{skilgreining+}\rm}
\newenvironment{sesk}{\begin{setningogskilgreining+}\rm}{\hfill$\square$\end{setningogskilgreining+}\rm}
\newenvironment{hs}{\begin{hjalparsetning+}\sl}{\hfill$\square$\end{hjalparsetning+}\rm}
\newenvironment{fs}{\begin{fylgisetning+}\sl}{\hfill$\square$\end{fylgisetning+}\rm}
\newenvironment{sy}{\begin{synidaemi+}\rm}{\hfill$\square$\end{synidaemi+}\rm}
\newenvironment{fo}{\begin{forrit+}\rm}{\hfill\end{forrit+}\rm}
\newenvironment{so}{\medbreak\noindent{\it Sönnun:}\rm}{\hfill$\blacksquare$\rm}
\newenvironment{sotx}[1]{\medbreak\noindent{\it #1:}\rm}{\hfill$\blacksquare$\rm}
\newcounter{daemateljari}
\newcommand{\aefing}{\section{Æfingardæmi} \setcounter{daemateljari}{1}}
\newcommand{\daemi}{
{\medskip\noindent{\bf \thedaemateljari.}}
\addtocounter{daemateljari}{1}
}

%\def\aefing{{\large\bf\bigskip\bigskip\noindent Æfingardæmi}}
%\def\daemi#1{\medskip\noindent{\bf #1.}}
\def\svar#1{\smallskip\noindent{\bf #1.} \ }
\def\lausn#1{\smallskip\noindent{\bf #1.} \ }
\def\ugrein#1{\medbreak\noindent{\bf #1.} }
\newcommand{\samantekt}{\noindent{\bf Samantekt.} }
%\newcommand{\proclaimbox}{\hfill$\square$}

%

\chapter {LAPLACE--UMMYNDUN}
 

\section{ Skilgreiningar og reiknireglur}

\subsection{ Skilgreiningar og reiknireglur}  

\noindent
Látum $f$ vera fall sem skilgreint er á 
$\R_+=\{t\in \R; t\geq 0\}$ með gildi í $\C$ og gerum ráð fyrir að
$f$ sé heildanlegt á sérhverju lokuðu og takmörkuðu bili $[0,b]$.


{\it Laplace--mynd} $f$, 
sem við táknum með $\L f$ eða $\L\{f\}$, er skilgreind með
formúlunni
 \begin{equation*}\L f(s)=\int_0\sp \infty e\sp{-st}f(t)\, dt.

.. _7.1.1:

 \end{equation*}
Skilgreiningarmengi fallsins $\L f$ samanstendur af öllum
tvinntölum $s$ þannig að heildið í hægri hliðinni sé samleitið.  


{\it Laplace-ummyndun} er vörpunin $\L$ sem úthlutar falli $f$ 
Laplace-mynd sinni $\L f$.



.. _skil10.1.1a:

\subsubsection{Skilgreining}
Við segjum að fallið $f:\R_+\to \C$ sé af
veldisvísisgerð :hover:`veldisvísisgerð` ef til eru
jákvæðir fastar $M$ og $c$ þannig að 
 \begin{equation*}|f(t)|\leq Me\sp{c t}, \qquad t\in \R_+.

.. _7.1.2:

 \end{equation*}


--------------



Ef $f$ er heildanlegt á sérhverju takmörkuðu bili $[0,b]$ og uppfyllir 
ójöfnuna, þá er $\L f$ skilgreint fyrir öll $s\in \C$ með $\Re s
>c$.   Við fáum að auki
vaxtartakmarkanir á $\L f$,
 \begin{equation*}
|\L f(s) |\leq \int_0\sp\infty e\sp{-\Re st} Me\sp{c t} \, dt =
\dfrac M{\Re\,  s-c}, \qquad \Re\,  s>c.


.. _7.1.3:

 \end{equation*}


Það er augljóst að Laplace-ummyndun er línuleg vörpun, en það þýðir að
$$
\L\{\alpha f+\beta g\}(s)=\alpha\L\{f\}(s)+\beta\L\{g\}(s)
$$
ef $f$ og $g$ eru föll af veldisvísisgerð, $\alpha$ og $\beta$ eru
tvinntölur og $s\in \C$ liggur í skilgreiningarmengi fallanna
$\L\{f\}$ og $\L\{g\}$.

\smallskip
Við þurfum að leiða út nokkrar reiknireglur fyrir Laplace-ummyndun.
Sú fyrsta segir okkur að Laplace-myndir falla af veldisvísisgerð séu
fáguð föll og hún segir okkur einnig að afleiður af Laplace-myndum 
af slíkum föllum  séu  einnig Laplace myndir: 



.. _se10.1.2a:

\subsubsection{Setning}  Látum $f:\R_+\to \C$ vera fall sem er heildanlegt á
sérhverju bili $[0,b]$ og   uppfyllir (:ref:`7.1.2`).  Þá er $\L f$ 
fágað á menginu $\{s\in \C;\Re s>c\}$ og 
 \begin{equation*}\dfrac{d^k}{ds^k}\L\{f\}(s)=
(-1)^k\L\{t^kf(t)\}(s), \qquad \Re s>c.


.. _7.1.4:

 \end{equation*}


--------------



\subsection*{Nokkur mikilvæg dæmi}


Reiknum nú út nokkrar Laplace-myndir:


Ef $\alpha\in \R$ og  $\alpha>-1$, þá er
\begin{align*}
\L\{t^\alpha\}(s)
&=\int_0^\infty e^{-st}t^\alpha \, dt =
\dfrac 1{s^{\alpha+1}} \int_0^\infty e^{-st}(st)^\alpha \, s dt \\
&=
\dfrac 1{s^{\alpha+1}} \int_0^\infty e^{-\tau}\tau^\alpha \,  d\tau =
\dfrac {\Gamma(\alpha+1)}{s^{\alpha+1}}.
\end{align*}
Ef ${\alpha}$ er heiltala, þá verður þessi formúla
$$
\L\{t^\alpha\}(s)
=\dfrac {\alpha!}{s^{\alpha+1}}.
 $$

\smallskip
Fyrir sérhvert $\alpha\in \C$ gildir
 $$
\L\{e^{\alpha t}\}(s)=
\int_0^{\infty}e^{-st}e^{\alpha t}\, dt =
\int_0^{\infty}e^{-(s-\alpha)t}\, dt =
\left[\dfrac {-e^{-(s-\alpha)t}} {s-\alpha}\right]_0^{\infty}=
\dfrac 1{s-\alpha},
 $$
og í framhaldi af þessu fáum við
\begin{align*}
\L\{\cos\beta t\}(s) &=
\frac 12 \L\{e^{i\beta t}\}(s) +\frac 12\L\{e^{-i\beta t}\}(s)\\
&=\frac 12\left[\dfrac 1{s-i\beta}+\dfrac 1{s+i\beta}\right]
=\dfrac s{s^2+\beta^2},\\
\L\{\sin\beta t\}(s) &=
\frac 1{2i}\L\{e^{i\beta t}\}(s) -\frac 1{2i}\L\{e^{-i\beta t}\}(s)\\
&=\frac 1{2i}\left[\dfrac 1{s-i\beta}-\dfrac 1{s+i\beta}\right]
=\dfrac {\beta}{s^2+\beta^2},\\
\L\{\cosh \beta t\}(s) &= 
\frac 12 \L\{e^{\beta t}\}(s) +\frac 12\L\{e^{-\beta t}\}(s)\\
&=\frac 12\left[\dfrac 1{s-\beta}+\dfrac 1{s+\beta}\right]
=\dfrac s{s^2-\beta^2},\\
\L\{\sinh \beta t\}(s) &= 
\frac 1{2}\L\{e^{\beta t}\}(s) -\frac 1{2}\L\{e^{-i\beta t}\}(s)\\
&=\frac 1{2}\left[\dfrac 1{s-\beta}-\dfrac 1{s+\beta}\right]
=\dfrac \beta{s^2-\beta^2}.
\end{align*}


Við höfum almenna reiknireglu:

\subsubsection{Setning} $\L\{e^{\alpha t}f\}(s) = \L\{f\}(s-\alpha)$.


--------------



Útreikninga okkar hér að framan getum við  nú  tekið saman í litla
töflu: 
\begin{align*}
\L\{e^{\alpha t}t^{\beta}\}(s)
&=\dfrac{\Gamma(\beta+1)}{(s-\alpha)^{\beta+1}},\\
\L\{e^{\alpha t}\cos \beta t\}(s)
&=\dfrac{s-\alpha}{(s-\alpha)^2+\beta^2},\\
\L\{e^{\alpha t}\sin \beta t\}(s)
&=\dfrac{\beta}{(s-\alpha)^2+\beta^2},\\
\L\{e^{\alpha t}\cosh \beta t\}(s)
&=\dfrac{s-\alpha}{(s-\alpha)^2-\beta^2},\\
\L\{e^{\alpha t}\sinh \beta t\}(s)
&=\dfrac{\beta}{(s-\alpha)^2-\beta^2}.
\end{align*}


\subsection*{Laplace-ummyndun er eintæk vörpun}



.. _set:10.1.2:

\subsubsection{Setning}
Gerum ráð fyrir að föllin $f,g\in C(\R_+)$ séu bæði af
veldisvísisgerð og að til sé fasti $c$ þannig að 
 $$\L f(s)=\L g(s), \qquad s\in \C, \quad \Re s\geq c.
 $$
Þá er $f(t)=g(t)$ fyrir öll $t\in \R_+$. 


--------------




Þessa setningu má einnig orða þannig að Laplace-ummyndun er eintæk
vörpun á mengi allra samfelldra falla af veldisvísisgerð.  
Ef við sjáum að eitthvert fall $F(s)$ er Laplace-mynd af samfelldu 
falli $f$, 
þá segir setningin okkur að $f$ er ótvírætt ákvarðað og við köllum þá
$f$ {\it andhverfa Laplace-mynd } af fallinu $F$ og skrifum
$f(t)=\L^{-1}\{F\}(t)$.


\subsection*{Heaviside-fallið}

Fallið $H:\R\to \R$, sem skilgreint er með
\begin{equation*}
H(t)=\begin{cases} 1, &t\geq 0,\\ 0, & t<0,\end{cases}

.. _7.1.5:

\end{equation*}
kallast {\it Heaviside--fall :hover:`Heaviside-fall`}.  Athugum að
hliðrun þess $H_a(t)=H(t-a)$ uppfyllir
\begin{equation*}
H_a(t)=\begin{cases} 1, &t\geq a,\\ 0, & t<a,\end{cases}

.. _7.1.6:

\end{equation*}
og því er Laplace-mynd þess
\begin{equation*}
\L H_a(s)= \int_a^{\infty} e^{-st}\, dt= \dfrac{e^{-as}} s, \qquad a>0.


.. _7.1.7:

\end{equation*}
Við fáum reyndar almenna reiknireglu:

\subsubsection{Setning}  Látum $f:\R_+\to \C$ vera fall af veldisvísisgerð.  Þá gildir um
sérhvert $a\geq 0$ að
$$
\L\{H(t-a)f(t-a)\}(s) = e^{-as}\L\{f\}(s).
$$
þar sem fallið $t\mapsto H(t-a)f(t-a)$ tekur gildið $0$ fyrir öll
$t<a$. 


--------------



\subsection*{Laplace-ummyndun af vigur- og fylkjagildum vörpunum}

Ef $u=(u_1,\dots,u_m): \R_+\to \C^m$ er vigurgilt fall á jákvæða
raunásnum, þá skilgreinum við Laplace-mynd $u$ með því að taka
Laplace-mynd af hnitaföllunum, 
$$\L u(s)=(\L u_1,\dots,\L u_m).
$$ 
Við
förum eins að við að skilgreina Laplace-mynd af $p\times m$-fylkjagildu
falli $U=(u_{jk})_{j,k=1}^{p,m}$, þar sem við skilgreinum $\L U(s)$ sem
$p\times m$ fylkjagilda fallið  
$$\L U(s)=(\L u_{jk}(s))_{j,k=1}^{p,m}.
$$
Ef $A$ er $p\times m$ fylki, þá er 
\begin{equation*}
\L\{Au\}(s)=A\L u(s).


.. _7.1.8:

\end{equation*}
Þessa reglu sönnum við  með því að líta á $v=Au$,
$v_j=a_{j1}u_1+\cdots+a_{jm}u_m$ og notfæra okkur að Laplace-ummyndunin
er línuleg vörpun.  Það gefur okkur $\L v_j(s)=a_{j1}\L u_1(s)+\cdots+a_{jm}\L
u_m(s)$.  Vinstri hliðin í þessari jöfnu er þáttur númer $j$ í vinstri
hlið jöfnunnar, en hægri hliðin er þáttur númer $j$ í hægri hlið hennar.


Ef hins $A$ er eitthvert $q\times p$ fylki, þá fæst reglan
\begin{equation*}
\L\{AU\}(s)=A\L U(s).


.. _7.1.9:

\end{equation*}



\section{Upphafsgildisverkefni}

\subsection{Upphafsgildisverkefni}

\noindent
Nú skulum við snúa okkur að kjarna málsins, en það er að 
taka  fall $f\in C\sp
1(\R_+)$ af veldisvísisgerð og reikna út heildið
\begin{align*}
\int_0\sp b e\sp{-st}f\dash(t)\, dt &=
\left[e\sp{-st}f(t)\right]_0\sp b+
\int_0\sp b se\sp{-st}f(t)\, dt \\
&=
s\int_0\sp b e\sp{-st}f(t)\, dt -f(0)+e\sp{-sb}f(b).
\end{align*}
Ef $\Re s$ er nógu stórt, þá getum við látið $b\to \infty$ og fáum því
 \begin{equation*}\L\{f\dash\}(s)=s\L\{f\}(s)-f(0).


.. _7.3.1:

 \end{equation*}
Ef við gerum ráð fyrir að $f\in C^2(\R_+)$ og að bæði $f$ og $f\dash$
séu af veldisvísisgerð, þá fáum við með því að
beita þessari formúlu tvisvar  að 
 \begin{equation*}\L\{f\ddash\}(s)=s\L\{f\dash\}(s)-f\dash(0)=s\sp 2\L\{f\}(s)
-sf(0)-f\dash(0),


.. _7.3.2:

 \end{equation*}
og með þrepun fáum við síðan:

\subsubsection{Setning} 
Ef $f\in C\sp m(\R_+)$ og  $f, f\dash, f\ddash,
\dots, f\sp{(m-1)}$,  eru af veldisvísisgerð, þá er $\L\{f\sp{(m)}\}(s)$
skilgreint fyrir öll $s\in \C$ með  $\Re s$ nógu stórt og
 \begin{equation*}\L\{f\sp{(m)}\}(s)=s\sp
m\L\{f\}(s)-s\sp{m-1}f(0)-\cdots-sf\sp{(m-2)}(0)-f\sp{(m-1)}(0).

.. _7.3.3:

 \end{equation*}


--------------





Áður en við snúum okkur að því að leysa afleiðujöfnuhneppi með
Laplace-ummyndun, skulum við líta á veldisvísisfylkið:


\subsubsection{Setning} 
Um sérhvert $m\times m$ fylki  $A$ gildir
\begin{equation*}
\L\{e^{tA}\}(s) = (sI-A)^{-1}.


.. _7.3.4:

\end{equation*}


--------------

  

\section{Green--fallið og  földun :hover:`Green-fall`}

\subsection{Green--fallið og  földun :hover:`Green-fall`}


\noindent
Lítum nú á afleiðujöfnu með fastastuðla
 \begin{equation*}
P(D)u=(a_mD^m+\cdots+a_1D+a_0)u=f(t),

.. _7.4.1:

 \end{equation*}
með upphafsskilyrðunum
 \begin{equation*}
u(a)=b_0, u\dash(a)=b_1,\  \dots,  \  u^{(m-1)}(a)=b_{m-1}.

.. _7.4.2:

 \end{equation*}
Með því að hliðra til tímaásnum, þ.e.~skipta á fallinu $u(t)$ og
$u(t-a)$, þá getum við gert ráð fyrir að $a=0$.  


Við höfum
sýnt fram á að fallið $u_p$ sem uppfyllir $P(D)u=f(t)$, með
óhliðruðu upphafsskilyrðunum $b_0=\cdots=b_{m-1}=0$ er gefið með
formúlunni
 \begin{equation*}u_p(t)=\int_0^tG(t,\tau) f(\tau)\, d\tau,

.. _7.4.3:

 \end{equation*}
þar sem $G$ er Green--fall virkjans $P(D)$.
Við skulum nú reikna út $U_p(s)=\L\{u_p\}(s)$.  Vegna
þess að upphafsgildin :hover:`Green-fall!fyrir upphafsgildisverkefni`
eru öll 0, þá er
\begin{equation*}
\L\{u_p\dash\}(s)=sU_p(s), \quad 
\L\{u_p\ddash\}(s)=s^2U_p(s),\dots,
\L\{u_p^{(m)}\}(s)=s^mU_p(s).
\end{equation*}
Þetta gefur okkur að 
$$ \L\{P(D)u_p\}(s)=(a_ms^m+\cdots+a_1s+a_0)U_p(s)=\L f(s), $$
sem er greinilega jafnan
$$ P(s)U_p(s)=\L f(s), $$
og við fáum 

.. _7.4.4:

\begin{equation*}
\L\{u_p\}(s)=\dfrac {\L f(s)}{P(s)}.
\end{equation*}
Nú er Green--fallið $G(t,\tau)=g(t-\tau)$, þar sem $g$ uppfyllir
$$
P(D)g=0, \  g(0)=g\dash(0)=\cdots=g^{(m-2)}(0)=0, \ 
g^{(m-1)}(0)=\dfrac 1{a_m}.  
$$ 
Ef við setjum  $U(s)=\L g(s)$, þá fáum við
\begin{align*}
\L\{g\dash\}(s) &= s\L\{g\}(s)-g(0)=sU(s),\\
\L\{g\ddash\}(s) &= s^2\L\{g\}(s)-sg(0)-g\dash(0)\\
&=s^2U(s),\\
&\qquad \vdots\qquad\qquad\vdots\qquad\qquad \vdots\\
\L\{g^{(m-1)}\}(s) &=
s^{m-1}\L\{g\}(s)-s^{m-2}g(0)-\cdots-g^{(m-2)}(0)\\
&=s^{m-1}U(s),\\
\L\{g^{(m)}\}(s) &=
s^m\L\{g\}(s)-s^{m-1}g(0)-\cdots-g^{(m-1)}(0)\\
&=s^mU(s)-\dfrac 1{a_m}.
\end{align*}
Við tökum nú Laplace-myndina af báðum hliðum jöfnunnar $P(D)g=0$ og fáum
$$ (a_ms^mU(s)-1)+a_{m-1}s^{m-1}U(s)+\cdots+a_1sU(s)+a_0U(s)=0, $$
og við fáum $P(s)U(s)=1$, sem jafngildir

.. _7.4.5:

\begin{equation*}
\L g(s)=\dfrac 1{P(s)}.
\end{equation*}
Við höfum því sýnt fram á að 
$$
\L\left\{\int_0^tg(t-\tau)f(\tau)\, d\tau\right\}(s)= \L\{u_p\}(s)=
\L\{g\}(s)\L\{f\}(s).
$$
Þessi formúla er engin tilviljun, því við höfum:


.. _set7.4.1:

\subsubsection{Setning}  Ef $f$ og $g$ eru föll af veldisvísisgerð
og heildanleg á sérhverju bili $[0,b]$, þá er
 $$\L\left\{\int_0^tf(t-\tau)g(\tau)\, d\tau\right\}(s)=
\L\{f\}(s)\L\{g\}(s).
 $$


--------------



Athugið að 
 $$\int_0^t f(t-\tau)g(\tau) \, d\tau=
\int_0^t f(\tau)g(t-\tau) \, d\tau.
 $$
Með því að velja $g(t)=1$ og nota að $\L\{1\}=1/s$, þá fæst:

\subsubsection{Fylgisetning} Ef $f$ er  af veldisvísisgerð og heildanlegt á sérhverju
bili $[0,b]$, þá er
 \begin{equation*}\L\left\{\int_0^t f(\tau) \, d\tau\right\}(s) = \dfrac 1s
\L\{f\}(s).


.. _7.4.6:

 \end{equation*}


--------------



Földun :hover:`földun` tveggja falla $f, g: \R\to \C$  er skilgreind
með formúlunni 
 $$f*g(t)=\int_{-\infty}^{+\infty}f(t-\tau)g(\tau) \, d\tau,
 $$
og talan $t$ liggur í skilgreiningarmengi $f*g$ ef heildið er
samleitið.  Ef $f$ er til dæmis heildanlegt  á $\R$ og $g$ er
takmarkað, þá er földunin vel skilgreind fyrir öll $t\in \R$.
Ef föllin $f$ og $g$ eru bæði skilgreind og heildanleg á  $\R_+$, þá
getum við framlengt  skilgreiningarsvæði þeirra yfir í allt $\R$ með
því að setja 
$f(t)=g(t)=0$ fyrir öll $t<0$.  Þá er $f*g(t)$  skilgreint fyrir öll
$t\in \R$ og $$ f*g(t)= \int_0^tf(t-\tau)g(\tau)\, d\tau.$$
Við getum því umritað síðustu  setningu í 

.. _7.4.7:

\begin{equation*}
\L\{f*g\}=\L\{f\}\L\{g\}.
\end{equation*}


