%% Generated by Sphinx.
\def\sphinxdocclass{report}
\documentclass[a4paper,10pt,icelandic]{sphinxmanual}
\ifdefined\pdfpxdimen
   \let\sphinxpxdimen\pdfpxdimen\else\newdimen\sphinxpxdimen
\fi \sphinxpxdimen=.75bp\relax

\PassOptionsToPackage{warn}{textcomp}
\usepackage[utf8]{inputenc}
\ifdefined\DeclareUnicodeCharacter
% support both utf8 and utf8x syntaxes
  \ifdefined\DeclareUnicodeCharacterAsOptional
    \def\sphinxDUC#1{\DeclareUnicodeCharacter{"#1}}
  \else
    \let\sphinxDUC\DeclareUnicodeCharacter
  \fi
  \sphinxDUC{00A0}{\nobreakspace}
  \sphinxDUC{2500}{\sphinxunichar{2500}}
  \sphinxDUC{2502}{\sphinxunichar{2502}}
  \sphinxDUC{2514}{\sphinxunichar{2514}}
  \sphinxDUC{251C}{\sphinxunichar{251C}}
  \sphinxDUC{2572}{\textbackslash}
\fi
\usepackage{cmap}
\usepackage[T1]{fontenc}
\usepackage{amsmath,amssymb,amstext}
\usepackage{babel}



\usepackage{times}
\expandafter\ifx\csname T@LGR\endcsname\relax
\else
% LGR was declared as font encoding
  \substitutefont{LGR}{\rmdefault}{cmr}
  \substitutefont{LGR}{\sfdefault}{cmss}
  \substitutefont{LGR}{\ttdefault}{cmtt}
\fi
\expandafter\ifx\csname T@X2\endcsname\relax
  \expandafter\ifx\csname T@T2A\endcsname\relax
  \else
  % T2A was declared as font encoding
    \substitutefont{T2A}{\rmdefault}{cmr}
    \substitutefont{T2A}{\sfdefault}{cmss}
    \substitutefont{T2A}{\ttdefault}{cmtt}
  \fi
\else
% X2 was declared as font encoding
  \substitutefont{X2}{\rmdefault}{cmr}
  \substitutefont{X2}{\sfdefault}{cmss}
  \substitutefont{X2}{\ttdefault}{cmtt}
\fi


\usepackage[Sonny]{fncychap}
\usepackage{sphinx}

\fvset{fontsize=\small}
\usepackage{geometry}

% Include hyperref last.
\usepackage{hyperref}
% Fix anchor placement for figures with captions.
\usepackage{hypcap}% it must be loaded after hyperref.
% Set up styles of URL: it should be placed after hyperref.
\urlstyle{same}

\usepackage{sphinxmessages}



\usepackage{amsmath}
\usepackage{amssymb}
\usepackage{hyperref}


\title{Stærðfræðigreining III}
\date{ágú. 18, 2020}
\release{2020}
\author{AUTHOR}
\newcommand{\sphinxlogo}{\sphinxincludegraphics{hi_horiz_raunvisindadeild.png}\par}
\renewcommand{\releasename}{Útgáfa}
\makeindex
\begin{document}

\ifdefined\shorthandoff
  \ifnum\catcode`\=\string=\active\shorthandoff{=}\fi
  \ifnum\catcode`\"=\active\shorthandoff{"}\fi
\fi

\pagestyle{empty}
\sphinxmaketitle
\pagestyle{plain}
\sphinxtableofcontents
\pagestyle{normal}
\phantomsection\label{\detokenize{index::doc}}



\chapter{Formáli}
\label{\detokenize{formali:formali}}\label{\detokenize{formali::doc}}
Þetta kennsluefni er haft til hliðsjónar í fyrirlestrum í áfanganum
Stærðfræðigreining III við Háskóla Íslands. Það er bæði aðgengilegt sem
vefsíða, \sphinxurl{http://notendur.hi.is/sigurdur/stae302}, og sem \sphinxhref{https://notendur.hi.is/sigurdur/stae302/stae302.pdf}{pdf-skjal} sem hentar
til útprentunar. Efnið er unnið úr glærupakka sem Rögnvaldur Möller útbjó og byggja þær glærur að miklu leyti á bókinni \sphinxstyleemphasis{Tvinnfallagreining, afleiðujöfnur, Fourier-greining og hlutafleiðujöfnur} eftir Ragnar Sigurðsson. Bók Ragnars er einnig kennslubók námskeiðsins.

\sphinxstylestrong{Ágúst 2019, Sigurður Örn Stefánsson}


\chapter{Tvinntölur}
\label{\detokenize{Kafli01:tvinntolur}}\label{\detokenize{Kafli01::doc}}
\sphinxstyleemphasis{There’s something that doesn’t make sense. Let’s go and poke it with a stick.}

- The Doctor, Doctor Who


\section{Tvinntölurnar}
\label{\detokenize{Kafli01:tvinntolurnar}}

\subsection{Skilgreining}
\label{\detokenize{Kafli01:skilgreining}}
Skilgreinum margföldun á \(\mathbb{R}^2\) þannig að
\begin{equation*}
\begin{split}(a,b)(c,d)=(ac-bd, ad+bc).\end{split}
\end{equation*}
Þegar hugsað er um \(\mathbb{R}^2\) með þessari margföldun og venjulegri
samlagningu þá eru stökin í \(\mathbb{R}^2\) kallaðar tvinntölur og mengi
þeirra er táknað með \({\mathbb{C}}\). Þegar við viljum leggja áherslu á að
líta má á tvinntölu sem punkt í planinu \(\mathbb{R}^2\) þá er talað um
tvinntalnaplanið.


\subsection{Ritháttur.}
\label{\detokenize{Kafli01:rithattur}}
Þegar fjallað er um tvinntölur þá er stakið \((a,b)\)
venjulega ritað sem \(a+ib\).

Hugsum okkur að \(i^2=-1\). Notum svo venjulega dreifireglu og að
\(i\) víxlast við rauntölur til að reikna margfeldið
\begin{equation*}
\begin{split}(a+ib)(c+id)=ac+iad+ibc+i^2bd=ac-bd+i(ad+bc).\end{split}
\end{equation*}
Við höfum nú fengið aftur skilgreininguna á margfölduninni hér að ofan.


\subsection{Setning}
\label{\detokenize{Kafli01:setning}}
Eftirfarandi reiknireglur gilda um tvinntölur:
\begin{enumerate}
\sphinxsetlistlabels{\roman}{enumi}{enumii}{(}{)}%
\item {} 
\(\big((a+ib)+(c+id)\big)+(e+if)=(a+ib)+\big((c+id)+(e+if)\big)\) (tengiregla fyrir samlagningu)

\item {} 
\(\big((a+ib)(c+id)\big)(e+if)=(a+ib)\big((c+id)(e+if)\big)\) (tengiregla fyrir margföldun)

\item {} 
\((a+ib)+(c+id)=(c+id)+(a+ib)\) (víxlregla fyrir samlagningu)

\item {} 
\((a+ib)(c+id)=(c+id)(a+ib)\) (víxlregla fyrir margföldun)

\item {} 
\((a+ib)\big((c+id)+(e+if)\big)=(a+ib)(c+id)+(a+ib)(e+if)\) (dreifiregla)

\item {} 
Talan \(0=0+i0\) er samlagningarhlutleysa, þ.e.a.s. \((a+ib)+0=a+ib\).

\item {} 
Talan \(1=1+i0\) er margföldunarhlutleysa, þ.e.a.s. \(1(a+ib)=a+ib\).

\end{enumerate}


\subsection{Ritháttur.}
\label{\detokenize{Kafli01:id1}}
Þegar unnið er með tvinntölur þá er ekki gerður
greinarmunur á rauntölunni \(a\) og tvinntölunni \(a+i0.\) Því
getum við hugsað mengi rauntalna \(\mathbb{R}\) sem hlutmengi í mengi
tvinntalna \({\mathbb{C}}\). Sérhver rauntala er þannig líka tvinntala.


\subsection{Setning}
\label{\detokenize{Kafli01:id2}}
Ef \(z=a+ib\neq 0\) er tvinntala þá á \(z\) sér
margföldunarandhverfu sem er
\begin{equation*}
\begin{split}z^{-1}=\frac{1}{z}=\frac{a}{a^2+b^2}+\frac{-b}{a^2+b^2}i.\end{split}
\end{equation*}

\subsection{Skilgreining og setning}
\label{\detokenize{Kafli01:skilgreining-og-setning}}
Ef \(z\) er tvinntala þá getum við
skilgreint heiltöluveldi \(z^n\) af \(z\) þannig að
\(z^0=1\), og ef \(n>0\) þá er
\(z^n=z\cdot z\cdot\ldots\cdot z\) (\(n\) sinnum) og
\(z^{-n}=\big(z^{-1}\big)^n\). Venjulegar veldareglur gilda um
tvinntöluveldi, þ.e.a.s.
\begin{equation*}
\begin{split}z^n\cdot z^m=z^{n+m}\qquad z^n/z^m=z^{n-m}\qquad z^n\cdot w^n=(zw)^{n}
\qquad (z^n)^m=z^{nm}.\end{split}
\end{equation*}

\subsection{Skilgreining}
\label{\detokenize{Kafli01:id3}}
Ritum tvinntölu \(z\) sem \(z=x+iy\) þar sem
\(x\) og \(y\) eru rauntölur.

Talan \(x\) kallast raunhluti \(z\) og er táknaður með
\(\operatorname{Re\, } z\).

Talan \(y\) kallast þverhluti \(z\) og er táknaður með
\(\operatorname{Im\, } z\). (Athugið að þverhlutinn er rauntala.)

Sagt er að \(z\) sé rauntala ef \(\operatorname{Im\, } z=0\) en hrein þvertala ef
\(\operatorname{Re\, } z=0\).

Fyrir tvinntölu \(z=x+iy\) skilgreinum við samok \(z\) sem
tvinntöluna \(\overline{z}=x-iy\).


\subsection{Reiknireglur.}
\label{\detokenize{Kafli01:reiknireglur}}
Um tvinntölu \(z=x+iy\) gildir
\begin{equation*}
\begin{split}\begin{aligned}
\overline{(\overline{z})}&=z\\
z\overline{z}&=(x+iy)(x-iy)=x^2+y^2\\
z+\overline z&=2x=2\operatorname{Re\, } z\\
z-\overline z&=2yi=(2\operatorname{Im\, } z)i\\
\overline{z+w}&=\overline{z}+\overline{w}\\
\overline{zw}&=\overline{z}\,\overline{w}\\
\overline{z/w}&=\overline{z}/\overline{w}\end{aligned}\end{split}
\end{equation*}

\subsection{Skilgreining}
\label{\detokenize{Kafli01:id4}}
Lengd tvinntölu \(z=x+iy\) er skilgreind sem
rauntalan \(|z|=\sqrt{x^2+y^2}=\sqrt{z\overline{z}}\).

Hugsum nú tvinntöluna \(z=x+iy\) sem punkt \((x,y)\) í planinu.
Setjum \(r=\sqrt{x^2+y^2}=|z|\). Ritum nú punktinn \((x,y)\) sem
\((x,y)=r(\cos \theta, \sin\theta)\) (pólhnit). Þá er
\(z=|z|(\cos\theta+i\sin\theta)\) og \(\theta\) kallast
stefnuhorn tvinntölunnar \(z\). (Athugið að stefnuhorn er ekki
ótvírætt ákvarðað því ef \(\theta\) er stefnuhorn þá er
\(\theta+k\cdot 2\pi\) líka stefnuhorn.)


\subsection{Formúla.}
\label{\detokenize{Kafli01:formula}}
Lát \(z=x+iy\neq 0\) vera tvinntölu í
\({\mathbb{C}}\setminus \mathbb{R}_-\) (\(\mathbb{R}_-\) er mengi allra neikvæðra
rauntalna sem við samsömum við mengi allra tvinntalna á forminu
\(a+ib\) með \(b=0\) og \(a<0\)). Stefnuhorn \(z\) er
gefið með formúlunni
\begin{equation*}
\begin{split}\theta=2\arctan\left(\tfrac{y}{|z|+x}\right).\end{split}
\end{equation*}
Athugið að þessi formúla gefur gildi á \(\theta\) þannig að
\(-\pi<\theta<\pi\).


\subsection{Skilgreining}
\label{\detokenize{Kafli01:id5}}
Ef \(z\) og \(w\) eru tvær tvinntölur þá er
fjarlægðin á milli þeirra skilgreind sem rauntalan \(|z-w|\).


\subsection{Setning}
\label{\detokenize{Kafli01:id6}}
Fyrir sérhverjar tvinntölur \(z\) og \(w\) gildir
að
\begin{equation*}
\begin{split}|z+w|\leq |z|+|w|.\end{split}
\end{equation*}
Athugið að \(|z+w|=|z|+|w|\) ef og aðeins ef til er jákvæð rauntala
\(a\) þannig að \(w=az\).


\subsection{Rúmfræðileg túlkun margföldunar.}
\label{\detokenize{Kafli01:rumfraeileg-tulkun-margfoldunar}}
Ef \(z\) og \(w\) eru tvær
tvinntölur með lengdir \(|z|\) og \(|w|\) og stefnuhornin
\(\alpha\) og \(\beta\), þá er
\begin{equation*}
\begin{split}zw=|z||w|\big(\cos(\alpha+\beta)+i\sin(\alpha+\beta)\big).\end{split}
\end{equation*}
Það segir okkur að lengd margfeldisins er margfeldi lengda \(z\) og
\(w\) (þ.e.a.s. \(|zw|=|z||w|\)) og að stefnuhorn margfeldisins
sé summa stefnuhorna \(z\) og \(w\).

Sérstaklega gildir Regla de Moivre sem segir að
\begin{equation*}
\begin{split}(\cos \theta+i\sin\theta)^n=\cos(n\theta)+i\sin(n\theta).\end{split}
\end{equation*}

\subsection{Skilgreining}
\label{\detokenize{Kafli01:id7}}
Lína í tvinntalnaplaninu \({\mathbb{C}}\) er mengi allra
tvinntalna \(z=x+iy\) sem uppfylla jöfnu af taginu
\(ax+by+c=0\), þar sem \(a,b,c\) eru rauntölur.

Hringur í tvinntalnaplaninu er mengi allra punkta sem er í gefinni
fastri fjarlægð (geisli, radíus) frá gefnum föstum punkti \(m\)
(miðjunni). Hringur með miðju í \(m\) og geisla \(r\) er mengið
\(\{z\mid |z-m|=r\}\).


\subsection{Skilgreining}
\label{\detokenize{Kafli01:id8}}
Einingarhringurinn \(\mathbb{T}\) í \({\mathbb{C}}\) er mengi
allra tvinntalna sem hafa lengd 1. (Einnig má lýsa honum sem mengi allra
tvinntalna sem eru í fjarlægð 1 frá \(0\). Einingarhringurinn er
hringur með miðju í 0 og geisla 1.)


\subsection{Setning}
\label{\detokenize{Kafli01:id9}}
Sérhverri línu og sérhverjum hring má lýsa með jöfnu af
taginu
\begin{equation*}
\begin{split}\alpha|z|^2+\overline{\beta} z+\beta\overline{z}+\gamma=0,\end{split}
\end{equation*}
þar sem \(\alpha\) og \(\gamma\) eru rauntölur og \(\beta\)
er tvinntala.

Öfugt, ef við fáum slíka jöfnu þá lýsir hún:
\begin{enumerate}
\sphinxsetlistlabels{\roman}{enumi}{enumii}{(}{)}%
\item {} 
línu ef \(\alpha=0\) og \(\beta \neq 0\)

\item {} 
hring ef \(\alpha\neq 0\) og \(|\beta|^2-\alpha\gamma>0\) (og miðjan er \(m=-\beta/\alpha\) og geislinn er \(r=\sqrt{|\beta|^2-\alpha\gamma}/|\alpha|\));

\item {} 
stökum punkti ef \(\alpha\neq 0\) og \(|\beta|^2-\alpha\gamma=0\) (punkturinn er \(m=-\beta/\alpha\))

\item {} 
tóma menginu ef \(\alpha\neq 0\) og \(|\beta|^2-\alpha\gamma<0\);

\item {} 
öllu planinu \({\mathbb{C}}\) ef \(\alpha=\beta=\gamma=0\).

\end{enumerate}


\section{Margliður, ræð föll og veldisvísisföll}
\label{\detokenize{Kafli01:margliur-rae-foll-og-veldisvisisfoll}}

\subsection{Skilgreining (Sjá \S{}1.4)}
\label{\detokenize{Kafli01:skilgreining-sja-1-4}}
Getum skilgreint margliður með tvinntölustuðlum á sama hátt og margliður með rauntölustuðlum. Margliða
með tvinntölustuðlum er stærðtákn á forminu
\begin{equation*}
\begin{split}P(z)=a_nz^n+a_{n-1}z^{n-1}+\cdots+a_1z+a_0,\end{split}
\end{equation*}
þar sem stuðlarnir \(a_0, a_1, \ldots, a_{n-1}, a_n\) eru
tvinntölur.

Þegar sett er inn ákveðin tvinntala í stað \(z\) í þessari stæðu og
reiknað þá fæst út tvinntala. Margliðan gefur því fall
\(P:{\mathbb{C}}\rightarrow {\mathbb{C}}\).


\subsection{Margliður. (Sjá \S{}1.4)}
\label{\detokenize{Kafli01:margliur-sja-1-4}}
Tvinntölumargliður hegða sér
um flest eins og rauntölumargliður. Sérstaklega þá virkar deiling
tvinntölumargliða eins og deiling rauntölumargliða.

Fáum að ef \(P\) er margliða af stigi \(n\) og \(Q\) er
margliða af stigi \(m\) þá eru til ótvírætt ákvarðaðar margliður
\(S\) og \(R\) þannig að stig \(R(z)\) er minna en \(m\)
og
\begin{equation*}
\begin{split}P(z)=Q(z)S(z)+R(z).\end{split}
\end{equation*}
Sagt er að \(Q\) gangi upp í \(P\) ef \(R\) er
núllmargliðan.

Sérstaklega gildir að \(\alpha\) er núllstöð eða rót margliðunnar
\(P\) (þ.e.a.s. \(P(\alpha)=0\)) ef og aðeins ef
\(z-\alpha\) gengur upp í \(P\).


\subsection{Setning (Undirstöðusetning algebrunnar)}
\label{\detokenize{Kafli01:setning-undirstousetning-algebrunnar}}
Sérhver margliða af stigi \(\geq 1\) með tvinntölustuðla hefur núllstöð í
\({\mathbb{C}}\).


\subsection{Skilgreining og setning (Sjá \S{}1.4)}
\label{\detokenize{Kafli01:skilgreining-og-setning-sja-1-4}}
Hugsum okkur að \(\alpha\) sé núllstöð margliðu \(P\) og \(j\) sé hæsta talan þannig að
\((z-\alpha)^j\) gengur upp í \(P\),
þ.e.a.s. \(P(z)=(z-\alpha)^jQ(z)\) þar sem
\(Q(\alpha)\neq 0\). Þá segjum við að \(\alpha\) sé
\(j\)-föld núllstöð og köllum \(j\) margfeldni núllstöðvarinnar
\(\alpha\).

Það er afleiðing af Undirstöðusetningu algebrunnar að ef \(P\) er
margliða af stigi \(m\geq 1\) með núllstöðvar
\(\beta_1, \ldots, \beta_k\) sem hafa margfeldni
\(m_1,\ldots, m_k\) þá er \(m=m_1+\cdots+m_k\) og
\begin{equation*}
\begin{split}P(z)=A(z-\beta_1)^{m_1}\cdots(z-\beta_k)^{m_k},\end{split}
\end{equation*}
þar sem \(A\) er fasti.


\subsection{Skilgreining og setning (Sjá \S{}1.3)}
\label{\detokenize{Kafli01:skilgreining-og-setning-sja-1-3}}\begin{enumerate}
\sphinxsetlistlabels{\roman}{enumi}{enumii}{(}{)}%
\item {} 
Jafnan \(z^n=1\) hefur \(n\) ólíkar lausnir sem kallast \(n\)-tu einingarrætur og þær eru

\end{enumerate}
\begin{equation*}
\begin{split}z_k=\cos (k\cdot 2\pi/n)+i\sin (k\cdot 2\pi/n),\qquad k=0, 1, \ldots, n-1.\end{split}
\end{equation*}\begin{enumerate}
\sphinxsetlistlabels{\roman}{enumi}{enumii}{(}{)}%
\setcounter{enumi}{1}
\item {} 
Jafna af taginu \(z^n=\alpha=|\alpha|(\cos\phi+i\sin\phi)\) hefur \(n\) ólíkar lausnir og þær eru

\end{enumerate}
\begin{equation*}
\begin{split}z_k=|\alpha|^{1/n}\big(\cos (\phi/n+k\cdot 2\pi/n)+
i\sin (\phi/n+k\cdot 2\pi/n)\big),\qquad k=0, 1, \ldots, n-1.\end{split}
\end{equation*}\begin{enumerate}
\sphinxsetlistlabels{\roman}{enumi}{enumii}{(}{)}%
\setcounter{enumi}{2}
\item {} 
Jafna af taginu \(z^2=w=u+iv\) hefur tvær lausnir sem við köllum kvaðratrætur \(w\). Ef \(v\neq 0\) má rita þær:

\end{enumerate}
\begin{equation*}
\begin{split}z= \pm\left(\sqrt{\tfrac{1}{2}(|w|+u)}+i\;\mathrm{sign}(v)\sqrt{\tfrac{1}{2}(|w|-u)}\right).\end{split}
\end{equation*}
þar sem
\begin{equation*}
\begin{split}{{\operatorname{sign}}}(t)=
 \begin{cases}
 1, &t\geq 0,\\
 -1,&t<0.
 \end{cases}\end{split}
\end{equation*}
Ef \(v=0\) fæst tilfellið í liðnum á undan.
\begin{enumerate}
\sphinxsetlistlabels{\roman}{enumi}{enumii}{(}{)}%
\setcounter{enumi}{3}
\item {} 
(Sjá \S{}1.4)  Jafnan \(az^2+bz+c=0\) með \(a\neq 0\) (og \(a, b, c\) tvinntölur) hefur lausnir

\end{enumerate}
\begin{equation*}
\begin{split}z_1=\frac{-b+\sqrt{D}}{2a}\qquad\mbox{ og }\qquad z_2=\frac{-b-\sqrt{D}}{2a}\end{split}
\end{equation*}
þar sem \(D=b^2-4ac\) og \(\sqrt{D}\) táknar aðra lausn jöfnunnar \(z^2=D\) (sjá aðvörun fyrir neðan). Ef \(D\) er rauntala og \(D\geq 0\) tökum við kvaðratrót eins og við erum vön en ef \(D<0\) má rita lausnirnar
\begin{equation*}
\begin{split}z_1=\frac{-b+i\sqrt{|D|}}{2a}\qquad\mbox{ og }\qquad z_2=\frac{-b-i\sqrt{|D|}}{2a}\end{split}
\end{equation*}
\begin{sphinxadmonition}{warning}{Aðvörun:}
Ef \(z\) er tvinntala hefur táknmálið \(\sqrt{z}\) almennt ekki merkingu. Ef það er notað þarf ávallt að tilgreina fyrir hvað það stendur.
\end{sphinxadmonition}


\subsection{Skilgreining}
\label{\detokenize{Kafli01:id10}}
Rætt fall er kvóti tveggja margliða, \(R(z)=P(z)/Q(z)\).


\subsection{Stofnbrotaliðun. (Sjá \S{}1 1.5)}
\label{\detokenize{Kafli01:stofnbrotaliun-sja-1-1-5}}
Látum \(R(z)=P(z)/Q(z)\) vera rætt fall þar sem stig \(P(z)\) er lægra en stig \(Q(z)\).

Ef \(Q(z)=a(z-\alpha_1)\cdots(z-\alpha_m)\) þar sem \(\alpha_1, \ldots, \alpha_k\) eru ólíkar tvinntölur þá eru til fastar \(A_1, \ldots, A_k\) þannig að
\begin{equation*}
\begin{split}R(z)=\frac{A_1}{z-\alpha_1}+\cdots+\frac{A_k}{z-\alpha_k}.\end{split}
\end{equation*}
Stuðlarnir eru gefnir með
\begin{equation*}
\begin{split}A_j = \frac{P(\alpha_j)}{Q'(\alpha_j)},\end{split}
\end{equation*}
\(j=1,..k\).

\begin{sphinxadmonition}{important}{Mikilvægt:}
Við getum diffrað tvinngildar margliður líkt og raungildar margliður með því að nota
\begin{equation*}
\begin{split}\frac{dz^n}{dz} = n z^{n-1}\end{split}
\end{equation*}
ásamt því að diffrun er línuleg. Réttlæting kemur síðar.
\end{sphinxadmonition}

Ef \(Q(z)=a(z-\alpha_1)^{m_1}\cdots(z-\alpha_k)^{m_k}\)  og \(\alpha_1, \ldots, \alpha_k\) eru ólíkar tvinntölur þá eru til fastar
\(A_{11},\ldots, ,A_{m_11}, A_{12},\ldots, ,A_{m_12}, \ldots, A_{1k},\ldots, ,A_{m_1k}\) þannig að
\begin{equation*}
\begin{split}\begin{aligned}
 \dfrac{P(z)}{Q(z)}&=
 \dfrac{A_{1,0}}{(z-\alpha_1)^{m_1}}+\cdots+\dfrac{A_{1,m_1-1}}{(z-\alpha_1)}\\
 &+\dfrac{A_{2,0}}{(z-\alpha_2)^{m_2}}+\cdots+\dfrac{A_{2,m_2-1}}{(z-\alpha_2)}
 \\
 &\qquad \vdots\qquad\qquad\vdots\qquad \qquad \vdots\\
 &+\dfrac{A_{k,0}}{(z-\alpha_k)^{m_k}}+\cdots+\dfrac{A_{k,m_k-1}}{(z-\alpha_k)}\end{aligned}\end{split}
\end{equation*}
Stuðlarnir eru gefnir með
\begin{equation*}
\begin{split}A_{j,\ell}=\left.\dfrac 1{\ell!}
 \bigg(\dfrac {d}{dz}\bigg)^{\ell}\bigg(
 \dfrac{P(z)}{q_j(z)}\bigg)\right|_{z=\alpha_j},\end{split}
\end{equation*}
\(j=1,\dots,k, \ell=0,\dots,m_k-1\) þar sem \(q_j(z) = Q(z)/(z-\alpha_j)^{m_j}\).


\subsection{Skilgreining (Sjá \S{}1.6)}
\label{\detokenize{Kafli01:skilgreining-sja-1-6}}
Ritum tvinntölu \(z\) sem
\(z=x+iy\) þar sem \(x\) og \(y\) eru rauntölur. Skilgreinum
veldisvísisfallið
\begin{equation*}
\begin{split}e^z=e^{x+iy}=e^x(\cos y+i\sin y).\end{split}
\end{equation*}

\subsection{Reiknireglur. (Sjá \S{}1.6)}
\label{\detokenize{Kafli01:reiknireglur-sja-1-6}}
Látum \(z\) og \(w\) vera
tvinntölur. Þá gildir að
\begin{equation*}
\begin{split}e^ze^w=e^{z+w}.\end{split}
\end{equation*}
Ef \(k\) er heiltala þá er \(e^{z+k\cdot(2\pi i)}=e^z\), þanng
að \(e^z\) er lotubundið fall með lotu \(2\pi i\). Ennfremur
gildir að
\begin{equation*}
\begin{split}\overline{e^z}=e^{\overline{z}}\qquad |e^z|=e^{\operatorname{Re\, } z}\qquad |e^{iy}|=1.\end{split}
\end{equation*}

\subsection{Fallegasta jafna stærðfræðinnar.}
\label{\detokenize{Kafli01:fallegasta-jafna-staerfraeinnar}}
\begin{sphinxadmonition}{important}{Mikilvægt:}\begin{equation*}
\begin{split}e^{i\pi}+1=0\end{split}
\end{equation*}\end{sphinxadmonition}


\subsection{Jöfnur Eulers. (Sjá \S{}1.6)}
\label{\detokenize{Kafli01:jofnur-eulers-sja-1-6}}\begin{quote}

Fyrir rauntölu \(\theta\) er
\end{quote}
\begin{equation*}
\begin{split}\cos\theta=\frac{e^{i\theta}+e^{-i\theta}}{2}\qquad\mbox{ og }\qquad
\sin\theta=\frac{e^{i\theta}-e^{-i\theta}}{2i}.\end{split}
\end{equation*}

\subsection{Skilgreining  (Sjá \S{}1.6)}
\label{\detokenize{Kafli01:id11}}
Hægt er að útvíkka hornaföllin og
breiðbogaföllin yfir á allt tvinntalnaplanið með formúlunum
\begin{align*}\!\begin{aligned}
\cos z=\frac{e^{iz}+e^{-iz}}{2}\qquad\mbox{ og }\qquad
  \sin z=\frac{e^{iz}-e^{-iz}}{2i},\\
og\\
\end{aligned}\end{align*}\begin{equation*}
\begin{split}\cosh z=\frac{e^{z}+e^{-z}}{2}\qquad\mbox{ og }\qquad
\sinh z=\frac{e^{z}-e^{-z}}{2},\end{split}
\end{equation*}
og síðan eru \(\tan z, \cot z, \tanh z\) og \(\coth z\)
skilgreind á augljósan hátt. (Ef \(z\) er rauntala þá fást sömu
gildi og við þekkjum.)


\section{\protect\(\mathbb{R}\protect\)- og \protect\({\mathbb{C}}\protect\)-línulegar varpanir}
\label{\detokenize{Kafli01:mathbb-r-og-mathbb-c-linulegar-varpanir}}

\subsection{Skilgreining og setning (Sjá \S{}1.7)}
\label{\detokenize{Kafli01:skilgreining-og-setning-sja-1-7}}
Vörpun \(L:{\mathbb{C}}\rightarrow {\mathbb{C}}\) er sögð línuleg (nákvæmar,
\(\mathbb{R}\)-línuleg) ef um sérhverjar tvinntölur \(z\) og \(w\)
og sérhverja rauntölu \(c\) gildir að
\begin{equation*}
\begin{split}L(z+w)=L(z)+L(w)\qquad \mbox{ og }\qquad L(cz)=cL(z).\end{split}
\end{equation*}

\subsection{Setning}
\label{\detokenize{Kafli01:id12}}
Látum \(L:{\mathbb{C}}\rightarrow {\mathbb{C}}\) vera
línulega vörpun. Samsömum tvinntölu \(x+iy\) við vigur
\((x,y)\in \mathbb{R}^2\). Nú má hugsa \(L\) sem vörpun
\(\mathbb{R}^2\rightarrow \mathbb{R}^2\). Þá er \(L\) línuleg vörpun og til eru
rauntölur \(a, b, c, d\) þannig að fyrir allar rauntölur \(x\)
og \(y\) er (ef ekki er gerður munur á dálkvigrum og línuvigrum)
\begin{equation*}
\begin{split}L(x,y)=(ax+by, cx+dy)=\begin{bmatrix}a&b\\c&d\end{bmatrix}
\begin{bmatrix}x\\y\end{bmatrix}.\end{split}
\end{equation*}
Ef við ritum \(A=\frac{1}{2}\big((a+d)+i(c-b)\big)\) og
\(B=\frac{1}{2}\big((a-d)+i(c+b)\big)\) þá gildir fyrir sérhverja
tvinntölu \(z=x+iy\) að
\begin{equation*}
\begin{split}L(z)=Az+B\overline{z}.\end{split}
\end{equation*}

\subsection{Skilgreining}
\label{\detokenize{Kafli01:id13}}
Vörpun \(L:{\mathbb{C}}\rightarrow {\mathbb{C}}\) er sögð
\({\mathbb{C}}\)-línuleg ef um sérhverjar tvinntölur \(z\) og \(w\) og
sérhverja tvinntölu \(c\) gildir að
\begin{equation*}
\begin{split}L(z+w)=L(z)+L(w)\qquad \mbox{ og }\qquad L(cz)=cL(z).\end{split}
\end{equation*}
(Athugið að sérhver \({\mathbb{C}}\)-línuleg vörpun er líka
\(\mathbb{R}\)-línuleg, en ekki öfugt.)


\subsection{Setning}
\label{\detokenize{Kafli01:id14}}
Sérhverja \({\mathbb{C}}\)-línulega vörpun má
rita sem \(L(z)=Az\) þar sem \(A\) er tvinntala.


\subsection{Skilgreining}
\label{\detokenize{Kafli01:id15}}
Vörpun \({\mathbb{C}}\to {\mathbb{C}}\) af gerðinni
\(z\mapsto z+a\), þar sem \(a\in {\mathbb{C}}\) nefnist hliðrun.

Vörpun af gerðinni \(z\mapsto az\), nefnist snúningur, ef
\(a\in {\mathbb{C}}\) og \(|a|=1\), hún nefnist stríkkun ef \(a\in
\mathbb{R}\) og \(|a|>1\) og herping, ef \(a\in \mathbb{R}\) og \(|a|<1\), en
almennt nefnist hún snústríkkun ef \(a\in {\mathbb{C}}\setminus \{0\}\).

Vörpunin \({\mathbb{C}}\setminus \{0\} \to {\mathbb{C}}\setminus \{0\}\),
\(z\mapsto 1/z\) nefnist umhverfing.


\subsection{Skilgreining}
\label{\detokenize{Kafli01:id16}}
Vörpun \(f:{\mathbb{C}}\rightarrow{\mathbb{C}}\) af gerðinni
\begin{equation*}
\begin{split}f(z)=\dfrac{az+b}{cz+d}, \qquad ad-bc\neq 0, \quad a,b,c,d\in {\mathbb{C}},\end{split}
\end{equation*}
kallast brotin línuleg vörpun (eða brotin línuleg færsla eða
Möbiusarvörpun). Við sjáum að \(f(z)\) er skilgreint fyrir öll
\(z\in {\mathbb{C}}\), ef \(c=0\), en fyrir öll \(z\neq -d/c\), ef
\(c\neq 0\).


\subsection{Setning}
\label{\detokenize{Kafli01:id17}}
Sérhver brotin línuleg vörpun er samskeyting af hliðrunum,
snústríkunum og umhverfingum.


\subsection{Setning}
\label{\detokenize{Kafli01:id18}}
Sérhver brotin línuleg vörpun varpar hring í \({\mathbb{C}}\) á
hring eða línu og hún varpar línu á hring eða línu.


\begin{center}
\includegraphics[width=4cm,keepaspectratio=true]{stikaferill.png}
\end{center}



\begin{center}
\includegraphics[width=4cm,keepaspectratio=true]{polarggb.png}
\end{center}


Hér má sjá hvert brotin línuleg vörpun \(f\) með stika \(a,b,c,d\) líkt og að ofan, varpar línunni Form1 og hringnum Form2 í tvinntalaplaninu. Hægt er að breyta gildum stikanna með því að draga þá til með músinni.


\chapter{Fáguð föll}
\label{\detokenize{Kafli02:fagu-foll}}\label{\detokenize{Kafli02::doc}}
\sphinxstyleemphasis{Come on, Rory! It isn’t rocket science, it’s just quantum physics!}

- The Doctor, Doctor Who


\section{Markgildi og samfelldni}
\label{\detokenize{Kafli02:markgildi-og-samfelldni}}

\subsection{Skilgreining (Sjá \S{}2.1)}
\label{\detokenize{Kafli02:skilgreining-sja-2-1}}
Opin skífa með miðju \(\alpha\) og
geisla \(\varrho\) er skilgreind sem mengið
\begin{equation*}
\begin{split}S(\alpha,\varrho)=\{z\in {\mathbb{C}}\mid |z-\alpha|<\varrho\},\end{split}
\end{equation*}
lokuð skífa með miðju \(\alpha\) og geisla \(\varrho\) er
mengið
\begin{equation*}
\begin{split}\overline S(\alpha,\varrho)=\{z\in {\mathbb{C}}\mid |z-\alpha|\leq\varrho\}\end{split}
\end{equation*}
og götuð opin skífa með miðju \(\alpha\) og geisla \(\varrho\)
er mengið
\begin{equation*}
\begin{split}S^*(\alpha,\varrho)=\{z\in {\mathbb{C}}\mid 0<|z-\alpha|<\varrho\}.\end{split}
\end{equation*}

\subsection{Skilgreining}
\label{\detokenize{Kafli02:skilgreining}}
Hlutmengi \(X\) í \({\mathbb{C}}\) er sagt vera opið ef
um sérhvern punkt \(a\in X\) gildir að til er opin skífa
\(S(a,r)\), með \(r>0\) sem er innihaldin í \(X\).

Hlutmengi \(X\) í \({\mathbb{C}}\) er sagt vera lokað ef fyllimengi þess
\({\mathbb{C}}\setminus X\) er opið.

Jaðar hlutmengis \(X\) í \({\mathbb{C}}\) samanstendur af öllum punktum
\(a\in {\mathbb{C}}\) þannig að sérhver opin skífa \(S(a,r)\) með
\(r>0\) sker bæði \(X\) og \({\mathbb{C}}\setminus X\). Við táknum
jaðar \(X\) með \(\partial X\).

Punktur \(a\in {\mathbb{C}}\) nefnist þéttipunktur mengisins \(X\) ef um
sérhvert \(r>0\) gildir að gataða opna skífan \(S^*(a,r)\)
inniheldur punkta úr \(X\).

Opið hlutmengi í \({\mathbb{C}}\) kallast svæði ef það er samanhangandi.


\subsection{Skilgreining}
\label{\detokenize{Kafli02:id1}}
Látum \(X\) vera hlutmengi í \({\mathbb{C}}\) og
\(f:X\rightarrow {\mathbb{C}}\) vera fall. Ef \(a\) er þéttipunktur
\(X\) þá segjum við að \(f(z)\) stefni á tvinntölu \(L\)
þegar \(z\) stefnir á \(a\), og ritum
\begin{equation*}
\begin{split}\lim_{z\rightarrow a} f(z)=L\end{split}
\end{equation*}
ef um sérhvert \(\epsilon>0\) gildir að til er \(\delta >0\)
þannig að ef \(0<|z-a|<\delta\) þá er \(|f(z)-L|<\epsilon\).

Segjum að fallið \(f\) sé samfellt í punkti \(a\in X\) ef
\begin{equation*}
\begin{split}\lim_{z\rightarrow a} f(z)=f(a).\end{split}
\end{equation*}

\subsection{Setning}
\label{\detokenize{Kafli02:setning}}
Ef \(f\) og \(g\) eru tvinntölugild föll sem
skilgreind eru á menginu \(X\subseteq {\mathbb{C}}\),
\(\lim_{z\to a}f(z)=L\) og \(\lim_{z\to a}g(z)=M\), þá er
\begin{equation*}
\begin{split}\begin{gathered}
\lim_{z\to a}(f(z)+g(z))=\lim_{z\to a}f(z)+\lim_{z\to a}g(z)=L+M,\\
\lim_{z\to a}(f(z)-g(z))=\lim_{z\to a}f(z)-\lim_{z\to a}g(z)=L-M,\\
\lim_{z\to a}(f(z)g(z))=\big(\lim_{z\to a}f(z)\big)\big(\lim_{z\to
a}g(z)\big)=LM\\
\lim_{z\to a}\dfrac{f(z)}{g(z)}=\dfrac{\lim_{z\to a}f(z)}{\lim_{z\to
a}g(z)}=\dfrac LM.\end{gathered}\end{split}
\end{equation*}
Í síðustu formúlunni þarf að gera ráð fyrir að \(M\neq 0\). Ef
\(f\) og \(g\) eru föll á mengi \(X\) með gildi í \({\mathbb{C}}\)
sem eru samfelld í punktinum \(a\in X\), þá eru föllin \(f+g\),
\(f-g\), \(fg\) og \(f/g\) samfelld í \(a\) og
\begin{equation*}
\begin{split}\begin{gathered}
\lim_{z\to a}(f(z)+g(z))=f(a)+g(a),\\
\lim_{z\to a}(f(z)-g(z))=f(a)-g(a),\\
\lim_{z\to a}(f(z)g(z))=f(a)g(a),\\
\lim_{z\to a}\dfrac{f(z)}{g(z)}=\dfrac{f(a)}{g(a)},
\qquad \text{ef } \ g(a)\neq 0.\end{gathered}\end{split}
\end{equation*}
Ef \(f:X\to {\mathbb{C}}\) og \(g:Y\to {\mathbb{C}}\) eru föll,
\(f(X)\subset Y\), \(a\) er þéttipunkur \(X\),
\(b=\lim_{z\to a}f(z)\) er þéttipunktur mengisins \(Y\) og
\(g\) er samfellt í \(b\), þá er
\begin{equation*}
\begin{split}\lim_{z\to a} g\circ f(z)=g(\lim_{z\to a}f(z)).\end{split}
\end{equation*}
\begin{sphinxadmonition}{attention}{Athugið:}
Skilgreining 2.3 er sambærileg skilgreiningu á markgildi úr Stærðfræðigreiningu I og II og Setning 2.4 er sambærileg og reiknireglur fyrir markgildi raungildra falla í Stærðfræðigreiningu I og II.
\end{sphinxadmonition}


\section{Fáguð föll}
\label{\detokenize{Kafli02:id2}}

\subsection{Ritháttur (Sjá \S{}2.1)}
\label{\detokenize{Kafli02:rithattur-sja-2-1}}
Til samræmis við nótur Ragnars notum við annan
rithátt fyrir hlutafleiður en í Stærðfræðigreiningu II. Ef \(f\) er
fall af raunbreytistærðum \(x\) og \(y\), þá skrifum við
\begin{equation*}
\begin{split}{\partial}_xf=\dfrac{\partial f}{\partial x}, \qquad
{\partial}_yf=\dfrac{\partial f}{\partial y}, \qquad
{\partial}_x^2f=\dfrac{\partial^2f}{\partial x^2}, \qquad
{\partial}_{xy}^2f=\dfrac{\partial^2f}{\partial x\partial y}, \qquad
{\partial}_{xxy}^3f=\dfrac{\partial^3f}{\partial x^2\partial y}, \ \dots.\end{split}
\end{equation*}

\subsection{Skilgreining (Sjá \S{}2.1)}
\label{\detokenize{Kafli02:id3}}
Ef \(X\) er opið hlutmengi í \({\mathbb{C}}\)
þá látum við \(C(X)\) tákna mengi allra samfelldra falla
\(f:X\to {\mathbb{C}}\). Við látum \(C^m(X)\) tákna mengi allra
\(m\) sinnum samfellt deildanlegra falla. Hér er átt við að allar
hlutafleiður fallsins \(f\) af stigi \(\leq m\) eru til og þar
að auki samfelldar. Við skrifum \(C^0(X)=C(X)\) og táknum mengi
óendanlega oft deildanlegra falla með \(C^{\infty}(X)\).


\subsection{Skilgreining (Sjá Skilgreining 2.2.1)}
\label{\detokenize{Kafli02:skilgreining-sja-skilgreining-2-2-1}}
Látum \(f:X\to {\mathbb{C}}\) vera
fall skilgreint á opnu hlutmengi \(X\) af \({\mathbb{C}}\). Við segjum að
\(f\) sé \({\mathbb{C}}\)\textendash{}deildanlegt í punktinum \(a\in X\) ef
markgildið
\begin{equation*}
\begin{split}\lim _{\substack{ h\to 0\\ h\in{\mathbb{C}}}}
 \dfrac{f(a+h)-f(a)}h  \label{4.2.3}\end{split}
\end{equation*}
er til. Markgildið er táknað með \(f'(a)\) og kallað
\({\mathbb{C}}\)\textendash{}afleiða fallsins \(f\) í punktinum \(a\).

Fall \(f:X\to {\mathbb{C}}\) er sagt vera fágað á opna menginu \(X\) ef
\(f\in
C^1(X)\) og \(f\) er \({\mathbb{C}}\)\textendash{}deildanlegt í sérhverjum punkti í
\(X\).

Við látum \({\cal O}(X)\) tákna mengi allra fágaðra falla á \(X\).

Við segjum að \(f\) sé fágað í punktinum \(a\) ef til er opin
grennd \(U\) um \(a\) þannig að \(f\) sé fágað í \(U\).

Fallið \(f\) er sagt vera heilt fall (e. entire function) ef það er
fágað á öllu \({\mathbb{C}}\).


\subsection{Setning (Sjá Setningu 2.2.3)}
\label{\detokenize{Kafli02:setning-sja-setningu-2-2-3}}
Látum \(f,g:X\to {\mathbb{C}}\) vera föll,
\(a\in X\), \(\alpha,\beta\in {\mathbb{C}}\) og gerum ráð fyrir að
\(f\) og \(g\) séu \({\mathbb{C}}\)\textendash{}deildanleg í \(a\). Þá gildir
\begin{enumerate}
\sphinxsetlistlabels{\roman}{enumi}{enumii}{(}{)}%
\item {} 
\(\alpha f+\beta g\) er \({\mathbb{C}}\)\textendash{}deildanlegt í \(a\) og

\end{enumerate}
\begin{equation*}
\begin{split}(\alpha f+\beta g)'(a)=\alpha f'(a)+\beta g'(a).\end{split}
\end{equation*}\begin{enumerate}
\sphinxsetlistlabels{\roman}{enumi}{enumii}{(}{)}%
\setcounter{enumi}{1}
\item {} 
(Leibniz-regla). \(fg\) er \({\mathbb{C}}\)\textendash{}deildanlegt í \(a\) og

\end{enumerate}
\begin{equation*}
\begin{split}(fg)'(a)=f'(a)g(a)+f(a)g'(a).\end{split}
\end{equation*}\begin{enumerate}
\sphinxsetlistlabels{\roman}{enumi}{enumii}{(}{)}%
\setcounter{enumi}{2}
\item {} 
Ef \(g(a)\neq 0\), þá er \(f/g\) \({\mathbb{C}}\)\textendash{}deildanlegt í \(a\) og

\end{enumerate}
\begin{equation*}
\begin{split}(f/g)'(a)=\dfrac{f'(a)g(a)-f(a)g'(a)}{g(a)^2}.\end{split}
\end{equation*}

\subsection{Setning (Sjá Setningu 2.2.6)}
\label{\detokenize{Kafli02:setning-sja-setningu-2-2-6}}
Látum \(X\) og \(Y\) vera opin
hlutmengi af \({\mathbb{C}}\). Lát \(f:X\to {\mathbb{C}}\) og \(g:Y\to {\mathbb{C}}\) vera
föll, þannig að \(f(X)\subset Y\), \(a\in X\), \(b\in Y\),
\(b=f(a)\) og setjum
\begin{equation*}
\begin{split}h=g\circ f.\end{split}
\end{equation*}\begin{enumerate}
\sphinxsetlistlabels{\roman}{enumi}{enumii}{(}{)}%
\item {} 
Ef \(f\) er \({\mathbb{C}}\)\textendash{}deildanlegt í \(a\) og \(g\) er \({\mathbb{C}}\)\textendash{}deildanlegt í \(b\), þá er \(h\) líka \({\mathbb{C}}\)\textendash{}deildanlegt í \(a\) og

\end{enumerate}
\begin{equation*}
\begin{split}h'(a)=g'(b)f'(a).\end{split}
\end{equation*}\begin{enumerate}
\sphinxsetlistlabels{\roman}{enumi}{enumii}{(}{)}%
\setcounter{enumi}{1}
\item {} 
Ef \(g\) er \({\mathbb{C}}\)\textendash{}deildanlegt í \(b\), \(g'(b)\neq 0\), \(h\) er \({\mathbb{C}}\)\textendash{}deildanlegt í \(a\) og \(f\) er samfellt í \(a\), þá er \(f\) einnig \({\mathbb{C}}\)\textendash{}deildanlegt í \(a\) og

\end{enumerate}
\begin{equation*}
\begin{split}f'(a)=h'(a)/g'(b).\end{split}
\end{equation*}

\subsection{Fylgisetning (Sjá Fylgisetningu 2.2.7)}
\label{\detokenize{Kafli02:fylgisetning-sja-fylgisetningu-2-2-7}}
Látum \(X\) og \(Y\) vera opin hlutmengi af \({\mathbb{C}}\), og \(f:X\to Y\) vera gagntækt fall. Ef \(f\) er \({\mathbb{C}}\)\textendash{}deildanlegt í \(a\) og \(f'(a)\neq 0\), þá er andhverfa fallið \(f^{[-1]}\) líka \({\mathbb{C}}\)\textendash{}deildanlegt í \(b=f(a)\) og
\begin{equation*}
\begin{split}\left(f^{[-1]}\right)'(b)= \dfrac 1{f'(a)}.\label{4.2.4}\end{split}
\end{equation*}

\subsection{Setning (Sjá Setningu 2.2.8)}
\label{\detokenize{Kafli02:setning-sja-setningu-2-2-8}}
Látum \(f=u+iv:X\to {\mathbb{C}}\) vera fall af \(z=x+iy\) á opnu hlutmengi \(X\) í \({\mathbb{C}}\). Ef \(f\) er \({\mathbb{C}}\)\textendash{}deildanlegt í \(a\in X\), þá eru báðar hlutafleiðurnar \(\partial_xf(a)\) og \(\partial_yf(a)\) til og
\begin{equation*}
\begin{split}f'(a)=\partial_xf(a)=-i\partial_yf(a).\end{split}
\end{equation*}
Þar með gildir Cauchy\textendash{}Riemann\textendash{}jafnan
\begin{equation*}
\begin{split}\tfrac 12\big(\partial_xf(a)+i\partial_yf(a)\big)=0,\end{split}
\end{equation*}
og hún jafngildir hneppinu
\begin{equation*}
\begin{split}\partial_xu(a)=\partial_yv(a), \qquad \partial_yu(a)=-\partial_xv(a).\end{split}
\end{equation*}

\subsection{Skilgreining (Sjá \S{}2.2)}
\label{\detokenize{Kafli02:skilgreining-sja-2-2}}
Við skilgreinum fyrsta stigs hlutafleiðuvirkjana \({\partial}_z={\partial}/{\partial}z\) og \({\partial}_{\bar z}={\partial}/{\partial}\bar z\) með
\begin{equation*}
\begin{split}{\partial}_zf=\dfrac{{\partial}f}{{\partial} z}
=\tfrac 12\big({\partial}_xf-i{\partial}_yf\big) \quad \text{ og } \quad
{\partial}_{\bar z}f=\dfrac{{\partial}f}{{\partial}\bar z}
=\tfrac 12\big({\partial}_xf+i{\partial}_yf\big)
\label{4.2.14}\end{split}
\end{equation*}
Tölurnar \({\partial}_zf(a)\) og \({\partial}_{\bar z}f(a)\)
nefnast Wirtinger\textendash{}afleiður fallsins \(f\) í punktinum \(a\) og
virkinn \({\partial}_{\bar z}\) nefnist Cauchy\textendash{}Riemann\textendash{}virki


\subsection{Setning (Sjá Setningu 2.2.10)}
\label{\detokenize{Kafli02:setning-sja-setningu-2-2-10}}
Látum \(X\) vera opið hlutmengi í \({\mathbb{C}}\), \(a\in X\) og \(f:X\to {\mathbb{C}}\) vera fall. Þá gildir:
\begin{enumerate}
\sphinxsetlistlabels{\roman}{enumi}{enumii}{(}{)}%
\item {} 
\(f\) er \({\mathbb{C}}\)\textendash{}deildanlegt í \(a\) þá og því aðeins að \(f\) sé deildanlegt í \(a\) og \({\partial}_{\bar z}f(a)=0\). Þá er \(f'(a)={\partial}_zf(a)\).

\item {} 
\(f\) er fágað í \(X\) þá og því aðeins að \(f\) sé samfellt deildanlegt í \(X\) og uppfylli Cauchy\textendash{}Riemann\textendash{}jöfnuna \({\partial}_{\bar z}f=0\) í \(X\). Við höfum þá

\end{enumerate}
\begin{equation*}
\begin{split}f'=\dfrac{df}{dz}=\dfrac{\partial f}{\partial z}=\dfrac 12\bigg(
\dfrac{\partial f}{\partial x}-i\dfrac{\partial f}{\partial y}\bigg).\end{split}
\end{equation*}

\subsection{Tenging við línulegar varpanir.}
\label{\detokenize{Kafli02:tenging-vi-linulegar-varpanir}}
Afleiða samfellt deildanlegrar vörpunar
\(f:\mathbb{R}^2\rightarrow\mathbb{R}^2\) í punkti \(a\) er línuleg vörpun
\(Df(a):\mathbb{R}^2\rightarrow\mathbb{R}^2\). Ef við hugsum \(f\) sem vörpun
\({\mathbb{C}}\rightarrow{\mathbb{C}}\) þá er \(Df(a)\) almennt bara
\(\mathbb{R}\)-línuleg vörpun en \(f\) er \({\mathbb{C}}\)-deildanlegt í
\(a\) nákvæmlega þegar \(Df(a)\) er \({\mathbb{C}}\)-línuleg vörpun.


\section{Veldaraðir, veldisvísisfallið og lograr}
\label{\detokenize{Kafli02:veldarair-veldisvisisfalli-og-lograr}}

\subsection{Upprifjun úr Stærðfræðigreiningu I}
\label{\detokenize{Kafli02:upprifjun-ur-staerfraeigreiningu-i}}
Veldaraðir þar sem stuðlar og breyta eru tvinntölur ,,virka‘‘ eins og veldaraðir með rauntölustuðlum og rauntölubreytu. Það eina sem þarf að breyta er að í stað samleitnibils er talað um samleitniskífu.


\bigskip\hrule\bigskip

\begin{enumerate}
\sphinxsetlistlabels{\Alph}{enumi}{enumii}{(}{)}%
\item {} 
Fáum í hendurnar röð \(\sum_{n=1}^\infty a_n\) þannig að \(a_1, a_2, \ldots\) eru tölur. Skilgreinum

\end{enumerate}
\begin{equation*}
\begin{split}s_n=a_1+a_2+\cdots+a_n\end{split}
\end{equation*}
(summa fyrstu \(n\) liða raðarinnar). Segjum að röðin \(\sum_{n=1}^\infty a_n\) sé samleitin með summu \(s\) ef \(\lim_{n\rightarrow\infty}s_n=s\), það er að segja, röðin er samleitin með summu \(s\) ef
\begin{equation*}
\begin{split}\lim_{n\rightarrow \infty}(a_1+a_2+\cdots+a_n)=s.\end{split}
\end{equation*}
Ritað \(\sum_{n=1}^\infty a_n=s\).


\bigskip\hrule\bigskip

\begin{enumerate}
\sphinxsetlistlabels{\Alph}{enumi}{enumii}{(}{)}%
\setcounter{enumi}{1}
\item {} 
Um sérhverja veldaröð \(\sum_{n=0}^\infty a_n(z-\alpha)^n\) gildir eitt af þrennu:

\end{enumerate}
\begin{enumerate}
\sphinxsetlistlabels{\roman}{enumi}{enumii}{(}{)}%
\item {} 
Röðin er aðeins samleitin fyrir \(z=\alpha\).

\item {} 
Til er jákvæð tala \(\varrho\) þannig að veldaröðin er alsamleitin fyrir öll \(z\) þannig að \(|z-\alpha|<\varrho\) og ósamleitin fyrir öll \(z\) þannig að \(|z-\alpha|>\varrho\). Talan \(\varrho\) kallast samleitnigeisli veldaraðarinnar.

\item {} 
Röðin er samleitin fyrir allar tvinntölur \(z\).

\end{enumerate}


\bigskip\hrule\bigskip

\begin{enumerate}
\sphinxsetlistlabels{\Alph}{enumi}{enumii}{(}{)}%
\setcounter{enumi}{2}
\item {} 
Stundum má reikna út samleitnigeislann með eftirfarandi aðferðum:

\end{enumerate}
\begin{enumerate}
\sphinxsetlistlabels{\roman}{enumi}{enumii}{(}{)}%
\item {} 
Gerum ráð fyrir að \(L=\lim_{n\rightarrow\infty}\left|\frac{a_{n+1}}{a_n}\right|\) sé til eða \(\infty\). Þá hefur veldaröðin \(\sum_{n=0}^\infty a_n(z-\alpha)^n\) samleitnigeisla

\end{enumerate}
\begin{equation*}
\begin{split}\varrho=\left\{\begin{array}{ll}
\infty & \text{ef }L=0,\\
\frac{1}{L} & \text{ef }0<L<\infty,\\
0 & \text{ef }L=\infty.\\
\end{array}
\right.\end{split}
\end{equation*}\begin{enumerate}
\sphinxsetlistlabels{\roman}{enumi}{enumii}{(}{)}%
\setcounter{enumi}{1}
\item {} 
Gerum ráð fyrir að \(L=\lim_{n\rightarrow\infty}\sqrt[n]{|a_n|}\) sé til eða \(\infty\). Þá hefur veldaröðin \(\sum_{n=0}^\infty a_n(z-\alpha)^n\) samleitnigeisla

\end{enumerate}
\begin{equation*}
\begin{split}\varrho=\left\{\begin{array}{ll}
\infty & \mbox{ef }L=0,\\
\frac{1}{L} & \mbox{ef }0<L<\infty,\\
0 & \mbox{ef }L=\infty.\\
\end{array}
\right.\end{split}
\end{equation*}

\subsection{Setning}
\label{\detokenize{Kafli02:id4}}
Látum \(X\subseteq {\mathbb{C}}\) vera opið mengi og látum
\(f\) vera fall skilgreint á \(X\).
\begin{enumerate}
\sphinxsetlistlabels{\roman}{enumi}{enumii}{(}{)}%
\item {} 
(Sjá Setningu 2.3.2) Ef fyrir sérhvert \(\alpha\in X\) er til tala \(\varrho>0\) þannig að fyrir öll \(z\in S(\alpha, \varrho)\) er

\end{enumerate}
\begin{equation*}
\begin{split}f(z)= \sum_{n=0}^\infty a_n(z-\alpha)^n\end{split}
\end{equation*}
þá er fallið \(f\) fágað á \(X\) og fyrir \(z\in S(\alpha, \varrho)\) er
\begin{equation*}
\begin{split}f'(z)= \sum_{n=1}^\infty na_n(z-\alpha)^{n-1}.\end{split}
\end{equation*}\begin{enumerate}
\sphinxsetlistlabels{\roman}{enumi}{enumii}{(}{)}%
\setcounter{enumi}{1}
\item {} 
(Sjá Setningu 2.3.5) Ef fallið \(f\) er fágað þá er til fyrir sérhvern punkt \(\alpha\in X\) tala \(\varrho>0\) og veldaröð \(\sum_{n=0}^\infty a_n(z-\alpha)^n\) sem er alsamleitin á \(S(\alpha, \varrho)\) þannig að um alla punkta \(z\in S(\alpha, \varrho)\) gildir að \(f(z)=\sum_{n=0}^\infty a_n(z-\alpha)^n\).

\end{enumerate}


\subsection{Setning (Sjá Fylgisetningu 2.3.6)}
\label{\detokenize{Kafli02:setning-sja-fylgisetningu-2-3-6}}
Ef \(f\in {\cal O}(X)\) þá er \(f'\in {\cal O}(X)\).


\subsection{Setning (Samsemdarsetning fyrir samleitnar veldaraðir)}
\label{\detokenize{Kafli02:setning-samsemdarsetning-fyrir-samleitnar-veldarair}}
Gerum ráð fyrir að \(f,g\in {\cal O}(S(\alpha,\varrho))\) séu gefin með samleitnum veldaröðum
\begin{equation*}
\begin{split}f(z)=\sum\limits_{n=0}^\infty a_n(z-\alpha)^n, \qquad
g(z)=\sum\limits_{n=0}^\infty b_n(z-\alpha)^n, \qquad
z\in S(\alpha,\varrho),\end{split}
\end{equation*}
og gerum ráð fyrir að til sé runa \(\{\alpha_j\}\) af ólíkum punktum í \(S(\alpha,\varrho)\) þannig að \(\alpha_j\to \alpha\) og \(f(\alpha_j)=g(\alpha_j)\) fyrir öll \(j\). Þá er \(a_n=b_n\) fyrir öll \(n\) og þar með \(f(z)=g(z)\) fyrir öll \(z\in S(\alpha,\varrho)\).


\subsection{Setning (Sjá \S{}2.4)}
\label{\detokenize{Kafli02:setning-sja-2-4}}
Fyrir sérhverja tvinntölu \(z\) er
\begin{equation*}
\begin{split}e^z=\sum_{n=0}^\infty \frac{z^n}{n!}.\end{split}
\end{equation*}

\subsection{Skilgreining (Sjá Skilgreiningu 2.5.1)}
\label{\detokenize{Kafli02:skilgreining-sja-skilgreiningu-2-5-1}}
Látum \(X\) vera opið hlutmengi af \({\mathbb{C}}\). Samfellt fall \(\lambda:X\to {\mathbb{C}}\) kallast logri á \(X\) ef
\begin{equation*}
\begin{split}e^{\lambda(z)}=z, \qquad z\in X.\end{split}
\end{equation*}
Samfellt fall \(\varrho:X\to {\mathbb{C}}\) kallast \(n\)\textendash{}ta rót á \(X\) ef
\begin{equation*}
\begin{split}\big(\varrho(z)\big)^n=z, \qquad z\in X.\end{split}
\end{equation*}
Samfellt fall \(\theta:X\to \mathbb{R}\) kallast horn á \(X\) ef
\begin{equation*}
\begin{split}z=|z|e^{i\theta(z)}, \qquad z\in X.\end{split}
\end{equation*}

\subsection{Setning (Sjá Setningu 2.5.2)}
\label{\detokenize{Kafli02:setning-sja-setningu-2-5-2}}\begin{enumerate}
\sphinxsetlistlabels{\roman}{enumi}{enumii}{(}{)}%
\item {} 
Ef \(\lambda\) er logri á \(X\), þá er \(0\not\in X\), \(\lambda\in {\cal O}(X)\) og

\end{enumerate}
\begin{equation*}
\begin{split}\lambda'(z)=\frac 1z, \qquad z\in X.\end{split}
\end{equation*}
Föllin \(\lambda(z)+i2\pi k\), \(k\in \mathbb{Z}\) eru einnig lograr á \(X\).
\begin{enumerate}
\sphinxsetlistlabels{\roman}{enumi}{enumii}{(}{)}%
\setcounter{enumi}{1}
\item {} 
Ef \(\lambda\) er logri á \(X\), þá er

\end{enumerate}
\begin{equation*}
\begin{split}\lambda(z)=\ln
|z|+i\theta(z), \qquad z\in X,\end{split}
\end{equation*}
þar sem \(\theta:X\to \mathbb{R}\) er horn á \(X\). Öfugt, ef \(\theta:X\to \mathbb{R}\) er horn á \(X\), þá er \(\lambda(z)=\ln|z|+i\theta(z)\) logri á \(X\).
\begin{enumerate}
\sphinxsetlistlabels{\roman}{enumi}{enumii}{(}{)}%
\setcounter{enumi}{2}
\item {} 
Ef \(\varrho\) er \(n\)\textendash{}ta rót á \(X\) þá er \(\varrho\in {\cal O}(X)\) og

\end{enumerate}
\begin{equation*}
\begin{split}\varrho'(z)=\frac {\varrho(z)}{nz}, \qquad z\in X.\end{split}
\end{equation*}\begin{enumerate}
\sphinxsetlistlabels{\roman}{enumi}{enumii}{(}{)}%
\setcounter{enumi}{3}
\item {} 
Ef \(\lambda\) er logri á \(X\), þá er \(\varrho(z)=e^{\lambda(z)/n}\) \(n\)\textendash{}ta rót á \(X\).

\end{enumerate}


\subsection{Skilgreining og setning (Sjá \S{}2.5)}
\label{\detokenize{Kafli02:skilgreining-og-setning-sja-2-5}}
Fyrir sérhverja tvinntölu \({\alpha}\) er hægt að skilgreina fágað veldisfall með veldisvísi \(\alpha\) með
\begin{equation*}
\begin{split}z^\alpha=\exp(\alpha\lambda(z)), \qquad z\in X,\end{split}
\end{equation*}
þar sem \(\lambda\) er gefinn logri á \(X\) og við fáum að
\begin{equation*}
\begin{split}\begin{aligned}
\dfrac d{dz}z^\alpha=&\dfrac d{dz}e^{\alpha\lambda(z)}=e^{\lambda(z)}\frac
\alpha z =\alpha e^{\alpha\lambda(z)}e^{-\lambda(z)}\\
=&
\alpha e^{(\alpha-1)\lambda(z)}=\alpha z^{\alpha-1}.\end{aligned}\end{split}
\end{equation*}

\subsection{Skilgreining og setning (Sjá \S{}2.5)}
\label{\detokenize{Kafli02:id5}}
Ef \(\lambda\) er logri á opið mengi \(X\subseteq {\mathbb{C}}\) og \(\alpha \in X\), þá skilgreinum við veldisvísisfall með grunntölu \(\alpha\) sem fágaða fallið á \({\mathbb{C}}\), sem gefið er með
\begin{equation*}
\begin{split}\alpha^z=e^{z\lambda(\alpha)}.\end{split}
\end{equation*}
Athugið að skilgreiningin er háð valinu á logranum. Keðjureglan gefur
\begin{equation*}
\begin{split}\dfrac d{dz}\alpha^z=
\dfrac d{dz}e^{z\lambda(\alpha)}=e^{z\lambda(\alpha)}\cdot
\lambda(\alpha)=\alpha^z\lambda(\alpha).\end{split}
\end{equation*}

\subsection{Skilgreining (Sjá \S{}2.5)}
\label{\detokenize{Kafli02:skilgreining-sja-2-5}}
Lítum nú á mengið \(X={\mathbb{C}}\setminus \mathbb{R}_-\), sem fæst með því að skera neikvæða raunásinn og \(0\) út úr tvinntalnaplaninu. Við skilgreinum síðan pólhnit í \(X\) og veljum hornið \(\theta(z)\) þannig að \(-\pi<\theta(z)<\pi\), \(z\in X\). Fallið
\begin{equation*}
\begin{split}{\operatorname{Arg}} :{\mathbb{C}}\setminus \mathbb{R}_-\to \mathbb{R}, \qquad
{\operatorname{Arg}} z=\theta(z),\quad z\in X\end{split}
\end{equation*}
0 er kallað höfuðgrein hornsins og formúla þess er í grein 1.1.10 (og bók \S{}1.2.6.2),
\begin{equation*}
\begin{split}{\operatorname{Arg}}\, z=2\arctan\bigg(\dfrac y{|z|+x}\bigg), \qquad z=x+iy\in X.\end{split}
\end{equation*}
Fallið
\begin{equation*}
\begin{split}{\operatorname{Log}} :{\mathbb{C}}\setminus \mathbb{R}_-\to {\mathbb{C}}, \qquad
{\operatorname{Log}} z=\ln |z| +i{\operatorname{Arg}}(z),\quad z\in X,\end{split}
\end{equation*}
er kallað höfuðgrein lografallsins. Fallið
\begin{equation*}
\begin{split}z^\alpha = e^{\alpha{\operatorname{Log}} z}, \qquad z\in {\mathbb{C}}\setminus \mathbb{R}_-,\end{split}
\end{equation*}
kallast höfuðgrein veldisfallsins með veldisvísi \(\alpha\).


\chapter{Cauchy-setningin og Cauchy-formúlan}
\label{\detokenize{Kafli03:cauchy-setningin-og-cauchy-formulan}}\label{\detokenize{Kafli03::doc}}
\sphinxstyleemphasis{A straight line may be the shortest distance between two points, but it is by no means the most interesting.}

- The Doctor, Doctor Who


\section{Vegheildi}
\label{\detokenize{Kafli03:vegheildi}}

\subsection{Upprifjun úr Stærðfræðigreiningu II}
\label{\detokenize{Kafli03:upprifjun-ur-staerfraeigreiningu-ii}}
Heildun eftir vegum er mjög mikilvæg í tvinnfallagreiningu. Fjallað var um vegheildi í Stærðfræðigreiningu II. Hér er sú umfjöllun rifjuð upp og sett í samhengi.


\bigskip\hrule\bigskip

\begin{enumerate}
\sphinxsetlistlabels{\Alph}{enumi}{enumii}{(}{)}%
\item {} 
Samfelld vörpun \(\gamma:[a,b]\rightarrow {\mathbb{C}}\) kallast stikaferill. Myndmengi stikaferils, mengið \(\mbox{mynd}(\gamma)=\{\gamma(t)\mid t\in [a,b]\}\), kallast ferill.

Stikaferill sem er samfellt deildanlegur á köflum kallast vegur. (Stikaferill er samfellt deildanlegur á köflum ef til eru tölur \(a=b_0<b_1<\cdots<b_n=b\) þannig að stikaferillinn er samfellt diffranlegur á opnu bilunum \((b_i, b_{i+1})\) og báðar einhliða afleiður eru skilgreindar í punktunum \(b_i\).)

Stikaferill \(\gamma:[a,b]\rightarrow {\mathbb{C}}\) er sagður lokaður ef \(\gamma(a)=\gamma(b)\). Lokaður ferill, eða stikaferill, er sagður einfaldur ef hann sker ekki sjálfan sig nema í endapunktunum, þ.e.a.s. ekki eru til ólíkar tölur \(t_1\in (a,b)\) og \(t_2\in [a,b]\) þannig að \(\gamma(t_1)=\gamma(t_2)\).

\end{enumerate}


\bigskip\hrule\bigskip

\begin{enumerate}
\sphinxsetlistlabels{\Alph}{enumi}{enumii}{(}{)}%
\setcounter{enumi}{1}
\item {} 
Ef \(\gamma:[a,b]\rightarrow {\mathbb{C}}\) er vegur sem stikar feril \(C\) þá má skilgreina lengd ferilsins með formúlunni
\begin{equation*}
\begin{split}L(C)=L(\gamma)=\int_a^b|\gamma'(t)|\,dt.\end{split}
\end{equation*}
Ef \(f\) er samfellt tvinntölugilt fall á \(C\) þá er heildi \(f\) með tilliti til bogalengdar skilgreint sem
\begin{equation*}
\begin{split}\int_C f\,ds=\int_a^b f(\gamma(t))|\gamma'(t)|\,dt.\end{split}
\end{equation*}
Líka táknað
\begin{equation*}
\begin{split}\int_\gamma f\,ds, \qquad \int_C f\,|dz|,\qquad \int_\gamma f\,|dz|.\end{split}
\end{equation*}
Heildi með tilliti til bogalengdar eru óháð vali á stikun og stefnu stikunar.

\end{enumerate}


\bigskip\hrule\bigskip

\begin{enumerate}
\sphinxsetlistlabels{\Alph}{enumi}{enumii}{(}{)}%
\setcounter{enumi}{2}
\item {} 
Heildi vigursviðs \(\mathbf{F}:{\mathbb R}^2\rightarrow {\mathbb R}^2\) eftir veg \(\gamma:[a,b]\rightarrow {\mathbb R}^2\) er skilgreint sem heildið
\begin{equation*}
\begin{split}\int_\gamma \mathbf{F}\cdot d\gamma=\int_a^b\mathbf{F}(\gamma(t))\cdot \gamma'(t)\,dt.\end{split}
\end{equation*}
Ef við ritum \(\mathbf{F}(x,y)=(f(x,y),g(x,y))\) og \(\gamma(t)=(\alpha(t), \beta(t))\) þá má líka rita heildið sem
\begin{equation*}
\begin{split}\int_C f\,dx+g\,dy=\int_a^b f(\alpha(t), \beta(t))\alpha'(t)\,dt+g(\alpha(t), \beta(t))\beta'(t)\,dt.\end{split}
\end{equation*}
Heildi vigursviðs yfir veg er háð stikun að því marki að ef stefnu stikunar er breytt breytist formerki heildis.

\end{enumerate}


\subsection{Skilgreining (Sjá \S{}3.1)}
\label{\detokenize{Kafli03:skilgreining-sja-3-1}}
Ritum \(z=x+iy\) og látum \(f\) vera fall af \(z\). Látum \(\gamma: [a,b]\rightarrow {\mathbb{C}}\) vera veg sem stikar feril \(C\) og ritum
\(\gamma(t)=\alpha(t)+i\beta(t)\). Heildi \(f\) yfir \(C\) er skilgreint sem
\begin{equation*}
\begin{split}\int_C f\,dz=\int_\gamma f\,dz=\int_a^b f(\gamma(t))\gamma'(t)\,dt.\end{split}
\end{equation*}
Ritum nú \(f(z)=u(z)+iv(z)\) og fáum þá að
\begin{equation*}
\begin{split}\begin{aligned}
\int_C f\,dz&=\int_a^b f(\gamma(t))\gamma'(t)\,dt\\
&=\int_a^b f(\gamma(t)(\alpha'(t)+i\beta'(t))\,dt\\
&=\int_a^b f(\gamma(t))\alpha'(t)\,dt
+if(\gamma(t))\beta'(t)\,dt\\
&= \int_C f\,dx+if\,dy.\end{aligned}\end{split}
\end{equation*}
Athugið að ef stefnu stikunar er breytt þá breytist formerki á heildi.


\subsection{Dæmi.}
\label{\detokenize{Kafli03:daemi}}\begin{enumerate}
\sphinxsetlistlabels{\roman}{enumi}{enumii}{(}{)}%
\item {} 
Línustrikið frá \(z_0\) til \(z_1\) má stika með stikaferli \(\gamma:[0,1]\rightarrow {\mathbb{C}}\) þannig að

\end{enumerate}
\begin{equation*}
\begin{split}\gamma(t)=z_0+t(z_1-z_0)=(1-t)z_0+tz_1.\end{split}
\end{equation*}\begin{enumerate}
\sphinxsetlistlabels{\roman}{enumi}{enumii}{(}{)}%
\setcounter{enumi}{1}
\item {} 
Hring með miðju í punkti \(\alpha\) og geisla \(r\) má stika með stikaferli \(\gamma:[0,2\pi]\rightarrow {\mathbb{C}}\) þannig að

\end{enumerate}
\begin{equation*}
\begin{split}\gamma(t)=m+re^{i\theta}=m+r(\cos\theta+i\sin\theta).\end{split}
\end{equation*}

\subsection{Setning (Sjá \S{}3.1)}
\label{\detokenize{Kafli03:setning-sja-3-1}}\begin{equation*}
\begin{split}\left|\int_C f(z)\,dz\right|\leq \int_C |f(z)|\, |dz|\leq
\max_{z\in C}|f(z)|\int_C|dz|= \max_{z\in C}|f(z)|L(C).\end{split}
\end{equation*}

\subsection{Setning (Sjá Setningu 3.1.2)}
\label{\detokenize{Kafli03:setning-sja-setningu-3-1-2}}
Gerum ráð fyrir að \(X\) sé opið mengi og \(f\in C(X)\). Ef \(f\) hefur stofnfall \(F\), þ.e.a.s. ef til er fall \(F\in {\cal O}(X)\) þannig að \(F'=f\), þá er
\begin{equation*}
\begin{split}\int_\gamma f(z)\, dz = F(e_\gamma)-F(u_\gamma)\end{split}
\end{equation*}
fyrir sérhvern veg \(\gamma\) í \(X\) þar sem \(u_\gamma\) er upphafspunktur \(\gamma\) og \(e_\gamma\) er endapunkturinn. Sérstaklega gildir
\begin{equation*}
\begin{split}\int_\gamma f(z)\, dz = 0\end{split}
\end{equation*}
fyrir sérhvern lokaðan veg \(\gamma\) í \(X\).


\subsection{Fylgisetning. (Sjá Setning 3.1.2)}
\label{\detokenize{Kafli03:fylgisetning-sja-setning-3-1-2}}
Látum \(X\) vera svæði í \({\mathbb{C}}\) (\(X\) er opið samanhangandi hlutmengi í \({\mathbb{C}}\)). Ef \(f\) er fágað á \(X\) og \(f'(z)=0\) fyrir öll \(z\in X\), þá er \(f\) fastafall.


\subsection{Setning Green (Upprifjun úr Stærðfræðigreiningu II)}
\label{\detokenize{Kafli03:setning-green-upprifjun-ur-staerfraeigreiningu-ii}}
Látum \(\Omega\) vera opið mengi í planinu með jaðar \(\partial \Omega\) sem við gerum ráð fyrir að samanstandi af endanlega mörgum lokuðum ferlum sem hver um sig er samfellt deildanlegur á köflum. Áttum jaðarinn jákvætt þannig að ef gengið er eftir jaðri samkvæmt gefinni stefnu þá er \(\Omega\) á vinstri hönd. Ef \(f\) og \(g\) eru samfellt deildanleg föll þá er
\begin{equation*}
\begin{split}\int_{\partial \Omega} f\,dx+g\,dy=\int\!\!\int_\Omega \left(\partial_x g-\partial_y f\right)\,dx\,dy.\end{split}
\end{equation*}
Föllin \(f\) og \(g\) mega líka vera tvinntölugild því þá reiknar maður raun- og þverhluta heildis sitt í hvoru lagi og setningin gildir um hvort tveggja.


\subsection{Skilgreining og upprifjun. (Sjá \S{}2.2)}
\label{\detokenize{Kafli03:skilgreining-og-upprifjun-sja-2-2}}
Ritum \(z=x+iy\) og \(f=u+iv\). Setjum nú
\begin{equation*}
\begin{split}\partial_x f=\partial_x u+i\partial_xv\qquad\mbox{ og }\qquad
\partial_y f=\partial_y u+i\partial_yv.\end{split}
\end{equation*}
Rifjum upp að Wirtinger-afleiðurnar eru skilgreindar með formúlunum
\begin{equation*}
\begin{split}\partial_z f=\tfrac{1}{2}(\partial_xf-i\partial_yf)\qquad\mbox{ og }\qquad
\partial_{\overline{z}} f=\tfrac{1}{2}(\partial_xf+i\partial_yf).\end{split}
\end{equation*}
Cauchy-Riemann jöfnurnar \(\partial_xu=\partial_yv\) og \(\partial_yu=-\partial_xv\) jafngilda því að
\begin{equation*}
\begin{split}\partial_{\overline{z}} f=\tfrac{1}{2}(\partial_xf+i\partial_yf)=0.\end{split}
\end{equation*}

\subsection{Cauchy-setning. (Sjá Setning 3.3.1)}
\label{\detokenize{Kafli03:cauchy-setning-sja-setning-3-3-1}}
Látum \(X\) vera opið hlutmengi í \({\mathbb{C}}\). Gerum ráð fyrir að \(\Omega\) sé opið hlutmengi af \(X\) og að \(\partial \Omega\subseteq X\). Gerum enn fremur ráð fyrir að jaðarinn \(\partial \Omega\) samanstandi af endanlega mörgum sundurlægum lokuðum einföldum vegum sem eru áttaðir jákvætt með tilliti til \(\Omega\). Ef \(f\in C^1(X)\), þá er
\begin{equation*}
\begin{split}\int_{\partial\Omega}f\, dz = i\iint_\Omega
(\partial_xf+i\partial_yf)\, dxdy.\end{split}
\end{equation*}
Ef \(f\in {\cal O}(X)\), þá er
\begin{equation*}
\begin{split}\int_{\partial\Omega}f\, dz = 0.\end{split}
\end{equation*}

\subsection{Skilgreining (Sjá Skilgreiningu 3.3.2)}
\label{\detokenize{Kafli03:skilgreining-sja-skilgreiningu-3-3-2}}
Opið mengi \(X\) kallast stjörnusvæði með tilliti til punktsins \(\alpha\in X\), ef línustrikið \(\langle \alpha, z \rangle\) er innihaldið í \(X\) fyrir sérhvert \(z\in X\). Við segjum að \(X\) sé stjörnusvæði ef það er stjörnusvæði með tilliti til einhvers punkts.


\subsection{Setning (Sjá Setningu 3.3.3)}
\label{\detokenize{Kafli03:setning-sja-setningu-3-3-3}}
Ef \(X\) er stjörnusvæði með tilliti til punktsins \(\alpha\), þá hefur sérhvert \(f\in {\cal O}(X)\) stofnfall \(F\) (þ.e.a.s. \(F'=f\)), sem gefið er með formúlunni
\begin{equation*}
\begin{split}F(z)=\int_{\langle \alpha, z\rangle} f(\zeta)\, d\zeta, \qquad z\in X.\end{split}
\end{equation*}
og þar með gildir
\begin{equation*}
\begin{split}\int_\gamma f\, dz =0\end{split}
\end{equation*}
fyrir sérhvern lokaðan veg \(\gamma\) í \(X\).


\subsection{Cauchy-formúlan. (Sjá Setningu 3.3.4)}
\label{\detokenize{Kafli03:cauchy-formulan-sja-setningu-3-3-4}}
Gerum ráð fyrir sömu forsendum og í Cauchy-setningunni. Ef \(f\in C^1(X)\), þá gildir um sérhvert \(z\in \Omega\) að
\begin{equation*}
\begin{split}\begin{aligned}
f(z)=&\dfrac 1{2 \pi i}\int_{\partial\Omega}\dfrac
{f(\zeta)}{\zeta-z}\, d\zeta \\
&-\dfrac 1{2\pi}\iint_{\Omega}
\dfrac{(\partial_\xi+i\partial_\eta)f(\zeta)}
{\zeta-z}\, d\xi d\eta,
\end{aligned}\end{split}
\end{equation*}
þar sem breytan í heildinu er \({\zeta}={\xi}+i\eta\). Ef
\(f\in {\cal O}(X)\), þá er
\begin{equation*}
\begin{split}f(z)=\dfrac 1{2 \pi i}\int_{\partial\Omega}\dfrac
{f(\zeta)}{\zeta-z}\, d\zeta.\end{split}
\end{equation*}

\section{Afleiðingar Cauchy-setningarinnar}
\label{\detokenize{Kafli03:afleiingar-cauchy-setningarinnar}}

\subsection{Meðalgildissetning (Sjá Setningu 3.3.5)}
\label{\detokenize{Kafli03:mealgildissetning-sja-setningu-3-3-5}}
Látum \(X\) vera opið mengi í \({\mathbb{C}}\), \(f\in {\cal O}(X)\), \(z\in X\) og gerum ráð fyrir að \(\overline S(z,r)\subset X\). Þá gildir
\begin{equation*}
\begin{split}f(z)=\dfrac 1{2\pi} \int_0^{2\pi}f(z+re^{it})\, dt.\end{split}
\end{equation*}

\subsection{Setning (Sjá Setningu 3.3.6)}
\label{\detokenize{Kafli03:setning-sja-setningu-3-3-6}}
Gerum ráð fyrir að forsendur Cauchy-setningarinnar séu uppfylltar og að \(Q\) sé margliða með einfaldar núllstöðvar \(\alpha_1,\dots,\alpha_m\) og að engin þeirra liggi á \(\partial\Omega\). Þá er
\begin{equation*}
\begin{split}\int_{\partial\Omega} \dfrac{f(z)}{Q(z)} \, dz =
2\pi i\sum_{\alpha_j\in \Omega}
\dfrac{f(\alpha_j)}{Q'(\alpha_j)}.\end{split}
\end{equation*}

\subsection{Setning Morera (Sjá Setningu 3.4.5)}
\label{\detokenize{Kafli03:setning-morera-sja-setningu-3-4-5}}
Látum \(X\) vera opið mengi í \({\mathbb{C}}\), \(f\in C(X)\) og gerum ráð fyrir að
\begin{equation*}
\begin{split}\int_{\partial\Omega} f\, dz =0\end{split}
\end{equation*}
fyrir sérhvert þríhyrningssvæði \(\Omega\) þannig að \(\Omega\cup \partial \Omega\subset X\). Þá er \(f\in {\cal O}(X)\).


\subsection{Setning Goursat}
\label{\detokenize{Kafli03:setning-goursat}}
Látum \(f\) vera tvinntölugilt fall skilgreint á opnu mengi \(X\) í \({\mathbb{C}}\). Gerum ráð fyrir að \(f\) sé \({\mathbb{C}}\)-deildanlegt í sérhverjum punkti í \(X\). Þá er \(f\) fágað á \(X\).


\subsection{Cauchy-formúlur fyrir afleiður. (Sjá Setningu 3.4.1)}
\label{\detokenize{Kafli03:cauchy-formulur-fyrir-afleiur-sja-setningu-3-4-1}}
Látum \(X\) og \(\Omega\) vera eins og í Cauchy-setningunni og tökum \(z\in \Omega\). Þá er sérhvert \(f\) í \({\cal O}(X)\) óendanlega oft deildanlegt á \(X\), allar hlutafleiður af \(f\) eru fáguð föll og
\begin{equation*}
\begin{split}f^{(n)}(z)=
\dfrac {n!}{2\pi i}\int_{\partial\Omega}
\dfrac {f(\zeta)}{(\zeta-z)^ {n+1}}\, d\zeta.\end{split}
\end{equation*}

\subsection{Cauchy-ójöfnur. (Sjá Fylgisetningu 3.4.2)}
\label{\detokenize{Kafli03:cauchy-ojofnur-sja-fylgisetningu-3-4-2}}
Ef \(X\) er opið hlutmengi af \({\mathbb{C}}\), \(\bar S(\alpha,\varrho)\subset X\), \(f\in {\cal O}(X)\) og \(|f(z)|\leq M\) fyrir öll \(z\in \partial
S(\alpha,\varrho)\), þá er
\begin{equation*}
\begin{split}|f^{(n)}(\alpha)|\leq
Mn!/\varrho^ n.\end{split}
\end{equation*}

\subsection{Setning Liouville (Sjá Setningu 3.4.6)}
\label{\detokenize{Kafli03:setning-liouville-sja-setningu-3-4-6}}
Látum \(f\in {\cal O}({\mathbb{C}})\) og gerum ráð fyrir að \(f\) sé takmarkað fall (þ.e.a.s. til er fasti \(M\) þannig að \(|f(z)|\leq M\) fyrir öll \(z\in {\mathbb{C}}\)) . Þá er \(f\) fasti.


\subsection{Undirstöðusetning algebrunnar (Sjá Setningu 3.4.7)}
\label{\detokenize{Kafli03:undirstousetning-algebrunnar-sja-setningu-3-4-7}}
Sérhver margliða af stigi \(\geq 1\) hefur núllstöð í \({\mathbb{C}}\).


\section{Fleiri afleiðingar Cauchy-setningarinnar}
\label{\detokenize{Kafli03:fleiri-afleiingar-cauchy-setningarinnar}}

\subsection{Fræðilegur bakgrunnur. (Sjá \S{}3.5)}
\label{\detokenize{Kafli03:fraeilegur-bakgrunnur-sja-3-5}}
Nú munum við fást við spurninguna um hvenær má víxla röðinni á diffrun og summu og heildun og summu þegar fengist er við veldaraðir. Svarið er ekki augljóst og til að gera þetta almennilega þarf ný hugtök og þónokkra vinnu.


\bigskip\hrule\bigskip

\begin{enumerate}
\sphinxsetlistlabels{\Alph}{enumi}{enumii}{(}{)}%
\item {} 
Látum \(A\subseteq {\mathbb{C}}\) og \(f_n:A\rightarrow{\mathbb{C}}\) vera föll.

Segjum að \(f_n\rightarrow f\) með \(f:A\rightarrow{\mathbb{C}}\) ef fyrir sérhvert \(z\in A\) gildir að \(f_n(z)\rightarrow f(z)\), þ.e.a.s. ef \(z\in A\) þá er til fyrir sérhvert \(\epsilon>0\) tala \(N_z\) (hugsanlega háð \(z\)) þannig að ef \(n\geq N_z\) þá er \(|f(z)-f_n(z)|<\epsilon\).

Segjum að \(f_n\rightarrow f\) í jöfnum mæli þar sem \(f:A\rightarrow{\mathbb{C}}\) ef fyrir sérhvert \(\epsilon>0\) er til tala \(N\) þannig að ef \(n\geq N\) þá er \(|f(z)-f_n(z)|<\epsilon\) fyrir öll \(z\in A\). (Sama \(N\) dugar fyrir öll \(z\in A\).)

\end{enumerate}


\bigskip\hrule\bigskip

\begin{enumerate}
\sphinxsetlistlabels{\Alph}{enumi}{enumii}{(}{)}%
\setcounter{enumi}{1}
\item {} 
Látum nú \(X\) vera opið mengi í \({\mathbb{C}}\) og \(f_n:X\rightarrow {\mathbb{C}}\) vera föll. Ef \(f_n\rightarrow f\) í jöfnum mæli á sérhverju lokuðuð takmörkuðu hlutmengi í \(X\) og föllin \(f_n\) eru öll samfelld þá er markgildið \(f\) líka samfellt á \(X\).

Ef \(\gamma\) er vegur í \(X\) þá er
\begin{equation*}
\begin{split}\lim_{n\rightarrow \infty}\int_\gamma f_n(z)\,dz=
\int_\gamma \left(\lim_{n\rightarrow \infty}f_n(z)\right)\,dz=\int_\gamma f(z)\,dz.\end{split}
\end{equation*}
Ef föllin \(f_n\) eru öll fáguð þá er markgildið \(f\) líka fágað og \(f_n'\rightarrow f'\) (í jöfnum mæli á lokuðum takmörkuðum hlutmengjum í \(X\)).

\end{enumerate}


\bigskip\hrule\bigskip

\begin{enumerate}
\sphinxsetlistlabels{\Alph}{enumi}{enumii}{(}{)}%
\setcounter{enumi}{2}
\item {} 
(\(M\)-próf Weierstrass) Látum \(f_n\) vera runu falla sem öll eru skilgreind á mengi \(A\). Gerum ráð fyrir að \(M_k\) sé tala þannig að \(|f_k(z)|\leq M_k\) fyrir öll \(z\in A\) og að röðin \(\sum_{n=0}^\infty M_k\) sé samleitin. Þá er röðin \(\sum_{n=0}^\infty f_n\) samleitin í jöfnum mæli á \(A\) að fallinu \(f(z)=\sum_{n=0}^\infty f_n(z)\). (Þ.e.a.s. fallarunan \(g_k=\sum_{n=0}^k f_n\) stefnir á \(f\) í jöfnum mæli á \(A\).)

\end{enumerate}


\subsection{Setning Abels.}
\label{\detokenize{Kafli03:setning-abels}}
Skoðum veldaröð \(\sum_{n=0}^\infty a_n(z-\alpha)^n\) með samleitnigeisla \(\varrho>0\). Ef \(0<r<\varrho\) þá er veldaröðin samleitin í jöfnum mæli á opnu hringskífunni \({S}(\alpha,r)\).


\bigskip\hrule\bigskip


Samkvæmt ofangreindu gildir að ef \(f(z)=\sum_{n=0}^\infty a_n(z-\alpha)^n\) þá er
\begin{equation*}
\begin{split}f'(z)=\sum_{n=1}^\infty na_n(z-\alpha)^{n-1}\end{split}
\end{equation*}
fyrir öll \(x\in S(\alpha,\varrho)\) og ef \(\gamma\) er vegur í \(S(\alpha, \varrho)\) þá er
\begin{equation*}
\begin{split}\int_\gamma f(z)\,dz
=\int_\gamma \left(\sum_{n=0}^\infty a_n(z-\alpha)^n\right)\,dz
=\sum_{n=0}^\infty\int_\gamma a_n(z-\alpha)^n\,dz.\end{split}
\end{equation*}

\subsection{Skilgreining (Sjá Skilgreiningu 3.6.2)}
\label{\detokenize{Kafli03:skilgreining-sja-skilgreiningu-3-6-2}}
Ef \(X\) er opið hlutmengi af \({\mathbb{C}}\), \(\alpha\in X\) og \(f\in {\cal O}(X)\), þá kallast veldaröðin
\begin{equation*}
\begin{split}\sum\limits_{n=0}^\infty \dfrac{f^{(n)}(\alpha)}{n!}(z-\alpha)^n,\end{split}
\end{equation*}
Taylor-röð fágaða fallsins \(f\) í punktinum \(\alpha\). Ef \(\alpha=0\), þá kallast hún Maclaurin-röð fágaða fallsins \(f\).


\subsection{Setning (Sjá Setningu 3.6.1)}
\label{\detokenize{Kafli03:setning-sja-setningu-3-6-1}}
Látum \(X\) vera opið hlutmengi af \({\mathbb{C}}\), \(\alpha\in X\), \(\overline S(\alpha,\varrho)\subset X\) og \(f\in {\cal O}(X)\), þá er unnt að setja \(f\) fram með samleitinni veldaröð á skífunni \(S(\alpha,\varrho)\),
\begin{equation*}
\begin{split}f(z)=\sum_{n=0}^ \infty a_n(z-\alpha)^ n,
\qquad z\in S(\alpha,\varrho),\end{split}
\end{equation*}
þar sem stuðlarnir \(a_n\) eru ótvírætt ákvarðaðir og eru gefnir með
\begin{equation*}
\begin{split}a_n=\dfrac {f^{(n)}(\alpha)}{n!}.\end{split}
\end{equation*}
Samleitnigeisli raðarinnar er stærri en eða jafn fjarlægðinni frá \(\alpha\) út á jaðar \(X\).

Fyrir \(z\in S(\alpha, \varrho)\) er
\begin{equation*}
\begin{split}f'(z)= \sum_{n=1}^\infty na_n(z-\alpha)^{n-1}.\end{split}
\end{equation*}

\subsection{Skilgreining (Sjá Skilgreiningu 3.6.3)}
\label{\detokenize{Kafli03:skilgreining-sja-skilgreiningu-3-6-3}}
Látum \(f\in {\cal O}(X)\). Segjum að \(\alpha\) sé núllstöð \(f\) af stigi \(m\) (eða núllstöð af margfeldni \(m\)) ef \(f(\alpha)=f'(\alpha)=\cdots=f^{(m-1)}(\alpha)=0\) en \(f^{(m)}(\alpha)\neq 0\).


\subsection{Setning (Sjá Setningu 3.6.4)}
\label{\detokenize{Kafli03:setning-sja-setningu-3-6-4}}
Fall \(f\in {\cal O}(X)\) hefur núllstöð af stigi \(m>0\) í punktinum \(\alpha\in X\) þá og því aðeins að til sé \(g\in {\cal O}(X)\) þannig að \(g(\alpha)\neq 0\) og
\begin{equation*}
\begin{split}f(z)=(z-\alpha)^ mg(z), \qquad z\in X.\end{split}
\end{equation*}

\subsection{Samsendarsetning I (Sjá Setningu 3.7.1)}
\label{\detokenize{Kafli03:samsendarsetning-i-sja-setningu-3-7-1}}
Ef \(X\) er svæði í \({\mathbb{C}}\), \(f,g\in {\cal O}(X)\) og til er punktur \({\alpha}\) í \(X\) þannig að \(f^{(n)}({\alpha})=g^{(n)}({\alpha})\) fyrir öll \(n\geq 0\), þá er \(f(z)=g(z)\) fyrir öll \(z\in X\).


\subsection{Fylgisetning. (Sjá Setningu 3.7.2)}
\label{\detokenize{Kafli03:fylgisetning-sja-setningu-3-7-2}}
Ef \(X\) er svæði og \(f\in {\cal O}(X)\) er ekki núllfallið, þá er núllstöðvamengi \({\cal N}(f)=\{z\in X; f(z)=0\}\) fallsins \(f\) dreift hlutmengi af \(X\). (Þ.e.a.s. fyrir sérhvern punkt \(\alpha\in X\) er til tala \(\varrho>0\) þannig að hringskífan \(S(\alpha,\varrho)\) inniheldur enga núllstöð \(f\), nema hugsanlega \(\alpha\).)


\subsection{Samsemdarsetning II. (Sjá Setningu 3.7.3)}
\label{\detokenize{Kafli03:samsemdarsetning-ii-sja-setningu-3-7-3}}
Ef \(X\) er svæði, \(f,g\in {\cal O}(X)\) og \(f(a_j)=g(a_j)\) þar sem \(\{a_j\}\) er runa af ólíkum punktum, sem hefur markgildi \(a\in X\), þá er \(f(z)=g(z)\) fyrir öll \(z\in X\).


\subsection{Hágildislögmál I (Sjá Setningu 3.8.1)}
\label{\detokenize{Kafli03:hagildislogmal-i-sja-setningu-3-8-1}}
Ef \(X\) er svæði og \(f\in {\cal O}(X)\), þá getur \(|f(z)|\) ekki haft staðbundið hágildi í \(X\) nema \(f\) sé fastafall.


\subsection{Hágildislögmál II (Sjá Setning 3.8.2)}
\label{\detokenize{Kafli03:hagildislogmal-ii-sja-setning-3-8-2}}
Látum \(X\) vera takmarkað svæði og \(f\in {\cal O}(X)\cap C(\overline X)\) (samfellt á lokuninni \(\overline X\)). Þá tekur \(|f(z)|\) hágildi á jaðri svæðisins \(\partial X\).


\chapter{Leifareikningur}
\label{\detokenize{Kafli04:leifareikningur}}\label{\detokenize{Kafli04::doc}}
\sphinxstyleemphasis{Big flashy things have my name written all over them. Well… not yet, give me time and a crayon.}

- The Doctor, Doctor Who


\section{Laurent-raðir og sérstöðupunktar}
\label{\detokenize{Kafli04:laurent-rair-og-serstoupunktar}}

\subsection{Skilgreining (Sjá \S{}4.1)}
\label{\detokenize{Kafli04:skilgreining-sja-4-1}}
Mengi af gerðinni
\begin{equation*}
\begin{split}A(\alpha,\varrho_1,\varrho_2)=\{z\in {\mathbb{C}}\mid
\varrho_1<|z-\alpha|<\varrho_2\}\end{split}
\end{equation*}
þar sem \(0\leq\varrho_1<\varrho_2\leq +\infty\) kallast opinn hringkragi með miðju í \(\alpha\), innri geisla \(\varrho_1\), og ytri geisla \(\varrho_2\).


\subsection{Setning  (Sjá Setningu 4.1.1) (Laurent)}
\label{\detokenize{Kafli04:setning-sja-setningu-4-1-1-laurent}}
Látum \(X\) vera opið hlutmengi af \({\mathbb{C}}\) og gerum ráð fyrir að \(A(\alpha,\varrho_1,\varrho_2)\subset X\). Ef \(f\in {\cal O}(X)\), þá er unnt að skrifa \(f\) sem
\begin{equation*}
\begin{split}f(z)=\sum_{n=-\infty}^{+\infty}a_n(z-\alpha)^ n, \qquad z\in
A(\alpha,\varrho_1,\varrho_2),\end{split}
\end{equation*}
stuðlar raðarinnar \(a_n\) eru gefnir með formúlunni
\begin{equation*}
\begin{split}a_n=\dfrac 1{2\pi i}\int_{\partial S(\alpha,r)} \dfrac{f(\zeta)}
{(\zeta-\alpha)^{n+1}} \, d\zeta,\end{split}
\end{equation*}
og \(r\) getur verið hvaða tala sem er á bilinu
\(]\varrho_1,\varrho_2[\). Röðin
\begin{equation*}
\begin{split}\sum_{n=0}^{+\infty}a_n(z-\alpha)^ n\end{split}
\end{equation*}
er samleitin ef \(|z-\alpha|<\varrho_2\) og röðin
\begin{equation*}
\begin{split}\sum_{n=-\infty}^{-1}a_n(z-\alpha)^ n\end{split}
\end{equation*}
er samleitin ef \(|z-\alpha|>\varrho_1\). Báðar raðir eru
samleitnar á opna hringkraganum \(A(\alpha,\varrho_1, \varrho_2)\).


\subsection{Skilgreining (Sjá Skilgreiningu 4.1.2)}
\label{\detokenize{Kafli04:skilgreining-sja-skilgreiningu-4-1-2}}
Röð af gerðinni
\begin{equation*}
\begin{split}\sum_{-\infty}^{+\infty}a_n(z-\alpha)^ n\end{split}
\end{equation*}
kallast Laurent-röð. Innri samleitnigeisli raðarinnar \(\varrho_1\) er skilgreindur sem neðra mark yfir \(\varrho=|z-\alpha|\) þannig að
\begin{equation*}
\begin{split}\sum_{-\infty}^{-1} a_n(z-{\alpha})^ n\end{split}
\end{equation*}
er samleitin, ytri samleitnigeisli raðarinnar \(\varrho_2\) er skilgreindur sem efra mark yfir öll \(\varrho=|z-\alpha|\) þannig að
\begin{equation*}
\begin{split}\sum_{n=0}^{+\infty}a_n(z-{\alpha})^ n\end{split}
\end{equation*}
er samleitin. Ef \(\varrho_1<\varrho_2\) þá segjum við að Laurent-röðin sé samleitin.


\subsection{Skilgreining (Sjá \S{}4.2)}
\label{\detokenize{Kafli04:skilgreining-sja-4-2}}
Gefin er Laurent-röð
\begin{equation*}
\begin{split}\sum_{-\infty}^{+\infty}a_n(z-\alpha)^ n\end{split}
\end{equation*}
fyrir fágað fall \(f\). Stuðullinn \(a_{-1}\) kallast leif Laurent-raðarinnar eða leif \(f\) í \(\alpha\) og er táknaður \(\operatorname{Res}(f,\alpha)\) og röðin
\begin{equation*}
\begin{split}\sum_{n=-\infty}^{-1}a_n(z-{\alpha})^ n\end{split}
\end{equation*}
kallast höfuðhluti Laurent-raðarinnar eða höfuðhluti fallsins \(f\) í punktinum \(\alpha\).


\subsection{Skilgreining  (Sjá \S{}4.2)}
\label{\detokenize{Kafli04:id1}}
Punktur \(\alpha\) í mengi \(A\) kallast einangraður punktur í \(A\) ef til er opin hringskífa með miðju í \(\alpha\) sem inniheldur engan punkt úr \(A\) nema \(\alpha\).

Látum \(f\in{\cal O}(X)\). Ef \(\alpha\in\mathbb{C}\setminus X\) er einangraður punktur í \(A=\mathbb{C}\setminus X\) þá nefnist \(\alpha\) einangraður sérstöðupunktur \(f\).


\subsection{Skilgreining (Sjá \S{}4.2)}
\label{\detokenize{Kafli04:id2}}
Látum \(\alpha\) vera einangraðan sérstöðupunkt fyrir fágað fall \(f\). Ritum Laurent-röð \(f\) í \(\alpha\) sem
\begin{equation*}
\begin{split}\sum_{-\infty}^{+\infty}a_n(z-\alpha)^ n.\end{split}
\end{equation*}\begin{enumerate}
\sphinxsetlistlabels{\roman}{enumi}{enumii}{(}{)}%
\item {} 
\(\alpha\) er sagður afmáanlegur sérstöðupunktur ef höfuðhluti Laurent-raðarinnar er \(0\), þ.e.a.s. \(a_n=0\) fyrir öll \(n\leq -1\).

\item {} 
\(\alpha\) er sagt vera skaut ef höfuðhluti Laurent-raðarinnar er endanlegur en ekki 0. Skautið er sagt hafa stig \(m\) ef \(a_{-m}\neq 0\) en \(a_n=0\) fyrir öll \(n<-m\).

\item {} 
\(\alpha\) er sagt vera verulegur sérstöðupunktur ef höfuðhluti Laurent-raðarinnar er óendanlegur.

\end{enumerate}


\subsection{Setning (Sjá \S{}4.2 og Setningu 4.2.1)}
\label{\detokenize{Kafli04:setning-sja-4-2-og-setningu-4-2-1}}
Einangraður sérstöðupunktur \({\alpha}\) fágaða fallsins \(f\) sem skilgreint er á opnu mengi \(X\) er afmáanlegur ef og aðeins ef til er \(r>0\) og \(g\in {\cal O}(S({\alpha},r))\) þannig að \(S^*({\alpha},r)\subset X\) og \(f(z)=g(z)\) fyrir öll \(z\in S^*({\alpha},r)\).


\subsection{Setning Riemanns.}
\label{\detokenize{Kafli04:setning-riemanns}}
Ef \(\alpha\) er einangraður sérstöðupunktur
fágaða fallsins \(f\), og
\(\lim_{z\to \alpha}(z-\alpha)f(z)= 0\), þá er \(\alpha\)
afmáanlegur sérstöðupunktur.


\subsection{Setning (Sjá \S{}4.2)}
\label{\detokenize{Kafli04:setning-sja-4-2}}
Látum \(f\) vera fágað fall á opnu mengi \(X\) og \(\alpha\) vera einangraðan sérstöðupunkt fallsins \(f\). Sérstöðupunkturinn \(\alpha\) er skaut af stigi \(m>0\), ef og aðeins ef til er fágað fall \(g\in {\cal O}(U)\), þar sem \(U\) er grennd um \(\alpha\), þannig að \(g(\alpha)\neq 0\) og
\begin{equation*}
\begin{split}f(z)=\dfrac{g(z)}{(z-\alpha)^ m}, \qquad z\in U\setminus\{\alpha\}.\end{split}
\end{equation*}

\subsection{Setning}
\label{\detokenize{Kafli04:setning}}
Fall \(f\) hefur skaut í \(\alpha\) ef og
aðeins ef \(|f(z)|\to +\infty\) þegar \(z\to \alpha\).


\subsection{Setning (Stóra Picard-setningin.)}
\label{\detokenize{Kafli04:setning-stora-picard-setningin}}
Ef \(\alpha\) er verulegur sérstöðupunktur fágaðs falls \(f\) þá gildir að fyrir sérhvert \(\delta>0\) að mengið
\begin{equation*}
\begin{split}f(S^*(\alpha, \delta))=\{f(z)\mid z\in S^*(\alpha, \delta)\}\end{split}
\end{equation*}
er annaðhvort allt \({\mathbb{C}}\) eða til jafnt og \({\mathbb{C}}\setminus\{z_0\}\) þar sem \(z_0\) er einhver föst tvinntala.


\subsection{Setning (Sjá \S{}4.4, jöfnur 4.4.3 og 4.4.4)}
\label{\detokenize{Kafli04:setning-sja-4-4-jofnur-4-4-3-og-4-4-4}}
Látum \(f\) vera fágað fall og \(\alpha\) skaut \(f\).
\begin{enumerate}
\sphinxsetlistlabels{\roman}{enumi}{enumii}{(}{)}%
\item {} 
Ef skautið er einfalt (af stigi 1) þá er

\end{enumerate}
\begin{equation*}
\begin{split}\operatorname{Res}(f,\alpha)=\lim_{z\to \alpha}(z-\alpha)f(z).\end{split}
\end{equation*}\begin{enumerate}
\sphinxsetlistlabels{\roman}{enumi}{enumii}{(}{)}%
\setcounter{enumi}{1}
\item {} 
Ef skautið er af stigi \(m\) og við ritum \(f(z)=g(z)/(z-\alpha)^m\) þannig að \(g(\alpha)\neq 0\) þá er

\end{enumerate}
\begin{equation*}
\begin{split}\operatorname{Res}(f,\alpha)=\dfrac{g^{(m-1)}(\alpha)}{(m-1)!}.\end{split}
\end{equation*}

\section{Leifasetningin}
\label{\detokenize{Kafli04:leifasetningin}}

\subsection{Leifasetningin (Sjá Setningu 4.3.1)}
\label{\detokenize{Kafli04:leifasetningin-sja-setningu-4-3-1}}
Látum \(X\) vera opið hlutmengi í \({\mathbb{C}}\) og látum \(\Omega\) vera opið hlutmengi af \(X\) sem uppfyllir sömu forsendur og í Cauchy-setningunni. Látum \(A\) vera dreift hlutmengi af \(X\) sem sker ekki jaðarinn \(\partial\Omega\) á \(\Omega\). Ef \(f\in {\cal O}(X\setminus A)\), þá er
\begin{equation*}
\begin{split}\int_{\partial\Omega}f(z)\, dz = 2\pi i \sum_{\alpha\in \Omega\cap A}
\operatorname{Res}(f,\alpha).\end{split}
\end{equation*}
(Sjá \S{}4.4, jöfnur 4.4.3 og 4.4.4) Látum \(f\) vera fágað fall og \(\alpha\) skaut \(f\).
\begin{enumerate}
\sphinxsetlistlabels{\roman}{enumi}{enumii}{(}{)}%
\item {} 
Ef skautið er einfalt (af stigi 1) þá er

\end{enumerate}
\begin{equation*}
\begin{split}\operatorname{Res}(f,\alpha)=\lim_{z\to \alpha}(z-\alpha)f(z).\end{split}
\end{equation*}\begin{enumerate}
\sphinxsetlistlabels{\roman}{enumi}{enumii}{(}{)}%
\setcounter{enumi}{1}
\item {} 
Ef skautið er af stigi \(m\) og við ritum \(f(z)=g(z)/(z-\alpha)^m\) fyrir \(z\) í gataðri grennd um \(\alpha\) þannig að \(g(\alpha)\neq 0\) þá er

\end{enumerate}
\begin{equation*}
\begin{split}\operatorname{Res}(f,\alpha)=\dfrac{g^{(m-1)}(\alpha)}{(m-1)!}.\end{split}
\end{equation*}

\subsection{Setning (Sjá \S{}4.4, jöfnur 4.4.6 og 4.4.7)}
\label{\detokenize{Kafli04:setning-sja-4-4-jofnur-4-4-6-og-4-4-7}}
Gerum ráð fyrir að \(f(z)=g(z)/h(z)\) í grennd við punkt \(\alpha\) þar sem \(g(\alpha)\neq 0\) og \(\alpha\) er \(m\)-föld núllstöð fallsins \(h\) og \(h(z)=(z-\alpha)^mh_1(z)\) þar sem \(h_1(\alpha)\neq 0\). Þá er \(f\) með skaut af stigi \(m\) í \(\alpha\).

Ef \(m=1\) þá er
\begin{equation*}
\begin{split}\operatorname{Res}(f,\alpha)=\frac{g(\alpha)}{h'(\alpha)}.\end{split}
\end{equation*}
Ef \(m>1\) þá er
\begin{equation*}
\begin{split}\operatorname{Res}(f,\alpha)=\dfrac 1{(m-1)!}\cdot
\left.\dfrac {d^{m-1}}{dz^{m-1}}\bigg(\dfrac
{g(z)}{h_1(z)}\bigg)\right|_{z=\alpha}. \label{11.1.7}\end{split}
\end{equation*}

\subsection{Setning (Sjá \S{}4.5)}
\label{\detokenize{Kafli04:setning-sja-4-5}}
Látum \(f(x,y)\) vera fall af tveimur breytum sem er skilgreint á opnu mengi sem inniheldur einingarhringinn \(x^2+y^2=1\). Gerum ráð fyrir að til sé dreift mengi \(A\) sem inniheldur enga punkta úr einingarhringnum \(\partial S(0,1)\) og opið mengi \(X\) sem inniheldur \(\overline{S}(0,1)\) þannig að fallið
\begin{equation*}
\begin{split}g(z)=f\left(\frac{z^2+1}{2z}, \frac{z^2-1}{2iz}\right)\frac{1}{iz}\end{split}
\end{equation*}
sé fágað á \(X\setminus A\). Þá er
\begin{equation*}
\begin{split}\int_0^{2\pi}f(\cos\theta, \sin\theta)\,d\theta
=\int_{\partial S(0,1)}g(z)\,dz\\
= 2\pi i\sum_{\alpha\in A\cap S(0,1)}\operatorname{Res}(g(z),\alpha).\end{split}
\end{equation*}

\subsection{Setning (Sjá \S{}4.5)}
\label{\detokenize{Kafli04:id3}}
Látum \(f\) vera fall sem er fágað á menginu \({\mathbb{C}}\setminus A\) þar sem \(A\) er dreift mengi. Gerum ráð fyrir að í menginu \(A\) séu engar rauntölur. Fyrir rauntölu \(r>0\) látum við \(\gamma_r(\theta)=re^{i\theta}\) með \(0\leq\theta\leq \pi\) vera stikunn á hringboganum í efra hálfplaninu \(H_+\) frá \(r\) til \(-r\). Ef
\begin{equation*}
\begin{split}\int_{\gamma_r}f(z)\,dz\xrightarrow[r\rightarrow\infty]{} 0\end{split}
\end{equation*}
þá er
\begin{equation*}
\begin{split}\int_{-\infty}^\infty f(x)\,dx=2\pi i\sum_{\alpha\in A\cap H_+}\operatorname{Res}(f,\alpha).\end{split}
\end{equation*}
(Efra hálfplanið \(H_+\) er mengi allra tvinntalna \(z\) þannig
að \(\operatorname{Im\, } z>0\). Hægt er að setja fram álíka setningu þar sem er
tekinn sá hringbogi sem liggur í neðra hálfplaninu
\(H_-=\{z\in {\mathbb{C}}\mid \operatorname{Im\, } z<0\}\).)


\chapter{Þýð föll og fágaðar varpanir}
\label{\detokenize{Kafli05:y-foll-og-fagaar-varpanir}}\label{\detokenize{Kafli05::doc}}
\sphinxstyleemphasis{Bow ties are cool!}

- The Doctor, Doctor Who


\section{Þýð föll}
\label{\detokenize{Kafli05:y-foll}}
Skilgreining (Sjá \S{}5.1) Nabla-virkinn í tveimur víddum er diffurvirki sem er skilgreindur sem vigur
\begin{equation*}
\begin{split}\nabla=\Big(\frac{\partial}{\partial x},  \frac{\partial}{\partial y}\Big).\end{split}
\end{equation*}
Ef \(\varphi(x,y)\) er diffranlegt fall þá er stigull \(f\) skilgreindur sem
\begin{equation*}
\begin{split}\nabla f=\Big(\frac{\partial \varphi}{\partial x},  \frac{\partial \varphi}{\partial y}\Big),\end{split}
\end{equation*}
og ef \({\mathbf F}=(F_1, F_2)\) er vigursvið þá er sundurleitni \({\mathbf F}\) skilgreind sem
\begin{equation*}
\begin{split}\nabla\cdot{\mathbf F}=\frac{\partial F_1}{\partial x}+
\frac{\partial F_2}{\partial y}.\end{split}
\end{equation*}

\subsection{Skilgreining (Sjá \S{}5.1)}
\label{\detokenize{Kafli05:skilgreining-sja-5-1}}
Laplace-virkinn í tveimur víddum er diffurvirki sem er skilgreindur með formúlunni
\begin{equation*}
\begin{split}\Delta=\nabla^2 = \frac{\partial^2 }{\partial x^2}+\frac{\partial^2}{\partial y^2}\end{split}
\end{equation*}
þannig að ef \(\varphi(x,y)\) er diffranlegt fall þá er
\begin{equation*}
\begin{split}\Delta\varphi=\frac{\partial^2 \varphi}{\partial x^2}
+\frac{\partial^2\varphi}{\partial y^2}.\end{split}
\end{equation*}
Laplace-jafnan er hlutafleiðujafnan \(\Delta\varphi=0\) eða
\begin{equation*}
\begin{split}\frac{\partial^2 \varphi}{\partial x^2}
+\frac{\partial^2\varphi}{\partial y^2}=0.\end{split}
\end{equation*}

\subsection{Skilgreining}
\label{\detokenize{Kafli05:skilgreining}}
Látum \(X\) vera opið hlutmengi í \({\mathbb{C}}\). Ritum \(z=x+iy\). Fall \(u:X\to {\mathbb{R}}\) sem er þannig að allar 2. stigs hlutafleiður eru skilgreindar og samfelldar á öllu \(X\) er sagt vera þýtt ef í öllum punktum \(z\in  X\) er
\begin{equation*}
\begin{split}\frac{\partial^2 u}{\partial x^2}(z)
+\frac{\partial^2 u}{\partial y^2}(z)=0.\end{split}
\end{equation*}
(Oft er hentugt að samsama \({\mathbb{C}}\) og \({\mathbb{R}}^2\) og hugsa um \(u\) sem fall af tveimur raunbreytum.)


\subsection{Setning (Sjá Setning 5.1.2)}
\label{\detokenize{Kafli05:setning-sja-setning-5-1-2}}
Ef \(f\) er fágað fall á opnu mengi \(X\) í \({\mathbb{C}}\), þá eru \(u=\operatorname{Re\, } f\) og \(v=\operatorname{Im\, } f\) þýð föll og stiglar þeirra eru hornréttir í sérhverjum punkti í \(X\). Ef \(X\) er svæði og annað hvort \(u\) eða \(v\) er fastafall, þá er hitt fallið það líka.


\subsection{Skilgreining}
\label{\detokenize{Kafli05:id1}}
Svæði \(X\) í \({\mathbb{C}}\) er sagt vera einfaldlega samanhangandi ef ekki er til einfaldur lokaður vegur í \(X\) þannig að punktur úr \({\mathbb{C}}\setminus X\) er innan ferilsins. (Á mannamáli þá er mengi einfaldlega samanhangandi ef það hefur engin ,,göt‘‘.)


\subsection{Setning (Sjá Setning 5.1.5)}
\label{\detokenize{Kafli05:setning-sja-setning-5-1-5}}
Látum \(X\) vera einfaldlega samanhangandi svæði í \({\mathbb{C}}\) og \(u:{\mathbb{C}}\to {\mathbb{R}}\) þýtt fall á \(X\). Þá er til fágað fall \(f\) skilgreint á \(X\) þannig að \(u=\operatorname{Re\, } f\). Fallið \(f\) hefur formúlu
\begin{equation*}
\begin{split}f(z)=u(a)+ic+2\int_{\gamma_z}\frac{\partial u}{\partial \zeta}(\zeta)\,d\zeta,\end{split}
\end{equation*}
þar sem \(a\in X\) er einhver fastur punktur, \(c\) er rauntölufasti og \(\gamma_z\) er einhver vegur í \(X\) með upphafspunkt \(a\) og lokapunkt \(\zeta\).

Ef \(X\) er svæði í \({\mathbb{C}}\) sem er ekki einfaldlega samanhangandi þá er alltaf til þýtt fall \(u\) skilgreint á \(X\) þannig að ekki er til neitt fágað fall \(f\) á \(X\) með \(u=\operatorname{Re\, } f\).


\subsection{Fylgisetning}
\label{\detokenize{Kafli05:fylgisetning}}
Látum \(x\) vera opið mengi og \(u\) þýtt fall á \(x\). Látum \(\alpha\in X\). Ef \(r>0\) er tala þannig að \(S(\alpha, r)\subseteq X\) þá er til fágað fall \(f\), skilgreint á \(S(\alpha, r)\) þannig að fyrir öll \(z\in S(\alpha,r)\) er \(u(z)=\operatorname{Re\, } f(z)\).


\subsection{Meðalgildissetningin fyrir þýð föll}
\label{\detokenize{Kafli05:mealgildissetningin-fyrir-y-foll}}
Látum \(u\) vera þýtt fall skilgreint á opnu mengi \(X\) og \(\alpha= x_0+iy_0\in X\). Látum \(\varrho>0\) vera tölu þannig að \({S}(\alpha, \varrho)\subseteq X\). Fyrir \(0<r<\varrho\) er
\begin{equation*}
\begin{split}u(\alpha)=\frac{1}{2\pi}\int_0^{2\pi}u(\alpha+re^{it})\,dt.\end{split}
\end{equation*}
Ef við samsömum tvinntalnaplanið við \({\mathbb{R}}^2\) þá verður formúlan
svona
\begin{equation*}
\begin{split}u(x_0,y_0)=\frac{1}{2\pi}\int_0^{2\pi}u(x_0+r\cos t, y_0+r\sin t)\,dt.\end{split}
\end{equation*}

\subsection{Hágildislögmálið fyrir þýð föll}
\label{\detokenize{Kafli05:hagildislogmali-fyrir-y-foll}}
Látum \(u\) vera þýtt fall skilgreint á svæði \(X\). Fallið \(u\) tekur engin staðbundin útgildi á \(X\), nema þegar \(u\) er fastafall.


\section{Hagnýtingar í straumfræði}
\label{\detokenize{Kafli05:hagnytingar-i-straumfraei}}

\subsection{Skilgreining (Sjá Stærðfræðigreiningu II)}
\label{\detokenize{Kafli05:skilgreining-sja-staerfraeigreiningu-ii}}
Látum \(X\) vera opið mengi í planinu \({\mathbb{R}}^2\). Vigursvið \({\mathbf V}(x,y)=(p(x,y), q(x,y))\) er skilgreint á \(X\) og við gerum ráð fyrir að allar þær hlutafleiður sem við munum þurfa á að halda séu skilgreindar og samfelldar á öllu \(X\).


\subsection{Túlkun}
\label{\detokenize{Kafli05:tulkun}}
Við hugsum okkur að vigursviðið lýsi vökvaflæði í planinu þannig að í punkti \((x,y)\) þá er \({\mathbf V}(x,y)\) hraðavigur agnar sem berst með vökvanum. Vökvaflæðið hér breytist ekki með tíma.


\subsection{Straumlínur}
\label{\detokenize{Kafli05:straumlinur}}
Straumlína er ferill í \({\mathbb{R}}^2\) sem gefur braut agnar sem berst með vökvanum. Mest lýsandi fyrir straumlínur er að finna stikunn \(\gamma(t)\) þannig að ef ögnin er í punktinum \((x_0, y_0)=\gamma(0)\) á tíma \(t=0\) þá er hún í punktinum \(\gamma(t)\) á tíma \(t\).


\subsection{Ósamþjappanlegur vökvi}
\label{\detokenize{Kafli05:osamjappanlegur-vokvi}}
Gerum ráð fyrir að vökvinn sé hvergi að þjappast saman eða þenjast út. Þetta segir að ef við látum \(\Omega\) vera svæði í \(X\) þannig að jaðar \(\Omega\) er einfaldur lokaður ferill þá er ,,nettó‘‘flæðið út úr \(\Omega\) jafnt of 0. Með vísan til Sundurleitnisetningarinnar þá má lýsa þessum eiginleika með því að \({\mathbf V}\) sé sundurleitnilaus, þ.e.a.s.
\begin{equation*}
\begin{split}\frac{\partial p}{\partial x}+ \frac{\partial q}{\partial y}=0.\end{split}
\end{equation*}

\subsection{Engir hvirflar}
\label{\detokenize{Kafli05:engir-hvirflar}}
Gerum einnig ráð fyrir að engir hvirflar séu í vökvaflæðinu. Við viljum að hringstreymið eftir sérhverjum lokuðum einföldum ferli sé 0. Með vísan til Setningar Green (eða Setningar Stokes) þá er þetta jafngilt því að krefjast þess að vigursviðið sé rótlaust, þ.e.a.s.
\begin{equation*}
\begin{split}\frac{\partial q}{\partial x}-\frac{\partial p}{\partial y}=0.\end{split}
\end{equation*}

\subsection{Mætti}
\label{\detokenize{Kafli05:maetti}}
Fall \(\varphi:X\to {\mathbb{R}}\) kallast (raun)mætti fyrir \({\mathbf V}\) ef \({\mathbf V}(x,y)=\nabla\varphi(x,y)\) í öllum punktum \((x,y)\in X\).


\subsection{Skipt um umhverfi}
\label{\detokenize{Kafli05:skipt-um-umhverfi}}
Lítum nú á tvinntalnaplanið \({\mathbb{C}}\) og planið \({\mathbb{R}}^2\) sem sama hlutinn. Ritum nú \(z=x+iy\) og \({\mathbf V}(z)=p(z)+iq(z)\). Hér lítum við á \({\mathbf V}\) sem fall \({\mathbf V}:X\to {\mathbb{C}}\).


\subsection{Setning (Sjá \S{}5.2)}
\label{\detokenize{Kafli05:setning-sja-5-2}}
Fallið \(\overline{\mathbf V}:X\to {\mathbb{C}}\) þar sem \(\overline{\mathbf V}=p-iq\) er fágað á \(X\).

Ef \(f=\varphi+i\psi\) er fágað fall á \(X\) þannig að \(f'=\overline{\mathbf V}\) þá er \(\nabla \varphi={\mathbf V}\). Straumlínur \({\mathbf V}\) eru svo jafnhæðarlínur fallsins \(\psi\).


\subsection{Skilgreining (Sjá \S{}5.2)}
\label{\detokenize{Kafli05:skilgreining-sja-5-2}}
Fallið \(f\) kallast tvinnmætti fyrir \({\mathbf V}\), fallið \(\varphi:X\to {\mathbb{R}}\) kallast raunmætti fyrir \({\mathbf V}\) og fallið \(\psi:X\to {\mathbb{R}}\) kallast streymisfall.


\chapter{Undirstöðuatriði um afleiðujöfnur}
\label{\detokenize{Kafli06:undirstouatrii-um-afleiujofnur}}\label{\detokenize{Kafli06::doc}}
\sphinxstyleemphasis{I love humans. Always seeing patterns in things that aren’t there.”}

- The Doctor, Doctor Who


\section{Afleiðujöfnur}
\label{\detokenize{Kafli06:afleiujofnur}}

\subsection{Skilgreining (Sjá \S{}6.1)}
\label{\detokenize{Kafli06:skilgreining-sja-6-1}}
Venjuleg afleiðujafna (eða bara afleiðujafna eða diffurjafna) er jafna sem lýsir sambandinu á milli gilda falls af einni breytistærð og gilda afleiðu þess.


\subsection{Uppsetning (Sjá \S{}6.1)}
\label{\detokenize{Kafli06:uppsetning-sja-6-1}}
Sérhverja afleiðujöfnu má rita á forminu
\begin{equation*}
\begin{split}F(t,u,u',u'',\dots,u^{(m)})=0\end{split}
\end{equation*}
þar sem við hugsum okkur að \(t\) sé breytistærð, sem tekur gildi í einhverju hlutmengi \(A\) af \(\mathbb{R}\) og að \(u\) sé óþekkt fall sem skilgreint er á \(A\) og tekur gildi í \(\mathbb{R}\), \({\mathbb{C}}\) eða jafnvel \(\mathbb{R}^m\).

Lausn á afleiðujöfnunni er fall \(u\) skilgreint á opnu bili \(I\) í \(A\) þannig að fyrir öll \(t\in  I\) er
\begin{equation*}
\begin{split}F(t,u(t),u'(t),u''(t),\dots,u^{(m)}(t))=0.\end{split}
\end{equation*}

\subsection{Skilgreining (Sjá \S{}6.1)}
\label{\detokenize{Kafli06:id1}}
Stig afleiðujöfnu er hæsta stig á afleiðu, sem kemur fyrir í jöfnunni. Við segjum að \(m\)-ta stigs afleiðujafna sé á staðalformi þegar hún hefur verið umrituð yfir í jafngilda jöfnu af taginu
\begin{equation*}
\begin{split}u^{(m)}=G(t,u,u',\dots,u^{(m-1)}).\end{split}
\end{equation*}

\subsection{Grundvallarspurningar}
\label{\detokenize{Kafli06:grundvallarspurningar}}
Ef gefin er afleiðujafna er þá endilega til lausn?

Er hægt að finna lausn sem uppfyllir tiltekin viðbótarskilyrði, t.d. lausn \(u\) þannig að \(u(a)=b\)?

Ef til er lausn er hún þá ótvírætt ákvörðuð? Hvernig viðbótarskilyrðum þarf að bæta við til að fá ótvírætt ákvarðaða lausn?

Hvernig finnur maður lausn?

Ef maður getur ekki fundið beina formúlu fyrir lausn er samt hægt að
álykta eitthvað um eiginleika lausnar?


\subsection{Skilgreining (Sjá \S{}6.1)}
\label{\detokenize{Kafli06:id2}}
Afleiðujafna af gerðinni
\begin{equation*}
\begin{split}a_m(t)u^{(m)}+a_{m-1}(t)u^{(m-1)}+\cdots+a_0(t)u
 =f(t),\end{split}
\end{equation*}
þar sem föllin \(a_0,\dots,a_m,f\) eru skilgreind á bili \(I\subset \mathbb{R}\), er sögð vera línuleg. Línuleg afleiðujafna er sögð óhliðruð ef fallið \(f(t)\) í hægri hlið er fastafallið 0 en hliðruð annars.


\subsection{Skilgreining (Sjá \S{}6.3)}
\label{\detokenize{Kafli06:skilgreining-sja-6-3}}
Afleiðujöfnuhneppi (nákvæmar, venjulegt afleiðujöfnuhneppi) er safn af jöfnum sem lýsa sambandi milli gilda óþekktra falla \(u_1, \ldots, u_k\) af einni breytistærð og gilda á einstökum afleiðum þeirra. Venjulegt afleiðujöfnuhneppi er alltaf hægt að umrita yfir í jöfnur af gerðinni
\begin{equation*}
\begin{split}F_j(t,u_1,\dots,u_k,u_1',\dots,u_k',\dots,
u_1^{(m)},\dots,u_k^{(m)})=0,\qquad
j=1,\dots,l,\end{split}
\end{equation*}
þar sem \(t\) táknar breytistærðina, \(u_1,\dots,u_k\) eru óþekktu föllin og föllin \(F_1,\dots,F_l\) taka gildi í \(\mathbb{R}\) eða \({\mathbb{C}}\). Til þess að einfalda ritháttinn, þá skilgreinum við vigurgildu föllin \(u=(u_1,\dots,u_k)\) og \(F=(F_1,\dots,F_l)\). Þá eru jöfnurnar hér að ofan jafngildar vigurjöfnunni
\begin{equation*}
\begin{split}F(t,u,u',\dots,u^{(m)})=0.\end{split}
\end{equation*}
Lausn jöfnunnar er vigurfall \(u=(u_1,\dots,u_k)\) þar sem föllin \(u_1, \cdots, u_k\) eru öll skilgreind á opnu bili \(I\), þannig að vigurinn \(u(t)\) er í skilgreiningarmengi fallsins \(F\) fyrir öll \(t\in I\) og uppfyllir jöfnuna.

Stig afleiðujöfnuhneppis er skilgreint sem hæsta stig á afleiðu sem
kemur fyrir í jöfnunni.


\subsection{Skilgreining (Sjá \S{}6.3)}
\label{\detokenize{Kafli06:id3}}
Við segjum að hneppið sé á staðalformi, ef fjöldi jafna og fjöldi óþekktra falla er sá sami og það má rita á forminu
\begin{equation*}
\begin{split}\begin{aligned}
u_1'&= G_1(t, u_1,\dots, u_m),\\
u_2'&= G_2(t, u_1,\dots, u_m),\\
&\quad \vdots\\
u_m'&= G_m(t, u_1,\dots, u_m),\end{aligned}\end{split}
\end{equation*}

\subsection{Skilgreining (Sjá \S{}6.3)}
\label{\detokenize{Kafli06:id4}}
Við segjum að fyrsta stigs afleiðujöfnuhneppi sé línulegt ef það má rita á forminu
\begin{equation*}
\begin{split}\begin{aligned}
u_1'&=a_{11}(t)u_1+\cdots+a_{1m}(t)u_m+f_1(t),\\
u_2'&=a_{21}(t)u_1+\cdots+a_{2m}(t)u_m+f_2(t),\\
&\qquad \qquad \vdots\qquad \qquad \qquad \qquad \vdots\\
u_m'&=a_{m1}(t)u_1+\cdots+a_{mm}(t)u_m+f_m(t).\end{aligned}\end{split}
\end{equation*}
Við segjum að hneppið sé óhliðrað ef \(f_i\) er núllfallið fyrir öll \(i\) og við segjum að það sé hliðrað annars.


\subsection{Setning (Sjá \S{}6.3)}
\label{\detokenize{Kafli06:setning-sja-6-3}}
Sérhverja venjulega afleiðujöfnu á staðalformi
\begin{equation*}
\begin{split}v^{(m)}=G(t,v,v',\dots,v^{(m-1)})\end{split}
\end{equation*}
má umrita sem jafngilt afleiðujöfnuhneppi (lausnir afleiðujöfnunnar gefa lausnir á hneppinu og öfugt) sem er fundið þannig að við setjum
\begin{equation*}
\begin{split}u_1=v, \qquad u_2=u_1', \quad \ldots\quad u_m=v^{(m-1)},\end{split}
\end{equation*}
og jöfnuhneppið er
\begin{equation*}
\begin{split}\begin{aligned}
u_1'&=u_2\\
u_2'&=u_3\\
&\ \,\vdots\\
u_m'&=G(t,u_1,u_2,\dots,u_m).\end{aligned}\end{split}
\end{equation*}

\subsection{Skilgreining (Sjá \S{}6.4)}
\label{\detokenize{Kafli06:skilgreining-sja-6-4}}
Upphafsgildisverkefni snúast um að leysa afleiðujöfnu eða afleiðujöfnuhneppi með því hliðarskilyrði að lausnin og einhverjar afleiður hennar taki fyrirfram gefin gildi í ákveðnum punkti.


\subsection{Upphafsgildisverkefni fyrir línulega afleiðujöfnu (Sjá \S{}6.4)}
\label{\detokenize{Kafli06:upphafsgildisverkefni-fyrir-linulega-afleiujofnu-sja-6-4}}
Upphafsgildisverkefni fyrir línulega \(m\)-ta stigs afleiðujöfnu er sett fram sem
\begin{equation*}
\begin{split}\begin{cases} a_m(t)v^{(m)}+\cdots+a_1(t)v'+a_0(t)v=g(t), & t\in I,\\
v(a)=b_0, \quad v'(a)=b_1, \quad \dots \quad  v^{(m-1)}(a)=b_{m-1}.&
\end{cases}\end{split}
\end{equation*}
Það að leysa upphafgildis verkefnið felst í því að finna lausn \(v\) á afleiðujöfnunni sem uppfyllir skilyrðin um gildi á \(v(a),\ldots, v^{(m-1)}(a)\).


\subsection{Skilgreining (Sjá \S{}6.5)}
\label{\detokenize{Kafli06:skilgreining-sja-6-5}}
Jaðargildisverkefni snúast um að leysa afleiðujöfnu
\begin{equation*}
\begin{split}u^{(m)}=f(t,u,u',\dots,u^{(m-1)})\end{split}
\end{equation*}
af stigi \(m\) á takmörkuðu bili \(I=[a,b]\) með skilyrðum á einhver gildanna (ekki endilega öll)
\begin{equation*}
\begin{split}u(a), \ u'(a),\dots,  \ u^{(m-1)}(a)\qquad \text{ og }
\qquad  u(b), \ u(b),\dots, \ u^{(m-1)}(b).\end{split}
\end{equation*}

\subsection{Útfærsla jaðargildisverkefna}
\label{\detokenize{Kafli06:utfaersla-jaargildisverkefna}}
Skilyrði eru venjulega sett fram þannig að ákveðnar línulegar samantektir af þessum fallgildum og afleiðum eigi að taka fyrirfram gefin gildi. Fyrir annars stigs jöfnu geta jaðarskilyrðin til dæmis verið
\begin{equation*}
\begin{split}u(a)=0, \qquad u'(b)=0.\end{split}
\end{equation*}
Lotubundin jaðarskilyrði eru af gerðinni
\begin{equation*}
\begin{split}u(a)=u(b), \qquad u'(a)=u'(b).\end{split}
\end{equation*}

\subsection{Skilgreining}
\label{\detokenize{Kafli06:skilgreining}}
Hlutafleiðujafna er jafna sem lýsir sambandinu á milli gilda falls af fleiri en einni breytistærð og einstakra hlutafleiða þess.


\section{Upprifjun á lausnaaðferðum og hagnýtingar}
\label{\detokenize{Kafli06:upprifjun-a-lausnaaferum-og-hagnytingar}}

\subsection{Línulegar fyrsta stigs jöfnur}
\label{\detokenize{Kafli06:linulegar-fyrsta-stigs-jofnur}}
Almenna línulega fyrsta stigs afleiðujöfnu má rita á forminu
\begin{equation*}
\begin{split}u'+p(t)u=q(t).\end{split}
\end{equation*}
Skilgreinum \(\mu(t)=\int p(t)\,dt\) (eitthvert stofnfall). Þá er
\begin{equation*}
\begin{split}u(t)=e^{-\mu(t)}\int e^{\mu(t)}q(t)\,dt\end{split}
\end{equation*}
lausn á afleiðujöfnunni. (Þegar þið reiknið \(\mu(t)=\int p(t)\,dt\) þá megið þið sleppa heildunarfasta, en ekki þegar þið reiknið heildið \(\int e^{\mu(t)}q(t)\,dt\).)


\subsection{Fyrsta stigs aðgreinanlegar afleiðujöfnur}
\label{\detokenize{Kafli06:fyrsta-stigs-agreinanlegar-afleiujofnur}}
Fyrsta stigs afleiðujafna sem hægt er að rita á forminu
\begin{equation*}
\begin{split}\frac{du}{dt}=f(t)g(u)\end{split}
\end{equation*}
kallast aðgreinanleg (e. seperable). (Hægri hlið má þátta þannig að annar þátturinn er bara fall af \(t\) og hinn þátturinn er bara fall af \(u\).)

Umritum jöfnuna yfir á formið
\begin{equation*}
\begin{split}\frac{du}{g(u)}=f(t)\,dt.\end{split}
\end{equation*}
(Ekkert \(t\) í vinstri hlið, ekkert \(u\) í hægri hlið.) Síðan smellum við heildum á báðar hliðar og fáum að
\begin{equation*}
\begin{split}\int\frac{du}{g(u)}=\int f(t)\,dt.\end{split}
\end{equation*}
Reiknum stofnföll og munum eftir að setja inn heildunarfasta (einn er nóg). Þá höfum við jöfnu sem tengir saman \(t\) og \(u\) og út frá þeirri jöfnu má fá upplýsingar um eiginleika lausnarinnar \(u\).

Stundum er hægt að einangra \(u\) og fá þannig formúlu fyrir lausn afleiðujöfnunnar.


\subsection{Annars stigs óhliðraðar línulegar afleiðujöfnur með fastastuðlum}
\label{\detokenize{Kafli06:annars-stigs-ohliraar-linulegar-afleiujofnur-me-fastastulum}}
Finna á lausn á afleiðujöfnu \(au''+bu'+cu=0\). Kennijafna hennar er \(a\lambda^2+b\lambda+c=0\).
\begin{enumerate}
\sphinxsetlistlabels{\roman}{enumi}{enumii}{(}{)}%
\item {} 
Kennijafnan \(a\lambda^2+b\lambda+c=0\) hefur tvær ólíkar rauntölulausnir \(\lambda_1\) og \(\lambda_2\). Fallið

\end{enumerate}
\begin{equation*}
\begin{split}u(t)=Ae^{\lambda_1t}+Be^{\lambda_2t}\end{split}
\end{equation*}
er alltaf lausn sama hvernig fastarnir \(A\) og \(B\) eru valdir og sérhverja lausn má rita á þessu formi.
\begin{enumerate}
\sphinxsetlistlabels{\roman}{enumi}{enumii}{(}{)}%
\setcounter{enumi}{1}
\item {} 
Kennijafnan \(a\lambda^2+b\lambda+c=0\) hefur bara eina rauntölulausn \(k=-\frac{b}{2a}\). Fallið

\end{enumerate}
\begin{equation*}
\begin{split}u(t)=Ae^{kt}+Bte^{kt}\end{split}
\end{equation*}
er alltaf lausn sama hvernig fastarnir \(A\) og \(B\) eru valdir og sérhverja lausn má rita á þessu formi.
\begin{enumerate}
\sphinxsetlistlabels{\roman}{enumi}{enumii}{(}{)}%
\setcounter{enumi}{2}
\item {} 
Kennijafnan \(a\lambda^2+b\lambda+c=0\) hefur engar rauntölulausnir. Setjum \(k=-\frac{b}{2a}\) og \(\omega=\frac{\sqrt{4ac-b^2}}{2a}\). Rætur kennijöfnunnar eru \(\lambda_1=k+i\omega\) og \(\lambda_2=k-i\omega\). Fallið

\end{enumerate}
\begin{equation*}
\begin{split}u(t)=Ae^{kt}\cos(\omega t)+Be^{kt}\sin(\omega t)\end{split}
\end{equation*}
er alltaf lausn sama hvernig fastarnir \(A\) og \(B\) eru
valdir og sérhverja lausn má rita á þessu formi.


\section{Tilvist og ótvíræðni lausna}
\label{\detokenize{Kafli06:tilvist-og-otviraeni-lausna}}

\subsection{Setning Peano (Sjá Setningu 6.6.1)}
\label{\detokenize{Kafli06:setning-peano-sja-setningu-6-6-1}}
Gerum ráð fyrir að \(\Omega\) sé grennd um punktinn \((a,b)\in \mathbb{R}\times\mathbb{R}^m\) og að \(f\in C(\Omega,\mathbb{R}^m)\). Þá er til opið bil \(I\) sem inniheldur punktinn \(a\) og fall \(u:I\to \mathbb{R}^m\), þannig að \((t,u(t))\in \Omega\), \(u'(t)=f(t,u(t))\) fyrir öll \(t\in I\) og \(u(a)=b\).


\subsection{Dæmi (Sjá Sýnidæmi 6.6.2)}
\label{\detokenize{Kafli06:daemi-sja-synidaemi-6-6-2}}
Athugum upphafsgildisverkefnið \(u'=3u^{2/3}\), \(u(0)=0\). Fyrir sérhvert \(\alpha>0\) fáum við lausnina \(u_\alpha\), sem skilgreind er með
\begin{equation*}
\begin{split}u_\alpha(t)=\begin{cases}
(t+\alpha)^3, &t<-\alpha,\\
0, &-\alpha\leq t<\alpha,\\
(t-\alpha)^3, &\alpha\leq t.
\end{cases}\end{split}
\end{equation*}
Þetta dæmi sýnir okkur að til þess að fá ótvírætt ákvarðaða lausn þurfum við að setja einhver strangari skilyrði á \(f\) en samfelldni.


\subsection{Skilgreining (Sjá Skilgreiningu 6.6.3)}
\label{\detokenize{Kafli06:skilgreining-sja-skilgreiningu-6-6-3}}
Látum \(f:\Omega\to\mathbb{R}^m\) vera fall, þar sem \(\Omega\subset \mathbb{R}\times \mathbb{R}^m\) og \(A\subset \Omega\). Ef til er fasti \(C\) þannig að
\begin{equation*}
\begin{split}|f(t,x)-f(t,y)|\leq C|x-y|,\qquad (t,x), (t,y)\in
 A,\end{split}
\end{equation*}
þá segjum við að \(f\) uppfylli Lipschitz\textendash{}skilyrði í menginu \(A\).


\subsection{Setning (Sjá Setningu 6.6.5) (Picard. Víðfeðm útgáfa.)}
\label{\detokenize{Kafli06:setning-sja-setningu-6-6-5-picard-vifem-utgafa}}
Látum \(I\subset \mathbb{R}\) vera opið bil, \(a\in I\), \(b\in \mathbb{R}^m\), \(f\in C(I\times \mathbb{R}^m,\mathbb{R}^m)\) og gerum ráð fyrir að \(f\) uppfylli Lipschitz\textendash{}skilyrði í \(J\times \mathbb{R}^m\) fyrir sérhvert lokað og takmarkað hlutbil \(J\) í \(I\). Þá er til ótvírætt ákvörðuð lausn \(u\in C^1(I,\mathbb{R}^ m)\) á upphafsgildisverkefninu
\begin{equation*}
\begin{split}u'=f(t,u), \qquad u(a)=b.\end{split}
\end{equation*}

\subsection{Fylgisetning. (Sjá Fylgisetningu 6.6.6)}
\label{\detokenize{Kafli06:fylgisetning-sja-fylgisetningu-6-6-6}}
Látum \(I\subset \mathbb{R}\) vera opið bil, \(a\in I\), \(b\in {\mathbb{C}}^m\), \(A\in C(I,{\mathbb{C}}^{m\times m})\) og \(f\in C(I,{\mathbb{C}}^m)\). Þá er til ótvírætt ákvörðuð lausn \(u\in C^1(I,{\mathbb{C}}^ m)\) á upphafsgildisverkefninu
\begin{equation*}
\begin{split}u'=A(t)u+f(t) \qquad u(a)=b.\end{split}
\end{equation*}

\subsection{Fylgisetning. (Sjá Fylgisetningu 6.6.7)}
\label{\detokenize{Kafli06:fylgisetning-sja-fylgisetningu-6-6-7}}
Látum \(I\subset \mathbb{R}\) vera opið bil, \(a\in I\), \(b_0,\dots,b_{m-1} \in {\mathbb{C}}\), \(a_0,\dots,a_m, g\in C(I)\) og \(a_m(t)\neq 0\) fyrir öll \(t\in I\). Þá er til ótvírætt ákvörðuð lausn \(u\in C^m(I)\) á upphafsgildisverkefninu
\begin{equation*}
\begin{split}\begin{gathered}
a_m(t)u^{(m)}+\cdots+a_1(t)u'+a_0(t)u=g(t),\\
u(a)=b_0, u'(a)=b_1,\dots, u^{(m-1)}(a)=b_{m-1}.\end{gathered}\end{split}
\end{equation*}

\subsection{Setning (Sjá Setningu 6.6.8) (Picard. Staðbundin útgáfa.)}
\label{\detokenize{Kafli06:setning-sja-setningu-6-6-8-picard-stabundin-utgafa}}
Látum \(\Omega\) vera opið hlutmengi í \(\mathbb{R}\times \mathbb{R}^{m}\), \(a\in \mathbb{R}\), \(b\in \mathbb{R}^m\), \((a,b)\in \Omega\) og \(f\in C(\Omega,\mathbb{R}^m)\). Gerum ráð fyrir að til sé grennd \(U\) um punktinn \((a,b)\) innihaldin í \(\Omega\) og að fallið \(f\) uppfylli Lipschitz\textendash{}skilyrði í \(U\). Þá er til opið bil \(I\) á \(\mathbb{R}\) sem inniheldur \(a\) og ótvírætt ákvörðuð lausn \(u\in C^1(I, \mathbb{R}^m)\) á upphafsgildisverkefninu
\begin{equation*}
\begin{split}u'=f(t,u), \qquad u(a)=b.\end{split}
\end{equation*}

\subsection{Picard-nálgun. (Sjá Sýnidæmi 6.6.10 og 6.6.11)}
\label{\detokenize{Kafli06:picard-nalgun-sja-synidaemi-6-6-10-og-6-6-11}}\begin{enumerate}
\sphinxsetlistlabels{\roman}{enumi}{enumii}{(}{)}%
\item {} 
Upphafsgildisverkefnið

\end{enumerate}
\begin{equation*}
\begin{split}u'=f(t,u),\qquad u(a)=b,\end{split}
\end{equation*}
er jafngilt heildisjöfnunni (lausn upphafsgildisverkefnis er lausn heildisjöfnu og öfugt)
\begin{equation*}
\begin{split}u=b+\int_a^t f(\tau,u)\,d\tau.\end{split}
\end{equation*}\begin{enumerate}
\sphinxsetlistlabels{\roman}{enumi}{enumii}{(}{)}%
\setcounter{enumi}{1}
\item {} 
Gerum ráð fyrir að fallið \(f(t,u)\) uppfylli Lipschitz-skilyrði eins og í Skilgreiningu 15.3. Látum \(u_0(t)=b\) fyrir öll \(t\in I\). Skilgreinum svo runu falla \(u_1, u_2, \ldots\) með þrepun þannig að fyrir \(n\geq 1\) er

\end{enumerate}
\begin{equation*}
\begin{split}u_n(t)=b+\int_a^t f(\tau, u_{n-1}(\tau))\,d\tau.\end{split}
\end{equation*}
Runan \(u_0, u_1, u_2, \ldots\) hefur sem markgildi fall \(u\) sem er lausn á upphafsgildisverkefninu
\begin{equation*}
\begin{split}u'=f(t,u),\qquad u(a)=b.\end{split}
\end{equation*}

\subsection{Merking tilvistar- og ótvíræðnisetninga}
\label{\detokenize{Kafli06:merking-tilvistar-og-otviraenisetninga}}
Skoðum upphafsgildisverkefni
\begin{equation*}
\begin{split}u'=f(t,u),\qquad u(a)=b,\end{split}
\end{equation*}
þar sem fallið \(f(t,u)\) uppfyllir Lipschitz-skilyrði.  Hugsum okkur að afleiðujafnan lýsi ,,kerfi‘‘ og að við þekkjum ástand þess ,,núna‘‘ þegar tíminn er \(t=a\).
\begin{enumerate}
\sphinxsetlistlabels{\roman}{enumi}{enumii}{(}{)}%
\item {} 
Lausn er til!

\item {} 
Lausnin er ótvírætt ákvörðuð. Ef við vitum ástand kerfisins núna þá getum við sagt fyrir með ótvíræðum hætti fyrir um ástand þess í framtíðinni.

\item {} 
Upphafsgildið er oft fengið með mælingum og þá má búast við mæliskekkju. Aðferðina við að sanna ótvíræðni lausnar má nýta til að sýna að ,,lausnin breytist samfellt” ef upphafsgildi er breytt. Hundalógikin er að ,,lítil‘‘ skekkja í upphafsgildi leiðir til ,,lítillar‘‘ skekkju í lausn. (Athugið að þegar gildi lausna í punkti \(t\) ,,langt‘‘ frá \(t=a\) eru skoðuð þá getur verið mikill munur á ,,réttri‘‘ lausn og lausn sem fengin er út frá mæligildi.)

\item {} 
Alla jafna má líka segja að ,,lausnin breytist samfellt” með \(f(t,u)\). Stuðlar sem koma fyrir í \(f(u,t)\) eru oft fengnir með mælingum. Þetta er flókið viðfangsefni og mögulegt að hegðun lausnar gjörbreytist við smá breytingu í stuðlum í afleiðujöfnu.

\end{enumerate}


\chapter{Línulegar afleiðujöfnur}
\label{\detokenize{Kafli07:linulegar-afleiujofnur}}\label{\detokenize{Kafli07::doc}}
\sphinxstyleemphasis{This,’ whispered the Doctor to Romana, ‘is going to be like trying to find a book about needles in a room full of books about haystacks.}

- The Doctor, Doctor Who


\section{Línulegar afleiðujöfnur}
\label{\detokenize{Kafli07:id1}}

\subsection{Athugasemd}
\label{\detokenize{Kafli07:athugasemd}}
Umfjöllun á þessu efni verður skýrari ef við leyfum
föllunum sem fengist er við að taka ekki bara rauntölugildi heldur líka
tvinntölugildi. Fastar og stuðlar sem koma fyrir geta því verið
tvinntölur og þurfa ekki að vera rauntölur. Breytistæðirnar i föllunum,
oftast táknaðar með \(t\) eða \(x\), eru hinsvegar alltaf
rauntölur.


\subsection{Skilgreining (Sjá \S{}7.1)}
\label{\detokenize{Kafli07:skilgreining-sja-7-1}}
Afleiðujafna af gerðinni
\begin{equation*}
\begin{split}a_m(t)u^{(m)}+a_{m-1}(t)u^{(m-1)}+\cdots+a_1(t)u'+a_0(t)u=f(t),\end{split}
\end{equation*}
þar sem föllin \(a_0,\dots,a_m,f\) eru skilgreind á bili \(I\subset \mathbb{R}\), er sögð vera línuleg afleiðujafna. Við segjum að jafnan sé óhliðruð ef \(f\) er núllfallið, annars er sagt að hún sé hliðruð.

Fyrir sérhvern punkt \(t\in I\) fæst margliða í einni breytistærð \(\lambda\)
\begin{equation*}
\begin{split}P(t,\lambda)= a_m(t)\lambda^{m}+a_{m-1}(t)\lambda^{m-1}+
\cdots+a_1(t)\lambda+a_0(t),\end{split}
\end{equation*}
sem er kölluð kennimargliða jöfnunnar.


\subsection{Skilgreining og setning (Sjá \S{}7.1)}
\label{\detokenize{Kafli07:skilgreining-og-setning-sja-7-1}}
Látum \(a_0(t), \ldots, a_m(t)\) vera samfelld föll sem eru öll skilgreind á bili \(I\). Rifjum upp að mengi samfelldra falla sem eru skilgreind á \(I\) er táknað \(C(I)\) og mengi falla á \(I\) sem eru \(m\)-sinnum diffranleg með samfellda \(m\)-tu afleiðu er táknað \(C^m(I)\). Skilgreinum afleiðuvirkja \(L:C^m(I)\to C(I)\) þannig að fyrir fall \(u(t)\in C^m(I)\) er
\begin{equation*}
\begin{split}Lu(t)=a_m(t)u^{(m)}(t)+a_{m-1}(t)u^{(m-1)}(t)+\cdots+a_1(t)u'(t)+
a_0(t)u(t).\end{split}
\end{equation*}
Þegar mengin \(C(I)\) og \(C^m(I)\) eru skoðuð sem vigurrúm yfir \({\mathbb{C}}\) þá er vörpunin \(L\) línuleg.


\subsection{Skilgreining (Sjá \S{}7.1)}
\label{\detokenize{Kafli07:id2}}
Skilgreinum virkja \(D:C^1(I)\to C(I)\) þar sem \(I\) er bil þannig að \(Du=u'\). Látum \(a_0(t), \ldots, a_m(t)\) vera samfelld föll sem eru öll skilgreind á bili \(I\) og setjum
\begin{equation*}
\begin{split}P(t,D)= a_m(t)D^{m}+a_{m-1}(t)D^{m-1}+
\cdots+a_1(t)D+a_0(t).\end{split}
\end{equation*}
Ef \(L\) er afleiðuvirkinn sem er skilgreindur hér að ofan þá er \(Lu=P(t,D)u\) fyrir öll föll \(u\in C^m(I)\).


\subsection{Setning (Sjá \S{}7.1)}
\label{\detokenize{Kafli07:setning-sja-7-1}}
Kjarni eða núllrúm virkjans \(P(t,D)\) samanstendur af öllum lausnum \(u\) á óhliðruðu jöfnunni \(P(t,D)u=0\). Fyrst \(P(t,D)\) er línulegur virki, þá er núllrúmið vigurrúm yfir tvinntölurnar. Við táknum það með \({\cal N}(P(t,D))\).


\subsection{Fylgisetning (Sjá Setningu 7.1.3)}
\label{\detokenize{Kafli07:fylgisetning-sja-setningu-7-1-3}}
Látum \(a_0(t), \ldots, a_m(t)\) vera samfelld föll sem eru öll skilgreind á bili \(I\). Gerum ráð fyrir að \(a_m(t)\neq 0\) fyrir öll \(t\in I\). Til eru föll \(u_1, \ldots, u_m\) skilgreind á \(I\) þannig að sérhverja lausn á afleiðujöfnunni
\begin{equation*}
\begin{split}a_m(t)u^{(m)}+a_{m-1}(t)u^{(m-1)}+\cdots+a_1(t)u'+a_0(t)u=0,\end{split}
\end{equation*}
má rita sem \(u=c_1u_1+\cdots+c_mu_m\) þar sem \(c_1, \ldots, c_m\) eru fastar. Föllin \(u_1, \ldots, u_m\) mynda grunn fyrir núllrúm virkjans \(P(t,D)\).


\subsection{Fylgisetning (Sjá Setningu 7.1.4)}
\label{\detokenize{Kafli07:fylgisetning-sja-setningu-7-1-4}}
Látum \(a_0(t), \ldots, a_m(t), f(t)\) vera samfelld föll sem eru öll skilgreind á bili \(I\). Ef \(v(t)\) er einhver ein lausn (,,sérlausn‘‘) afleiðujöfnunnar
\begin{equation*}
\begin{split}a_m(t)u^{(m)}+a_{m-1}(t)u^{(m-1)}+\cdots+a_1(t)u'+a_0(t)u=f(t),\end{split}
\end{equation*}
þá má rita sérhverja lausn sem \(u=c_1u_1+\cdots+c_mu_m+v\) þar sem \(c_1, \ldots, c_m\) eru fastar og föllin \(u_1, \ldots, u_m\) mynda grunn fyrir núllrúm virkjans \(P(t,D)\).

Línulega samantektin \(c_1u_1+\cdots+c_mu_m\) gefur almenna lausnarformúlu fyrir afleiðujöfnuna
\begin{equation*}
\begin{split}a_m(t)u^{(m)}+a_{m-1}(t)u^{(m-1)}+\cdots+a_1(t)u'+a_0(t)u=0.\end{split}
\end{equation*}
Sérhverja lausn má því rita sem (lausn óhliðraðar)+(ákveðin sérlausn).


\subsection{Skilgreining  (Sjá \S{}7.2)}
\label{\detokenize{Kafli07:skilgreining-sja-7-2}}
Afleiðujafna á forminu
\begin{equation*}
\begin{split}a_mu^{(m)}+\cdots + a_1u'+a_0u=f(t)\end{split}
\end{equation*}
þar sem \(a_0, a_1, \ldots, a_m\) eru fastar kallast línuleg \(m\)-ta stigs afleiðujafna með fastastuðla. Kennimargliða hennar er
\begin{equation*}
\begin{split}P(\lambda)=a_m\lambda^{m}+\cdots + a_1\lambda+a_0.\end{split}
\end{equation*}

\subsection{Setning (Sjá Setningu 7.2.1)}
\label{\detokenize{Kafli07:setning-sja-setningu-7-2-1}}
Gerum ráð fyrir að \(P(D)\) sé línulegur afleiðuvirki af \(m\) með fastastuðla og að kennimargliðan \(P(\lambda)\) hafi \(\ell\) ólíkar núllstöðvar \(\lambda_1,\dots,\lambda_\ell\in {\mathbb{C}}\) með margfeldnina \(m_1,\dots,m_\ell\). Þá mynda föllin
\begin{equation*}
\begin{split}\begin{gathered}
e^{\lambda_1t}, te^{\lambda_1t},\dots, t^{m_1-1}e^{\lambda_1t},\\
e^{\lambda_2t}, te^{\lambda_2t},\dots, t^{m_2-1}e^{\lambda_2t},\\
\quad \vdots\qquad \vdots \qquad \qquad \vdots\\
e^{\lambda_\ell t}, te^{\lambda_\ell t},\dots, t^{m_\ell-1}e^{\lambda_\ell t},\end{gathered}\end{split}
\end{equation*}
grunn í núllrúmi virkjans \(P(D)\) og sérhvert stak í núllrúminu má rita sem
\begin{equation*}
\begin{split}q_1(t)e^{\lambda_1t}+\cdots+q_\ell(t)e^{\lambda_\ell t},\end{split}
\end{equation*}
þar sem \(q_j\) eru margliður af stigi \(<m_j\),
\(1\leq j\leq \ell\).


\subsection{Athugasemd}
\label{\detokenize{Kafli07:id3}}
Látum \(a_0, \ldots, a_m\) vera rauntölur. Viljum
finna raungildar lausnir
\begin{equation*}
\begin{split}a_nu^{(n)}+\cdots+a_1u'+a_0u=0.\end{split}
\end{equation*}
Hugsum okkur að \(\lambda=\alpha+i\beta\) sé \(m\)-föld rót kennimargliðu afleiðujöfnunar. Þá er \(\mu=\overline{\lambda}=\alpha-i\beta\) líka \(m\)-föld rót kennijöfnu. Þegar grunnur fyrir lausnarúmið er skrifaður má í stað
\begin{equation*}
\begin{split}e^{\lambda t}, te^{\lambda t},\dots, t^{m-1}e^{\lambda t},
e^{\mu t}, te^{\mu t},\dots, t^{m-1}e^{\mu t}\end{split}
\end{equation*}
hafa í grunninum raungildu föllin
\begin{equation*}
\begin{split}e^{\alpha t}\cos(\beta t), te^{\alpha t}\cos(\beta t), \ldots,
t^{m-1}e^{\alpha t}\cos(\beta t), e^{\alpha t}\sin(\beta t),
te^{\alpha t}\sin(\beta t), \ldots, t^{m-1}e^{\alpha t}\sin(\beta t).\end{split}
\end{equation*}

\subsection{Skilgreining (Sjá \S{}7.3)}
\label{\detokenize{Kafli07:skilgreining-sja-7-3}}
Afleiðujafna af gerðinni
\begin{equation*}
\begin{split}a_mx^mu^{(m)}+\cdots+a_1xu'+a_0u=0,\end{split}
\end{equation*}
þar sem stuðlarnir \(a_0,\ldots, a_m\) eru tvinntölur, kallast \(m\)-ta stigs Euler-jafna (sumsstaðar kallaðar Cauchy-Euler jöfnur).


\subsection{Setning (Sjá Setningu 7.3.1)}
\label{\detokenize{Kafli07:setning-sja-setningu-7-3-1}}
Gefin er afleiðujafna
\begin{equation*}
\begin{split}a_mx^mu^{(m)}+\cdots+a_1xu'+a_0u=0,\end{split}
\end{equation*}
Skilgreinum margliðu
\begin{equation*}
\begin{split}Q(r)=a_m r(r-1)\cdots(r-m+1)+\cdots+a_1r+a_0.\end{split}
\end{equation*}
Almenn lausn afleiðujöfnunnar á jákvæða raunásnum er línuleg samantekt fallanna
\begin{equation*}
\begin{split}\begin{gathered}
x^{r_1}, \big(\ln x \big) x^{r_1}, \dots,
\big(\ln x\big)^{m_1-1}x^{r_1},\\
x^{r_2}, \big(\ln x\big)x^{r_2}, \dots,
\big(\ln x \big)^{m_2-1} x^{r_2},\\
\vdots \qquad \qquad \qquad \vdots \qquad \qquad \qquad \vdots\\
x^{r_\ell}, \big(\ln x \big)x^{r_\ell}, \dots,
\big(\ln x\big)^{m_\ell-1} x^{r_\ell},\end{gathered}\end{split}
\end{equation*}
þar sem \(r_1,\dots,r_\ell\) eru ólíkar núllstöðvar margliðunnar \(Q\) með margfeldni \(m_1,\dots,m_\ell\).


\subsection{Athugasemd}
\label{\detokenize{Kafli07:id4}}
Gerum nú ráð fyrir að \(a_0, \ldots, a_m\) séu rauntölur. Hugsum okkur að \(\lambda=\alpha+i\beta\) sé \(m\)-föld rót margliðunni \(Q(r)\). Þá er \(\mu=\overline{\lambda}=\alpha-i\beta\) líka \(m\)-föld rót \(Q(r)\). Þegar grunnur fyrir lausnarúmið er skrifaður má í stað
\begin{equation*}
\begin{split}x^{\lambda}, (\ln x)x^{\lambda},\dots, (ln x)^{m-1}x^{\lambda},
x^{\mu}, (\ln x)x^{\mu},\dots, (\ln x)^{m-1}x^{\mu}\end{split}
\end{equation*}
hafa í grunninum raungildu föllin
\begin{equation*}
\begin{split}x^{\alpha}\cos(\ln(\beta x)), (\ln x)x^{\alpha}\cos(\ln(\beta x)), \ldots,
(\ln x)^{m-1}x^{\alpha}\cos(\ln(\beta x)),\end{split}
\end{equation*}\begin{equation*}
\begin{split}x^{\alpha}\sin(\ln(\beta x)),
(\ln x)x^{\alpha t}\sin(\ln(\beta x)), \ldots, (\ln x)^{m-1}x^{\alpha}\sin(\ln(\beta x)).\end{split}
\end{equation*}

\section{Sérlausnir og Green-föll}
\label{\detokenize{Kafli07:serlausnir-og-green-foll}}

\subsection{Upprifjun}
\label{\detokenize{Kafli07:upprifjun}}
Viljum leysa afleiðujöfnu af taginu
\begin{equation*}
\begin{split}a_m(t)u^{(m)}+a_{m-1}(t)u^{(m-1)}+\cdots+a_1(t)u'+a_0(t)u=f(t).\end{split}
\end{equation*}
Fyrst er að finna grunn fyrir lausnarúm óhliðruðu jöfnunnar
\begin{equation*}
\begin{split}a_m(t)u^{(m)}+a_{m-1}(t)u^{(m-1)}+\cdots+a_1(t)u'+a_0(t)u=0.\end{split}
\end{equation*}
Svo finnum við einhverja eina lausn (,,sérlausn‘‘)
\begin{equation*}
\begin{split}a_m(t)u^{(m)}+a_{m-1}(t)u^{(m-1)}+\cdots+a_1(t)u'+a_0(t)u=f(t).\end{split}
\end{equation*}
Þá getum við lýst öllum lausnum afleiðujöfnunnar.

Það að finna sérlausnina getur verið erfitt.


\subsection{Ágiskun (Sjá \S{}7.4)}
\label{\detokenize{Kafli07:agiskun-sja-7-4}}
Giskum á sérlausn sem er fall af ,,sömu gerð‘‘ og fallið \(f(t)\), nema hvað ekki settar inn ákveðnar tölur fyrir stuðla sem koma fyrir. Stungið inn í jöfnu og reynt að ákvarða stuðla.


\subsection{Góðar ágiskanir}
\label{\detokenize{Kafli07:goar-agiskanir}}
Höfum línulega afleiðujöfnu með fastastuðlum
\begin{equation*}
\begin{split}a_mu^{(m)}+a_{m-1}u^{(m-1)}+\cdots+a_1u'+a_0u=f(t).\end{split}
\end{equation*}
Látum \(P_n(t)\) standa fyrir einhverja \(n\)-ta stigs margliðu og látum \(A_n(t)\) og \(B_n(t)\) tákna \(n\)-ta stigs margliður með óákveðnum stuðlum.

Ef \(f(t)=P_n(t)\) þá giskað á \(u_{\rm p}(t)=t^lA_n(t)\).

Ef \(f(t)=P_n(t)e^{rt}\) þá giskað á \(u_{\rm p}(t)=t^lA_n(t)e^{rt}\).

Ef \(f(t)=P_n(t)e^{rt}\sin(kt)\) þá giskað á \(u_{\rm p}(t)=t^le^{rt}[A_n(t)\cos(kt)+B_n(t)\sin(kt)]\).

Ef \(f(t)=P_n(t)e^{rt}\cos(kt)\) þá giskað á \(u_{\rm p}(t)=t^le^{rt}[A_n(t)\cos(kt)+B_n(t)\sin(kt)]\).

Hér táknar \(l\) minnstu töluna af tölunum \(0, 1, \ldots, m-1\) sem tryggir að enginn liður í ágiskuninni sé lausn á óhliðruðu jöfnunni \(a_mu^{(m)}+a_{m-1}u^{(m-1)}+\cdots+a_1u'+a_0u=0\).


\subsection{Dæmi - Deyfð sveifla með drifkrafti}
\label{\detokenize{Kafli07:daemi-deyf-sveifla-me-drifkrafti}}
Lítum á diffurjöfnuna \(mx''+cx'+kx=A\cos(\omega t)\).


\begin{center}
\includegraphics[width=4cm,keepaspectratio=true]{polarggb.png}
\end{center}



\subsection{Sérlausnir fundnar með virkjareikningi (Sjá \S{}7.4)}
\label{\detokenize{Kafli07:serlausnir-fundnar-me-virkjareikningi-sja-7-4}}
Aðferðin snýst um að nýta sér ákveðin mynstur sem koma upp þegar línulegum afleiðuvirkja með fastastuðla er beitt á ákveðnar gerðir falla. Lykilformúlur eru:
\begin{equation*}
\begin{split}P(D)e^{\alpha t}=P(\alpha)e^{\alpha t}.\end{split}
\end{equation*}\begin{equation*}
\begin{split}(D-\alpha)v(t)e^{\alpha t}=v'(t)e^{\alpha t}\quad\mbox{og
  almennar}\quad (D-\alpha)^kv(t)  e^{\alpha t}=v^{(k)}(t)e^{\alpha
  t},\end{split}
\end{equation*}\begin{equation*}
\begin{split}\mbox{ef } P(D)=Q(D)(D-\alpha)^k\mbox{ þá }
P(D)\frac{t^ke^{\alpha t}}{k!Q(\alpha)}=e^{\alpha t}.\end{split}
\end{equation*}

\subsection{Hjálparsetning (Sjá Hjálparsetningu 7.5.1)}
\label{\detokenize{Kafli07:hjalparsetning-sja-hjalparsetningu-7-5-1}}
Ef \(I\) er bil á raunásnum, \(a\in I\), \(f\in C(I)\) og \(g\in C(I\times I)\), er samfellt deildanlegt fall af fyrri breytistærðinni, þ.e. \({\partial}_tg\in C(I\times I)\), þá er fallið \(h\), sem gefið er með formúlunni
\begin{equation*}
\begin{split}h(t)=\int_a^t g(t, \tau)f(\tau) \, d\tau, \qquad t\in I,\end{split}
\end{equation*}
í \(C^1(I)\) og afleiða þess er
\begin{equation*}
\begin{split}h'(t)=g(t,t)f(t)+\int_a^t \partial_tg(t,\tau)f(\tau) \, d\tau,
\qquad t\in I.\end{split}
\end{equation*}

\subsection{Skilgreining og umræða (Sjá \S{}7.5)}
\label{\detokenize{Kafli07:skilgreining-og-umraea-sja-7-5}}
Skoðum afleiðujöfnu
\begin{equation*}
\begin{split}P(t,D)u=\big(a_m(t)D^m+\cdots+a_1(t)D+a_0(t)\big)u=f(t)\end{split}
\end{equation*}
þar sem föllin \(a_0(t),\dots,a_m(t),f(t)\) eru í \(C(I)\) og \(a_m(t)\neq 0\) fyrir öll \(t\in I\).

Samkvæmt \sphinxhref{./Kafli06.html\#fylgisetning-sja-fylgisetningu-6-6-7}{Fylgisetningu 6.3.6.}  gildir fyrir sérhvern punkt \(\tau\in I\) að til er ótvírætt ákvörðuð lausn

\(u_\tau\) á upphafsgildisverkefninu \(P(t,D)u=0\) þannig að
\begin{equation*}
\begin{split}u_\tau(\tau)=u_\tau'(\tau)=\cdots=u_\tau^{(m-2)}(\tau)=0\qquad\mbox{og}\qquad
u_\tau^{(m-1)}(\tau)=1/a_m(\tau).\end{split}
\end{equation*}
Skilgreinum Green-fall virkjans \(P(t, D)\) sem fallið
\(G(t,\tau)\) þannig að fyrir öll \(t,\tau\in I\) er
\(G(t,\tau)=u_\tau(t)\).


\subsection{Setning (Sjá \S{}7.5)}
\label{\detokenize{Kafli07:setning-sja-7-5}}
Um Green-fall línulegs afleiðuvirkja
\begin{equation*}
\begin{split}P(t,D)=a_m(t)D^m+\cdots+a_1(t)D+a_0(t)\end{split}
\end{equation*}
þar sem föllin \(a_0(t),\dots,a_m(t),f(t)\) eru í \(C(I)\) og \(a_m(t)\neq 0\) fyrir öll \(t\in I\) gildir:
\begin{equation*}
\begin{split}P(t,D_t)G(t,\tau)=0,  \qquad \mbox{fyrir öll }t,\tau\in I,\label{2.5.2}\end{split}
\end{equation*}\begin{equation*}
\begin{split}\begin{gathered}
G(\tau,\tau)=\partial_tG(\tau,\tau)=\cdots=
\partial_t^{(m-2)}G(\tau,\tau)=0,\\
\partial_t^{(m-1)}G(\tau,\tau)=1/a_m({\tau})\label{2.5.3}.
\end{gathered}\end{split}
\end{equation*}
Green-fallið ákvarðast ótvírætt af þessum skilyrðum.

Fallið \(G(t,\tau)\) er \(m\)-sinnum samfellt deildanlegt fall af \(t\) fyrir sérhvert \(\tau\in I\) og \(\partial_t^jG\in C(I\times I)\) fyrir \(j=0,\dots,m\).


\subsection{Setning (Sjá Setningu 7.5.2)}
\label{\detokenize{Kafli07:setning-sja-setningu-7-5-2}}
Látum \(P(t,D)\) vera línulegan afleiðuvirkja á forminu
\begin{equation*}
\begin{split}P(t,D)u=(a_m(t)D^m+\cdots+a_1(t)D+a_0(t))u\end{split}
\end{equation*}
þar sem föllin \(a_0(t),\dots,a_m(t),f(t)\) eru í \(C(I)\) og \(a_m(t)\neq 0\) fyrir öll \(t\in I\).

Ef \(a\) er einhver punktur í \(I\) þá hefur upphafsgildisverkefnið
\begin{equation*}
\begin{split}P(t,D)u=f(t),\end{split}
\end{equation*}
með
\begin{equation*}
\begin{split}u(a)=u'(a)=\cdots=u^{(m-1)}(a)=0,\end{split}
\end{equation*}
ótvírætt ákvarðaða lausn \(u_p\in C^m(I)\) sem gefin er með formúlunni
\begin{equation*}
\begin{split}u_p(t) = \int_a^t G(t,\tau)f(\tau) \, d\tau, \qquad t\in I,\end{split}
\end{equation*}
og \(G(t,\tau)\) er Green-fall virkjans \(P(t, D)\).


\subsection{Fylgisetning (Sjá Fylgisetningu 7.5.4)}
\label{\detokenize{Kafli07:fylgisetning-sja-fylgisetningu-7-5-4}}
Gerum ráð fyrir að \(P(D)=a_mD^m+\cdots+a_1D+a_0\) sé línulegur afleiðuvirki með fastastuðla. Látum \(g\in C^{\infty}(\mathbb{R})\) vera fallið sem uppfyllir
\begin{equation*}
\begin{split}P(D)g=0,\  \text{með }
g(0)=g'(0)=\cdots=g^{(m-2)}(0)=0,\mbox{ og }
g^{(m-1)}(0)=1/a_m.\end{split}
\end{equation*}
Þá er \(G(t,\tau)=g(t-\tau)\) Green-fall virkjans \(P(D)\).


\section{Green-föll og Wronski-ákveður}
\label{\detokenize{Kafli07:green-foll-og-wronski-akveur}}

\subsection{Reikniaðferð}
\label{\detokenize{Kafli07:reikniafer}}
Finna skal Green-fall \(G(t,\tau)\) fyrir \(m\)-ta stigs línulegan afleiðuvirkja \(P(t,D)\).


\bigskip\hrule\bigskip

\begin{enumerate}
\sphinxsetlistlabels{\Alph}{enumi}{enumii}{(}{)}%
\item {} 
Stuðlar afleiðuvirkjans eru fastar, þ.e.a.s. \(P(D)=a_mD^m+\cdots+a_1D+a_0\).

\end{enumerate}

Fyrst er fundinn grunnur \(u_1(t), \ldots, u_m(t)\) fyrir lausnarúm jöfnunnar \(P(D)u=0\). Almenna lausnin er á á forminu
\begin{equation*}
\begin{split}u=c_1u_1+\cdots+c_mu_m,\end{split}
\end{equation*}
þar sem \(c_1, \ldots, c_m\) eru fastar.

Næst er fundin ein ákveðin lausn \(g(t)\) á jöfnunni \(P(D)u=0\) sem uppfyllir skilyrðin \(g(0)=\cdots=g^{(m-2)}(0)=0\) og \(g^{(m-1)}(0)=1/a_m\).

Green-fallið er svo gefið með formúlunni \(G(t,\tau)=g(t-\tau)\).


\bigskip\hrule\bigskip

\begin{enumerate}
\sphinxsetlistlabels{\Alph}{enumi}{enumii}{(}{)}%
\setcounter{enumi}{1}
\item {} 
Stuðlar í \(P(t,D)\) eru föll \(a_0(t), \ldots, a_m(t)\) skilgreind á bili \(I\) þannig að

\end{enumerate}
\begin{equation*}
\begin{split}P(t,D)=a_m(t)D^m+\cdots+a_1(t)D+a_0(t).\end{split}
\end{equation*}
Fyrst er fundinn grunnur \(u_1(t), \ldots, u_m(t)\) fyrir lausnarúm \(P(t,D)u=0\). Almenna lausnin er á forminu
\begin{equation*}
\begin{split}u=c_1u_1+\cdots+c_mu_m,\end{split}
\end{equation*}
þar sem \(c_1, \ldots, c_m\) eru fastar.

Svo finnum við fyrir almennan punkt \(\tau\in I\) gildi á fastana \(c_1(\tau), \ldots, c_m(\tau)\) þannig að um lausnina \(u_\tau(t)= c_1(\tau)u_1(t)+\cdots+c_m(\tau)u_m(t)\) á \(P(t,D)u=0\) gildi að
\begin{equation*}
\begin{split}u_\tau(\tau)=u_\tau'(\tau)=\cdots=u_\tau^{(m-2)}(\tau)=0\qquad\mbox{og}\qquad
u_\tau^{(m-1)}(\tau)=1/a_m(\tau).\end{split}
\end{equation*}
Green-fallið er þá gefið með formúlunni \(G(t,\tau)=u_\tau(t)\).


\subsection{Skilgreining (Sjá \S{}7.6)}
\label{\detokenize{Kafli07:skilgreining-sja-7-6}}
Látum \(u_1, \ldots, u_m\) vera vera \((m-1)\)-sinni deildanleg föll skilgreind á bili I. Wronski-fylki fallanna \(u_1, u_2, \ldots, u_m\) er skilgreint sem fylkið
\begin{equation*}
\begin{split}V(u_1, u_2, \ldots, u_m)=\begin{bmatrix}
u_1(t)&u_2(t)&\cdots&u_m(t)\\
u_1'(t)&u_2'(t)&\cdots&u_m'(t)\\
\vdots&\vdots&\dots&\vdots\\
u_1^{(m-1)}(t)&u_2^{(m-1)}(t)&\cdots&u_m^{(m-1)}(t)
\end{bmatrix}.\end{split}
\end{equation*}
Ákveða þessa fylkis er kölluð Wronski-ákveða fallanna \(u_1, u_2, \ldots, u_m\).

\begin{sphinxadmonition}{attention}{Athugið:}
Stuðlarnir í Wronski-fylkinu eru föll af breytunni \(t\) og sömuleiðis er Wronski-ákveðan fall af breytunni \(t\).
\end{sphinxadmonition}


\subsection{Setning (Sjá Setningu 7.6.3)}
\label{\detokenize{Kafli07:setning-sja-setningu-7-6-3}}
Látum \(P(t,D)=a_m(t)D^m+\cdots+a_1(t)D+a_0(t)\) vera afleiðuvirkja með samfellda stuðla, \(u_1,\dots,u_m\) vera lausnir á óhliðruðu jöfnunni \(P(t,D)u=0\) og táknum Wronski-ákveðu þeirra með \(W(t)\). Þá uppfyllir \(W\) fyrsta stigs afleiðujöfnuna
\begin{equation*}
\begin{split}a_m(t) W'+a_{m-1}(t)W=0\end{split}
\end{equation*}
og þar með gildir formúlan
\begin{equation*}
\begin{split}W(t)=W(a)\exp\bigg(-\int_a^t\dfrac{a_{m-1}(\tau)}{a_m(\tau)}\,
d\tau\bigg)\end{split}
\end{equation*}
fyrir öll \(a\) og \(t\) á bili \(J\) þar sem \(a_m\) er núllstöðvalaust.


\subsection{Setning (Sjá Setningu 7.6.3)}
\label{\detokenize{Kafli07:id5}}
Látum \(u_1,\dots,u_m\) vera lausnir á óhliðruðu jöfnunni \(P(t,D)u=0\), þar sem
\begin{equation*}
\begin{split}P(t,D)=a_m(t)D^m \cdots+a_1(t)D+a_0(t)\end{split}
\end{equation*}
og föllin \(a_0(t), \ldots, a_m(t)\) eru skilgreind og samfelld á bili \(I\) og við gerum ráð fyrir að \(a_m\) sé núllstöðvalaust á opnu bili \(J\subseteq I\). Þá eru eftirfarandi skilyrði jafngild:
\begin{enumerate}
\sphinxsetlistlabels{\roman}{enumi}{enumii}{(}{)}%
\item {} 
Föllin \(u_1,\dots,u_m\) eru línulega óháð á bilinu \(J\).

\item {} 
\(W(u_1,\dots,u_m)(t)\neq 0\) fyrir sérhvert \(t\in J\).

\item {} 
\(W(u_1,\dots,u_m)(a)\neq 0\) fyrir eitthvert \(a\in J\).

\end{enumerate}


\subsection{Setning (Sjá Setningu 7.6.4)}
\label{\detokenize{Kafli07:setning-sja-setningu-7-6-4}}
Látum \(P(t,D)=a_m(t)D^m+\cdots+a_1(t)D+a_0(t)\) vera afleiðuvirkja með stuðla sem eru samfelld föll skilgreind á bili \(I\) og \(u_1,\dots,u_m\) vera grunn í \({\cal N}(P(t,D))\). Green-fall virkjans er gefið með formúlunni
\begin{equation*}
\begin{split}G(t,\tau)=c_1(\tau)u_1(t)+\cdots+c_m(\tau)u_m(t), \qquad t,\tau\in I,\end{split}
\end{equation*}
þar sem vigurinn \(a_m({\tau})(c_1(\tau),\dots,c_m(\tau))\) myndar aftasta dálkinn í andhverfu Wronski-fylkisins \(V(u_1,\dots,u_m)(\tau)\),
\begin{equation*}
\begin{split}c_j(\tau)=(-1)^{m+j} \dfrac{\det V_{mj}(u_1,\dots,u_m)(\tau)}
{a_m({\tau})W(u_1,\dots, u_m)(\tau)},\end{split}
\end{equation*}
þar sem \(V_{mj}(u_1,\dots,u_m)(\tau)\) táknar \((m-1)\times (m-1)\) fylkið sem fæst með því að fella niður neðstu línuna og dálk númer \(j\) í \(V(u_1,\dots,u_m)(\tau)\).


\subsection{Fylgisetning (Er hluti af Setningu 7.6.4)}
\label{\detokenize{Kafli07:fylgisetning-er-hluti-af-setningu-7-6-4}}
Sérlausn á afleiðujöfnunni \(P(t,D)u=f(t)\) er gefin með formúlunni
\begin{equation*}
\begin{split}u_p(t)=v_1(t)u_1(t)+\cdots+v_m(t)u_m(t), \qquad  t\in I,\end{split}
\end{equation*}
þar sem stuðlaföllin \(v_j\) eru gefin með formúlunni
\begin{equation*}
\begin{split}v_j(t)=\int_a^t c_j(\tau)f(\tau) \, d\tau.\end{split}
\end{equation*}

\subsection{Fylgisetning (Sjá Fylgisetning 7.6.5)}
\label{\detokenize{Kafli07:fylgisetning-sja-fylgisetning-7-6-5}}
Látum \(P(t,D)=a_2(t)D^2+a_1(t)D+a_0(t)\) vera annars stigs afleiðuvirkja á bilinu \(I\) með samfellda stuðla og \(a_2(t)\neq 0\) fyrir öll \(t\in I\). Gerum nú ráð fyrir að \(u_1\) og \(u_2\) séu línulega óháðar lausnir á óhliðruðu jöfnunni \(P(t,D)u=0\). Þá er
\begin{equation*}
\begin{split}G(t,\tau)
=a_2(\tau)^{-1}
\left|\begin{matrix}
u_1(\tau) & u_1(t)\\
u_2(\tau) & u_2(t)
\end{matrix}\right|\bigg /
\left|\begin{matrix}
u_1(\tau) & u_2({\tau})\\
u_1'(\tau) & u_2'({\tau})
\end{matrix}\right|.\end{split}
\end{equation*}

\chapter{Veldaraðalausnir á afleiðujöfnum}
\label{\detokenize{Kafli08:veldaraalausnir-a-afleiujofnum}}\label{\detokenize{Kafli08::doc}}
Rose: \sphinxstyleemphasis{If you are an alien, how come you sound like you’re from the north?}

Doctor: \sphinxstyleemphasis{Lots of planets have a north!}

- Doctor Who


\section{Veldaraðalausnir}
\label{\detokenize{Kafli08:veldaraalausnir}}

\subsection{Skilgreining (Sjá Skilgreiningu 8.1.1)}
\label{\detokenize{Kafli08:skilgreining-sja-skilgreiningu-8-1-1}}
Fall \(f:X\to {\mathbb{C}}\) skilgreint á opnu mengi \(X\) á raunásnum, er sagt vera raunfágað á \(X\) ef fyrir sérhvern punkt \(a\in X\) er til tala \(\rho>0\) þannig að bilið \(]a-\rho, a+\rho[\subseteq X\) og til er veldaröð \(\sum_{n=0}^\infty c_n(x-a)^n\) þannig að fyrir öll \(x\in ]a-\rho, a+\rho[\) er \(f(x)=\sum_{n=0}^\infty c_n(x-a)^n\).


\subsection{Umræða (Sjá \S{}8.1)}
\label{\detokenize{Kafli08:umraea-sja-8-1}}
Mörg hugtök og skilgreiningar fyrir raunfáguð föll eru eins og fyrir fáguð föll af tvinntölubreytu. Látum nú \(f(x)\) vera raunfágað fall skilgreint á opnu mengi \(X\) í \(\mathbb{R}\).

Gerum ráð fyrir að \(f(a)=0\) og \(f(x)=\sum_{n=0}^\infty c_n(x-a)^n\). Ef \(n\) er minnsta tala þannig að \(c_n\neq 0\) þá segjum við að \(f(x)\) hafi núllstöð af stigi \(n\) í \(a\).

Segjum að punktur \(a\) sé einangraður sérstöðupunktur ef til er tala \(\delta>0\) þannig að fallið \(f(x)\) er skilgreint á öllu bilinu \(]a-\delta, a+\delta[\) nema í punktinum \(a\).

Einangraður sérstöðupunktur \(a\) er sagður afmáanlegur ef hægt er að gefa \(f(a)\) gildi þannig að útvíkkaða fallið verði raunfágað á \(X\cup\{a\}\).


\subsection{Skilgreining (Sjá Skilgreiningu 8.2.6)}
\label{\detokenize{Kafli08:skilgreining-sja-skilgreiningu-8-2-6}}
Látum \(a_0(x), a_1(x), a_2(x)\) vera föll sem eru raunfáguð á bili \(I\). Segjum að punktur \(a\in I\) sé venjulegur punktur fyrir afleiðujöfnuna
\begin{equation*}
\begin{split}a_2(x)u''+a_1(x)u'+a_0(x)u=0,\end{split}
\end{equation*}
ef \(a_2(a)\neq 0\) eða ef \(a_2(a)=0\) þá hafi föllin \(P(x)=a_1(x)/a_2(x)\) og \(Q(x)=a_0(x)/a_2(x)\) afmáanlegan sérstöðupunkt í \(a\).


\subsection{Setning (Samanber Setning 8.2.8)}
\label{\detokenize{Kafli08:setning-samanber-setning-8-2-8}}
Gerum ráð fyrir að \(a\) sé venjulegur punktur afleiðujöfnunnar
\begin{equation*}
\begin{split}a_2(x)u''+a_1(x)u'+a_0(x)u=0.\end{split}
\end{equation*}
Þá er sérhver lausn \(u\) á afleiðujöfnunni raunfáguð á bili umhverfis \(a\).


\subsection{Reikniaðferð (Sjá \S{}8.2)}
\label{\detokenize{Kafli08:reikniafer-sja-8-2}}
Gerum ráð fyrir að \(a\) sé venjulegur punktur afleiðujöfnunnar
\begin{equation*}
\begin{split}a_2(x)u''+a_1(x)u'+a_0(x)u=0.\end{split}
\end{equation*}

\bigskip\hrule\bigskip


Skref 0: Setjum \(P(x)=a_1(x)/a_2(x)\) og \(Q(x)=a_0(x)/a_2(x)\) og ritum afleiðujöfnuna sem
\begin{equation*}
\begin{split}u''+P(x)u'+Q(x)u=0.\end{split}
\end{equation*}

\bigskip\hrule\bigskip


Skref 1: Finnum veldaraðir með miðju í \(a\) fyrir föllin \(P(x)\) og \(Q(x)\):
\begin{equation*}
\begin{split}P(x)=\sum_{n=0}^\infty P_n(x-a)^n\qquad\mbox{ og }\qquad
Q(x)=\sum_{n=0}^\infty Q_n(x-a)^n.\end{split}
\end{equation*}

\bigskip\hrule\bigskip


Skref 2: Setjum inn í afleiðujöfnuna
\begin{equation*}
\begin{split}u(x)=\sum_{n=0}^\infty c_n(x-a)^n,\quad
u'(x)=\sum_{n=0}^\infty (n+1)c_{n+1}(x-a)^n,\quad
u''(x)=\sum_{n=0}^\infty (n+2)(n+1)c_{n+2}(x-a)^n.\end{split}
\end{equation*}

\bigskip\hrule\bigskip


Skref 3: Tökum saman í eina veldaröð og fáum að
\begin{equation*}
\begin{split}\sum_{n=0}^\infty
\bigg((n+2)(n+1)c_{n+2} +
\sum_{k=0}^{n} \big((k+1)P_{n-k}c_{k+1}+
Q_{n-k} c_k\big)\bigg)(x-a)^n=0.\end{split}
\end{equation*}

\bigskip\hrule\bigskip


Skref 4: Allir stuðlar í þessari síðustu veldaröð eru 0. Stuðlana \(c_0\) og \(c_1\) má velja frjálst og svo fæst rakningarformúla fyrir \(c_n\) þannig að
\begin{equation*}
\begin{split}c_{n+2} = \dfrac{-1}{(n+2)(n+1)}
\sum_{k=0}^n \big[(k+1)P_{n-k}c_{k+1}+ Q_{n-k}c_k\big].\end{split}
\end{equation*}

\bigskip\hrule\bigskip


Skref 5: Þegar rakningarformúlan er fengin þá er oft hægt að átta sig á hvaða fall er lausn eða reikna má fyrstu stuðlana í veldaröðinni og fá þannig Taylor-margliðu fallsins \(u\) sem má nota til að reikna nálgunargildi. Athugið einnig að \(c_0=u(a)\) og \(c_1=u'(a)\) þannig að oft ákvarðast því \(c_0\) og \(c_1\) af upphafsgildum.


\subsection{Skilgreining (Sjá \S{}8.3)}
\label{\detokenize{Kafli08:skilgreining-sja-8-3}}
\(\Gamma\)-fallið er skilgreint með formúlunni
\begin{equation*}
\begin{split}\Gamma(z)=\int_0^\infty e^{-t}t^{z-1}\, dt, \quad z\in {\mathbb{C}}, \quad \operatorname{Re\, }
z>0.\end{split}
\end{equation*}

\subsection{Nokkrar formúlur (Sjá \S{}8.3)}
\label{\detokenize{Kafli08:nokkrar-formulur-sja-8-3}}\begin{equation*}
\begin{split}\begin{aligned}
\Gamma(z+1)&=z\Gamma(z)\\
\Gamma(z+n)&=z(z+1)\cdots(z+n-1)\Gamma(z)\\
\Gamma(1)&=1\\
\Gamma(n)&=(n-1)!\\
\Gamma(1/2)&=\sqrt{\pi}\\
\Gamma(-1/2)&=2\sqrt{\pi}\\
\Gamma(n+1/2)&=\frac{(2n-1)!}{2^{2n-1}(n-1)!}\sqrt{\pi}.\end{aligned}\end{split}
\end{equation*}

\section{Aðferð Frobeniusar}
\label{\detokenize{Kafli08:afer-frobeniusar}}

\subsection{Umræða}
\label{\detokenize{Kafli08:umraea}}
Afleiðujafnan
\begin{equation*}
\begin{split}x^2u''+xu'+(x^2-\alpha^2)u=0\end{split}
\end{equation*}
kallast jafna Bessel. Besseljafnan og lausnir hennar, sem kallaðar eru Bessel-föll, koma upp í rafsegulfræði, varmafræði, skammtafræði, …

Punkturinn \(a=0\) er ekki venjulegur punktur. Aðeins í undantekningartilfellum fæst lausn með aðferðinni úr síðasta fyrirlestri við að prófa veldaraðarlausn með miðju í \(a=0\) og í engu tilfelli fæst grunnur fyrir lausnarúmið. Samt er hægt að nota aðferð sem er áþekk því sem lýst var í síðasta fyrirlestri.


\subsection{Skilgreining (Sjá Skilgreiningu 8.4.1)}
\label{\detokenize{Kafli08:skilgreining-sja-skilgreiningu-8-4-1}}
Látum \(f\) vera raunfágað fall á opnu mengi \(X\) í \(\mathbb{R}\). Við segjum að einangraður sérstöðupunktur \(a\) raunfágaða fallsins \(f\) sé skaut af stigi \(m>0\), ef til er tala \(\varrho>0\) og raunfágað fall \(g\) skilgreint á bilinu \(\{x\mid |x-a|<\varrho\}\), þannig að \(\{x\mid 0<|x-a|<{\varrho}\}\subseteq X\), \(g(a)\neq 0\) og
\begin{equation*}
\begin{split}f(x)=\dfrac {g(x)}{(x-a)^m}\qquad \mbox{ef }0<|x-a|<\varrho.\end{split}
\end{equation*}

\subsection{Skilgreining (Sjá Skilgreiningu 8.4.2)}
\label{\detokenize{Kafli08:skilgreining-sja-skilgreiningu-8-4-2}}
Við segjum að \(a\) sé reglulegur sérstöðupunktur afleiðujöfnunnar
\begin{equation*}
\begin{split}a_2(x)u''+a_1(x)u'+a_0(x)u=0\end{split}
\end{equation*}
ef \(a\) er sérstöðupunktur jöfnunnar, fallið \(P=a_1(x)/a_2(x)\) hefur annað hvort afmáanlegan sérstöðupunkt í \(a\) eða skaut af stigi \(\leq 1\) og \(Q=a_0(x)/a_2(x)\) hefur annað hvort afmáanlegan sérstöðupunkt í \(a\) eða skaut af stigi \(\leq 2\).


\subsection{Skilgreining (Sjá Skilgreiningu 8.4.4)}
\label{\detokenize{Kafli08:skilgreining-sja-skilgreiningu-8-4-4}}
Gerum ráð fyrir að \(a\) sé reglulegur sérstöðupunktur afleiðujöfnu sem rituð er á forminu
\begin{equation*}
\begin{split}(x-a)^2u''+(x-a)p(x)u'+q(x)u=0.\label{3.4.7}\end{split}
\end{equation*}
Þá kallast margliðan
\begin{equation*}
\begin{split}\varphi(\lambda)=\lambda(\lambda-1)+p(a)\lambda+q(a)\end{split}
\end{equation*}
vísamargliða afleiðujöfnunnar í punktinum \(a\), og jafnan \(\varphi(\lambda)=0\) kallast vísajafna afleiðujöfnunnar í punktinum \(a\). Núllstöðvarnar kallast vísar jöfnunnar í punkti \(a\).


\subsection{Setning Frobeniusar (Sjá Setningu 8.4.5)}
\label{\detokenize{Kafli08:setning-frobeniusar-sja-setningu-8-4-5}}
Gerum ráð fyrir því að \(a\) sé reglulegur sérstöðupunktur afleiðujöfnunnar
\begin{equation*}
\begin{split}(x-a)^2u''+ (x-a)p(x)u'+q(x)u=0\end{split}
\end{equation*}
og gerum ráð fyrir að föllin \(p\) og \(q\) séu sett fram með veldaröðunum
\begin{equation*}
\begin{split}p(x)=\sum_{n=0}^\infty p_n(x-a)^n, \qquad\quad
q(x)=\sum_{n=0}^\infty q_n(x-a)^n,\end{split}
\end{equation*}
og að þær séu samleitnar ef \(|x-a|<\varrho\). Látum \(r_1\) og \(r_2\) vera núllstöðvar vísajöfnunnar
\begin{equation*}
\begin{split}\varphi(\lambda)=\lambda(\lambda-1)+p(a)\lambda+q(a)=0\end{split}
\end{equation*}
og gerum ráð fyrir að \(\operatorname{Re\, } r_1\geq \operatorname{Re\, } r_2\). Þá gildir:
\begin{enumerate}
\sphinxsetlistlabels{\roman}{enumi}{enumii}{(}{)}%
\item {} 
Til er lausn \(u_1\) á jöfnunni sem gefin er með

\end{enumerate}
\begin{equation*}
\begin{split}u_1(x)=|x-a|^{r_1}\sum_{n=0}^\infty a_n(x-a)^n.\end{split}
\end{equation*}
Röðin er samleitin fyrir öll \(x\) sem uppfylla \(0<|x-a|<\varrho\). Valið á \(a_0\) er frjálst, en hinir stuðlar raðarinnar fást með rakningarformúlunni
\begin{equation*}
\begin{split}a_n=\dfrac{-1}{\varphi(n+r_1)}
\sum_{k=0}^{n-1}((k+r_1)p_{n-k}+q_{n-k})a_k, \qquad n=1,2,3,\dots.\end{split}
\end{equation*}\begin{enumerate}
\sphinxsetlistlabels{\roman}{enumi}{enumii}{(}{)}%
\setcounter{enumi}{1}
\item {} 
Ef \(r_1-r_2\neq 0,1,2,\dots\), þá er til önnur línulega óháð lausn \(u_2\) á jöfnunni sem gefin er með

\end{enumerate}
\begin{equation*}
\begin{split}u_2(x)=|x-a|^{r_2}\sum_{n=0}^\infty b_n(x-a)^n.\end{split}
\end{equation*}
Röðin er samleitin fyrir öll \(x\) sem uppfylla \(0<|x-a|<\varrho\). Valið á \(b_0\) er frjálst, en hinir stuðlar raðarinnar fást með rakningarformúlunni
\begin{equation*}
\begin{split}b_n=\dfrac{-1}{\varphi(n+r_2)}
\sum_{k=0}^{n-1}((k+r_2)p_{n-k}+q_{n-k})b_k, \qquad n=1,2,3,\dots.\end{split}
\end{equation*}\begin{enumerate}
\sphinxsetlistlabels{\roman}{enumi}{enumii}{(}{)}%
\setcounter{enumi}{2}
\item {} 
Ef \(r_1-r_2=0\), þá er til önnur línulega óháð lausn \(u_2\) á jöfnunni sem gefin er með

\end{enumerate}
\begin{equation*}
\begin{split}u_2(x)=|x-a|^{r_1+1}\sum_{n=0}^\infty b_n(x-a)^n+
u_1(x)\ln|x-a|.\end{split}
\end{equation*}
Röðin er samleitin fyrir öll \(x\) sem uppfylla \(0<|x-a|<\varrho\) og stuðlar raðarinnar fást með innsetningu í jöfnuna.
\begin{enumerate}
\sphinxsetlistlabels{\roman}{enumi}{enumii}{(}{)}%
\setcounter{enumi}{3}
\item {} 
Ef \(r_1-r_2=N\), þar sem \(N\) er jákvæð heiltala, þá er til önnur línulega óháð lausn \(u_2\) á upphaflegu jöfnunni sem gefin er með

\end{enumerate}
\begin{equation*}
\begin{split}u_2(x)=|x-a|^{r_2}\sum_{n=0}^\infty b_n(x-a)^n+
\gamma u_1(x)\ln|x-a|.\end{split}
\end{equation*}
Röðin er samleitin fyrir öll \(x\) sem uppfylla \(0<|x-a|<\varrho\). Stuðlar raðarinnar og \(\gamma\) fást með innsetningu í jöfnuna.


\subsection{Skilgreining (Sjá Skilgreiningu 8.5.1)}
\label{\detokenize{Kafli08:skilgreining-sja-skilgreiningu-8-5-1}}
Lausn á Bessel-jöfnunni \(x^2u''+xu'+(x^2-\alpha^2)u=0\), sem gefin er með formúlunni
\begin{equation*}
\begin{split}J_\alpha(x)=\left|\dfrac x2\right|^\alpha\sum_{k=0}^\infty
\dfrac{(-1)^k}{k!\Gamma(\alpha+k+1)}\left( \dfrac x2\right)^{2k}\end{split}
\end{equation*}
er kölluð Bessel-fall af fyrstu gerð með vísi \(\alpha\).


\subsection{Skilgreining (Sjá Skilgreiningu 8.5.2)}
\label{\detokenize{Kafli08:skilgreining-sja-skilgreiningu-8-5-2}}
Fallið \(Y_{\alpha}\), \({\alpha}=1,2,3,\dots\) sem skilgreint er með
\begin{equation*}
\begin{split}\begin{aligned}
Y_{\alpha}(x)=\dfrac 2{\pi}\bigg[&
J_{\alpha}(x)\bigg(\ln \dfrac {|x|}2+{\gamma}\bigg)\\
&+x^{\alpha}\sum\limits_{k=0}^{\infty}
\dfrac{(-1)^{k-1}\big(h_k+h_{k+\alpha}\big)}
{2^{2k+\alpha+1}k!(k+{\alpha})!} x^{2k}\\
&-x^{-\alpha}\sum\limits_{k=0}^{\alpha-1}
\dfrac{(\alpha-k-1)!}{2^{2k-\alpha+1}k!}x^{2k}\bigg],\end{aligned}\end{split}
\end{equation*}
þar sem \(h_k=1+1/2+1/3+\cdots+1/k\) og \({\gamma}\) táknar fasta Eulers, nefnist Bessel-fall af annarri gerð með vísi \({\alpha}\).


\chapter{Línuleg afleiðujöfnuhneppi}
\label{\detokenize{Kafli09:linuleg-afleiujofnuhneppi}}\label{\detokenize{Kafli09::doc}}
Leela: \sphinxstyleemphasis{„To be, or not to be, that is the question.“ That is a very stupid question!}

The Doctor: \sphinxstyleemphasis{It’s Shakespeare.}

Leela: \sphinxstyleemphasis{And that is a very stupid name. You do not shake a spear, you throw it! Throwspeare, now that is a name.}


\section{Línuleg afleiðujöfnuhneppi}
\label{\detokenize{Kafli09:id1}}

\subsection{Skilgreining (Sjá \S{}9.1)}
\label{\detokenize{Kafli09:skilgreining-sja-9-1}}
Afleiðujöfnuhneppi af gerðinni
\begin{equation*}
\begin{split}\begin{aligned}
u_1'&=a_{11}(t)u_1+\cdots+a_{1m}(t)u_m+f_1(t),\\
u_2'&=a_{21}(t)u_1+\cdots+a_{2m}(t)u_m+f_2(t),\\
\vdots&\qquad \qquad \vdots\qquad \qquad \qquad \qquad \vdots\\
u_m'&=a_{m1}(t)u_1+\cdots+a_{mm}(t)u_m+f_m(t).\end{aligned}\end{split}
\end{equation*}
er kallað línulegt fyrsta stig afleiðujöfnuhneppi. Á fylkjaformi má rita þetta sem
\begin{equation*}
\begin{split}\begin{bmatrix}u_1'\\u_2'\\\vdots\\u_m'\end{bmatrix}
=\begin{bmatrix}a_{11}(t)&a_{12}(t)&\cdots&a_{1m}(t)\\
a_{11}(t)&a_{12}(t)&\cdots&a_{1m}(t)\\
\vdots&\vdots&\ddots&\vdots\\
a_{m1}(t)&a_{m2}(t)&\cdots&a_{mm}(t)\end{bmatrix}
\begin{bmatrix}u_1 \\u_2 \\\vdots\\u_m\end{bmatrix}
+\begin{bmatrix}f_1(t) \\f_2(t) \\\vdots\\f_m(t)\end{bmatrix},\end{split}
\end{equation*}
og ef við notum \(u(t)\) til að tákna vigurinn \((u_1(t), u_2(t), \ldots, u_m(t))\), og svo \(A(t)\) til að tákna fylkið og \(f(t)\) til að tákna vigurinn \((f_1(t), f_2(t), \ldots, f_m(t))\) þá má rita jöfnuna hér að ofan sem \(u'=A(t)u+f(t)\).

Hér er gert ráð fyrir að föllin sem koma fyrir í fylkinu \(A(t)\) og sem hnit í \(f(t)\) séu öll skilgreind á einhverju opnu bili \(I\) í \(\mathbb{R}\) og að þau séu öll samfelld.

Í framhaldinu er gert ráð fyrir að \(u, A, f\) séu á því formi sem lýst er hér að ofan.


\subsection{Skilgreining (Sjá \S{}9.1)}
\label{\detokenize{Kafli09:id2}}
Afleiðujafna af taginu \(u'=A(t)u+f(t)\) er sögð óhliðruð ef \(f(t)\) er núllfallið
(útkoman er alltaf vigurinn sem hefur 0 í öllum hnitum), en hliðruð annars. Talað er um jöfnuhneppi með fastastuðlum ef stuðlarnir í fylkinu \(A\) eru allir fastar.


\subsection{Setning (Sjá \S{}9.1)}
\label{\detokenize{Kafli09:setning-sja-9-1}}
Upphafsgildisverkefnið
\begin{equation*}
\begin{split}u'=A(t)u+f(t), \qquad u(a)=b,\label{5.1.2}\end{split}
\end{equation*}
hefur ótvírætt ákvarðaða lausn, þar sem \(a\) er einhver gefinn punktur í \(I\) og \(b\) er einhver gefinn vigur í \({\mathbb{C}}^m\). Sjá \sphinxhref{./Kafli06.html\#fylgisetning-sja-fylgisetningu-6-6-6}{Fylgisetningu 6.3.5}


\subsection{Setning (Sjá \S{}9.1)}
\label{\detokenize{Kafli09:id3}}
Látum \(I\) vera opið bil á rauntalnaásnum. Rifjum upp að \(C(I, {\mathbb{C}}^m)\) er mengi allra samfelldra falla skilgreindra á \(I\) með gildi í \({\mathbb{C}}^m\) og \(C^1(I, {\mathbb{C}}^m)\) er mengi allra falla skilgreindra á \(I\) með gildi í \({\mathbb{C}}^m\) sem hafa samfellda fyrstu afleiðu. Bæði \(C(I, {\mathbb{C}}^m)\) og \(C^1(I, {\mathbb{C}}^m)\) eru vigurrúm.

Vörpunin \(L:C^1(I, {\mathbb{C}}^m)\to C(I, {\mathbb{C}}^m)\) þannig að \(Lu=u'-A(t)u\) er línuleg.


\subsection{Fylgisetning (Sjá Setningu 9.1.3)}
\label{\detokenize{Kafli09:fylgisetning-sja-setningu-9-1-3}}\begin{enumerate}
\sphinxsetlistlabels{\roman}{enumi}{enumii}{(}{)}%
\item {} 
Lausnamengi óhliðraðar jöfnu \(u'=A(t)u\) er hlutrúm í \(C^1(I, {\mathbb{C}}^m)\) af vídd \(m\). Lausnamengið, eða núllrúm A, er táknað með \({\cal N}(A)\).

\item {} 
Sérhver lausn á \(u'=A(t)u+f(t)\) er af gerðinni

\end{enumerate}
\begin{equation*}
\begin{split}u(t)=c_1u_1(t)+\cdots+c_mu_m(t)+u_p(t),\end{split}
\end{equation*}
þar sem \(u_1,\dots,u_m\) er einhver grunnur \({\cal N}(A)\), \(c_1,\dots,c_m\in{\mathbb{C}}\) og \(u_p\) er einhver lausn á hliðruðu jöfnunni.


\subsection{Setning (Sjá Hjálparsetningu 9.3.1)}
\label{\detokenize{Kafli09:setning-sja-hjalparsetningu-9-3-1}}
Látum \(u_1,\dots,u_m\) vera föll í \({\cal N}(A)\). Þá eru eftirfarandi skilyrði jafngild:
\begin{enumerate}
\sphinxsetlistlabels{\roman}{enumi}{enumii}{(}{)}%
\item {} 
Vigurföllin \(u_1,\dots,u_m\) eru línulega óháð á bilinu \(I\).

\item {} 
Vigrarnir \(u_1(t),\dots,u_m(t)\) eru línulega óháðir í \(\mathbb{R}^ m\) (eða \({\mathbb{C}}^ m\)) fyrir sérhvert \(t\in I\).

\item {} 
Vigrarnir \(u_1(a),\dots,u_m(a)\) eru línulega óháðir í \(\mathbb{R}^ m\) (eða \({\mathbb{C}}^ m\)) fyrir eitthvert \(a\in I\).

\end{enumerate}


\subsection{Setning (Sjá \S{}9.1)}
\label{\detokenize{Kafli09:id4}}
Línuleg afleiðujafna af taginu
\begin{equation*}
\begin{split}P(t,D)v= v^{(m)}+a_{m-1}(t)v^{(m-1)}+\cdots+a_1(t)v'
+a_0(t)=g(t)\end{split}
\end{equation*}
er jafngild afleiðujöfnuhneppinu
\begin{equation*}
\begin{split}u_1'=u_2,\quad u_2'=u_3,\quad  \ldots,\quad u_{m-1}'=u_m\end{split}
\end{equation*}\begin{equation*}
\begin{split}u_m' =-a_0(t)u_1-a_1(t)u_2-\cdots-a_{m-1}(t)u_m+g(t).\end{split}
\end{equation*}
Þegar jöfnuhneppið ritað á fylkjaformi fæst
\begin{equation*}
\begin{split}\begin{bmatrix}u_1'\\u_2'\\\vdots\\u_{m-1}'\\u_m'\end{bmatrix}
=\begin{bmatrix}
0&1&0&\dots&0\\
0&0&1&\dots&0\\
\vdots&\vdots&\vdots&\ddots&\vdots\\
0&0&0&\dots&1\\
-a_0(t)&-a_1(t)&-a_2(t)&\dots&-a_{m-1}(t)
\end{bmatrix}\begin{bmatrix}u_1 \\u_2 \\\vdots\\u_{m-1}\\u_m\end{bmatrix}
+\begin{bmatrix}0 \\0 \\\vdots\\0\\g(t)\end{bmatrix}.\end{split}
\end{equation*}
Ef við ritum \(P(t,D)=D^ m+a_{m-1}(t)D^{m-1}+\cdots+a_1(t)D+a_0(t)\) og fylkið \(A(t)\) er skilgreint eins og hér að ofan þá er
\begin{equation*}
\begin{split}\det(\lambda I-A(t))=P(t,\lambda).\end{split}
\end{equation*}

\subsection{Setning (Sjá Hjálparsetningu 9.2.1)}
\label{\detokenize{Kafli09:setning-sja-hjalparsetningu-9-2-1}}
Látum \(A\) vera \(m\times m\) fylki og \(\varepsilon\) vera eiginvigur þess með tilliti til eigingildisins \(\lambda\). Þá uppfyllir vigurfallið \(u(t)=e^{\lambda t}\varepsilon\) jöfnuna \(u'=Au\).


\subsection{Setning (Sjá Setningu 9.2.2)}
\label{\detokenize{Kafli09:setning-sja-setningu-9-2-2}}
Látum \(A\) vera \(m\times m\) fylki og gerum ráð fyrir að \(\varepsilon_1,\dots,\varepsilon_\ell\) séu eiginvigrar þess með tilliti til eigingildanna \(\lambda_1,\dots,\lambda_\ell\). Ef \(a \in I\), \(b\in {\mathbb{C}}^m\) og unnt er að skrifa \(b=\beta_1\varepsilon_1+\cdots+\beta_\ell\varepsilon_\ell\) og \(f(t)=g_1(t)\varepsilon_1+\cdots+g_\ell(t)\varepsilon_\ell\), þá er lausnin á upphafsgildisverkefninu
\begin{equation*}
\begin{split}u'=Au+f(t), \qquad \qquad u(a)=b,\end{split}
\end{equation*}
gefin með \(u(t)=v_1(t)\varepsilon_1+\cdots+v_\ell(t)\varepsilon_\ell\), þar sem stuðullinn \(v_j\) uppfyllir
\begin{equation*}
\begin{split}v_j'(t)=\lambda_jv_j(t)+g_j(t), \qquad v_j(a)=\beta_j,\end{split}
\end{equation*}
og er þar með
\begin{equation*}
\begin{split}v_j(t)=\beta_je^{\lambda_j(t-a)}+e^{\lambda_jt}\int_a^t e^{-\lambda_j
\tau}g_j(\tau) \, d\tau.\end{split}
\end{equation*}

\subsection{Skilgreining (Sjá \S{}9.2)}
\label{\detokenize{Kafli09:skilgreining-sja-9-2}}
Fyrir tölur \(t_1, t_2, \ldots, t_m\) er \({\operatorname{diag}}(t_1, t_2, \ldots, t_m)\) skilgreint sem \(m\times m\) hornalínufylkið sem hefur tölurnar \(t_1, t_2, \ldots, t_m\) á hornalínunni.


\subsection{Setning (Sjá \S{}9.2.2)}
\label{\detokenize{Kafli09:setning-sja-9-2-2}}
Látum \(A\) vera \(m\times m\) fylki. Gerum ráð fyrir að \(T\) sé \(m\times m\) fylki þannig að \(T^{-1}AT=\Lambda\) þar sem \(\Lambda\) er hornalínufylki með stökin \(\lambda_1, \lambda_2, \ldots, \lambda_m\) á hornalínunni. (Athugið að \(A=T\Lambda T^{-1}\).)

Látum \(I\) vera bil á \(\mathbb{R}\), \(a\in I\), \(f\in C(I,{\mathbb{C}}^m)\) og \(b\in {\mathbb{C}}^m\). Þá hefur upphafsgildisverkefnið
\begin{equation*}
\begin{split}u'=Au+f(t), \qquad u(a)=b\end{split}
\end{equation*}
ótvírætt ákvarðaða lausn á \(I\), sem gefin er með formúlunni
\begin{equation*}
\begin{split}\begin{aligned}
u(t)&=T{\operatorname{diag}}(e^{\lambda_1(t-a)},\dots,e^{\lambda_m(t-a)})T^{-1}b\\
&+\int_a^t T{\operatorname{diag}}(e^{\lambda_1(t-\tau)},\dots,e^{\lambda_m(t-\tau)})
T^{-1}f(\tau)\, d\tau.\end{aligned}\end{split}
\end{equation*}

\section{Veldisvísisfylkið}
\label{\detokenize{Kafli09:veldisvisisfylki}}

\subsection{Skilgreining (Sjá Skilgreining 9.3.2)}
\label{\detokenize{Kafli09:skilgreining-sja-skilgreining-9-3-2}}
Fylki af gerðinni
\begin{equation*}
\begin{split}\Phi(t)=[u_1(t),\dots,u_m(t)], \qquad t\in I,\end{split}
\end{equation*}
þar sem dálkavigrarnir \(u_1,\dots,u_m\) mynda grunn í núllrúminu \({\cal N}(A)\) fyrir afleiðujöfnuhneppið \(u'=A(t)u\), kallast grunnfylki fyrir afleiðujöfnuhneppið.


\subsection{Setning (Sjá Setningu 9.3.3)}
\label{\detokenize{Kafli09:setning-sja-setningu-9-3-3}}
Lát \(\Phi\) og \(\Psi\) vera tvö grunnfylki fyrir jöfnuhneppið \(u'=A(t)u\). Þá er til andhverfanlegt fylki \(B\) þannig að
\begin{equation*}
\begin{split}\Psi(t)=\Phi(t)B.\label{5.3.2}\end{split}
\end{equation*}

\subsection{Setning (Sjá Setningu 9.3.4)}
\label{\detokenize{Kafli09:setning-sja-setningu-9-3-4}}
Lát \(\Phi(t)\) vera grunnfylki fyrir jöfnuhneppið \(u' =A(t)u\).
\begin{enumerate}
\sphinxsetlistlabels{\roman}{enumi}{enumii}{(}{)}%
\item {} 
Sérhvert stak í \({\cal N}(A)\) er af gerðinni \(u(t)=\Phi(t)c\), þar sem \(c\) er vigur í \({\mathbb{C}}^ m\).

\item {} 
Vigurfallið \(u_p\), sem gefið er með formúlunni

\end{enumerate}
\begin{equation*}
\begin{split}u_p(t)=\Phi(t)\int_a^ t \Phi(\tau)^{-1}f(\tau)\, d\tau,\end{split}
\end{equation*}
uppfyllir \(u'=A(t)u+f(t)\) og \(u(a)=0\).
\begin{enumerate}
\sphinxsetlistlabels{\roman}{enumi}{enumii}{(}{)}%
\setcounter{enumi}{2}
\item {} 
Lausnin á upphafsgildisverkefninu \(u'=A(t)u+f(t)\), \(u(a)=b\) er gefin með formúlunni

\end{enumerate}
\begin{equation*}
\begin{split}u(t)=\Phi(t)\Phi(a)^{-1}b+
\Phi(t)\int_a^ t \Phi(\tau)^{-1}f(\tau)\, d\tau.\end{split}
\end{equation*}

\subsection{Skilgreining (Sjá \S{}9.4)}
\label{\detokenize{Kafli09:skilgreining-sja-9-4}}
Runa \(\{C_n\}_{n=0}^\infty\), af \(\ell\times m\) fylkjum \(C_n=\big(c_{jkn}\big)_{j=1,k=1}^{\ell, m}\) er sögð vera samleitin með markgildi \(C=\big(c_{jk}\big)_{j=1,k=1}^{\ell, m}\) ef fyrir öll gildi á \(j, k\) gildir að
\begin{equation*}
\begin{split}\lim\limits_{n\to\infty}c_{jkn}=c_{jk}.\end{split}
\end{equation*}
Óendanleg summa \(\sum_{n=0}^\infty C_n\) af \(\ell\times m\) fylkjum er sögð vera samleitin, ef runan af hlutsummum \(\{\sum_{n=0}^N C_n\}_{N=0}^\infty\) er samleitin.


\subsection{Skilgreining (Sjá \S{}9.4)}
\label{\detokenize{Kafli09:id5}}
Fyrir \(m\times m\)-fylki \(A\) skilgreinum við
\begin{equation*}
\begin{split}e^A=\sum_{n=0}^\infty \frac{1}{n!}A^n=I+A+\frac{1}{2}A^2+\frac{1}{3!}A^3+\cdots.\end{split}
\end{equation*}
\begin{sphinxadmonition}{attention}{Athugið:}
Með tiltölulega lítilli fyrirhöfn (gert í hefti Ragnars) má sýna að röðin hér að ofan er samleitin fyrir öll \(m\times m\) fylki \(A\). Einnig má skilgreina á sama hátt \(\sin A, \cos A, \ldots\).
\end{sphinxadmonition}


\subsection{Setning (Sjá \S{}9.5)}
\label{\detokenize{Kafli09:setning-sja-9-5}}\begin{enumerate}
\sphinxsetlistlabels{\roman}{enumi}{enumii}{(}{)}%
\item {} 
Fyrir rauntölu \(t\) er

\end{enumerate}
\begin{equation*}
\begin{split}\frac{d}{dt}e^{tA}=Ae^{tA}.\end{split}
\end{equation*}\begin{enumerate}
\sphinxsetlistlabels{\roman}{enumi}{enumii}{(}{)}%
\setcounter{enumi}{1}
\item {} 
(Sjá Setningu 9.5.1) Fylkjafallið \(\Phi(t)= e^{tA}\) er hin ótvírætt ákvarðaða lausn upphafsgildisverkefnisins

\end{enumerate}
\begin{equation*}
\begin{split}\Phi'(t) = A\Phi(t), \qquad t\in \mathbb{R}, \qquad \Phi(0)=I.\end{split}
\end{equation*}

\subsection{Fylgisetning}
\label{\detokenize{Kafli09:fylgisetning}}
Fylkið \(e^{tA}\) er grunnfylki fyrir afleiðujöfnuhneppið \(u'=Au\).


\subsection{Setning (Sjá Setningu 9.5.2)}
\label{\detokenize{Kafli09:setning-sja-setningu-9-5-2}}\begin{enumerate}
\sphinxsetlistlabels{\roman}{enumi}{enumii}{(}{)}%
\item {} 
Ef \(A\) og \(B\) eru \(m\times m\) fylki og \(AB=BA\), þá er

\end{enumerate}
\begin{equation*}
\begin{split}e^{A+B}=e^ Ae^ B=e^Be^A.\label{5.5.1}\end{split}
\end{equation*}\begin{enumerate}
\sphinxsetlistlabels{\roman}{enumi}{enumii}{(}{)}%
\setcounter{enumi}{1}
\item {} 
Fylkið \(e^ {tA}\) hefur andhverfuna \(e^{-tA}\).

\end{enumerate}


\subsection{Setning}
\label{\detokenize{Kafli09:setning}}
Látum \(A\) vera \(m\times m\) fylki. Gerum ráð fyrir að að \(\varepsilon_1, \dots, \varepsilon_m\) séu eiginvigrar tilheyrandi eigingildum \(\lambda_1, \dots \lambda_m\) og að þessir vigrar myndi grunn. Látum \(T\) vera fylkið sem hefur vigrana \(\varepsilon_1, \dots, \varepsilon_m\) sem dálkvigra í þessari röð. Þá er
\begin{equation*}
\begin{split}e^{tA}=T{\operatorname{diag}}(e^{\lambda_1t}, \ldots, e^{\lambda_mt})T^{-1}.\end{split}
\end{equation*}

\section{Útreikningur lausna}
\label{\detokenize{Kafli09:utreikningur-lausna}}

\subsection{Verkefni (Sjá \S{}9.6)}
\label{\detokenize{Kafli09:verkefni-sja-9-6}}
Fyrir gefið \(m\times m\) fylki \(A\) skal reikna \(e^{tA}\).


\subsection{Setning Cayley-Hamilton (Sjá \S{}9.6)}
\label{\detokenize{Kafli09:setning-cayley-hamilton-sja-9-6}}
Látum \(A\) vera \(m\times m\) fylki. Kennimargliða \(A\) er margliðan \(p(\lambda)=p_A(\lambda)=\det(\lambda I-A)\). Þá er \(p_A(A)=0\).


\subsection{Afleiðing Setningar Cayley-Hamilton}
\label{\detokenize{Kafli09:afleiing-setningar-cayley-hamilton}}
Hægt er að finna föll \(f_0(t), f_1(t), \ldots, f_{m-1}(t)\) þannig að
\begin{equation*}
\begin{split}e^{tA}= f_0(t)I+f_1(t)A+\cdots+f_{m-1}(t)A^{m-1}.\end{split}
\end{equation*}

\subsection{Brúunarverkefni (Sjá \S{}9.7)}
\label{\detokenize{Kafli09:bruunarverkefni-sja-9-7}}
Látum \(f\in {\cal O}({\mathbb{C}})\) vera gefið fall, látum \(\alpha_1,\dots,\alpha_\ell\) vera ólíka punkta í \({\mathbb{C}}\), látum \(m_1,\dots,m_\ell\) vera jákvæðar heiltölur og setjum \(m=m_1+\cdots+m_\ell\). Viljum finna margliðu \(r\) af stigi \(<m\), sem uppfyllir
\begin{equation*}
\begin{split}f^{(j)}(\alpha_k) = r^{(j)}(\alpha_k), \qquad
 j=0,\dots,m_k-1, \quad
k=1,\dots, \ell.\end{split}
\end{equation*}
Þetta er alltaf hægt. Margliðan \(r\) er ótvírætt ákvörðuð.


\subsection{Skilgreining (Sjá \S{}9.7)}
\label{\detokenize{Kafli09:skilgreining-sja-9-7}}
Við skilgreinum rununa \(\lambda_1,\dots,\lambda_m\) með því að telja \(\alpha_1,\dots,\alpha_\ell\) með margfeldni, þannig að fyrstu \(m_1\) gildin á \(\lambda_j\) séu \(\alpha_1\), næstu \(m_2\) gildin á \(\lambda_j\) séu \(\alpha_2\) o.s.frv. Svo er
\begin{equation*}
\begin{split}p(z)=(z-\alpha_1)^{m_1}\cdots(z-\alpha_\ell)^{m_\ell}
=(z-\lambda_1)\cdots(z-\lambda_m).\end{split}
\end{equation*}

\subsection{Skilgreining (Sjá \S{}9.7)}
\label{\detokenize{Kafli09:id6}}
Látum \(\lambda_1,\dots,\lambda_m\) vera talnarunu eins og hér að ofan.

Mismunakvótar eru skilgreindir með formúlum
\begin{equation*}
\begin{split}f[\lambda_i,\dots,\lambda_{i+j}]=\dfrac{f^{(j)}(\lambda_i)}{j!},\end{split}
\end{equation*}
ef \(\lambda_i=\cdots=\lambda_{i+j}\), og
\begin{equation*}
\begin{split}f[\lambda_i,\dots,\lambda_{i+j}]=
\dfrac{f[\lambda_i,\dots,\lambda_{i+j-1}]-f[\lambda_{i+1},\dots,\lambda_{i+j}]}
{\lambda_i-\lambda_{i+j}},\end{split}
\end{equation*}
ef \(\lambda_i\neq \lambda_{i+j}\), fyrir \(i=1,\dots,m\) og \(j=0,\dots,m-i\) .


\subsection{Setning (Sjá \S{}9.7)}
\label{\detokenize{Kafli09:setning-sja-9-7}}
Látum \(f\in {\cal O}({\mathbb{C}})\), \(\alpha_1,\dots,\alpha_\ell\) vera ólíka punkta í \({\mathbb{C}}\), \(m_1,\dots,m_\ell\) vera jákvæðar heiltölur, setjum \(m=m_1+\cdots+m_\ell\) og skilgreinum \(p(z)\) eins og hér að ofan. Þá er til margliða \(r\) af stigi \(<m\) og \(g\in {\cal O}({\mathbb{C}})\) þannig að
\begin{equation*}
\begin{split}f(z)=r(z)+p(z)g(z), \qquad z\in {\mathbb{C}}.\end{split}
\end{equation*}
Margliðan \(r\) er lausn á brúunarverkefninu. Bæði \(r\) og \(g\) eru ótvírætt ákvörðuð og
\begin{equation*}
\begin{split}\begin{aligned}
r(z)=f[\lambda_1]&+f[\lambda_1,\lambda_2](z-\lambda_1)+\cdots\\
&+ f[\lambda_1,\dots,\lambda_m](z-\lambda_1)\cdots(z-\lambda_{m-1})\end{aligned}\end{split}
\end{equation*}
og
\begin{equation*}
\begin{split}g(z)=f[\lambda_1,\dots,\lambda_m,z](z-\lambda_1)\cdots(z-\lambda_m).\end{split}
\end{equation*}

\subsection{Reikniaðferð}
\label{\detokenize{Kafli09:reikniafer}}
Þegar reikna þarf mismunakvóta þá er gott að fylgja sama skema og hér á eftir:
\begin{equation*}
\begin{split}\begin{matrix}
f[\lambda_1]\\
            &f[\lambda_1,\lambda_2]\\
f[\lambda_2]&                       &f[\lambda_1, \lambda_2, \lambda_3]\\
        &f[\lambda_2,\lambda_3]& &f[\lambda_1,\lambda_2,\lambda_3,\lambda_4]\\
f[\lambda_3]&                       &f[\lambda_2, \lambda_3, \lambda_4]\\
            &f[\lambda_3,\lambda_4]\\
f[\lambda_4]
\end{matrix}\end{split}
\end{equation*}
Þegar \(\lambda_1=1=\lambda_2\) og \(\lambda_3=-1=\lambda_4\) og
\(f(z)=e^{tz}\):
\begin{equation*}
\begin{split}\begin{matrix}
\lambda_1=1 & e^t  \\
 & & te^t& \\
\lambda_2=1 & e^t  &  & \tfrac 12(te^t-\sinh t)\\
 & & \sinh t & & \tfrac 12(t\cosh t-\sinh t) \\
\lambda_3=-1 & e^{-t}  & & \tfrac 12(\sinh t -te^{-t})\\
 & & te^{-t}& \\
\lambda_4=-1 & e^{-t}
\end{matrix}\end{split}
\end{equation*}

\subsection{Reikniaðferð (Sjá \S{}9.7)}
\label{\detokenize{Kafli09:reikniafer-sja-9-7}}
Reikna á \(e^{tA}\) fyrir \(m\times m\) fylki \(A\) og/eða lausn \(u'=Au\) með ákveðið upphafsgildi \(u(0)=b\).

Skref 1: Reiknið eigingildi \(A\) með margfeldni.

Skref 2: Setjið upp mismunatöflu líkt og sýnt er hér að ofan.

Skref 3: Setjið upp formúlu \(e^{tA}\) með því að nota
brúunarmargliðuna \(r(z)\).

Skref 3: Ef beðið er um \(e^{tA}\) þá reiknið þið upp úr formúlunni, en ef bara þarf að finna lausnina \(u\) þá þarf ekki að reikna upp úr formúlunni fyrir \(e^{tA}\) heldur er nóg að stilla upp formúlunni með fylkjum og svo margfalda í gegn með vigrinum þannig að maður margfaldar aldrei saman tvö fylki heldur er alltaf að margfalda fylki og vigur.


\chapter{Laplace-ummyndunin}
\label{\detokenize{Kafli10:laplace-ummyndunin}}\label{\detokenize{Kafli10::doc}}
\sphinxstyleemphasis{Think you’ve seen it all? Think again. Outside those doors, we might see anything. We could find new worlds, terrifying monsters, impossible things. And if you come with me… nothing will ever be the same again!}

- The Doctor, Doctor Who


\section{Laplace-ummyndunin}
\label{\detokenize{Kafli10:id1}}

\subsection{Skilgreining (Sjá \S{}10.1)}
\label{\detokenize{Kafli10:skilgreining-sja-10-1}}
Látum \(f\) vera fall sem skilgreint er á menginu \(\mathbb{R}_+=\{t\in \mathbb{R}; t\geq 0\}\) með gildi í \({\mathbb{C}}\) og gerum ráð fyrir að \(f\) sé heildanlegt á sérhverju lokuðu og takmörkuðu bili \([0,b]\). Laplace\textendash{}mynd \(f\), sem við táknum með \({\cal L} f\) eða \({\cal L}\{f\}\), er skilgreind með formúlunni
\begin{equation*}
\begin{split}{\cal L} f(s)=\int_0^ \infty e^{-st}f(t)\, dt.\label{7.1.1}\end{split}
\end{equation*}
Skilgreiningarmengi fallsins \({\cal L} f\) samanstendur af öllum tvinntölum \(s\) þannig að heildið í hægri hliðinni sé samleitið. Laplace-ummyndun er vörpunin \({\cal L}\) sem úthlutar falli \(f\) Laplace-mynd sinni \({\cal L} f\).

\begin{sphinxadmonition}{attention}{Athugið:}
Stundum er skilgreind ,,tvíhliða Laplace-mynd‘‘ falls þar sem heildið er reiknað frá \(-\infty\) til \(\infty\) og þá gæti maður kallað okkar Laplace-mynd ,,einhliða Laplace-mynd‘‘.
\end{sphinxadmonition}


\subsection{Skilgreining og setning  (Sjá Skilgreiningu 10.1.1)}
\label{\detokenize{Kafli10:skilgreining-og-setning-sja-skilgreiningu-10-1-1}}
Við segjum að fallið \(f:\mathbb{R}_+\to {\mathbb{C}}\) sé af veldisvísisgerð ef til eru jákvæðir fastar \(M\) og \(c\) þannig að
\begin{equation*}
\begin{split}|f(t)|\leq Me^{c t}, \qquad t\in \mathbb{R}_+.\label{7.1.2}\end{split}
\end{equation*}
Ef \(f\) er þar að auki heildanlegt á sérhverju takmörkuðu bili \([0,b]\), þá er \({\cal L} f\) skilgreint fyrir öll \(s\in {\mathbb{C}}\) með \(\operatorname{Re\, } s>c\). Við fáum að auki vaxtartakmarkanir á \({\cal L} f\),
\begin{equation*}
\begin{split}|{\cal L} f(s) |\leq
\int_0^\infty e^{-\operatorname{Re\, } st} Me^{c t} \, dt =
\dfrac M{\operatorname{Re\, }\,  s-c}, \qquad \operatorname{Re\, }\,  s>c.\end{split}
\end{equation*}

\subsection{Setning (Sjá \S{}10.1)}
\label{\detokenize{Kafli10:setning-sja-10-1}}
Laplace-ummyndunin er línuleg vörpun, en það þýðir að
\begin{equation*}
\begin{split}{\cal L}\{\alpha f+\beta g\}(s)=\alpha{\cal L}\{f\}(s)+\beta{\cal L}\{g\}(s)\end{split}
\end{equation*}
ef \(f\) og \(g\) eru föll af veldisvísisgerð, \(\alpha\) og \(\beta\) eru tvinntölur og \(s\in {\mathbb{C}}\) liggur í skilgreiningarmengi fallanna \({\cal L}\{f\}\) og \({\cal L}\{g\}\).


\subsection{Setning (Sjá Setningu 10.1.6)}
\label{\detokenize{Kafli10:setning-sja-setningu-10-1-6}}
Gerum ráð fyrir að föllin \(f,g\in C(\mathbb{R}_+)\) séu bæði af veldisvísisgerð og að til sé fasti \(c\) þannig að
\begin{equation*}
\begin{split}{\cal L} f(s)={\cal L} g(s), \qquad s\in {\mathbb{C}}, \quad \operatorname{Re\, } s\geq c.\end{split}
\end{equation*}
Þá er \(f(t)=g(t)\) fyrir öll \(t\in \mathbb{R}_+\).


\subsection{Setning (Sjá \S{}10.1)}
\label{\detokenize{Kafli10:id2}}\begin{enumerate}
\sphinxsetlistlabels{\roman}{enumi}{enumii}{(}{)}%
\item {} 
Ef \(\alpha\in \mathbb{R}\) og \(\alpha>-1\), þá er

\end{enumerate}
\begin{equation*}
\begin{split}{\cal L}\{t^\alpha\}(s)=\dfrac {\Gamma(\alpha+1)}{s^{\alpha+1}},\end{split}
\end{equation*}
og sér í lagi gildir ef \(\alpha\) er heiltala að
\begin{equation*}
\begin{split}{\cal L}\{t^\alpha\}(s)=\dfrac {\alpha!}{s^{\alpha+1}}.\end{split}
\end{equation*}\begin{enumerate}
\sphinxsetlistlabels{\roman}{enumi}{enumii}{(}{)}%
\setcounter{enumi}{1}
\item {} 
Fyrir sérhvert \(\alpha\in {\mathbb{C}}\) gildir

\end{enumerate}
\begin{equation*}
\begin{split}{\cal L}\{e^{\alpha t}\}(s)=\dfrac{1}{s-\alpha}.\end{split}
\end{equation*}

\subsection{Setning (Sjá Setningu 10.1.5)}
\label{\detokenize{Kafli10:setning-sja-setningu-10-1-5}}
\({\cal L}\{e^{\alpha t}f\}(s) = {\cal L}\{f\}(s-\alpha)\).


\subsection{Tafla yfir Laplace-myndir (Sjá \S{}10.1)}
\label{\detokenize{Kafli10:tafla-yfir-laplace-myndir-sja-10-1}}\begin{equation*}
\begin{split}\begin{aligned}
{\cal L}\{\cos\beta t\}(s) &=
\frac 12 {\cal L}\{e^{i\beta t}\}(s) +\frac 12{\cal L}\{e^{-i\beta t}\}(s)\\
&=\frac 12\left[\dfrac 1{s-i\beta}+\dfrac 1{s+i\beta}\right]
=\dfrac s{s^2+\beta^2},\\
{\cal L}\{\sin\beta t\}(s) &=
\frac 1{2i}{\cal L}\{e^{i\beta t}\}(s) -\frac 1{2i}{\cal L}\{e^{-i\beta t}\}(s)\\
&=\frac 1{2i}\left[\dfrac 1{s-i\beta}-\dfrac 1{s+i\beta}\right]
=\dfrac {\beta}{s^2+\beta^2},\\
{\cal L}\{\cosh \beta t\}(s) &=
\frac 12 {\cal L}\{e^{\beta t}\}(s) +\frac 12{\cal L}\{e^{-\beta t}\}(s)\\
&=\frac 12\left[\dfrac 1{s-\beta}+\dfrac 1{s+\beta}\right]
=\dfrac s{s^2-\beta^2},\\
{\cal L}\{\sinh \beta t\}(s) &=
\frac 1{2}{\cal L}\{e^{\beta t}\}(s) -\frac 1{2}{\cal L}\{e^{-i\beta t}\}(s)\\
&=\frac 1{2}\left[\dfrac 1{s-\beta}-\dfrac 1{s+\beta}\right]
=\dfrac \beta{s^2-\beta^2},\\
{\cal L}\{e^{\alpha t}t^{\beta}\}(s)
&=\dfrac{\Gamma(\beta+1)}{(s-\alpha)^{\beta+1}},\\
{\cal L}\{e^{\alpha t}\cos \beta t\}(s)
&=\dfrac{s-\alpha}{(s-\alpha)^2+\beta^2},\\
{\cal L}\{e^{\alpha t}\sin \beta t\}(s)
&=\dfrac{\beta}{(s-\alpha)^2+\beta^2},\\
{\cal L}\{e^{\alpha t}\cosh \beta t\}(s)
&=\dfrac{s-\alpha}{(s-\alpha)^2-\beta^2},\\
{\cal L}\{e^{\alpha t}\sinh \beta t\}(s)
&=\dfrac{\beta}{(s-\alpha)^2-\beta^2}.\end{aligned}\end{split}
\end{equation*}

\subsection{Setning (Sjá Setning 10.2.1)}
\label{\detokenize{Kafli10:setning-sja-setning-10-2-1}}
Ef \(u\in C^ m(\mathbb{R}_+)\) og \(u, u', u'', \dots, u^{(m-1)}\), eru af veldisvísisgerð, þá er \({\cal L}\{u^{(m)}\}(s)\) skilgreint fyrir öll \(s\in {\mathbb{C}}\) með \(\operatorname{Re\, } s\) nógu stórt og
\begin{equation*}
\begin{split}{\cal L}\{u^{(m)}\}(s)=s^
m{\cal L}\{u\}(s)-s^{m-1}u(0)-\cdots
-su^{(m-2)}(0)-u^{(m-1)}(0).\end{split}
\end{equation*}
Sér í lagi gildir að ef \(U(s)={\cal L}\{u(t)\}(s)\), þá er
\begin{equation*}
\begin{split}{\cal L}\{u'\}(s)  = sU(s)-u(0),\qquad\mbox{og}\qquad
{\cal L}\{u''\}(s) =s^2U(s)-su(0)-u'(0).\end{split}
\end{equation*}

\subsection{Reikniaðferð (Sjá \S{}10.2)}
\label{\detokenize{Kafli10:reikniafer-sja-10-2}}
Leysa á upphafsgildisverkefni af taginu:
\begin{equation*}
\begin{split}a_mu^{(m)}+\cdots +a_1u'+a_0u=f(t), \qquad u(0)=b_0,\ u'(0)=b_1,\ldots,u^{(m-1)}(0)=b_{m-1}.\end{split}
\end{equation*}
Skref 1: Reiknið Laplace-mynd hvorrar hliðar fyrir sig. Setning 10.1.7. gefur aðferð til að reikna Laplace-mynd vinstri hliðarinnar.

Skref 2: Notið jöfnuna sem kemur út úr Skrefi 1 til að fá formúlu fyrir \(U(s)={\cal L}\{u(t)\}(s)\).

Skref 3: Notið töflu eða andhverfa Laplace-ummyndun til að finna samfellt fall \(u(t)\) sem hefur \(U(s)\) sem fundið var í Skrefi 2 sem Laplace-mynd. Gætuð t.d. þurft að nota stofnbrotaliðun.

Skref 4: Fallið \(u(t)\) sem fannst í Skrefi 3 er lausn upphafsgildisverkefnisins.


\section{Notkun Laplace-ummyndunar}
\label{\detokenize{Kafli10:notkun-laplace-ummyndunar}}

\subsection{Skilgreining (Sjá \S{}10.2)}
\label{\detokenize{Kafli10:skilgreining-sja-10-2}}
Fyrir vigurgilt fall \(u(t)=\big(u_1(t), \ldots, u_m(t)\big)\) má skilgreina Laplace-mynd þannig að tekin er Laplace-mynd í hverju hniti fyrir sig,
\begin{equation*}
\begin{split}{\cal L}\{u(t)\}(s)=\big({\cal L}\{u_1(t)\}(s), \ldots,{\cal L}\{u_m(t)\}(s)\big).\end{split}
\end{equation*}
Laplace-mynd fylkjagilds falls er reiknuð á sama hátt.


\subsection{Setning (Sjá Setningu 10.2.4)}
\label{\detokenize{Kafli10:setning-sja-setningu-10-2-4}}
Um sérhvert \(m\times m\) fylki \(A\) gildir
\begin{equation*}
\begin{split}{\cal L}\{e^{tA}\}(s) = (sI-A)^{-1}.\end{split}
\end{equation*}

\subsection{Upprifjun (Sjá \S{}10.3 og \S{}7.5)}
\label{\detokenize{Kafli10:upprifjun-sja-10-3-og-7-5}}
Gerum ráð fyrir að \(P(D)=a_mD^ m+\cdots+a_1D+a_0\) sé línulegur afleiðuvirki með fastastuðla. Látum \(g\in C^{\infty}(\mathbb{R})\) vera fallið sem uppfyllir
\begin{equation*}
\begin{split}P(D)g=0,\qquad \mbox{með }
g(0)=g'(0)=\cdots=g^{(m-2)}(0)=0,\mbox{ og }
g^{(m-1)}(0)=1/a_m.\end{split}
\end{equation*}
Þá er \(G(t,\tau)=g(t-\tau)\) Green-fall virkjans \(P(D)\).

Ef \(a\) er einhver punktur þá hefur upphafsgildisverkefnið
\begin{equation*}
\begin{split}P(D)u=f(t),\end{split}
\end{equation*}
með
\begin{equation*}
\begin{split}u(a)=u'(a)=\cdots=u^{(m-1)}(a)=0,\end{split}
\end{equation*}
ótvírætt ákvarðaða lausn \(u_p\in C^m(I)\) sem gefin er með
formúlunni
\begin{equation*}
\begin{split}u_p(t) = \int_a^ t G(t,\tau)f(\tau) \, d\tau, \qquad t\in I,\end{split}
\end{equation*}
og \(G(t,\tau)\) er Green-fall virkjans \(P(D)\).


\subsection{Setning (Sjá \S{}10.3)}
\label{\detokenize{Kafli10:setning-sja-10-3}}
Með sama táknmál og hér að ofan gildir að
\begin{equation*}
\begin{split}{\cal L}\{g\}(s)=\frac{1}{P(s)}.\end{split}
\end{equation*}
Einnig gildir
\begin{equation*}
\begin{split}{\cal L}\{u_p\}(s)={\cal L}\left\{\int_0^tg(t-\tau)f(\tau)\, d\tau\right\}(s)=
{\cal L}\{g\}(s){\cal L}\{f\}(s).\end{split}
\end{equation*}

\subsection{Setning (Sjá Setning 10.3.1)}
\label{\detokenize{Kafli10:setning-sja-setning-10-3-1}}
Ef \(f\) og \(g\) eru föll af veldisvísisgerð og heildanleg á sérhverju bili \([0,b]\), þá er
\begin{equation*}
\begin{split}{\cal L}\left\{\int_0^tf(t-\tau)g(\tau)\, d\tau\right\}(s)=
{\cal L}\{f\}(s){\cal L}\{g\}(s).\end{split}
\end{equation*}

\subsection{Fylgisetning (Sjá Fylgisetningu 10.3.2)}
\label{\detokenize{Kafli10:fylgisetning-sja-fylgisetningu-10-3-2}}
Ef \(f\) er af veldisvísisgerð og heildanlegt á sérhverju bili \([0,b]\), þá er
\begin{equation*}
\begin{split}{\cal L}\left\{\int_0^t f(\tau) \, d\tau\right\}(s) = \dfrac 1s
{\cal L}\{f\}(s).\end{split}
\end{equation*}

\subsection{Setning (Sjá Setningu 10.2.1)}
\label{\detokenize{Kafli10:setning-sja-setningu-10-2-1}}
Látum \(f:\mathbb{R}_+\to {\mathbb{C}}\) vera fall af veldisvísigerð sem er heildanlegt á sérhverju bili \([0,b]\). Þá er \({\cal L}f\) fágað á menginu \(\{s\in {\mathbb{C}}\mid \operatorname{Re\, } s>c\}\) (þar sem \(c\) er fastinn úr 10.1.2.) og
\begin{equation*}
\begin{split}\frac{d^k}{ds^k}{\cal L}\{f\}(s)=(-1)^k{\cal L}\{t^kf(t)\}(s),\qquad \operatorname{Re\, } s>c.\end{split}
\end{equation*}

\subsection{Skilgreining og setning}
\label{\detokenize{Kafli10:skilgreining-og-setning}}
Heaviside-fallið \(H(x)\) er skilgreint þannig að \(H(t)=1\) ef \(t\geq 0\) og \(H(t)=0\) ef \(t<0\).

Fallið \(H_a(t)=H(t-a)\) er þannig að \(H_a(t)=1\) ef \(t\geq a\) og \(H(t)=0\) ef \(t<a\).

Fyrir \(a\geq 0\) er
\begin{equation*}
\begin{split}{\cal L}\{H_a\}(s)=\frac{e^{-as}}{s}.\end{split}
\end{equation*}

\subsection{Setning}
\label{\detokenize{Kafli10:setning}}
Látum \(f:\mathbb{R}_+\to {\mathbb{C}}\) vera fall af veldisvísisgerð. Þá gildir um sérhvert \(a\geq 0\) að
\begin{equation*}
\begin{split}{\cal L}\{H(t-a)f(t-a)\}(s) = e^{-as}{\cal L}\{f\}(s).\end{split}
\end{equation*}
þar sem fallið \(t\mapsto H(t-a)f(t-a)\) tekur gildið \(0\) fyrir öll \(t<a\).


\subsection{Skilgreining og setning}
\label{\detokenize{Kafli10:id3}}
Látum \(a\) vera gefna rauntölu. Skilgreinum fall \(f_\epsilon(t)\) þannig að \(f_\epsilon(t)=1/\epsilon\) ef \(a<t<a+\epsilon\) en \(f(t)=0\) fyrir öll önnur gildi á \(t\). Athugið að ef \(a>0\) þá er
\begin{equation*}
\begin{split}\int_0^\infty f_\epsilon(t)\,dt=1.\end{split}
\end{equation*}
Skilgreinum nú \(\delta_a\) sem „markgildið“ \(\lim_{\epsilon\to 0}f_\epsilon\). Sérstaklega skilgreinum við Dirac delta fallið sem \(\delta=\delta_0\). Athugið að \(\delta_a(t)=\delta(t-a)\).

\begin{sphinxadmonition}{attention}{Athugið:}
Athugið að \(\delta_a\) er ekki venjulegt fall heldur útvíkkuð gerð af falli sem er kölluð dreififall.
\end{sphinxadmonition}

Hægt er að reikna heildi af \(\delta_a\) og við fáum að
\begin{equation*}
\begin{split}\int_c^d \delta_a(t)\,dt=\left\{\begin{array}{ll}
1 & \mbox{ef }c\leq a\leq d,\\
0 & \mbox{annars.}
\end{array}\right.\end{split}
\end{equation*}
Auðvelt er að sjá að ef \(c\leq a\leq d\) þá er \(\int_c^d f(t)\delta_a(t)\,dt=f(a)\). Sérstaklega gildir að
\begin{equation*}
\begin{split}{\cal L}\{\delta_a\}(s)=e^{-as}.\end{split}
\end{equation*}

\subsection{Setning (Andhvefuformúla Fourier-Mellin)}
\label{\detokenize{Kafli10:setning-andhvefuformula-fourier-mellin}}
Ef \(f:\mathbb{R}_+\to {\mathbb{C}}\) er samfellt deildanlegt á köflum og uppfyllir \(|f(t)|\leq Me^{ct}\), \(t\in \mathbb{R}_+\), þar sem \(M\) og \(c\) eru jákvæðir fastar, þá gildir um sérhvert \(\xi>c\) og sérhvert \(t>0\) að
\begin{equation*}
\begin{split}\begin{aligned}
\tfrac 12(f(t+)+f(t-)) &= \lim_{R\to +\infty} \dfrac 1{2\pi}
\int_{-R}^{+R}e^{(\xi+i\eta)t}{\cal L} f(\xi+i\eta) \, d\eta\\
&= \lim_{R\to +\infty} \dfrac 1{2\pi i}
\int_{\xi-iR}^{\xi+iR}e^{\zeta t}{\cal L} f(\zeta) \, d\zeta,\nonumber\end{aligned}\end{split}
\end{equation*}
þar sem \(\int_{\xi-iR}^{\xi+iR}\) táknar að heildað sé eftir línustrikinu með upphafspunktinn \(\xi-iR\) og lokapunktinn \(\xi+iR\). Ef \({\cal L} f(\xi+i\eta)\) er í \(L^ 1(\mathbb{R})\) sem fall af \(\eta\), þá er \(f\) samfellt í \(t\) og
\begin{equation*}
\begin{split}\begin{aligned}
f(t)&=  \dfrac 1{2\pi}
\int_{-\infty}^{+\infty}e^{(\xi+i\eta)t}{\cal L} f(\xi+i\eta) \, d\eta\\
&= \dfrac 1{2\pi i}
\int_{\xi-i\infty}^{\xi+i\infty}e^{\zeta t}{\cal L} f(\zeta) \,
d\zeta,\nonumber\end{aligned}\end{split}
\end{equation*}
þar sem \(\int_{\xi-i\infty}^{\xi+i\infty}\) táknar að heildað sé eftir línunni \(\{\xi+i\eta; \eta\in \mathbb{R}\}\) í stefnu vaxandi \(\eta\).


\subsection{Setning}
\label{\detokenize{Kafli10:id4}}
Látum \(f:\mathbb{R}_+\to {\mathbb{C}}\) vera samfellt deildanlegt á köflum og af veldisvísisgerð, með \(|f(t)|\leq Me^{ct}\), \(t>0\), og gerum ráð fyrir að hægt sé að framlengja \({\cal L} f\) yfir í fágað fall á \({\mathbb{C}}\setminus A\), þar sem \(A\) er endanlegt mengi. Ef \(\xi>c\), \(M_r\) táknar hálfhringinn sem stikaður er með \(\gamma_r(\theta)=\xi+ire^{i\theta}\), \(\theta\in [0,\pi]\) og
\begin{equation*}
\begin{split}\max_{\zeta\in M_r}|{\cal L} f(\zeta)|\to 0, \qquad r\to +\infty,\end{split}
\end{equation*}
þá er
\begin{equation*}
\begin{split}\frac 12(f(t+)+f(t-))=
\sum_{\alpha\in A}\operatorname{Res}(e^{\zeta t}{\cal L} f(\zeta),\alpha).\end{split}
\end{equation*}
Ef \(f\) er samfellt í \(t\), þá gildir
\begin{equation*}
\begin{split}f(t)= \sum_{\alpha\in A}\operatorname{Res}(e^{\zeta t}{\cal L} f(\zeta),\alpha).\end{split}
\end{equation*}

\subsection{Fylgisetning}
\label{\detokenize{Kafli10:fylgisetning}}
Notum áfram táknmálið í 10.2.3. Táknum með \(A\) núllstöðvamengi margliðunnar \(P(\zeta)\). Þá er
\begin{equation*}
\begin{split}g(t)= \sum\limits_{\alpha\in A}
\operatorname{Res}\bigg( \dfrac {e^{t\zeta}}{P(\zeta)},\alpha\bigg).\end{split}
\end{equation*}

\bigskip\hrule\bigskip


\sphinxstyleemphasis{If it’s time to go, remember what you’re leaving. Remember the best.}

- The Doctor, Doctor Who


\chapter{Viðauki}
\label{\detokenize{vidauki:viauki}}\label{\detokenize{vidauki::doc}}
\sphinxstyleemphasis{Do what I do. Hold tight and pretend it’s a plan!}

- The Doctor, Doctor Who


\section{Kennsluáætlun}
\label{\detokenize{vidauki:kennsluaaetlun}}
Með fyrirvara um smávægilegar breytingar.


\begin{savenotes}\sphinxatlongtablestart\begin{longtable}[c]{|*{3}{\X{1}{3}|}}
\hline
\sphinxstyletheadfamily 
Dagsetning
&\sphinxstyletheadfamily 
Efni
&\sphinxstyletheadfamily 
Lesefni
\\
\hline
\endfirsthead

\multicolumn{3}{c}%
{\makebox[0pt]{\sphinxtablecontinued{\tablename\ \thetable{} -- framhald frá fyrri síðu}}}\\
\hline
\sphinxstyletheadfamily 
Dagsetning
&\sphinxstyletheadfamily 
Efni
&\sphinxstyletheadfamily 
Lesefni
\\
\hline
\endhead

\hline
\multicolumn{3}{r}{\makebox[0pt][r]{\sphinxtablecontinued{Framhald á næstu síðu}}}\\
\endfoot

\endlastfoot

26.08.20.
&\begin{enumerate}
\sphinxsetlistlabels{\arabic}{enumi}{enumii}{}{.}%
\item {} 
Tvinntölur

\end{enumerate}
&
1.1, 1.2, 1.3,
\\
\hline
28.08.20.
&
2. Margliður, ræð
föll og
veldisvísisföll.
&
1.4, 1.5, 1.6.
\\
\hline
02.09.20
&
3. \(\mathbb{R}\)-
og \(\mathbb{C}\)-
línulegar varpanir.
Markgildi
og samfelldni.
&
1.7.1\textendash{}1.7.4, 2.1.
\\
\hline
04.09.20.
&\begin{enumerate}
\sphinxsetlistlabels{\arabic}{enumi}{enumii}{}{.}%
\setcounter{enumi}{3}
\item {} 
Fáguð föll.

\end{enumerate}
&
2.1, 2.2.
\\
\hline
09.09.20.
&
5. Veldisvísisfallið
og lograr.
&
2.3, 2.4, 2.4, 2.5.
\\
\hline
11.09.20.
&
6. Heildi,
Cauchy-setningin og
Cauchy-formúlan.
&
3.1, 3.2, 3.3.
\\
\hline
16.09.20.
&
7. Afleiðingar
Cauchy-setningarinnar
.
&
3.3, 3.4, 3.5, 3.6,
\\
\hline&&
3.7, 3.8, 3.9, 3.10.
\\
\hline
18.09.20.
&
8. Fleiri afleiðingar
Cauchy-setningarinnar
.
&
3.3, 3.4, 3.5, 3.6,
\\
\hline&&
3.7, 3.8, 3.9, 3.10.
\\
\hline
23.09.20.
&
9. Laurent-raðir og
sérstöðupunktar.
&
4.1, 4.2.
\\
\hline
25.09.20.
&\begin{enumerate}
\sphinxsetlistlabels{\arabic}{enumi}{enumii}{}{.}%
\setcounter{enumi}{9}
\item {} 
Leifasetningin.

\end{enumerate}
&
4.3, 4.4, 4.5, 4.6.
\\
\hline
30.09.20.
&\begin{enumerate}
\sphinxsetlistlabels{\arabic}{enumi}{enumii}{}{.}%
\setcounter{enumi}{10}
\item {} 
Þýð föll.

\end{enumerate}
&
5.1, 5.2.
\\
\hline
02.10.20.
&
12. Hagnýtingar í
straumfræði.
&
5.2.
\\
\hline
07.10.20.
&\begin{enumerate}
\sphinxsetlistlabels{\arabic}{enumi}{enumii}{}{.}%
\setcounter{enumi}{12}
\item {} 
Afleiðujöfnur.

\end{enumerate}
&
6.1, 6.2, 6.3, 6.4.
\\
\hline
09.10.20.
&
14. \sphinxstylestrong{Próf.} 30\% af
lokaeinkunn\(^
1\).
&
Tvinnfallagreining.
\\
\hline
14.10.20.
&
Upprifjun á
lausnaaðferðum.
Hagnýtingar.
&
6.3, 6.4, 6.5, 6.6.
\\
\hline
16.10.20.
&
15. Tilvist og
ótvíræðni lausna.
&
6.7, 6.8.
\\
\hline
21.10.20.
&
16. Línulegar
afleiðujöfnur.
&
7.1, 7.2, 7.3.
\\
\hline
23.10.20.
&
17. Sérlausnir.
Green-föll.
&
7.4, 7.5.
\\
\hline
28.10.20.
&
18. Green-föll og
Wronski-ákveður.
&
7.4, 7.5, 7.6.
\\
\hline
30.11.20.
&\begin{enumerate}
\sphinxsetlistlabels{\arabic}{enumi}{enumii}{}{.}%
\setcounter{enumi}{18}
\item {} 
Veldaraðalausnir.

\end{enumerate}
&
8.1, 8.2, 8.3.
\\
\hline
04.11.20.
&
20. Aðferð
Frobeniusar.
&
8.4, 8.5.
\\
\hline
06.11.20.
&
21. Línuleg
afleiðujöfnuhneppi.
&
9.1, 9.2, 9.3.
\\
\hline
11.11.20.
&\begin{enumerate}
\sphinxsetlistlabels{\arabic}{enumi}{enumii}{}{.}%
\setcounter{enumi}{21}
\item {} 
\(e^{tA}\).

\end{enumerate}
&
9.4, 9.5, 9.6.
\\
\hline
13.11.20.
&
23. Útreikningar
lausna.
&
9.6, 9.7.
\\
\hline
18.11.20.
&\begin{enumerate}
\sphinxsetlistlabels{\arabic}{enumi}{enumii}{}{.}%
\setcounter{enumi}{23}
\item {} 
Laplace-ummyndun.

\end{enumerate}
&
10.1, 10.2.
\\
\hline
20.11.20.
&
25. Notkun
Laplace-ummyndunar.
&
10.3, 10.4.
\\
\hline
25.11.20.
&
26. Dæmi um notkun
fræðanna.
&\\
\hline
27.11.20.
&
Dæmatími.
&\\
\hline
\end{longtable}\sphinxatlongtableend\end{savenotes}

Í dálkinum \sphinxstylestrong{Lesefni} er vísað í kennslubók Ragnars Sigurðssonar.
\begin{itemize}
\item {} 
\DUrole{xref,std,std-ref}{genindex}

\item {} 
\sphinxhref{stae302.pdf}{Pdf-útgáfa}

\end{itemize}



\renewcommand{\indexname}{Atriðaskrá}
\printindex
\end{document}